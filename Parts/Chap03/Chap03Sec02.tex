\section{Our Most Important Innovation}\index{Our Most Important Innovation}


Organization is humankind’s most important innovation. Humans, from their earliest beginning as part of the natural world, found it necessary to collaborate and cooperate. A look back at Figure 1 in the context of the world that we observe, lends credence to the concept of emergence.

A look back at Figure 1 in the context of the world that we observe, lends credence to the concept of emergence . . . 

Figure 3.1 Contributor’s Organizational Benefits and Burdens

There are advantages to be gained from group activity. An effective cooperative system accomplishes objectives far in excess of the simple sum of its parts. Then, the pooling of activity may permit each member of the group to satisfy his wants more expediently than alone and by individual action. All the many patterns of human behavior may be placed into one of these two categories; behavior that is independent of the activity of others, and behavior undertaken as part of a group effort and towards a common objective.

\subsection{Objective of Organized Activity}\index{Objective of Organized Activity}

Objectives pursued by organizations should be directed to the satisfaction of demands resulting from human wants. Therefore, the determination of appropriate objectives for organized activity must be preceded by an effort to determine precisely what these wants are. 

Fabrycky, W.J., Comment offered in 1960 for a graduate course taught by Prof. H.G. Thueson at OSU

Industrial organizations conduct market studies to learn what consumer goods should be produced. City commissions make surveys to ascertain what civic projects will be of most benefit. Highway commissions conduct traffic counts to learn what construction programs should be undertaken.

Organizations come into being as a means for creating and exchanging utility. Their success is dependent on the appropriateness of the series of acts contributed to the system. Most of these acts are purposeful; that is, they are directed to the accomplishment of some objective. These acts are physical in nature and find purposeful employment in the alteration of the physical environment. As a result, utility is created which, through the process of distribution, makes it possible for the cooperative system to endure.

Before the industrial revolution, most productive activity was accomplished in small owner-manager enterprises, usually with a single decision maker and simple organizational objectives. Increased technology and the growth of industrial organizations made necessary the establishment of a hierarchy of objectives.  This, in turn, required a division of the management function until today a hierarchy of decision makers exists in most organizations. Each decision maker is charged with the responsibility of meeting the objectives of his organizational division. Therefore, he may be expected to pursue these objectives in a manner consistent with his view of what is good for the organization as a whole.

The function of the management process is the delineation of organizational objectives and the coordination of activity toward the accomplishment of these objectives. To maintain this system in equilibrium, the decision maker must constantly choose from among a changing set of alternatives. Each member of a set of alternatives may contribute differently to the effectiveness with which organizational objectives are achieved and the contributors satisfied. It is evident from this that managerial talent is a valuable resource.

\subsection{The Benefits of Human Organization}\index{The Benefits of Human Organization}

Organized effort often leads to economy in the accomplishment of an objective. Suppose, for example, that two men adjacent to each other are confronted with the task of lifting a box onto a loading platform, and that each of the two boxes is too heavy for one man to lift but not too heavy for two men to lift. Assume that the only practical way for one man to accomplish his task is to obtain a hand winch with which the task can be accomplished in 30 minutes’ time. If there is no coordination of effort, the cost of getting the two boxes onto the platform will be 60 worker-minutes.

Suppose that the two men had coordinated their efforts to lift the two boxes in turn and the time consumed was 1 minute per box. The two tasks would have been accomplished at the expense of 4 worker-minutes or about one-fifteenth as much time as if there had been no coordination of effort.

Coordination of human effort is so effective a means of labor saving that it may be economical to pay for effort to bring about coordination of effort. In the preceding example, effort directed to bring about coordination would result in a net labor saving of 56 worker-minutes of effort.

As a further illustration of the economics of coordination of human effort, suppose that a water well would have a value of \$100 to each of 100 families in a village of a certain undeveloped country. The head of each of the families recognizes this and on inquiry finds that a well will cost \$1,000. Each family head, being oblivious of the opportunities for coordination, abandons the well-drilling project as unprofitable. If an entrepreneur could bring about a coordination of effort in the village, the net benefit might be as follows: (100 families X \$100) - \$1,000 = \$9,000. This illustrates that entrepreneurship is a worthwhile and necessary activity in most situations involving organized activity.

It is evident that desired ends may be obtained from the environment more easily by joint action than by individual action. For example, the utility of the harmonic sounds in music is usually increased by the precise coordination of the efforts of a group of musicians. The utility of steel is increased by a complex manufacturing process which ultimately results in an automobile. Even friendship is enhanced by participation in certain forms of organized activity.

\subsection{Efficiency of Organization}\index{Efficiency of Organization}

A person who is employed by an industrial organization may be presumed to value the wages and other benefits he gets more highly than the efforts he contributes to gain them. The person who sells material to the organization must value them less than the money he receives for them, or he or she would not sell. The same may be said of the seller of equipment. Similarly, a person who loans money to an organization will, in the long run, receive more in return than he or she advances or will cease to loan money. The customer who comes with money in hand to exchange for the products of the organization may be expected to part with his money only if he values it less than he values the products he can get for it. This situation is illustrated in Figure 3.1.

So that the organization illustrated be successful, not only must the total of the satisfactions exceed the total of contributions, but each contributor’s satisfaction must exceed his contribution as he subjectively evaluates them. In other words, contributors must realize their aspirations to a satisfactory degree, or they cease to contribute. Organizations are essentially devices to which people contribute what they desire less to gain what they desire more. Unless people receive more than they put into an organization, they withdraw from it. For an organization to endure, its efficiency (output divided by input) must exceed unity.

Efficiency, therefore, is a measure of the result of cooperative action for the contributors as subjectively evaluated by them. The effective pursuit of appropriate organizational objectives contributes directly to organizational efficiency. As used here, efficiency is a measure of the want-satisfying power of the cooperative system as a whole. Thus, efficiency is the summation of utilities received from the organization divided by the utilities given to the organization, as subjectively evaluated by each contributor.
