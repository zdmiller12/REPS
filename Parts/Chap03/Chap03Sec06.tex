\section{Systems Engineering Management}\index{Systems Engineering Management}

Given a comprehensive systems engineering plan and an organizational structure, as described and developed in , the challenge becomes one of implementation. The past provides many examples where relevant and timely planning was accomplished from the beginning, only to find that the subsequent implementation process did not follow the plan. As a result, the plan became impotent by neglect. For systems engineering objectives to be accomplished successfully, a two-step management approach is recommended. First, the planning and organization for systems engineering should be completed as presented in the section. Then, the follow-on program management, implementation and control, and evaluation activities described in this chapter should be activated.

Referring to , the SEMP defines the specific requirements for the implementation of a systems engineering program (or project) with the objective of designing, developing, and bringing a new (or reengineered) system into being. Program goals and objectives, required systems engineering tasks, a work breakdown structure (WBS), task schedules, and cost projections should be included in the SEMP. In addition, an organizational approach is described, along with structure and organizational interface requirements. The SEMP, in conjunction with the system specification (Type A), basically addresses the what requirements from an overall program perspective.

With the what requirements as input, the essential next step is to implement the plan. Accordingly, the objectives of this chapter are to respond to the SEMP requirements by describing the hows as they pertain to plan implementation. Through a study of the material in, the reader should become knowledgeable about the planning and organizational requirements for systems engineering as presented in, and also about the day-to-day program management, control, and evaluation requirements described in this chapter. Remaining topics that should be reviewed and assimilated are the following

\begin{itemize}
\item Establishing specific organizational goals and objectives
\item Outsourcing requirements and the identification of suppliers
\item Providing day-to-day program leadership and direction
\item Implementing a program evaluation and feedback capability
\item Conducting a risk analysis and management function
\end{itemize}

Of special interest in this chapter is an approach for conducting a performance evaluation of a systems engineering organization, including the determination of a level of maturity for the organization. This kind of organizational assessment is in keeping with the increasing necessity for accountability that now exists in most areas of human endeavor.

Successful implementation of the concepts, principles, models, and methods of systems engineering and analysis requires the coordination of numerous technical and managerial endeavors. planning, organizational, and management matters. Although the former dominates the latter, success with the technical is not possible in the absence of the managerial.

Systems engineering may appear to be appropriate and responsive to a given need and related stakeholder interests. However, the best technical approach may not be realized, regardless of how well it embraces systems engineering methods. Effective and efficient program implementation requires timely planning, the establishment of an appropriate organizational structure, a collaborative engineering environment, and continuously applied management controls. To ensure the realization of program deliverables that meet or exceed customer expectations, an appropriate blend of technical and management attention is required. illustrate, the proper implementation of systems engineering begins with the establishment of requirements by a planning process properly initiated during the conceptual design phase of the system life cycle. Inherent within this planning is the identification and scheduling of systems engineering functions and tasks, the development of an effective organizational structure, and the establishment of a sustaining oversight and control capability. Timely feedback regarding overall program visibility and status is necessary. Planning initiated early, coordinated with the development of a systems engineering management plan (SEMP), gives confidence that the systems engineering process can proceed in an effective manner to produce the desired outcome for stakeholders.

The objective of is to provide an overview of essential management functions that pertain to the implementation of a comprehensive systems engineering program. This is accomplished through two closely related chapters. is devoted to the establishment of systems engineering planning and organization, with emphasis on the SEMP. The section that the next section follows, with primary concentration on sustaining management, control, and evaluation activities that are critical to the effective implementation of the intent of systems engineering management. These sections are mutually supportive and should be considered together.