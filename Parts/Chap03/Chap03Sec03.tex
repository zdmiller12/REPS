\section{Enterprise and Enterprise Systems}\index{Enterprise and Enterprise Systems}

A history of man is, in part, a chronicle of the attempts of man to shape cooperative systems to accomplish his objectives. From earliest times, cooperative activity contributed to the survival of mankind. Many generations labored only to meet the basic requirements of food and protection from the elements and other forms of life. Gradually this success permitted a small surplus of activity which could be directed towards objectives beyond the barest necessities of life. It may have taken the larger part of the total existence of man before a significant surplus was able to be put aside to create a meaningful culture that could be passed on to succeeding generations.

As this surplus expanded, cooperative systems became more numerous, larger in size and more specialized in objectives. Participants acquired and refined skills unique to a cooperative system. Then communities became formalized, governments were established, armies were formed to protect the participants in these communities, and architectural works were undertaken to glorify these same people and governments. The buildings and monuments of early cooperative systems are the most obvious and permanent record of these systems. Many still exist today in diverse parts of the world as evidence of early cultures.

\subsection{Cooperative Activity}\index{Cooperative Activity}

Individuals engage in purposeful activity, a significant proportion of this activity is undertaken as a part of a cooperative system.
A cooperative system may be described as a group of people interacting, one with another, and pooling their efforts toward a common objective. Cooperative activity is the activity in which these individuals engage while a part of a cooperative system.

Individuals have been described as purposeful – as engaging in activity towards specific goals. So also are cooperative systems purposeful – they have objectives. The objectives of cooperative activity are usually restricted to the achievement of goals that either cannot be accomplished by individual activity or cannot be accomplished as economically through individual action1. In effect, cooperation is a second choice. A cooperative system might be formed to sponsor, conduct, and participate in a charity dance. Such an objective could not be accomplished by an individual because this individual could not interact with other people except through a cooperative system. Another system might be formed to manufacture automobiles. This activity could be accomplished by an individual working under a shade tree but it would probably take half of a lifetime to build one such car. A cooperative system can accomplish this objective more economically; more cars can be built per man-hour of input.

A measure of the material standard of living of a society can be obtained by assessing the extent to which many forms of activity are accomplished by collective rather than individual activity. In a  primitive society, the household furniture and utensils might be built by the individual who uses them. The house in which a man lives might even be built by this same individual, perhaps with the assistance of his immediate family - a small but unspecialized cooperative system. In a more advanced society, the house, the furniture, and the utensils will probably each have been fabricated or manufactured by a specialized cooperative system devoted exclusively to that activity. This will be a more economical method. The objectives of cooperative systems should be reserved for those objectives the individual cannot accomplish, or cannot complete as economically, and they will be either social or physical ends. One system may be established to build automobiles with an anticipated cooperative life of many years. Another may come into existence to sponsor a benefit tea and may last for only a few weeks. Each cooperative system has a goal and consists of a group of people pooling their efforts towards that objective.

1 See Chris Argyris, Integrating the Individual and the Organization, New York, John Wiley, 1964, p. 35, and Chester I. Barnard, op. cit., p. 23.

\subsection{Forms of Cooperation}\index{Forms of Cooperation}

Evidence of cooperative activity throughout the world can be seen in the remains of the early civilizations of the Mediterranean and then up into Europe during the Renaissance. Only within the last two centuries, however, has the emergence of science and the technology permitted the development of the complex enterprise systems that are so characteristic of Western civilization today. This technology and the art of directing complex cooperative systems are yielding a surplus of individual time as well as a high material standard of living. This in turn permits the formation of countless other cooperative systems directed toward social and leisure activities.

The inherent strength of cooperative activity carries no guarantee of the use to which this influence will be put. The dignity and the freedom of the individual may be immersed within the common good. The objective of this cooperative activity may be the exploitation of individuals not a part of the group.

Cooperative systems exist in a variety of forms today. A cooperative system can be considered to be any group undertaking wherein the activity or behavior of an individual must be directly coordinated with the activity of behavior of an individual must be directly coordinated with the activity or behavior of one or more other individuals toward some mutual objective. Systems that may initially come to mind include those of industry and commerce such as factories and banks. The varying levels and forms of governmental cooperative systems exist too.

Cooperative Activity is Pervasive. Cooperative activity is so prevalent and assumes such a variety of forms that is is easy to conclude that organized effort is usually successful. One might infer that the failure of a cooperative system is rather unusual. In reality, the opposite is nearer the truth. It will subsequently be demonstrated that the success of a cooperative system is the exception rather than the rule. The multitudes of observable cooperative systems are the successes remaining from a much larger number of attempts. Most cooperative systems die in infancy.

In Western civilization only a few cooperative systems have survived in essentially the same form more than a few hundred years. Some religious groups, a few universities, and a small number of national and municipal governments have enjoyed extended lives. Within the industrial arena, a few automobile companies remain today where once there were many more. The cooperative systems that do survive usually have to make a deliberate and conscious attempt to perpetuate themselves. The majority of churches actively seek converts. So also do social and fraternal organizations look for new members. Industrial concerns hire new employees and look for new products to manufacture or services to provide. All cooperative systems must maintain both an internal balance and enjoy a raison d’etre. The balance must be continually adjusted and new objectives south when old goals are accomplished. The reasons for the inherent instability of cooperative systems will subsequently be developed. For the present it will suffice to note the cooperative systems exist in a variety of forms, they are very numerous, and the state of cooperative activity need only be contrasted to the state of individual activity.

\subsection{The Plan and the Purpose}\index{The Plan and the Purpose}

It is suggested that there are constructs or explanations which are applicable to all cooperative systems. It is further suggested that an understanding of these constructs will facilitate cooperative activity and enhance both the likelihood of success of the cooperative system and the realization of anticipated satisfactions by the participants in that system.

Universal constructs are probably more characteristic of the subject-matter fields classified as physical sciences. As an example of such a construct, a physicist might express Newton’s second law of motion: the time rate of change of momentum (mass X velocity) is differentiated in regard to time and the mass is assumed to be constant, then the force can be shown to be equal to the product of the mass X the rate of acceleration.

The social sciences, e.g. psychology, economics, sociology, are also able to establish constructs or explanations. These, however, are not quite as precise and do not make up such a complete web of knowledge, and some explanations may even be partially contradictory one to another. One of the reasons for this difficulty is that the social scientist has chosen the more difficult task of understanding man. The fact remains there are more likely to be different and sometimes divergent theories to explain some of the fundamentals within the total sphere of knowledge.

The study of organizations and of the management might be placed in this latter category. There are constructs or explanations but they are not as precise as we might like and there are often exceptions or extenuating circumstances.

It is suggested that an understanding of this model or construct of cooperative systems will permit a better insight into the operation of these systems. The model will permit a better understanding of the behavior of people within cooperative systems and the role of the manager in cooperative systems.