\section{Transitioning Into the Systems Age}\index{Transitioning Into the Systems Age}

Evidence suggests that the advanced nations of the world are leaving one technological age and entering another. It appears that this transition is bringing about a change in the conception of the world in which we live. This conception is both a realization of the complexity of natural and human-made systems and a basis for improvement in humankind’s management of these systems. Long-term sustainability of both human-made systems and the natural world is becoming a common desideratum.

Transition to the Systems Age spawned the systems sciences and, driven by potent technologies, established a compelling need for systems engineering. Accordingly, selected definitions of systems engineering are given at the end of this chapter. Many others exist, but are not included herein. However, in conformity with the theme of this textbook, systems engineering may be described as a technologically-based interdisciplinary process for bringing systems into being. Systems engineering is an engineering interdisciplines in its own right, with important engineering domain manifestations.

\subsection{The Machine Age}\index{The Machine Age}

Two ideas have been dominant in the way people seek to understand the world in which we live. The first is called reductionism. It consists of the belief that everything can be reduced, decomposed, or disassembled to simple indivisible parts. These were taken to be atoms in physics; simple substances in chemistry; cells in biology; and monad, instincts, drives, motives, and needs in psychology.

Reductionism gives rise to an analytical way of thinking about the world, a way of seeking explanations and understanding. Analysis consists, first, of taking apart what is to be explained, disassembling it, if possible, down to the independent and indivisible parts of which it is composed; second, of explaining the behavior of these parts; and, finally, of aggregating these partial explanations into an explanation of the whole. For example, the analysis of a problem consists of breaking it down into a set of as simple problems as possible, solving each, and assembling their solutions into a solution of the whole. If the analyst succeeds in decomposing a problem into simpler problems that are independent of each other, aggregating the partial solutions is not required because the solution to the whole is simply the sum of the solutions to its independent parts. In the industrial or Machine Age, understanding the world was taken to be the sum, or result, of an understanding of its parts, which were conceptualized as independently of each other as was possible.

The second basic idea was that of mechanism. All phenomena were believed to be explainable by using only one ultimately simple relation, cause and effect. One thing or event was taken to be the cause of another (its effect) if it was both necessary and sufficient for the other. Because a cause was taken to be sufficient for its effect, nothing was required to explain the effect other than the cause. Consequently, the search for causes was environment free. It employed what is not called ``closed-system'' thinking. Laws such as that of freely falling bodies were formulated so as to exclude environmental effects. Specially designed environments, called laboratories, were used so as to exclude environmental effects on phenomena under study. Causal-based laws permit no exception. Effects are completely determined by causes. Hence, the prevailing view of the world was deterministic. It was also mechanistic because science found no need for teleological concepts (such as functions, goals, purposes, choice, and free will) in explaining any natural phenomenon. They considered such concepts to be unnecessary, illusory, or meaningless. The commitment to causal thinking yielded a conception of the world as a machine; it was taken to be like a hermetically sealed clock - a self-contained mechanism whose behavior was completely determined by its own structure.

The Industrial Revolution brought about mechanization, the substitution of machines for people as a source of physical work. This process affected the nature of work left for people to do. They no longer did all the things necessary to make a product; they repeatedly performed a simple operation in the production process. Consequently, the more machines were used as a substitute for people at work, the more workers were made to behave like machines. The dehumanization of work was an irony of the Industrial Revolution and Machine Age.

\subsection{The Systems Age}\index{The Systems Age}

Although eras do not have precise beginnings and endings, the 1940s can be said to have contained the beginning of the end of the Machine Age and the beginning of the Systems Age. This new age is the result of a new intellectual framework in which the doctrines of reductionism and mechanism and the analytical mode of thought are being supplemented by the doctrines of expansionism, teleology, and a new synthetic (or systems) mode of thought.

Expansionism is a doctrine that considers all objects and events, and all experiences of them, as parts of larger wholes. It does not deny that they have parts, but it focuses on the wholes of which they are part. It provides another way of viewing things, in a way that is different from, but compatible with, reductionism. It turns attention from ultimate elements to a whole with unrelated parts - to systems.

Preoccupation with systems brings with it the synthetic mode of thought. In the analytic mode, an explanation of the whole was derived from explanations of its parts. In synthetic thinking, something to be explained is viewed as part of a larger system and is explained in terms of its role in that larger system. The Systems Age is more interested in putting things together than in taking them apart.

Analytic thinking is outside-in thinking; synthetic thinking is inside-out thinking. Neither negates the value of the other, but by synthetic thinking on can gain understanding that cannot be obtained through analysis, particularly of collective phenomena.

The synthetic mode of thought, when applied to systems problems, is called systems thinking or the systems approach. This approach is based on the observation that when each part of a system performs as well as possible, the system as a whole may not perform as well as possible. This follows from the fact that the sum of the functioning of the parts is seldom equal to the functioning of the whole. Accordingly, the synthetic mode seeks to overcome the often observed predisposition to perfect details and ignore system outcomes.

Because the Systems Age is teleologically oriented, it is preoccupied with systems that are goal seeking or purposeful; that is, systems that offer the choice of either means or ends, or both. It is interested in purely mechanical systems only insofar as they can be used as enablers for purposeful systems. Furthermore, the Systems Age is largely concerned with purposeful systems, some of whose parts are purposeful; in the human domain, these are called social groups. The most important class of social groups is the one containing systems whose parts perform different functions that have a division of functional labor; these are called organizations.

In the Systems Age, attention in focused on groups and on organizations as parts of larger purposeful societal systems. Participantive management, collaboration, group decision making, and total quality management are new working arrangements within the organization. Among organizations is now found a keen concern for social and environmental factors, with economic competition continuing to increase worldwide.