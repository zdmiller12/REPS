\section{Sectors Comprising Our World}\index{Sectors Comprising Our World}\label{sec:sectorsComprisingWorld}

The world in which we live may be divided into the natural world and the human-made world.  Included in the former are all elements of the world that came into being by natural processes. The human-made world is made up of all human-originated systems, their product subsystems (including structures and services), and their other subsystems (such as those for production and support).  But it is the advent of the human-made that has resulted in a human-modified world as the actual world in which we live.

Systems are as pervasive as the universe in which they exist.  They are as grand as the universe itself or as infinitesimal as the atom.  This is, however, beyond the scope of STEA and requires reduction into sectors comprising our world.  Systems appeared first in natural forms, but with the advent of human beings a variety of human-made systems have come into existence.

The origin of systems gives the most important classification opportunity.  \textit{Natural systems} are those that came into being by natural processes.  \textit{Human-made systems} are those in which human beings have intervened through component, attributes, and relationships.  The \textit{human-modified system} is a natural system into which a human-made system has been integrated.

\subsection{The Natural World}\index{The Natural World}\label{sec:NaturalWorld}

The natural world is made up of all systems, including humans, that came into being by natural processes without human involvement. \textit{Natural systems} are those that came into being by natural processes. \textit{Human-made systems} are those in which human beings have intervened through components, attributes, and relationships. 

All human-made systems, when brought into being, are embedded into the natural world. Important interfaces often exist between human-made systems and natural systems. Each affects the other in some way. The effect of human-made systems on the natural world has only recently become a keen subject for study by concerned people, especially in those instances where the effect is undesirable. In some cases, this study is facilitated by analyzing the natural system as a human-modified system.

When designing a human-made system, undesirable effects can be minimized—and the natural system can sometimes be improved—by engineering the larger human-modified system instead of engineering only the human-made system. If analysis, evaluation, and validation of the human-modified system are appropriate, then the boundary of the environmental system—drawn to include the human-made system—should be considered the boundary of the human-modified system.

Natural systems exhibit a high degree of order and equilibrium. This is evidenced in the seasons, the food chain, the water cycle, and so on. Organisms and plant life adapt themselves to maintain equilibrium with the environment. Every event in nature is accompanied by an appropriate adaptation, one of the most important being that material flows are cyclic. In the natural environment there are no dead ends, no wastes, only continual re-circulation and regeneration.

Only recently have significant human-made systems appeared. These systems make up the human-made world, their chief engineer being human. The rapid evolution of human beings is not adequately understood, but their arrival has significantly affected the natural world, often in undesirable ways. Primitive beings had little impact on the natural world, for they had not yet developed potent and pervasive technologies.

An example of the impact of human-made systems on natural systems is the set of problems that arose from building the Aswan Dam on the Nile River. Construction of this massive dam ensures that the Nile will never flood again, solving an age-old problem. However, several new problems arose. The food chain was broken in the eastern Mediterranean, thereby reducing the fishing industry. Rapid erosion of the Nile Delta took place, introducing soil salinity into Upper Egypt. No longer limited by periodic dryness, the population of bilharzia (a waterborne snail parasite) has produced an epidemic of disease along the Nile. These side effects were not adequately anticipated by those responsible for the project. A system view encompassing both natural and human-made elements, as a human-modified system, might have led to a better solution to the problem of flooding. (Look for a better / more general example).


An example of the impact of human-made systems on natural systems is the set of problems that arose from building the Aswan Dam on the Nile River\footnote{\href{http://www.edc.uri.edu/temp/ci/ciip/FallClass/Docs_2006/UrbanWaterfronts/Abu-Zeid\%20and\%20El-Shibini.pdf}{Aswan Dam on the Nile}}. Construction of this massive dam ensures that the Nile will never flood again, solving an age-old problem. However, several new problems arose. The food chain was broken in the eastern Mediterranean, thereby reducing the fishing industry. Rapid erosion of the Nile Delta took place, introducing soil salinity into Upper Egypt. No longer limited by periodic dryness, the population of bilharzia (a waterborne snail parasite) has produced an epidemic of disease along the Nile. These side effects were not adequately anticipated by those responsible for the project. A system view encompassing both natural and human-made elements, as a human-modified system, might have led to a better solution to the problem of flooding.

\subsection{The Human-Made World}\index{The Human-Made World}\label{subsec:humanMadeWorld}

The human-made world is made up of all systems wherein humans have intervened through components, attributes, and relationships.

All human-made systems, when brought into being, are embedded into the natural world. Important interfaces often exist between human-made systems and natural systems. Each affects the other in some way. The effect of human-made systems on the natural world has only recently become a keen subject for study by concerned people, especially in those instances where the effect is undesirable. In some cases, this study is facilitated by analyzing the natural system as a human-modified system.

When designing a human-made system, undesirable effects can be minimized - and the natural system can sometimes be improved - by engineering the larger human-modified system instead of engineering only the human-made system. If analysis, evaluation, and validation of the human-modified system are appropriate, then the boundary of the environmental system - drawn to include the human-made system - should be considered a boundary of the human-modified system.
	
Natural systems exhibit a high degree of order and equilibrium. This is evidenced in the seasons, the food chain, the water cycle, and so on. Organisms and plant life adapt themselves to maintain equilibrium with the environment. Every event in nature is accompanied by and appropriate adaptation, one of the most important being that material flows are cyclic. In the natural environment there are no dead ends, no wastes, only continual recirculation and regeneration.

Only recently have significant human-made systems appeared. These systems make up the human-made world, their chief engineer being human. The rapid evolution of human beings is not adequately understood, but their arrival has significantly affected the natural world, often in undesirable ways. Primitive beings had little impact on the natural world, for they had not yet developed potent and pervasive technologies.

	Interesting. From a biologist’s point of view the following...
    
Anything a beaver, or even army ants, or colonial termites make is natural because it is made by a natural entity. Presumably evolution has very strongly selected for that which is made by them by eliminating many other alternatives as they arose. This ensures that at least for the near term, such innovations are fit within the environment. Until the environment changes which it usually does in the very long term.

But then why make a distinction for humans?  We have evolved also. We are natural entities. Why wouldn’t anything we make be burdened with the term “artificial” than any other thing made by a tool-using social organism?  Persons focusing only on these aspects would see the natural vs artificial controversy as empty and unnecessary.

Now, I did not make up that distinction but do use it often. All human systems including our socio-economic and socio-political institutions I consider immature artefacts. In fact, it was Nobel Laureate Herbert Simon, in whose honor NECSI grants an annual award, and whose famous book was titled, “Sciences of the Artificial” who first popularized use of the term.

Perhaps it is because scientists realize that anything man makes can be engineered so quickly that it is not subject to natural selection and evolution at first and then not even for a very long time afterward if at all. So, products that reflect more greed than adaptation to context, more arrogance than fit within natural parameters, surround our civilization. This might give some meaning to artificial. Further, the distinction might lead to a regime in which prescription and values become an important part of the process, recycling and fit in environment and cost-to-benefit an important part of the process in addition to making a buck. Len.

\subsection{Natural Versus Human-Made World}\index{Natural Versus Human-Made World}\label{subsec:naturalVsHumanWorld}

How do we classify a beaver dam?  Is it artificial or natural?  Clearly it is not designed and made by humans. This should keep the ontology folks busy.
	
To what extent is human activity not considered a part of nature? Or, do we define as synthetic anything done by the hand of man?

The universe is composed of stateful objects that undergo transformations via processes – patterns of object transformations that we as humans conceptualize in order to be able to think about cause-and-effect along the timeline.

We distinguish between natural and man-made systems (and I think they should be called by different names) because value and free will apply only to humans.

The definition of system then depends on whether it is natural or artificial.

An artificial system is an object that fulfills a function – a value-adding process from some beneficiary’s perspective.

A natural system is a collection of natural objects that are structured and behave following laws of nature.

In the discussion of “natural” vs. “man-made”, it is clear that we must decide whether to consider “man” to be a part of “nature” in the context of the discussion.

Men build dams, but we do not consider the Hoover Dam to be natural. Beavers make dams, but we deem them to be natural. Therefore, one might conclude the “we” do not consider man to be “natural” (neither man, nor his actions) in the context of the discussion. This could get downright philosophical and theological, but that is not the point.

So, are “natural” and “man-made” the words we want to use to define the concepts of “everything man DOES NOT DO” from “everything man DOES DO?”  Then, we must determine how to classify that which arises from an interaction between the two.
    
\subsection{The Human-Modified World}\index{The Human-Modified World}\label{subsec:humanModifiedWorld}

The Human-Modified World is composed of natural systems from which resources are extracted and human-made systems are embedded. We don’t live in the natural world. We don’t live in the human-made world. We live in a hybrid world at the intersection of the two already identified as the Human-Modified World. There is nothing that we do as humans that does not have one foot in the natural world and the other in the human-made world.

Since the very beginning of their existence, humans have struggled to cope with the world into which they…appeared. Knowledge, organization, and economics are the primary enablers. Human conceived existence enablers . . . 

A human-modified system is a natural system into which a human-made system has been integrated as a subsystem.

Twentieth Century man, more than his ancestors, must attempt to understand the varied peoples with whom he shares an increasingly small planet. To reach this understanding he needs to know the cultures which molded other peoples’ outlook, the history that carried them to this point.

How to select the civilizations that must be examined in a limited series of books on the history of the world’s cultures?  That is the subject of Jaques Barzun’s introduction to the Time-Life series entitled The Great ages of Man. Mr. Barzun, Dean of Faculties and Provost of Columbia University, is one of the pre-eminent cultural historians of this generation. He describes how the “revolution … in our conception of humanity” wrought by the emergence of “dozens of new peoples, new states, and new pasts” has made essential the realization that “nothing human is alien.”

In explaining the criteria for the selection of historic cultures examined in this series, he also suggests the path that present-day cultures may follow in the future.

THE EDITORS OF TIME-LIFE BOOKS (1965 Time Inc.) This book strives to justify the human-modified world as the ultimate system level at which the viability of all that is human-made should be judged. Systems thinking reveals the shortcomings of this obvious but limiting bifurcation. Consider stepping back from “The inherently presumptuous notion of human-made in favor of a continuum that starts with acceptance of the natural world as the original, or default state. Emanating from the ‘created’ state comes recognition of the advent of increasingly potent modifications by humans. The complex embedded systems that now serve humanity constitute a progressively advancing continuum between the beginning and now.”

As an example, the impact of human-made systems on natural systems is the set of problems that arose from building the Aswan Dam on the Nile River. Construction of this massive dam ensures that the Nile will never flood again, solving an age-old problem. However, several new problems arose. The food chain was broken in the eastern Mediterranean, thereby reducing the fishing industry. Rapid erosion of the Nile Delta took place, introducing soil salinity into Upper Egypt. No longer limited by periodic dryness, the population bilharzia (a waterborne snail parasite) has produced an epidemic of disease along the Nile. These side effects were not adequately anticipated by those responsible for the project. A systems view encompassing both natural and human-made elements, as a human-modified system, might have led to a better solution to the problem of flooding.

When brought into being, all that is human-made are embedded into the natural world. Important interfaces exist between human-made systems and natural systems. Each affects the other in many ways. The effect of human-made systems on the natural world has only recently become a keen subject for consideration by concerned people, especially when the effect is deemed undesirable.

Only recently have significant and complex human-made systems appeared. These systems make up the human-made world, their chief engineer being human. The rapid appearance of human beings is not adequately understood, but human presence has significantly affected the natural world, too often in undesirable ways. Primitive beings had little impact on the natural world, for they had not yet developed pervasive and potent enabling technologies.