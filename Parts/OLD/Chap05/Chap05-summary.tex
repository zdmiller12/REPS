In this chapter, some system definitions and systems science concepts were presented to provide a basis for the study of systems engineering and analysis. They include definitions of system characteristics, a classification of systems into various types, consideration of the current state of systems science, and a discussion of the transition to the Systems Age. Finally, the chapter presents technology and the nature and role of engineering in the Systems Age and ends with a number of commonly accepted definitions of systems engineering.</para>

Upon completion of , the reader will have obtained essential insight into systems and systems thinking, with an orientation toward systems engineering and analysis. The system definitions, classifications, and concepts presented in this chapter are intended to impart a general understanding about the following:

\begin{enumerate}
\item System classifications, similarities, and dissimilarities
\item The fundamental distinction between natural and human-made systems
\item The elements of a system and the position of the system in the hierarchy of systems
\item The domain of systems science, with consideration of cybernetics, general systems Theory, and systemology
\item Technology as the progenitor for the creation of technical systems, recognizing its impact on the natural world
\item The transition from the machine or industrial age to the Systems Age, with recognition of its impact upon people and society
\item System complexity and scope and the demands these factors make on engineering in the Systems Age
\item The range of contemporary definitions of systems engineering used within the profession
\end{enumerate}

Although this book focuses on the engineering of systems and on systems analysis, it would not be intellectually prudent to begin the discussion at that level. Upon examination, it is evident that both the engineering and the analysis aspects of the focus are directed to systems. Accordingly, this chapter is devoted to helping the reader gain essential insight into systems in general, and systems thinking in particular, with orientation toward the engineering and analysis of technical systems.

System definitions, a discussion of system elements, and a high-level classification of systems provide an opening panorama. It is here that a consideration of the origin of systems provides an orientation to natural and human-made domains as an overarching dichotomy. The importance of this dichotomy cannot be overemphasized in the study and application of systems engineering and analysis. It is the suggested frame of reference for considering and understanding the interface and impact of the human-made world on the natural world and on humans.

Individuals interested in obtaining as in-depth appreciation for this interface and the mitigation of environmental impacts are encouraged to read T.E. Graedel and B.R. Allenby, Industrial Ecology, 2nd ed., Prentice Hall, 2003. Also of contemporary interest is the issue of sustainability treated as part of an integrated approach to sustainable engineering by P. Stasinopoulos, M.H. Smith, K. Hargroves, and C. Desha, Whole System Design, Earthscan Publishing, 2009. These works are recommended as an extension to this chapter (as well as Chapter 16), because they illuminate and address the sensitive interface between the natural and the human-made.