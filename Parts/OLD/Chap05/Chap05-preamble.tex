Activities of analysis and design are not ends in themselves but are a means for satisfying human wants. Thus, modern engineering has two aspects. One aspect concerns itself with the materials and forces of nature; the other is concerned with the needs of people.

According to the definition of engineering adopted by the Accreditation Board for Engineering and Technology (ABET), ``Engineering is the profession in which a knowledge of the mathematical and natural sciences gained by study, experience, and practice is applied with judgment to develop ways to utilize economically the materials and forces of nature for the benefit of mankind.'' This definition is understood herein to encompass both systems and products, with the product often being a structure or service.

As unnecessary as it might seem, this chapter is intended to establish engineering as the foundation for systems engineering. Alternative equivalent routes to the study and practice of systems engineering are acknowledged. By, by accepting the principle that the alteration of physical factors is key to human want and purpose satisfaction, we are compelled to stay close to classical physics and its derivatives.