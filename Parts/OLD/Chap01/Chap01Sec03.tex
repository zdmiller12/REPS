Technology is broadly defined as the branch of knowledge that deals with the mechanical and industrial arts, applied science, and the engineering, or the sum of the ways in which social groups provide themselves with the material objects and the services of their civilization. A key attribute of a civilization is the inherent ability and the knowledge to maintain its store of technology. Modern civilizations possess pervasive and potent technology that makes possible needed systems manifested as products, which include structures and services.

\subsection{Technical Systems}\index{Technical Systems}

Science and technology is a phrase used often. Translated into its systems counterpart, this phrase prompts consideration of the link between systems science and technical systems. Technical or engineered systems have their foundation in both the natural and the systems sciences. They are a prominent and pervasive sector of the human-made world.

The phrase technical system may be used to represent all types of human-made artifacts, including technical products and processes. Accordingly, the technical system is the subject of the collection of activities that are performed by engineers within the processes of engineering design, including generating, retrieving, processing, and transmitting product information. It is also the subject of production processes, including work preparation and planning. It is also the subject of many economic considerations, both within enterprises and society.

In museums, thousands of technical objects are on display, and they are recognized as products of technology. Their variety of functions, form, size, and so forth tends to obscure common properties and features. But vast variety also exists in nature, and in those circumstances clearly defined kingdoms of natural objects have been defined for study in the natural sciences. Likewise, attempts have been made to define terms that conceptually describe classes of technical objects.

Technical objects can be referred to simply as objects, entities, things, machines, implements, products, documents, or technical works. The results of a manufacturing activity, as the conceptual content of technology, can be termed artifacts or instrumentum. Such definitions are meant to include all manner of machines, appliances, implements, structures, weapons, and vessels that represent the technical means by which humans achieve their ends. But, to be complete, this definition must recognize the hierarchical nature of systems and the interactions that occur between levels in the hierarchy. For example, the ``system'' of interest may be a transportation system, an airline system within the transportation system, or an aircraft system contained within the airline system.

Little difficulty exists in the classification of systems as either natural, technical (human-made), or human-modified. But it is difficult to classify technical systems accurately. One approach is to classify in accordance with the well-established subdivisions of technology in industry, for example, civil engineering, electrical engineering, and mechanical engineering. However, from a practical and organizational viewpoint, this does not permit a precise definition of a mechanical system or electrical system because no firm boundary can be established by describing these systems as outcomes of mechanical or electrical engineering.

Modern developments of technical systems have generally blurred the boundaries. Electronic and computer products, especially software, are increasingly used together with mechanical and human interfaces. Each acts as a subsystem to a system of greater complexity and purpose. Most systems in use today are hybrids of the simple systems of the past.

\subsection{Technological Growth and Change}\index{Technological Growth and Change}

Technological growth and change is occurring continuously and is initiated by attempts to respond to unmet deficiencies and by attempts to perform ongoing activities in a more effective and efficient manner. In addition, changes are being stimulated by political objectives, social factors, and ecological concerns.

As examples, environmental concerns have resulted in recent legislation and regulations requiring new methods for crop protection from insects, new means for the disposal of medical waste, and new methods for treating solid waste. Concern for shortages of fossil fuel sources as well as ecological impacts brought about a great focus on energy conservation and alternative energy sources. These and other comparable situations were created through both properly planned programs and as a result of panic situations. A common outcome is that all have stimulated beneficial technological innovation.

The world is increasing in complexity because of human intervention. Through the advent of advanced technologies, transportation times have been greatly reduced, and vastly more efficient means of communication have been introduced. Every aspect of human existence has become more intimate and interactive. The need for integration of ideas and conflict resolution becomes more important. At the same time, increasing populations and the desire for larger and better systems is leading to the accelerated exploitation of resources and increased environmental impact. A variety of technically literate specialists, if properly organized and incentivized, can meet most needs that arise from technological advancement and change.

Technology and Society. Human society is characterized by its culture. Each human culture manifests itself through the media of technology. In turn, the manifestation of culture is an important indicator of the degree to which a society is technologically advanced

The entire history of humankind is closely related to the progress of technology. But, technological progress is often stressful on people and their cultures alike. This need not be. The challenge should be to find ways for humans to live better lives as a result of new technological capability and social organization structure.

In general, the complexity of systems desired by societies is increasing. As new technologies become available, the pull of ``want'' is augmented by a push to incorporate these new capabilities into both new and existing systems. The desire for better systems produces an ever-changing set of requirements. The identification of the ``true'' need in answer to a problem and the elicitation of ``real'' requirements is, in itself, a technological challenge.

Transition from the past to present and future technological states is not a one-step process. Continuing technical advances become available to society as time unfolds. Societal response is often to make one transition and then to adopt a static pattern of behavior. A better response would be to seek new and well-thought-out possibilities for continuous advancement. Improvement in technological literacy should increase the population of individuals capable of participating in this desirable activity. One key to imparting this literacy is the communication technologies now expanding at a rapid pace. Thus, technology in this sphere may act favorably to aid the understanding and subsequent evaluation by society of technologies in other spheres.