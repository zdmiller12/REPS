\section{Detail Design and Development}\index{Detail Design and Development}

Having established the top-level requirements for the overall system as in Section 8.2 and the preliminary design requirements as in Section 8.3, the design process from this point forward is essentially evolutionary by nature. Referring to Figure 4.9, the design team has been established with the overall objective of integrating the various system elements into a final system configuration. Such elements include not only mission-related hardware and software, but people, real estate and facilities, data/information, consumables, and the resources necessary for the operation and sustaining support of the system throughout its planned life cycle. The integration of these elements is emphasized in Figure 5.2, which is an extension of Figure 4.9.

Include Figure 5.2 Here

Returning again to the example of a regional public transportation authority facing the problem of increasing the capacity for two-way traffic flow across a river to connect communities. The output from the Conceptual Design phase (in Chapter 3) was the tentative selection of an over-the-water approach from among several mutually exclusive river-crossing concepts. Its effect was to initiate and support Preliminary Design phase activities (in Chapter 4) to investigate a number of competing bridge configuration alternatives. The pier and superstructure configuration was indicated there as being the preferred approach. Choice of this approach now enables the Detail Design phase, with focus on static and operational subsystems and associated components.

Allocated requirements emanate from the tentatively chosen (preliminary) design and are of a lower-level appropriate for use in the design and development of components comprising the pier and superstructure bridge system. These lower-level requirements become the design criteria for subsystems and components, components that include the piers and superstructure plus other system components such as abutments, footings, toll collection, maintenance capability, and so forth.

Preliminary design evolves from ``system'' design decisions, which are determined through the definition of system operational requirements, the maintenance and support concept, and the identification and prioritization of TPMs. They are the criteria by which preliminary design alternatives are judged on the way to selecting a preferred preliminary design. These requirements are then documented through the preparation of the system specification (Type A), described in Section 3.9. From this top-level specification, requirements for the design of subsystems and the major elements of the system are defined through an extension of the functional analysis and allocation, the conduct of design trade-off studies, and so on. This involves an iterative process of top-down/bottom-up design, which continues until the next lowest level of system components are identified and configured.

Lower-level requirements then emanate from the allocated requirements from the tentatively chosen (preliminary) bridge design. These lower-level requirements become the design criteria for subsystems and components of the pier and superstructure bridge, components that include the piers and superstructure plus other system components such as abutments, footings, toll booths, and maintenance capability. These lower-level requirements are then specified through development, product, process, and/or material specifications.</title><section id=”ch05lev1fm” role=”fm”><title id=”ch05lev1fm.title”/></section>

The detail design and development phase of the system life cycle is a continuation of the iterative development process illustrated in  and . The definition of system requirements, the establishment of a top-level functional baseline, and the preparation of a system specification (Type A) are described in . This, in turn, leads to the definition and development of subsystems and major elements of the system as described in . Functional analysis, the allocation of design-to requirements below the system level, the accomplishment of synthesis and trade-off studies, the preparation of lower-level specifications (Types B-E), and the conduct of formal design reviews all provide a foundation upon which to base detailed design decisions that go down to the component level.

With the functional baseline developed as an output of the conceptual system design phase and with the allocated baseline derived during the preliminary design phase, the design team may now proceed with reasonable confidence in the realization of specific components as well as the ``make-up'' of the system configuration at the lowest level in the hierarchy. Realization includes the accomplishment of activities that (1) describe subsystems, units, assemblies, lower-level components, software modules, people, facilities, elements of maintenance and support, and so on, that make up the system and address their interrelationships; (2) prepare specifications and design data for all system components; and (3) acquire and integrate the selected components into a final system configuration.

This section addresses eight essential steps in the detail design and development process, while simultaneously providing an understanding of the intricacy of detail design and development within the systems engineering context. This includes

\begin{itemize}
\item Developing design requirements for all lower-level components of the system
\item Implementing the necessary technical activities to fulfill all design objectives
\item Integrating system elements and activities
\item Selecting and utilizing design tools and aids
\item Preparing design data and documentation
\item Developing engineering and prototype models
\item Implementing a design review, evaluation, and feedback capability
\item Incorporating design changes as appropriate
\end{itemize}

It is important for these steps to be thoroughly understood and reviewed, as they fall within the overall systems engineering process. As a learning objective, the intent is to provide a relatively comprehensive approach that addresses the detailed aspects of design. The chapter summary and extensions contribute insights by going beyond the general references cited in and by calling attention to applicable design standards and supporting documentation.

Specific requirements at this stage in the system design process are derived from the system specification (Type <emphasis>A</emphasis>) and evolve through applicable lower-level specifications (Types B–E), as illustrated in . Included within these specifications are applicable design-dependent parameters (DDPs), technical performance measures (TPMs), and supporting design-to criteria leading to identification of the specific characteristics (attributes) that must be incorporated into the design configuration of elements and components. This is influenced through the requirements allocation process illustrated in , where the appropriate built-in characteristics must be such that the allocated quantitative requirements in the figure will be met.

Given this top-down approach for establishing requirements at each level in the system hierarchical structure (refer to ), the design process evolves through the iterative steps of synthesis, analysis, and evaluation, and to the definition of components leading to the establishment of a <emphasis>product baseline</emphasis>, as shown in . At this point, the procurement and acquisition of system components begin, components are tested and integrated into a next higher entity (e.g., subassembly, assembly, and unit), and a physical model (or replica) of the system is constructed for test and evaluation. The integration, test, and evaluation steps constitute a bottom-up approach, and should result in a configuration that can be assessed for compliance with the initially specified customer requirements. This top-down/bottom-up approach is guided by the steps in the ``vee'' process model shown in 

Progressing through the system design and development process in an expeditious manner is essential in today’s competitive environment. Minimizing the time that it takes from the initial identification of a need to the ultimate delivery of the system to the customer is critical. This requires that certain design activities be accomplished on a concurrent basis. (an extension of ) further illustrates the importance of ``concurrency'' in system design.

Referring to , the designer(s) must think in terms of the four life cycles and their interrelationships, concurrently and in an integrated manner, in lieu of the sequential approach to design often followed in the past. The realization of this necessity became readily apparent in the late 1980s and resulted in concepts promulgated as <emphasis>simultaneous engineering, concurrent engineering, integrated product development</emphasis> (IPD), and others. These concepts must be inherent within the systems engineering process if the benefits of that process are to be realized.

\subsection{The Evolution of Detail Design}\index{The Evolution of Detail Design}

The evolution of detail design is based on the results from the requirements established during the conceptual and preliminary system design phases. As an example, for the river crossing problem addressed by a bridge system, the primary top-level requirements were identified in conceptual design (refer to  and ). These requirements were decomposed further in 

Illustration 1, expanded during the preliminary design phase through the functional analysis and allocation of requirements down to major subsystems to include the roadway and railbed, passenger walkway and bicycle path, toll collection facilities, and the maintenance and support infrastructure ; and then the more detailed design-to requirements for the various lower-level elements of the system are defined (in this chapter) down to the bridge substructures, foundations, piles and footings, retaining walls, toll collectors, lighting, and so on. This top-down/bottom-up process is illustrated in 

Referring to the figure, the design-to requirements are identified from the ``top down,'' with the cross-hatched block defined in conceptual design along with the allocation and requirements for the major subsystems noted by the shaded blocks (superstructure, substructure, etc.). These requirements are further expanded, through functional analysis and allocation, during the preliminary design phase to define the specific requirements for the lower-level elements of the bridge represented by the white blocks in the figure (road and railway deck, piles/piers, footings, toll facilities, etc.). The basic ``requirements'' for the river crossing bridge are defined and allocated (i.e., ``driven'') from the ``top down,'' while the detailed design and follow-on construction is accomplished from the ``bottom up.'' This is why one often invokes a top-down/bottom-up process versus assuming only a ``bottom-up'' approach (refer to 

Having established the basic top level requirements for the overall system as in and the preliminary design requirements as in the design process from this point forward is essentially evolutionary by nature as described earlier. Referring to , the design team has been established with the overall objective of integrating the various system elements into a final system configuration. Such elements include not only mission-related hardware and software but also people, real estate and facilities, data/information, consumables, and the materials and resources necessary for the operation and sustaining support of the system throughout its planned life cycle. The integration of these elements is emphasized in , an extension of 

Detail design evolution follows the basic sequence of activities shown in . The process is iterative, proceeding from system-level definition to a product configuration that can be constructed or produced in multiple quantities. There are “checks and balances” in the form of reviews at each stage of design progression and a feedback loop allows for corrective action as necessary. These reviews may be relatively informal and occur continuously, as compared with the formal design reviews scheduled at specific milestones. In this respect, the process is similar to the synthesis, analysis, evaluation, and product definition, accomplished in the preliminary design stage, except that the requirements are at a lower level (i.e., units, assemblies, subassemblies, etc.). As the level of detail increases, actual definition is accomplished through the development of data describing the item being designed. These data may be presented in the form of a digital description of the item(s) in an electronic format, design drawings in physical and electronic form, material and part lists, reports and analyses, computer programs, and so on.

The actual process of design iteration may occur through the use of the World Wide Web (WWW), a local area network (LAN), the use of telecommunications and compressed video conferences, or some equivalent form of communication. The design configuration selected may be the best possible in the eyes of the responsible designer. However, the results are practically useless unless properly documented, so that others can first understand what is being conveyed and then be able to translate and convert the output into an entity that can be constructed (as is the case for the river crossing bridge) or produced in multiple quantities (see on product and system categories).

\subsection{Integrating System Elements and Activities}\index{Integrating System Elements and Activities}

An important output from the functional analysis and allocation process is the identification of various elements of the system and the need for hardware (equipment), software, people, facilities, materials, data, or combinations thereof. The objective is to conduct the necessary trade-off analyses to determine the best way to respond to the hows. The result may take the form illustrated in and , which identify the major elements of a system.

Given the basic configuration of system elements, the designer must decide how best to meet the need in selecting a specific approach in responding to an equipment need, a software need, and so on. For example, there may be alternative approaches in selecting a specific resource, such as illustrated in , with the following steps taken (in order of precedence) in arriving at a satisfactory result:

\begin{enumerate}
\item Select a standard component that is commercially available and for which there are a number of viable suppliers; for example, a commercial off-the-shelf (COTS) item, or equivalent. The objective is, of course, to gain the advantage of competition (at reduced cost) and to provide the assurance that the appropriate maintenance and support will be readily available in the future and throughout the system life cycle when required, or
\item Modify an existing commercially available off-the-shelf item by providing a mounting for the purposes of installation, adding an adapter cable for the purposes of compatibility, providing a software interface module, and so on. Care must be taken to ensure that the proposed modification is relatively simple and inexpensive and doesn’t result in the introduction of a lot of additional problems in the process, or
\item Design and develop a new and unique component to meet a specific functional requirement. This approach will require that the component selected be properly integrated into the overall system design and development process in a timely and effective manner
\end{enumerate}

The most cost-effective solution seems to favor the utilization of commercial off-the-shelf (COTS) components, as the acquisition cost, item availability time, and risks associated with meeting a given system technical requirement are likely to be less. In any event, the decision-making process may occur at the subsystem level early in preliminary design, at the configuration-item level, and/or at the unit level. These decisions will be based on factors including the functions to be performed, availability and stability of current technology, number of sources of supply and supplier response times, reliability and maintainability requirements, supportability requirements, cost, and others. From the resulting decisions, the requirements for component acquisition (procurement) will be covered through the preparation of either a product specification (Type C) or <emphasis>process specification</emphasis> (Type D) for COTS components, or a development specification (Type B) for newly designed or modified components.

When the ultimate decision specifies the design and development of a new, or unique, system element, the follow-on design activity will include a series of steps, which evolve from the definition of need to the integration and test of the item as part of the system validation process. For example, the functional analysis in leads to the identification of <emphasis>hardware</emphasis>, <emphasis>software</emphasis>, and <emphasis>human</emphasis> requirements. The development of hardware usually leads to the identification of units, assemblies, modules, and down to the component-part level. The software acquisition process often involves the identification of computer software configuration items (CSCIs), computer software components (CSCs) and computer software units (CSUs), the development of code, and CSC integration and testing. The development of human requirements includes the identification of individual tasks, the combining of tasks into position descriptions, the determination of personnel quantities and skill levels, and personnel training requirements.

While  addresses only the three ``mini'' life cycles mentioned (i.e., hardware, software, and human), there may be other resource requirements evolving from the functional analysis to include facilities, data/documentation, elements of the maintenance and support infrastructure, and so on.. Associated with each will be a life cycle unique to the particular application.

There must be some form and degree of communication and integration throughout and across these component life cycles on a continuing basis. Quite often, design engineers become so engrossed in the development of hardware that they neglect the major interfaces and the impact that hardware design decisions have on the other elements of the system. Software engineers, like hardware engineers, often operate strictly within their own domain without considering the necessary interfaces with the hardware development process. On occasion, the hardware development process will lead to the need for software at a lower level. Further, consideration of the human being, as an element of the system, is often ignored in the design process altogether. Because of these practices in the past, the integration of the various system elements has not occurred until late in the detail design and development phase during systems integration and testing (refer to Block 2.3 in , often resulting in incompatibilities and the need for last-minute costly modifications in order to ``make it work!''

Thus, a primary objective of systems engineering is to ensure the proper coordination and timely integration of all system elements (and the activities associated with each) from the beginning. This can be accomplished with a good initial definition of requirements through the preparation of well-written specifications, followed by a structured and disciplined approach to design. This includes the scheduling of an appropriate number of formal design reviews to ensure the proper communications across the project and to check that all elements of the system are compatible at the time of review. In the event of incompatibilities, the hardware design effort should not be allowed to proceed without first ensuring that the software is compatible; the software development effort should not be allowed to proceed without first ensuring compatibility with the hardware; and so on.

\subsection{Design Tools and Aids}\index{Design Tools and Aids}

The successful completion of the design process depends on the availability of the appropriate tools and design aids that will help the design team in accomplishing its objectives in an effective and efficient manner. The application of computer-aided engineering (CAE) and computer-aided design (CAD) tools enables the projection of many different design alternatives throughout the life cycle. At the early stages of design, it is often difficult to visualize a system configuration (or element thereof) in its true perspective, whereas simulating a three-dimensional view of an item is possible through the use of CAD. In some instances, the validation of system requirements can be accomplished through the use of simulation methods during the preliminary system design and detail design and development phases prior to the introduction of hardware, software, and so on. Through the proper use of these and related technologies (to include the appropriate application of selected analytical models), the design team is able to produce a robust design more quickly, while reducing the overall program technical risks.

As an additional aid to the designer, physical three-dimensional scale models or <emphasis>mock-ups</emphasis> are sometimes constructed to provide a realistic simulation of a proposed system configuration. Models, or mock-ups, can be developed to any desired scale and to varying depths of detail depending on the level of emphasis desired. Mock-ups can be developed for large as well as small systems and may be constructed of heavy cardboard, wood, metal, or a combination of different materials. Mock-ups can be developed on a relatively inexpensive basis and in a short period of time when employing the right materials and personnel services. Industrial design, human factors, or model-shop personnel are usually available in many organizations, are well-oriented to this area of activity, and should be utilized to the greatest extent possible. Some of the uses and values of a mock-up are that they:

\begin{enumerate}
\item Provide the design engineer with the opportunity of experimenting with different facility layouts, packaging schemes, panel displays, cable runs, and so on, before the preparation of final design data. A mock-up or engineering model of the proposed river crossing bridge can be developed to better visualize the overall structure, its location, interfaces with the communities on each side of the river, and so on
\item Provide the reliability–maintainability–human factors engineer with the opportunity to accomplish a more effective review of a proposed design configuration for the incorporation of supportability characteristics. Problem areas readily become evident
\item Provide the maintainability-human factors engineer with a tool for use in the accomplishment of predictions and detailed task analyses. It is often possible to simulate operator and maintenance tasks to acquire task sequence and time data
\item Provide the design engineer with an excellent tool for conveying his or her final design approach during a formal design review.</para></listitem>
\item Serve as an excellent marketing tool
\item Can be employed to facilitate the training of system operator and maintenance personnel
\item Can be utilized by production and industrial engineering personnel in developing fabrication and assembly processes and procedures and in the design of factory tooling and associated test fixtures
\item Can serve as a tool at a later stage in the system life cycle for the verification of a modification kit design prior to the preparation of formal data and the development of kit hardware, software, and supporting materials
\end{enumerate}

In the domain of software development, in particular, designers are oriented toward the building of ``one-of-a-kind'' software packages. The issues in software development differ from those in other areas of engineering in that mass production is not the normal objective. Instead, the goal is to develop software that accurately portrays the features that are desired by the user (customer). For example, in the design of a complex workstation display, the user may not at first comprehend the implications of the proposed command routines and data format on the screen. When the system is ultimately delivered, problems occur, and the ``user interface'' is not acceptable for one reason or another. Changes are then recommended and implemented, and the costs of modification and rework are usually high.

The alternative is to develop a ``protoype'' early in the system design process, design the applicable software, involve the user in the operation of the prototype, identify areas that need improvement, incorporate the necessary changes, involve the user once again, and so on. This iterative and evolutionary process of software development, accomplished throughout the preliminary and detail design phases, is referred to as <emphasis>rapid prototyping</emphasis>. Rapid prototyping is a practice that is often implemented and is inherent within the systems engineering process, particularly in the development of large software-intensive systems.

\subsection{Incorporating Design Changes}\index{Incorporating Design Changes}

After a baseline has been established as a result of a formal design review, changes are frequently initiated for any one of a number of reasons such as to correct a design deficiency, improve a product, incorporate a new technology, improve the level of sustainability, respond to a change in operational requirements, compensate for an obsolete component, and so on. Changes may be initiated from within the project, or as a result of some new externally imposed requirement.

At first, it may appear that a change is relatively insignificant in nature and that it may constitute a change in the design of a prime equipment item, a software modification, a data revision, and/or a change in some process. However, what might initially appear to be minor often turns out to have a great impact on and throughout the system hierarchical structure. For instance, a change in the design configuration of prime equipment (e.g., a change in size, weight, repackaging, and added performance capability) will probably affect related software, design of test and support equipment, type and quantity of spares/repair parts, technical data, transportation and handling requirements, and so on. A change in software will likely have an impact on associated prime equipment, technical data, and test equipment. A change in any one item will likely have an impact on many other elements of the system. Further, if there are numerous changes being incorporated at the same time, the entire system configuration may be severely compromised in terms of maintaining some degree of requirements traceability.

Past experience with a variety of systems has indicated that many of the changes incorporated are introduced late in detail design and development, during production or construction, and/or early in the system utilization and sustaining support phase. As shown in , the accomplishment of changes this far downstream in the life cycle can be very costly; for example, a small change in an equipment item can result in a subsequent change in software, technical data, facilities, the various elements of support, or a production process.

While the incorporation of changes (for one reason or another) is certainly inevitable, the process for doing this must be formalized and controlled to ensure traceability from one configuration baseline to another. Also, it is necessary to ensure that the incorporation of a change is consistent with the requirements in the system specification (Type A). Referring to , a proposed change is initially submitted through the preparation of an engineering change proposal (ECP), which, in turn, is reviewed by a Change Control Board (or Configuration Control Board). Each proposed change must be thoroughly evaluated in terms of its impact on other elements of the system, the specified TPMs, life-cycle cost, and the various considerations that have been addressed throughout the earlier stages of design (e.g., reliability, maintainability, human factors, producibility, sustainability, and disposability). Approved changes will then lead to the development of the required modification kit, installation instructions, and the ultimate incorporation of the change in the system configuration. Accordingly, there needs to be a highly disciplined configuration management (CM) process from the beginning and throughout the entire system life cycle. This is particularly important in the successful implementation of the systems engineering process.