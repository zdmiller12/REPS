There are several phases through which the system design and development process must invariably pass. Foremost among them is identification and understanding of the customer-stated problem, with special emphasis on what the system is intended to do. This is followed by the solicitation of system-level requirements, technological innovation to discover and define feasible solutions, the design and development of system components, the fabrication of a prototype and/or engineering model, and the validation of system design through test and eval..

The coordinated sections that follow partition the system design and development process into major phases. Conceptual design, preliminary design, detail design and development, and test and evaluation may be incorrectly considered to be distinct from each other, rather than as part of a seamless process. This partition is made for convenience and communication purposes. The proper and timely application of feedback, iteration, and successive improvement will act to ensure integration of the essential activities.

System design is not systems engineering per se, but it is almost as pervasive. From the perspective of synthesis, system design reaches almost as far and deep as systems engineering. It is nominally comprised of conceptual, preliminary, and detail design. These are artificial categories that, along with test and evaluation, make up the four chapters in of this textbook. Included in these chapters is a high degree of connectivity among the design categories and an elaboration upon the theme of ``bringing into being,'' promulgated in Chapter 6.

The engine that drives the systems engineering process is system design. System design is a process itself, with a prominent position in systems engineering. It is an essential activity ensuring the orderly realization of the final configuration and composition of a system. It is also the basis for incorporating the designs for companion and post-realization capabilities such as maintenance, support, sustainment, recycling, and disposal.

Accordingly, the overall purpose of this chapter is to impart an in-depth understanding of system design as a process; a process that is greater than the sum of its artificial categories as identified by process phases.