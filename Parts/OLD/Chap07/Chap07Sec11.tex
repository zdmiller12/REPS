\section{The Systems Engineering Process}\index{The Systems Engineering Process}

Although there is general agreement regarding the principles and objectives of systems engineering, its actual implementation will vary from one system and engineering team to the next. The process approach and steps used will depend on the nature of the system application and the backgrounds and experiences of the individuals on the team. To establish a common frame of reference for improving communication and understanding, it is important that a “baseline” be defined that describes the systems engineering process, along with the essential life-cycle phases and steps within that process. Augmenting this common frame of reference are top-down and bottom-up approaches. And, there are other process models that have attracted various degrees of attention. Each of these topics is presented in this section.

\subsection{Development of Engineering Models}\index{Development of Engineering Models}

As the system design and development effort progresses, the basic process evolves from a description of the design in the form of drawings, documentation, and databases to the construction of a physical model or mock-up, to the construction of an engineering or laboratory model, to the construction of a prototype, and ultimately to the production of a final product. The purpose in proceeding through these steps is to provide a solid basis for design evaluation and/or validation. The earlier in the design process that one can accomplish this purpose, the better, since the incorporation of any necessary changes for corrective action will be more costly and disruptive later when the design progresses toward the production/construction phase.</para>
<para>The first two steps in this development sequence are discussed in the previous sections: <link linkend="ch05lev1sec4" preference="0" type="backward">Section <xref linkend="ch05lev1sec4" label="5.4"><inst>8.3.4</inst></xref></link> discusses the development of a mock-up and <link linkend="ch05lev1sec5" preference="0" type="backward">Section <xref linkend="ch05lev1sec5" label="5.5"><inst>8.3.5</inst></xref></link> discusses the development of data and documentation requirements. At this stage, the results have not produced an actual “working” model of the system.</para>
<para>At some point in the detail design and development phase, it may be appropriate to produce an <emphasis>engineering model</emphasis> (or a laboratory model). The objective is to demonstrate some (if not all) of the functions that the system is to ultimately perform for the customer by constructing an “operating” model and utilizing it in a research-oriented environment in an engineering shop, or equivalent. This model may be constructed using nonstandard and unqualified parts and will not necessarily reflect the design configuration that will ultimately be produced for the customer. The intent is to verify certain performance characteristics and to gain confidence that “all is well” at the time.

\subsection{Analytical Models and Modeling}\index{Analytical Models and Modeling}

<para>The design evaluation process may be further facilitated through the use of various analytical models, methods, and tools in support of the Macro-CAD objective. A model, in this context, is a simplified representation of the real world that abstracts features of the situation relative to the problem being analyzed. It is a tool employed by an analyst to assess the likely consequences of various alternative courses of action being examined. The model must be adapted to the problem at hand and the output must be oriented to the selected evaluation criteria. The model, in itself, is not the decision maker but is a tool that provides the necessary data in a timely manner in support of the decision-making process.</para>
<para>The extensiveness of the <emphasis>model</emphasis> will depend on the nature of the problem, the number of variables, input parameter relationships, number of alternatives being evaluated, and the complexity of operation. The ultimate objective in the selection and development of a model is simplicity and usefulness. The model used should incorporate the following features:</para>
<orderedlist numeration="arabic" spacing="normal" inheritnum="ignore" continuation="restarts"><listitem><inst>	1.	</inst><para>The model should represent the dynamics of the system configuration being evaluated in a way that is simple enough to understand and manipulate, and yet close enough to the operating reality to yield successful results.</para></listitem>
<listitem><inst>	2.	</inst><para>The model should highlight those factors that are most relevant to the problem at hand and suppress (with discretion) those that are not as important.</para></listitem>
<listitem><inst>	8.1.	</inst><para>The model should be comprehensive, by including <emphasis>all</emphasis> relevant factors, and be reliable in terms of repeatability of results.</para></listitem>
<listitem><inst>	8.2.	</inst><para>Model design should be simple enough to allow for timely implementation in problem solving. Unless the tool can be utilized in a timely and efficient manner by the analyst (or the manager), it is of little value. If the model is large and highly complex, it may be appropriate to develop a series of models where the output of one can be tied to the input of another. Also, it may be desirable to evaluate a specific element of the system independent of other elements.</para></listitem>
<listitem><inst>	5.	</inst><para>Model design should incorporate provisions for ease of modification or expansion to permit the evaluation of additional factors as required. Successful model development often includes a series of trials before the overall objective is met. Initial attempts may suggest information gaps, which are not immediately apparent and consequently may suggest beneficial changes.</para></listitem></orderedlist>
<para>The use of mathematical models offers significant benefits. In terms of system application, several considerations exist—operational considerations, design considerations, product/construction considerations, testing considerations, logistic support considerations, and recycling and disposal considerations. There are many interrelated elements that must be integrated as a system and not treated on an individual basis. The mathematical model makes it possible to deal with the problem as an entity and allows consideration of all major variables of the problem on a simultaneous basis. More specifically:</para>
    1. <orderedlist numeration="arabic" spacing="normal" inheritnum="ignore" continuation="restarts"><listitem><inst>	1.	</inst><para>The mathematical model will uncover relations between the various aspects of a problem that are not apparent in the verbal description.</para></listitem>
    2. <listitem><inst>	2.	</inst><para>The mathematical model enables a comparison of <emphasis>many</emphasis> possible solutions and aids in selecting the best among them rapidly and efficiently.</para></listitem>
    3. <listitem><inst>	8.1.	</inst><para>The mathematical model often explains situations that have been left unexplained in the past by indicating cause-and-effect relationships.</para></listitem>
    4. <listitem><inst>	8.2.	</inst><para>The mathematical model readily indicates the type of data that should be collected to deal with the problem in a quantitative manner.</para></listitem>
    5. <listitem><inst>	5.	</inst><para>The mathematical model facilitates the prediction of future events, such as effectiveness factors, reliability and maintainability parameters, logistics requirements, and so on.</para></listitem>
    6. <listitem><inst>	6.	</inst><para>The mathematical model aids in identifying areas of risk and uncertainty.</para></listitem></orderedlist>
    7. <para>When analyzing a problem in terms of selecting a mathematical model for evaluation purposes, it is desirable to first investigate the tools that are currently available. If a model already exists and is proven, then it may be feasible to adopt that model. However, extreme care must be exercised to relate the right technique with the problem being addressed and to apply it to the depth necessary to provide the sensitivity required in arriving at a solution. Improper application may not provide the results desired, and the consequence may be costly.</para>
<para>Conversely, it might be necessary to construct a new model. In accomplishing this task, one should generate a comprehensive list of system parameters that will describe the situation being simulated. Next, it is necessary to develop a matrix showing parameter relationships, each parameter being analyzed with respect to every other parameter to determine the magnitude of relationship. Model input–output factors and parameter feedback relationships must be established. The model is constructed by combining the various factors and then testing it for validity. Testing is difficult to do because the problems addressed primarily deal with actions in the future that are impossible to verify. However, it may be possible to select a known system or equipment item that has been in existence for several years and exercise the model using established parameters. Data and relationships are known and can be compared with historical experience. In any event, the analyst might attempt to answer the following questions: <emphasis>Can the model describe known facts and situations sufficiently well? When major input parameters are varied, do the results remain consistent and are they realistic? Relative to system application, is the model sensitive to changes in operational requirements, design, production/construction, and logistics and maintenance support? Can cause-and-effect relationships be established?</emphasis></para>
<para>Models and appropriate systems analysis methods are presented further in <link olinkend="part03" preference="0">Part <xref olinkend="part03" label="III"><inst>III</inst></xref></link>. Trade-off and system optimization relies on these methods and their application. A fundamental knowledge of probability and statistics, economic analysis methods, modeling and optimization, simulation, queuing theory, control techniques, and the other analytical techniques is essential to accomplishing effective systems analysis.

\subsection{Design, Data, Information, and Integration}\index{Design, Data, Information, and Integration}

As a result of advances in information systems technology, the methods for documenting design are changing rapidly. These advances have promoted the use of vast electronic databases for the purposes of information processing, storage, and retrieval. Through the use of CAD techniques, information can be stored in the form of three-dimensional representations, in regular two-dimensional line drawings, in digital format, or in combinations of these. By using computer graphics, word processing capability, e-mail communications, video-disc technology, and the like, design can be presented in more detail, in an easily modified format, and displayed faster.</para>
<para>Although many advances have been made in the application of computerized methods to data acquisition, storage, and retrieval, the need for some of the more conventional methods of design documentation remains. These include a combination of the following:</para>
<orderedlist numeration="arabic" spacing="normal" inheritnum="ignore" continuation="restarts"><listitem><inst>	1.	</inst><para><emphasis>Design drawings</emphasis>—assembly drawings, control drawings, logic diagrams, structural layouts, installation drawings, schematics, and so on.</para></listitem>
<listitem><inst>	2.	</inst><para><emphasis>Material and part lists</emphasis>—part lists, material lists, long-lead-item lists, bulk-item lists, provisioning lists, and so on.</para></listitem>
<listitem><inst>	3.	</inst><para><emphasis>Analyses and reports</emphasis>—trade-off study reports supporting design decisions, reliability and maintainability analyses and predictions, human factors analyses, safety reports, supportability analyses, configuration identification reports, computer documentation, installation and assembly procedures, and so on.</para></listitem></orderedlist>
<para>Design drawings, constituting a primary source of definition, may vary in form and function depending on the design objective; that is, the type of equipment being developed, the extent of development required, whether the design is to be subcontracted, and so on. Some typical types of drawings from the past are illustrated in 
<para>During the process of detail design, engineering documentation is rather preliminary and then gradually progresses to the depth and extent of definition necessary to enable product manufacture. The responsible designer, using appropriate design aids, produces a functional diagram of the overall system. The system functions are analyzed and initial packaging concepts are assumed. With the aid of specialists representing various disciplines (e.g., civil, electrical, mechanical, structural, components, reliability, maintainability, and sustainability) and supplier data, detail design layouts are prepared for subsystems, units, assemblies, and subassemblies. The results are analyzed and evaluated in terms of functional capability, reliability, maintainability, human factors, safety, producibility, and other design parameters to assure compliance with the allocated requirements and the initially established design criteria. This review and evaluation occurs at each stage in the basic design sequence and generally follows the steps presented in 
<para>Engineering data are reviewed against design standards and checklist criteria. Throughout the industrial and governmental sectors are design standards manuals and handbooks developed to cover preferred component parts and supplier data, preferred design and manufacturing practices, designated levels of quality for specified products, requirements for safety, and the like. These standards (ANSI, EIA, ISO, etc.), as applicable, may serve as a basis for design review and evaluation.</para>

Patterns Leading to Models
</para></section></section>
3. An Introduction to TRIZ (SE-110406, Blackburn, 15 April 2011, Page 6 of 25)

TRIZ began to appear in the West in the 1980s, and is evolving to be an inventive problem
solving approach for any system. (Stratton and Mann 2003) TRIZ (often pronounced trees) is a
Russian acronym for the theory of inventive problem solving, an approach to solving problems
systematically, creating innovation by identifying and eliminating system conflicts or
contradictions. (Stratton and Mann 2003; Harvard 2009) The Russian title for TRIZ
transliterated to English is Teoriya Resheniya Izobretatelskikh Zadatch. (Low, Lamvik et al.
2002) TRIZ uses generalized principles derived from patent data analysis to offer solutions for
system problems. (Stratton and Mann 2003)
TRIZ follows a scheme of abstracting a problem to identify an abstracted solution (ReVelle
2002) and offers abstracted inventive principles to solve problems based on a prior solution.
Savransky estimates that “about 95% of the inventive problems in any particular field have
already been solved in another field.” (Savransky 2000) Mann makes the case that TRIZ’
 “generic problem solving framework . . . allows engineers and scientists working in any one field to access the good practices of everyone working in not just their own, but every other field of science.” (Mann 2002) Savransky describes this as accumulating and condensing “all respective
human knowledge and then” applying “it to solve inventive problems.” (Savransky 2000)
As noted previously, TRIZ began when Genrich Altshuller studied trends in patents while
serving in the Russian Navy in 1946. He discovered that innovation is not a random process, but
is governed by learnable principles. (Savransky 2000; Silverstein, DeCarlo et al. 2007) He
identified patterns in patents, and later developed an approach that describes conflicting features
with mapped solutions. Specifically, his study identified 39 engineering or desired features (or
parameters) and 40 inventive solutions to apply when one or more parameters conflict – see
Appendices A-C. (ReVelle 2002; Stratton and Mann 2003) TRIZ is based on three premises as
follows. (ReVelle 2002; Stratton and Mann 2003)
1. Ideality or Ideal Final Result (IFR): The ultimate goal is to design a system with no
harmful functions. IFR is useful to describe what the desired final system state should be,
enabling the identification of system contradictions that will prevent success.
2. Contradictions (or conflicts): It is necessary to eliminate (wholly or partially) a system
contradiction to achieve higher success. A physical contradiction occurs when an aspect of a
system needs to be in opposite states (e.g. hot versus cold).
3. Resources: Maximize the utility of current resources prior to introducing any additional
complexity or resources to the system.
IFR can be useful to mitigate the impact of PI (Psychological Inertia), where it might be
difficult for the systems engineer to innovate beyond his or her normal experience. Altshuller
viewed PI as a barrier to innovation. (Low, Lamvik et al. 2002) PI relates to the likelihood that
one will limit the solution space inside a known or comfortable paradigm, or define a solution
path defined by his or her paradigm. (ReVelle 2002) PI has been described as “the sum of one’s
intellectual, emotional, academic, experiential, and other biases” (Silverstein, DeCarlo et al.
2007) and “the effort made by a system to preserve the current (meta-)stable state or to resist
change in that state.” (Savransky 2000)
SE-110406, Blackburn, 15 April 2011, Page 7 of 25
Starting at the ideal state and working backwards can stimulate new and innovative thinking.
The idea of eliminating contradictions is useful in that it enables the system to have breakthrough
performance. Last, focusing on maximizing current resources avoids adding additional cost and
complexity to the system, which can in turn trigger undesired emergent behaviors.
Using the findings from patents, TRIZ offers abstracted inventive principles to solve
problems. By aligning abstracted engineering or desired features with already successful
principles as observed in patents, system contradictions can be eliminated. While there are other
advanced TRIZ tools and methods, this paper will focus on the application of the classic TRIZ
Contradiction Matrix. The following presents a TRIZ Trade Study framework that uses the
TRIZ Contradiction Matrix.

RAW:
Conceptually sound system design derives from focusing on what the system is intended to do before determining what the system is, with form following function. This focus is most effective when based on essential design dependent parameters, recognizing the concurrent life-cycle factors of production, support, maintenance, phase-out, and disposal. It invokes integrating and iterating synthesis, analysis, and evaluation. These considerations are germane to system and product design when embedded within the systems engineering process. The purpose of this presentation is to provide an overview of the embedded relationship of design dependent parameters as key controllables in the effective, efficient, and orderly process of bringing cost-effective systems, products, structures, and services (the human-made world) into being.

\subsection{Patterns as Models for Systems Engineering}\index{Patterns as Models for Systems Engineering}

This group is probably quite advanced in the reflections on patterns and their role in systems science / systems thinking / systems engineering. I have been in contact with some members such as Len Troncale, Joseph Simpson a while ago in a discussion Len involved me in but it seems I lost the thread, and Peter Tuddenham, with whom I have been reflecting on the role of pattern literacy in support of systems literacy. In addition to whatwe presented at ISSS 2017on the topic we presented a paper at the pattern languages of programs conference, with pattern literacy in support of systems literacy seen from a pattern language perspective. An approach to patterns quite different than the systems science approach.
The issue you raise Aleksandar is key. Patterns can be everything or nothing, and can we really find a bottom line theory that is not a reduction from one or another perspective (sorry if I 'reduce' your point that much here, I am looking forward to get deeper in Dennett's understanding of patterns)? I am very interested in Len's work, because I think that what Len has done over the years can provide some responses although I do not know his work well enough to know exactly which piece. The isomorphies Len has been pursuing are patterns, and it seems to me that there would be ways of interconnecting patterns of different types on different types of criteria with some sort of 'semantic' if not semiotic relationship, to see clusters of 'probable' isomorphy emerge, around which conversations can take place. This is something we have discussed in Vienna, and that is outlined around slides 16 to 22 of the presentation linked above.

What I would very much like to do, with, and beyond this survey, is to gather research that has already been done on different approaches to patterns. When reading the paperDefining “System”: a Comprehensive Approach, [proceedings of?] 27th Annual INCOSE International Symposium (IS 2017) in Adelaide, with some of you I recognize as authors, I was wondering whether something like this type of research had been done on patterns within a systems context, and thought maybe some in this group had started looking at patterns similarly.

The survey is an attempt to scratch the surface. The input of your group will be very valuable, especially if key questions are asked there, and I offer to share here a compilation of responses I will have collected. I look forward to further discussions.

RAW:
Better for ST definition. From a cognitive perspective, systems thinking integrates analysis and synthesis. Natural science has been primarily reductionistic, studying the components of systems and using quantitative empirical verification. Human science, as a response to the use of positivistic methods for studying human phenomena, has embraced more holistic approaches, studying social phenomena through qualitative means to create meaning. Systems thinking bridges these two approaches by using both analysis and synthesis to create knowledge and understanding and integrating an ethical perspective. Analysis answers the ‘what’ and ‘how’ questions while synthesis answers the ‘why’ and ‘what for’ questions. Better for ST definition. From a cognitive perspective, systems thinking integrates analysis and synthesis. Natural science has been primarily reductionistic, studying the components of systems and using quantitative empirical verification. Human science, as a response to the use of positivistic methods for studying human phenomena, has embraced more holistic approaches, studying social phenomena through qualitative means to create meaning. Systems thinking bridges these two approaches by using both analysis and synthesis to create knowledge and understanding and integrating an ethical perspective. Analysis answers the ‘what’ and ‘how’ questions while synthesis answers the ‘why’ and ‘what for’ questions.