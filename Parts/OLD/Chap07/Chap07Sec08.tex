\section{Systems Engineering Process Models}\index{Systems Engineering Process Models}

Although there is general agreement regarding the principles and objectives of systems engineering, its actual implementation will vary from one system and engineering endeavor to the next. The process approach and steps used will depend on the nature of the system application and the backgrounds and experiences of the individuals on the team.
	To establish a common frame of reference for improving communication and understanding, it is important that a “baseline” be defined that describes the systems engineering process, along with the essential life-cycle phases and steps within that process. Augmenting this common frame of reference are top-down and bottom-up approaches. And, there are other process models that have attracted various degrees of attention. Each of these topics is presented in this section.</para>
 illustrates the major life-cycle process phases and selected milestones for a generic system. This is the “model” that will serve as a frame of reference for material presented in subsequent chapters. Included are the basic steps in the systems engineering process (i.e., requirements analysis, functional analysis and allocation, synthesis, trade-off studies, design evaluation, and so on).

<para>A newly identified need, or an evolving need, reveals a new system requirement. If a decision is made to seek a solution for the need, then a decision is needed whether to consider other needs in designing the solution. Based on an initial determination regarding the scope of needs, the basic phases of conceptual design and onward through system retirement and phase-out are then applicable, as described in the paragraphs that follow. The scope of needs may contract or expand, but the scope should be stabilized as early as possible during conceptual design, preferably based on an evaluation of value and cost by the customer.</para>
<para>Program phases described in are not intended to convey specific tasks, or time periods, or levels of funding, or numbers of iterations. Individual program requirements will vary from one application to the next. The figure exhibits an overall <emphasis>process</emphasis> that needs to be followed during system acquisition and deployment. Regardless of the type, size, and complexity of the system, there is a conceptual design requirement (i.e., to include requirements analysis), a preliminary design requirement, and so on. Also, to ensure maximum effectiveness, the concepts presented in must be properly “tailored” to the particular system application being addressed.</para>
shows the basic steps in the systems engineering process to be iterative in nature, providing a top-down definition of the system, and then proceeding down to the subsystem level (and below as necessary). Focused on the needs, and beginning with conceptual design, the completion of Block 0.2 defines the system in <emphasis>functional</emphasis> terms (having identified the “whats” from a requirements perspective). These “whats” are translated into an applicable set of “hows” through the iterative process of functional partitioning and requirements allocation, together with conceptual design synthesis, analysis, and evaluation. This conceptual design phase is where the initial configuration of the system (or system architecture) is defined.</para>
<para>During preliminary design, completion of Block 1.1 defines the system in <emphasis>refined functional</emphasis> terms providing a top-down definition of subsystems with preparation for moving down to the component level. Here the “whats” are extracted from (provided by) the conceptual design phase. These “whats” are translated into an applicable set of “hows” through the iterative process of functional partitioning and requirements allocation, together with preliminary design synthesis, analysis, and evaluation. This preliminary design phase is where the initial configuration of subsystems (or subsystem architecture) is defined.</para>
<para>Blocks 1.1–1.7 are an evolution from Blocks 0.1–0.8, Blocks 6.1–6.5 are an evolution from Blocks 1.1–1.7, and Blocks 3.1–3.6 are an evolution from Blocks 6.1–6.5. The overall process reflected in the figure constitutes an evolutionary design and development process. With appropriate feedback and design refinement provisions incorporated, the process should eventually converge to a successful design. The functional definition of the system, its subsystems, and its components serves as the baseline for the identification of resource requirements for production and then operational use (i.e., hardware, software, people, facilities, data, elements of support, or a combination thereof).</para></section>
</title>
	<para>The systems engineering process, and the steps illustrated in 7.X is now developed and presented and described in more detail. The overarching objective is to describe a <emphasis>process</emphasis> (as a frame of reference) that should be “tailored” to the specific program need.</para>
<para> 7.X7.X is not intended to emphasize any particular model, such as the “waterfall” model, the “spiral” model, the “vee” model, or equivalent. These well-known process models are illustrated and briefly described in, with related references to the literature found in
Figure 7.5 sheet 1 and sheet 2 HERE</para></section></section>

THEN DO MBSE - Use the NSF Designers Unification