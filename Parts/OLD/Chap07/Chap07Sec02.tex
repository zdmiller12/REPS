\section{Process and Decision Model Categories}\index{Process and Decision Model Categories}

Just as there is a fundamental difference at the science and engineering levels, there is a significant difference between models for the systems engineering process itself (as in bringing into being) and the system life-cycle level (as in evaluating design and operations). The difference is so profound and significantly important, that the material cannot be developed and conveyed effectively if the categories are presented together within the same chapter..
Accordingly, this section will serve the unique role of exposing the totality of topics and applications needed to convey understanding about models and modeling as it is needed in systems engineering and systems analysis. On purpose, this chapter in coordination with the next one on process phases, will end the aspect of modeling pertaining to process. Then, all of Part III (five chapters) on systems thinking and analysis will be devoted to decision models for design and operations; models of a mathematical kind. The role of design decision models is to bring forward into the system design phase insight about the predicted and estimated operational outcomes as projected from mutually exclusive design alternatives. But in both cases, the thinking pertaining to direct and indirect experimentation is fully applicable.

\subsection{Process Models for System Realization}\index{Process Models for System Realization}

The broadest classification of models for Systems Engineering derives from a high-level partition of the life cycle into design and operations; more specifically into models for process versus models for decision analysis.
	The first graphical model-like representation of the time-oriented unfolding of the developing system appeared in Figure 5.1. This is a process model exhibiting a left-to-right orientation of high-level system life-cycle phases. Variations and elaborations have appeared down through the years to convey thinking judged to be of importance. 
The basic models for the process of systems engineering are, understandably, life cycle oriented. Figure 5.1 is the most elementary of these. It is properly classified as a schematic model. But it provides the basis for two expanded elaborations.
An example conveying thinking of importance comes into view by partitioning Figure 5.1 into its concurrent life cycles. The overall system model is actually made up of four concurrent life cycles progressing in parallel, as is illustrated in. This conceptualization is one way of representing concurrent (or simultaneous) engineering<emphasis> </emphasis>.

Figure 7.1 Here (was 2.2) 

Modeling Concurrent Engineering. <para>System life-cycle engineering goes beyond the product life cycle viewed in isolation. It must simultaneously embrace the life cycle of the production or construction subsystem, the life cycle of the maintenance and support subsystem, and the life cycle for retirement, phase out, reuse, and disposal as another subsystem. The overall system is made up of four concurrent life cycles progressing in parallel, as is illustrated in. This conceptualization is the basis for concurrent engineering1
<para>The need for the product comes into focus first. This recognition initiates conceptual design to meet the need. Then, during conceptual design of the product, consideration should simultaneously be given to its production. This gives rise to a parallel life cycle for a production and/or construction capability. Many producer-related activities are needed to prepare for the production of a product, whether the production capability is a manufacturing plant, construction contractors, or a service activity.</para>
<para>Also shown in Figure 7.1 is another life cycle of considerable importance that is often neglected until product and production design is completed. This is the life cycle for support activities, including the maintenance, logistic support, and technical skills needed to service the product during use, to support the production capability during its duty cycle, and to maintain the viability of the entire system. Logistic, maintenance, technical support, and regeneration requirements planning should begin during product conceptual design in a coordinated manner.</para>
<para>As each of the life cycles is considered, design features should be integrated to facilitate phase out, regeneration, or retirement having minimal impact on interrelated systems. For example, attention to end-of-life recyclability, reusability, and disposability will contribute to environmental sustainability. Also, the system should be made ready for regeneration by anticipating and addressing changes in requirements, such as increases in complexity, incorporation of planned new technology, likely new regulations, market expansion, and others.
<para>In addition, the interactions between the product and system and any related systems should begin receiving compatibility attention during conceptual design to minimize the need for product and system redesign. Whether the interrelated system is a companion product sold by the same company, an environmental system that may be degraded, or a computer system on which a software product runs, the relationship with the product and system under development must be engineered concurrently.</para></section>
<section id=”ch02lev1sec2” label=”2.2”><title id=”ch02lev1sec2.title”><inst> </title><para>Experience over many decades indicates that a properly functioning system that is effective and economically competitive cannot be achieved through efforts applied largely after it comes into being. Accordingly, it is essential that anticipated outcomes during, as well as after, system utilization be considered during the early stages of design and development. Responsibility for <emphasis>life-cycle engineering,</emphasis> largely neglected in the past, must become the central engineering focus.</para><section id="ch02lev2sec6" label="2.2.2"><title id="ch02lev2sec6.title"><inst> </title>
	<para>Design within the system life-cycle context differs from design in the ordinary sense. Life-cycle-guided design is simultaneously responsive to customer needs (i.e., to requirements expressed in functional terms) and to life-cycle outcomes. Design should not only transform a need into a system configuration but should also ensure the design’s compatibility with related physical and functional requirements. Further, it should consider operational outcomes expressed as producibility, reliability, maintainability, usability, supportability, serviceability, disposability, sustainability, and others, in addition to performance, effectiveness, and affordability.</para>
<para>A detailed presentation of the elaborate technological activities and interactions that must be integrated over the system life cycle process is given in. The progression is iterative from left to right and not serial in nature, as might be inferred.

Figure 7.2 Here (was 2.3)

<para>Although the level of activity and detail may vary, the life-cycle functions described and illustrated are generic. They are applicable whenever a new need or changed requirement is identified, with the process being common to large as well as small-scale systems. It is essential that this process be implemented completely at an appropriate level of detail not only in the engineering of new systems but also in the re-engineering of existing or legacy systems.</para>
<para>Major technical activities performed during the design, production or construction, utilization, support, and phase-out phases of the life cycle are highlighted in. These are initiated when a new need is identified. A planning function is followed by conceptual, preliminary, and detail design activities. Producing and/or constructing the system are the function that completes the acquisition phase. System operation and support functions occur during the utilization phase of the life cycle. Phase-out and disposal are important final functions of utilization to be considered as part of design for the life cycle.
The numbered blocks in “map” and elaborate on the phases of the life cycles depicted in as follows:</para>
<para>The communication and coordination needed to design and develop the product, the production capability, the system support capability, and the relationships with interrelated systems—so that they traverse the life cycle together seamlessly—is not easy to accomplish. Progress in this area is facilitated by technologies that make more timely acquisition and use of design information possible. Computer-Aided Design (CAD) technology with internet/intranet connectivity enables a geographically dispersed multidiscipline team to collaborate effectively on complex physical designs.</para>
<para>Concern for the entire life cycle is particularly strong within the U.S. Department of Defense (DOD) and its non-U.S. counterparts. This may be attributed to the fact that acquired defense systems are owned, operated, and maintained by the DOD. This is unlike the situation most often encountered in the private sector, where the consumer or user is usually not the producer. Those private firms serving as defense contractors are obliged to design and develop in accordance with DOD directives, specifications, and standards. Because the DOD is the customer and also the user of the resulting system, considerable DOD intervention occurs during the acquisition phase.
<para>Many firms that produce for private-sector markets have chosen to design with the life cycle in mind. For example, design for energy efficiency is now common in appliances such as water heaters and air conditioners. Fuel efficiency is a required design characteristic for automobiles. Some truck manufacturers promise that life cycle maintenance costs will be within stated limits. These developments are commendable, but they do not go far enough. When the producer is not the consumer, it is less likely that potential operational problems will be addressed during development. Undesirable outcomes too often end up as problems for the user of the product instead of the producer.</para></section></section>
<para>The elaborate model of Figure 7.4 is not intended to emphasize any particular model, such as the “waterfall” model, the “spiral” model, the “vee” model, or equivalent. These well-known process models are illustrated and briefly described in , with related references to the literature found in 
	<para>The systems engineering process, and the steps illustrated in, is developed and presented and described in more detail in <link olinkend="part02" preference="0">the next chapter</inst></xref></link> of this textbook on SE process phases. The overarching objective is to describe a <emphasis>process</emphasis> model (as a frame of reference) that should be “tailored” to the specific program need.</para>

Insert Process Models Here

\subsection{Decision Models for System Evaluation}\index{Decision Models for System Evaluation}

Figure 7.3 Here (was3.26)

<section id="ch04lev2sec5" label="4.6.2"><title id="ch04lev2sec5.title"><inst>7.5.17</inst>Analytical Models and Modeling
</title>
<para>The design evaluation process may be further facilitated through the use of various analytical models, methods, and tools in support of the Macro-CAD objective. A model, in this context, is a simplified representation of the real world that abstracts features of the situation relative to the problem being analyzed. It is a tool employed by an analyst to assess the likely consequences of various alternative courses of action being examined. The model must be adapted to the problem at hand and the output must be oriented to the selected evaluation criteria. The model, in itself, is not the decision maker but is a tool that provides the necessary data in a timely manner in support of the decision-making process.</para>
<para>The extensiveness of the <emphasis>model</emphasis> will depend on the nature of the problem, the number of variables, input parameter relationships, number of alternatives being evaluated, and the complexity of operation. The ultimate objective in the selection and development of a model is simplicity and usefulness. The model used should incorporate the following features:</para>
<orderedlist numeration="arabic" spacing="normal" inheritnum="ignore" continuation="restarts"><listitem><inst>	1.	</inst><para>The model should represent the dynamics of the system configuration being evaluated in a way that is simple enough to understand and manipulate, and yet close enough to the operating reality to yield successful results.</para></listitem>
<listitem><inst>	2.	</inst><para>The model should highlight those factors that are most relevant to the problem at hand and suppress (with discretion) those that are not as important.</para></listitem>
<listitem><inst>	3.	</inst><para>The model should be comprehensive, by including <emphasis>all</emphasis> relevant factors, and be reliable in terms of repeatability of results.</para></listitem>
<listitem><inst>	4.	</inst><para>Model design should be simple enough to allow for timely implementation in problem solving. Unless the tool can be utilized in a timely and efficient manner by the analyst (or the manager), it is of little value. If the model is large and highly complex, it may be appropriate to develop a series of models where the output of one can be tied to the input of another. Also, it may be desirable to evaluate a specific element of the system independent of other elements.</para></listitem>
<listitem><inst>	5.	</inst><para>Model design should incorporate provisions for ease of modification or expansion to permit the evaluation of additional factors as required. Successful model development often includes a series of trials before the overall objective is met. Initial attempts may suggest information gaps, which are not immediately apparent and consequently may suggest beneficial changes.</para></listitem></orderedlist>
<para>The use of mathematical models offers significant benefits. In terms of system application, several considerations exist—operational considerations, design considerations, product/construction considerations, testing considerations, logistic support considerations, and recycling and disposal considerations. There are many interrelated elements that must be integrated as a system and not treated on an individual basis. The mathematical model makes it possible to deal with the problem as an entity and allows consideration of all major variables of the problem on a simultaneous basis. More specifically:</para>
<orderedlist numeration="arabic" spacing="normal" inheritnum="ignore" continuation="restarts"><listitem><inst>	1.	</inst><para>The mathematical model will uncover relations between the various aspects of a problem that are not apparent in the verbal description.</para></listitem>
<listitem><inst>	2.	</inst><para>The mathematical model enables a comparison of <emphasis>many</emphasis> possible solutions and aids in selecting the best among them rapidly and efficiently.</para></listitem>
<listitem><inst>	3.	</inst><para>The mathematical model often explains situations that have been left unexplained in the past by indicating cause-and-effect relationships.</para></listitem>
<listitem><inst>	4.	</inst><para>The mathematical model readily indicates the type of data that should be collected to deal with the problem in a quantitative manner.</para></listitem>
<listitem><inst>	5.	</inst><para>The mathematical model facilitates the prediction of future events, such as effectiveness factors, reliability and maintainability parameters, logistics requirements, and so on.</para></listitem>
<listitem><inst>	6.	</inst><para>The mathematical model aids in identifying areas of risk and uncertainty.</para></listitem></orderedlist>
<para>When analyzing a problem in terms of selecting a mathematical model for evaluation purposes, it is desirable to first investigate the tools that are currently available. If a model already exists and is proven, then it may be feasible to adopt that model. However, extreme care must be exercised to relate the right technique with the problem being addressed and to apply it to the depth necessary to provide the sensitivity required in arriving at a solution. Improper application may not provide the results desired, and the consequence may be costly.</para>
<para>Conversely, it might be necessary to construct a new model. In accomplishing this task, one should generate a comprehensive list of system parameters that will describe the situation being simulated. Next, it is necessary to develop a matrix showing parameter relationships, each parameter being analyzed with respect to every other parameter to determine the magnitude of relationship. Model input–output factors and parameter feedback relationships must be established. The model is constructed by combining the various factors and then testing it for validity. Testing is difficult to do because the problems addressed primarily deal with actions in the future that are impossible to verify. However, it may be possible to select a known system or equipment item that has been in existence for several years and exercise the model using established parameters. Data and relationships are known and can be compared with historical experience. In any event, the analyst might attempt to answer the following questions: <emphasis>Can the model describe known facts and situations sufficiently well? When major input parameters are varied, do the results remain consistent and are they realistic? Relative to system application, is the model sensitive to changes in operational requirements, design, production/construction, and logistics and maintenance support? Can cause-and-effect relationships be established?</emphasis></para>
<para>Models and appropriate systems analysis methods are presented further in <link olinkend="part03" preference="0">Part <xref olinkend="part03" label="III"><inst>III</inst></xref></link>. Trade-off and system optimization relies on these methods and their application. A fundamental knowledge of probability and statistics, economic analysis methods, modeling and optimization, simulation, queuing theory, control techniques, and the other analytical techniques is essential to accomplishing effective systems analysis.

\subsection{Analysis Models for Systems Engineering}\index{Analysis Models for Systems Engineering}