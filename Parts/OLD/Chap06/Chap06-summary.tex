The overarching goal of systems engineering is embodied in the title of this chapter. This goal is to bring successful systems and their products into being. Each and every definition of systems
engineering presented in <link olinkend="ch01lev1sec6" preference="0">Section <xref olinkend="ch01lev1sec6" label="1.6"><inst>1.6</inst></xref></link> supports this desired goal. Accordingly, it is appropriate to devote this chapter to a high-level presentation of the essentials involved in the engineering of systems. Subsequent chapters are devoted to specific topics needed to achieve the goal.</para>


4 This excerpt is from the plenary presentation “Beyond Systems Thinking: The role of beauty and love in the transformation of our world” by Dr. Kathia Laszlo at the 55th Meeting of the International Society for the Systems Sciences at the University of Hull, U.K., on July 21, 2011.


<para>This chapter introduces a technologically based interdisciplinary process encompassing an extension of engineering through all phases of the system life cycle; that is, design and development, production or construction, utilization and support, and phase-out and disposal. The process is derived from the systems concepts and general systems thinking presented in <link olinkend="ch01" preference="0">Chapter <xref olinkend="ch01" label="1"><inst>1</inst></xref></link>. Upon completion, <link preference="0" linkend="ch02">Chapter 6</link> should provide the reader with an in-depth understanding of the following:</para>
    1. <itemizedlist mark="bull" spacing="normal"><listitem><inst>	</inst><para>A more complete definition and description for the category of systems that are human-made, in contrast with the definition of general systems given in <link olinkend="ch01" preference="0">Chapter <xref olinkend="ch01" label="1"><inst>1</inst></xref></link>;</para></listitem>
    2. <listitem><inst>	</inst><para>The product as part of the engineered system, yet distinguishable from it, with emphasis on the system as the overarching entity to be brought into being;</para></listitem>
    3. <listitem><inst>	</inst><para>Product and system categories with life-cycle engineering and design as a generic paradigm for the realization of competitive products and systems;</para></listitem>
    4. <listitem><inst>	</inst><para>Engineering the relationships among systems to achieve sustainability of the product and the environment, synergy among human-made systems, and continuous improvement of human existence;</para></listitem>
    5. <listitem><inst>	</inst><para>System design evaluation and the multiple-criteria domain within which it is best pursued;</para></listitem>
    6. <listitem><inst>	</inst><para>Integration and iteration in system design, invoking the major activities of synthesis, analysis, and evaluation;</para></listitem>
    7. <listitem><inst>	</inst><para>A morphology for synthesis, analysis, and evaluation and its effective utilization within the systems engineering process;</para></listitem>
    8. <listitem><inst>	</inst><para>The importance of investing in systems thinking and engineering early in the life cycle and the importance thereto of systems engineering management; and</para></listitem>
    9. <listitem><inst>	</inst><para>Potential benefits to be obtained from the proper and timely implementation of systems engineering and analysis.</para></listitem></itemizedlist>
<para>The engineered or technical system is to be brought into being; it is a system destined to become part of the human-made world. Therefore, the definition and description of the engineered system is given early in this chapter. It narrows the conceptualization of systems set forth in <link olinkend="ch01lev1sec1" preference="0">Sections <xref olinkend="ch01lev1sec1" label="1.1"><inst>1.1</inst></xref></link> and <link olinkend="ch01lev1sec2" preference="0"><xref olinkend="ch01lev1sec2" label="1.2"><inst>1.2</inst></xref></link>. In most cases, there is a product coexistent with or within the system, and in others the system is the product. But in either case, there must exist a human need to be met.</para>
<para>Since systems are often known by their products, product and system categories are identified as frameworks for study in this and subsequent chapters. Major categories are single-entity product systems and multiple-entity population systems. Availability of these example categories is intended to help underpin and clarify the topics and steps in the process of bringing engineered or technical systems into being.</para>
<para>The product and system life cycle is the <emphasis>enduring paradigm</emphasis> used throughout this book. It is argued that the defense origin of this life-cycle paradigm has profitable applications in the private sector. The life cycle is first introduced in <link linkend="ch02lev1sec2" preference="0" type="backward">Section <xref linkend="ch02lev1sec2" label="2.2"><inst>6.2</inst></xref></link> with two simple diagrams; the first provides the product and the second gives an expanded concurrent life-cycle view. Then, designing for the life cycle is addressed with the aid of more elaborate life-cycle diagrams, showing many more activities and interactions. Other systems engineering process models are then exhibited to conclude an overview of the popular process structures for bringing systems and products into being.</para>
<para>Since design is the fundamental technical activity for both the product and the system, it is important to proceed with full knowledge of all system design considerations. The identification of DDPs and their counterparts, DIPs, follows. Emanating from DDPs are technical performance measures to be predicted and/or estimated. The deviation or difference between predicted TPMs and customer-specified criteria provides the basis for design improvement through iteration, with the expectation of convergence to a preferred design. During this design activity, criteria or requirements must be given center stage. Accordingly, the largest section of this chapter is devoted to an explanation of design evaluation based on customer-specified criteria. The explanation is enhanced by the development and presentation of a 10-block morphology for synthesis, analysis, and evaluation.</para>
<para>This chapter closes with some challenges and opportunities that will surely arise during the implementation of systems engineering. The available application domains are numerous. A general notion is that systems engineering is an engineering interdiscipline in its own right, with important engineering domain manifestations. It is hereby conjectured that the systems engineering body of systematic knowledge will not advance significantly without engineering domain opportunities for application. However, it is clear that significant improvements in domain-specific projects do occur when resources are allocated to systems engineering activities early in the life cycle. Two views of this observed benefit are illustrated in this chapter.</para>
<para>It is recognized that some readers may need and desire to probe beyond the content of this textbook. If so, we would recommend two edited works: The <emphasis>Handbook of Systems Engineering and Management</emphasis>, A. P. Sage and W. B. Rouse (Eds), John Wiley \& Sons, Inc., 2009, augments the technical and managerial topics encompassed by systems engineering. <emphasis>Design and Systems</emphasis>, A. Collen and W. W. Gasparski (Eds), Transaction Publishers, 1995, makes visible the pervasive nature of design in the many arenas of human endeavor from a philosophical and praxiological perspective.
Regarding the body of systems engineering knowledge, there is a timely project being pursued within the INCOSE the SEBoK (Systems Engineering Body of Knowledge) activity involving hundreds of members. Interested individuals may review the current state of development and/or make contributions to it by visiting . An earlier effort along this line was to engage the intellectual leaders of INCOSE (including the authors) in the writing of 16 seminal articles. These were published in the inaugural issue of <emphasis>Systems Engineering</emphasis>, Vol. 1, No. 1, July September 1994. Copies of this special issue and subsequent issues of the journal may be obtained through the INCOSE website.