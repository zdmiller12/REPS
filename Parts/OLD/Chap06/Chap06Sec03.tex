The systems engineering process is suggested as a preferred approach for bringing systems and their products into being that will be cost-effective and globally competitive. An essential technical activity within the process is that of system design evaluation. Evaluation must be inherent within the systems engineering process and be invoked regularly as the system design activity progresses. However, system evaluation should not proceed without either accepting guidance from customer requirements and applicable system design criteria, or direct involvement of the customer. When conducted with full recognition of customer requirements and design criteria, evaluation enhances assurance of continuous design improvement.

\subsection{Development of Design Criteria}\index{Development of Design Criteria}

As was depicted in 6.1, the definition of needs at the system level is the starting point for determining customer requirements and developing design criteria. The requirements for the system as an entity are established by describing the functions that must be performed. The operational and support functions (i.e., those required to accomplish a specified mission scenario, or series of missions, and those required to ensure that the system is able to perform the needed functions when required) must be described at the top level. Also, the general concepts and nonnegotiable requirements for production, systems integration, and retirement must be described.
<para>In design evaluation, an early step that fully recognizes design criteria is to establish a <emphasis>baseline</emphasis> against which a given alternative or design configuration may be evaluated (refer to ). This baseline is determined through the iterative process of requirements analysis (i.e., identification of needs, analysis of feasibility, definition of system operational and support requirements, and selection of concepts for production, systems integration, and retirement). A more specific baseline is developed for each <emphasis>system level</emphasis> in. The functions that the system must perform to satisfy a specific scope of customer needs should be described, along with expectations for cycle time, frequency, speed, cost, effectiveness, and other relevant factors. Functional requirements must be met by incorporating design characteristics within the system at the appropriate level.
<As part of this process, it is necessary to establish some system “metrics” related to performance, effectiveness, cost, and similar quantitative factors as required to meet customer expectations. For instance, what functions must the system perform, where are these to be accomplished, at what frequency, with what degree of reliability, and at what cost? Some of these factors may be more important than others by the customer which will, in turn, influence the design process by placing different levels of emphasis on meeting criteria. Candidate systems result from design synthesis and become the appropriate targets for design analysis and evaluation.</para>
<para>Evaluation is invoked to determine the degree to which each candidate system satisfies design criteria. Applicable criteria regarding the system should be expressed in terms of <emphasis>technical performance measures</emphasis> (TPMs) and exhibited at the system level. TPMs are measures for characteristics that are, or derive from, attributes inherent in the design itself. Attributes that depend directly on design characteristics are called <emphasis>design-dependent parameters</emphasis> (DDPs), with specific measures thereof being the TPMs. In contrast, relevant factors external to the design are called <emphasis>design-independent parameters</emphasis> (DIPs).</para>
<para>It is essential that the development of <emphasis>design criteria</emphasis> be based on an appropriate set of <emphasis>design considerations</emphasis>, considerations that lead to the identification of both <emphasis>design-dependent</emphasis> and <emphasis>design-independent parameters</emphasis> and that support the derivation of <emphasis>technical performance measures</emphasis>. More precise definitions for these terms are as follows:</para>
1.	Design considerations the full range of attributes and characteristics that could be exhibited by an engineered system, product, or service. These are of interest to both the producer and the customer.
2.	Design-dependent parameters (DDPs) attributes and/or characteristics inherent in the design for which predicted or estimated measures are required or desired (e.g., design life, weight, reliability, producibility, maintainability, pollutability, and others).
3.	Design-independent parameters (DIPs) factors external to the design that must be estimated and/or forecasted for use during design evaluation (e.g., fuel cost per pound, labor rates, material cost per pound, interest rates, and others). These depend upon the production and operating environment for the system.
4.	Technical performance measures (TPMs) predicted and/or estimated values for DDPs. They also include values for higher level (derived) performance considerations (e.g., availability, cost, flexibility, and supportability).
5.	Design criteria customer specified or negotiated target values for technical performance measures. Also, desired values for TPMs as specified by the customer as requirements.</para></listitem></orderedlist>

<para>The issue and impact of multiple criteria will be presented in the paragraphs that follow. Then, the next section will direct attention to design criteria as an important part of a morphology for synthesis, analysis, and evaluation. In so doing, the terms defined above will be better related to each other and to the system realization process.

\subsection{Considering Multiple Criteria}\index{Considering Multiple Criteria}

In, the prioritized TPMs at the top level reflect the overall performance characteristics of the system as it accomplishes its mission objectives in response to customer needs. There may be numerous factors, such as system size and weight, range and accuracy, speed of performance, capacity, operational availability, reliability and maintainability, supportability, cost, and so on. These <emphasis>measures of effectiveness</emphasis> (MOEs) must be specified in terms of a level of importance, as determined by the customer based on the criticality of the functions to be performed.</para>
<para>For example, there may be certain mission scenarios where system availability is critical, with reliability being less important if there are maintainability considerations built into the system that facilitate ease of repair. Conversely, for missions where the accomplishment of maintenance is not feasible, reliability becomes more important. Thus, the nature and criticality of the mission(s) to be accomplished will lead to the identification of specific requirements and the relative levels of importance of the applicable TPMs.</para>
<para>Given the requirements at the top level, it may be appropriate to develop a “design-objectives” tree similar to that presented in. First , second , and third order (and lower-level) considerations are noted. Based on the established MOEs for the system, a top-down breakout of requirements will lead to the identification of characteristics that should be included and made inherent within the design; for example, a first-order consideration may be <emphasis>system value</emphasis>, which, in turn, may be subdivided into <emphasis>economic</emphasis> factors and <emphasis>technical</emphasis> factors.</para>
<para>Technical factors may be expressed in terms of <emphasis>system effectiveness</emphasis>, which is a function of performance, operational availability, dependability, and so on. This leads to the consideration of such features as speed of performance, reliability and maintainability, size and weight, and flexibility. If maintainability represents a high priority in design, then such features as packaging, accessibility, diagnostics, mounting, and interchangeability should be stressed in the design. Thus, the <emphasis>criteria</emphasis> for design and the associated DDPs should be established early, during conceptual design, and then carried through the entire design cycle. The DDPs establish the exent and scope of the design space within which <emphasis>trade-off</emphasis> decisions may be made. During the process of making these trade-offs, requirements must be related to the appropriate hierarchical level in the system structure (i.e., system, subsystem, and configuration item) as in 

\subsection{System(s) of Systems and System(s) of Systems Engineering}\index{System(s) of Systems and System(s) of Systems Engineering}

System of Systems:Modern systems that comprise system of systems problems are not monolithic; rather they have five common characteristics: operational independence of the individual systems, managerial independence of the systems, geographical distribution, emergent behavior and evolutionary development:
Source:Sage, A.P., and C.D. Cuppan. "On the Systems Engineering and Management of Systems of Systems and Federations of Systems," Information, Knowledge, Systems Management, Vol. 2, No. 4, 2001, pp. 325-345.

System of Systems:System of systems problems are a collection of trans-domain networks of heterogeneous systems that are likely to exhibit operational and managerial independence, geographical distribution, and emergent and evolutionary behaviors that would not be apparent if the systems and their interactions are modeled separately.
Source:DeLaurentis, D. "Understanding Transportation as a System of Systems Design Problem," 43rd AIAA Aerospace Sciences Meeting, Reno, Nevada, January 10-13, 2005. AIAA-2005-0123.

Robert's Comments:I like the above definitions of system of systems a lot, because they make a useful distinction between systems in general, and systems, the engineering of which gives rise to particular challenges and emphases, because of the distinguishing features stated in the definitions. Of course, in a purely literal sense, almost any system is a system of systems (things constructed of two or more interacting parts).

System-of-Systems Engineering (SoSE):System-of-Systems Engineering (SoSE) is a set of developing processes, tools, and methods for designing, re-designing and deploying solutions to System-of-Systems challenges.
Robert's Comments: The above definition of System-of-Systems Engineering (SoSE) sits perfectly with the two definitions of system of systems.

System of Systems Engineering (SoSE):System of Systems (SoS) Engineering is an emerging interdisciplinary approach focusing on the effort required to transform capabilities into SoS solutions and shape the requirements for systems. SoS Engineering ensures that:
1. Individually developed, managed, and operated systems function as autonomous constituents of one or more SoS and provide appropriate functional capabilities to each of those SoS
2. Ppolitical, financial, legal, technical, social, operational, and organizational factors, including the stakeholders' perspectives and relationships, are considered in SoS development, management, and operations
3. A SoS can accommodate changes to its conceptual, functional, physical, and temporal boundaries without negative impacts on its management and operations
a SoS collective behavior, and its dynamic interactions with its environment to adapt and respond, enables the SoS to meet or exceed the required capability.
Source:Systems of Systems Engineering Center of Excellence

Robert's Comments:The first bullet point above seems to capture the essence of System-of-Systems Engineering (SoSE). Regarding the second bullet point, one would hope and expect that applicable political, financial, legal, technical, social, operational, and organizational factors, if any, including the stakeholders' perspectives and relationships, are considered in system development without exception, regardless of the system. To do otherwise is bad systems engineering. The third bullet point seems obtuse. The last bullet point would seem to apply to every engineered system, no exceptions.
System-of-Systems Engineering (SoSE):System of Systems (SoS) Engineering deals with planning, analyzing, organizing, and integrating the capabilities of a mix of existing and new systems into a SoS capability greater than the sum of the capabilities of the constituent parts.
SoSs should be treated and managed as a system in their own right, and should therefore be subject to the same systems engineering processes and best practices as applied to individual systems.
Differs from the engineering of a single system. The considerations should include the following factors or attributes:
Larger scope and greater complexity of integration efforts;
Collaborative and dynamic engineering;
Engineering under the condition of uncertainty;
Emphasis on design optimization;
Continuing architectural reconfiguration;
Simultaneous modeling and simulation of emergent System of Systems behavior; and
Rigorous interface design and management.
Source:Defense Acquisition Guidebook (DAG) - 2006 Definition of SoSE

Robert's Comments:The above definition of System of Systems (SoS) Engineering would seem to apply to the engineering of all systems. Of the list of so-called differences, the first is arguable; continuing architectural reconfiguration may or may not apply to any system depending on circumstances, and the rest of the alleged differences would seem to be factors in engineering of any system.

I am not convinced that a differentiation of "system of systems engineering" and "systems engineering" is helpful, at least with respect to choice of words. However, recognition of the specific additional challenges of engineering systems from subsystems that are subject to operational and managerial independence, may be geographically distributed, and may well exhibit evolutionary behaviors not under the control of the developers of the parent (SoS) system – that is extremely valuable. Research in this area from participants such as Purdue University's College of Engineering (USA), National Centers for System of Systems Engineering (Old Dominion University – USA), and others will contribute to mankind's ability to successfully engineer complex socio-technical systems.