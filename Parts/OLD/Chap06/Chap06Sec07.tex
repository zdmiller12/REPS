Within the context of synthesis, analysis, and evaluation is the opportunity to implement systems engineering over the system life cycle in measured ways that can help ensure its effectiveness. These measured implementations are necessary because the complexity of technical systems continues to increase, and many systems in use today are not meeting the needs of the customer in terms of performance, effectiveness, and overall cost. New technologies are being introduced on a continuing basis, while the life cycles for many systems are being extended. The length of time that it takes to develop and acquire a new system needs to be reduced, the costs of modifying existing systems are increasing, and available resources are dwindling. At the same time, there is more international cooperation, and competition is increasing worldwide.</para>
	<para>There are many categories of human-made systems, and there are several application domains where the concepts and principles of systems engineering can be effectively implemented. Every time that there is a newly identified need to accomplish some function, a new <emphasis>system</emphasis> requirement is established. In each instance, there is a new design and development effort that must be accomplished at the <emphasis>system</emphasis> level. This, in turn, may lead to a variety of approaches at the subsystem level and below (i.e., the design and development of new equipment and software, the selection and integration of new commercial off-the-shelf items, the modification of existing items already in use, or combinations thereof).</para>
<para>Accordingly, for every new customer requirement, there is a needed design effort for the system overall, to which the steps described in Section are applicable. Although the extent and depth of effort will vary, the concepts and principles for bringing a system into being are basically the same. Some specific application areas are highlighted in, and application domains include the following:
1.	</inst><para>Large-scale systems with many components, such as a space-based system, an urban transportation system, a hydroelectric power-generating system, or a health-care delivery system.</para></listitem>
<listitem><inst>	2.	</inst><para>Small-scale systems with relatively few components, such as a local area communications network, a computer system, a hydraulic system, or a mechanical braking system, or a cash receipt system.</para></listitem>
<listitem><inst>3. 	</inst><para>Manufacturing or production systems where there is input–output relationships, processes, processors, control software, facilities, and people.</para></listitem>
<listitem><inst>	4.	</inst><para>Systems where a great deal of new design and development effort is required (e.g., in the introduction of advanced technologies).</para></listitem>
<listitem><inst>	5.	</inst><para>Systems where the design is based largely on the use of existing equipment, commercial software, or existing facilities.</para></listitem>
<listitem><inst>	6.	</inst><para>Systems that are highly equipment, software, facilities, or data intensive.</para></listitem>
<listitem><inst>	7.	</inst><para>Systems where there are several suppliers involved in the design and development process at the local, and possibly international, level.</para></listitem>
<listitem><inst>	8.	</inst><para>Systems being designed and developed for use in the defense, civilian, commercial, or private sectors separately or jointly.</para></listitem>
<listitem><inst>	9.	</inst><para>Human-modified systems wherein a natural system is altered or augmented to make it serve human needs more completely, while being retained/sustained largely in its natural state.

\subsection{Recognizing and Managing Life Cycle Impacts}\index{Recognizing and Managing Life Cycle Impacts}

In evaluating past experiences regarding the development of technical systems, it is discovered that most of the problems experienced have been the direct result of not applying a <emphasis>disciplined</emphasis> top-down “systems approach.” The overall requirements for the system were not defined well from the beginning; the perspective in terms of meeting a need has been relatively “short term” in nature; and, in many instances, the approach followed has been to “deliver it now and fix it later,” using a bottom-up approach to design. The systems design and development process has suffered from the lack of good early planning and the subsequent definition and allocation of requirements in a complete and methodical manner. Yet, it is at this early stage in the life cycle when decisions are made that have a large impact on the overall effectiveness and cost of the system. This is illustrated conceptually in 
<para>Referring to, experience indicates that there can be a large commitment in terms of technology applications, the establishment of a system configuration and its performance characteristics, the obligation of resources, and potential life-cycle cost at the early stages of a program. It is at this point when system-specific knowledge is limited, but when major decisions are made pertaining to the selection of technologies, the selection of materials and potential sources of supply, equipment packaging schemes and levels of diagnostics, the selection of a manufacturing process, the establishment of a maintenance approach, and so on. It is estimated that from 50\% to 75\% of the projected life-cycle cost for a given system can be committed (i.e., “locked in”) based on engineering design and management decisions made during the early stages of conceptual and preliminary design. Thus, it is at this stage where the implementation of systems engineering concepts and principles is critical. It is essential that one start off with a good understanding of the customer need and a definition of system requirements.</para>
<para>The systems engineering process is applicable over all phases of the life cycle, with the greatest benefit being derived from its emphasis on the early stages, as illustrated in . The objective is to influence design early, in an effective and efficient manner, through a comprehensive needs analysis, requirements definition, functional analysis and allocation, and then to address the follow-on activities in a logical and progressive manner with the provision of appropriate feedback. As conveyed in , the overall objective is to influence design in the early phases of system acquisition, leading to the identification of individual discipline-based design needs. These should be applied in a timely manner as one evolves from system level requirements to the design of various subsystems and components thereof.

\subsection{Systems Thinking for Systems Being}\index{Systems Thinking for Systems Being}

Systems thinking is a gateway to seeing interconnections. Once we see a new reality, we cannot go back and ignore it. More importantly, that “seen” has an emotional connection, beautifully captured in the statement by Rusty Schweickart after his experience of seeing his home planet from space. OWH, a man’s mind one expanded cannot go back to …
This is the leading edge of the sustainability movement: the realization that no matter how many solar panels we install, how many green products we consume, how much CO2 we remove from the atmosphere, we will not be living better lives if we do not transform ourselves, our lifestyles, choices and priorities. Sustainability is an inside job, a learning journey to live lightly, joyfully, peacefully, meaningfully.
Systems being involves embodying a new consciousness, an expanded sense of self, a recognition that we cannot survive alone, that a future that works for humanity needs also to work for other species and the planet. It involves empathy and love for the greater human family and for all our relationships – plants and animals, earth and sky, ancestors and descendents, and the many peoples and beings that inhabit our Earth. This is the wisdom of many indigenous cultures around the world, this is part of the heritage that we have forgotten, and we are in the process of recovering.
Systems being and systems living brings it all together: linking head, heart and hands. The expression of systems being is an integration of our full human capacities. It involves rationality with reverence to the mystery of life, listening beyond words, sensing with our whole being, and expressing our authentic self in every moment of our life. The journey from systems thinking to systems being is a transformative learning process of expansion of consciousness—from awareness to embodiment. Kathia Laszlo, Ph.D., directs Saybrook University's program in Leadership of Sustainable Systems.