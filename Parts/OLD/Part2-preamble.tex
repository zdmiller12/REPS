Basic requirements for successful engineering in the Systems Age are then addressed as a prerequisite for introducing, describing, and defining systems engineering in Chapter 5 that begins with science and engineering relationships, distinguishing them by natural systems and the human made. Within this description are the important symbiotic relationships between the product and the system, a classification of product and system categories, and the recognition of the economic importance of product competitiveness. System life-cycle engineering is then offered as a generic paradigm for bringing engineered systems into being, implemented by designing for the life cycle and guided by a time-phased engineering process. Then, a unifying morphology for synthesis, analysis, and evaluation invoking integration and iteration brings order to the process of system realization.

System design is not systems engineering per se, but it is almost as pervasive. From the perspective of synthesis, system design reaches almost as far and deep as systems engineering. It is nominally comprised of conceptual, preliminary, and detail design. These are artificial categories that, along with test and evaluation, make up third chapters in Part II of this textbook. Included in these chapters is a high degree of connectivity among the design categories, as well as an elaboration upon the theme of ``bringing into being,'' promulgated in Chapter 6.

The closely related topics in Chapter 5 begin with a definition and description of engineered systems, distinguishing them from natural systems. Within this description are the important symbiotic relationships between the product and the system, a classification of product and system categories, and the recognition of the economic importance of product competitiveness. System life-cycle engineering is then offered as a generic paradigm for bringing engineered systems into being, implemented by designing for the life cycle and guided by a system engineering process. For completeness, some other popular systems engineering process models are introduced. Then, ad unifying morphology for synthesis, analysis, and evaluation invoking integration and iteration brings order to the process of system realization. The importance and benefit of investing in and implementing systems thinking and engineering early in the system life cycle is the central emphasis of Part II.