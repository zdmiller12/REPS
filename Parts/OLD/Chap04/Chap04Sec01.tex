Economic considerations embrace many of the subtleties and complexities characteristic of people. Economics is derived from the behavior of humans individually and collectively, particularly as their behavior relates to the satisfaction of wants.

Human wants have always exceeded the means of satisfying them. The relative scarcity of good and services has been and will continue to be an economic dilemma that all must face. Some wants cannot be satisfied at all, other can be partially satisfied, and only a few wants can be fully satisfied. Added to the elusiveness in satisfying wants are changes that occur.  As certain wants are satisfied, additional wants develop.

Some human wants are more predictable than others. The demand for food, clothing, and shelter needed for bare physical existence is more stable and predictable than the demand for those things that satisfy people’s emotional needs. The number of calories of energy needed to sustain life may be determined fairly accurately. Clothing and shelter requirements may be predicted within narrow limits from climatic conditions. Once people are assured of physical existence, they demand satisfactions of a less predictable nature resulting from being members and products of a society rather than only biological organisms.

The wants of people are motivated largely by emotional drives and tensions and, to a lesser extent, by logical reasoning processes. A part of human wants can be satisfied by physical goods and services, but people are rarely satisfied by physical things alone. In food, sufficient calories to meet physical needs will rarely satisfy. People want the food they eat to satisfy both energy needs and emotional needs. In consequence, people are concerned with the flavor of food, its consistency, the china and silverware with which it is served, the person or persons who serve it, the people in whose company it is eaten, and the atmosphere of the room in which it is served. Similarly, there are many desires associated with clothing and shelter in addition to those required merely to meet physical needs.

Much or little progress has been made in obtaining knowledge on which to base predictions of human behavior, depending on one’s viewpoint. The idea that human reactions will someday be well enough understood to be predictable is accepted by many people; even though this has been the objective of the thinkers of the world since the beginning of time, it appears the progress in psychology has been meager compared with the rapid progress made in the physical sciences. Even though human behavior can be neither predicted nor explain, it must be considered by those who are concerned with economic decision analysis.

What does ``scarce'' mean?  It means that what everybody wants ads up to more than there is. This may seem like a simple thing, but its implications are often grossly misunderstood, even by highly educated people. For example, a feature article in the New York Times laid out the economic woes and worries of middle-class Americans – one of the most affluent groups of human beings ever to inhabit this planet. Although this story included a structure of a middle-class American family in their own swimming pool, main headline read: ``The American Middle, Just Getting By''

In short, middle-class Americans’ desires exceed what they can comfortably afford, even though wat they already have would be considered unbelievable prosperity by people in many other countries around the world – or even by earlier generations of Americans. Yet both they and the reporter regarded them as “just getting by” and a Harvard sociologist was quoted as saying “how budget-constrained these people really are.”  But it is not something as man-made as a budget which constrains them: Reality constrains them. There has never been enough to satisfy everyone completely. That is the real constraint. That is what scarcity means.

To all people – from academia and journalism, as well as the middle-class people themselves – is apparently seemed strange somehow that there should be such a thing as scarcity and that this should imply a need for both productive efforts on their part and personal responsibility in spending the resulting income. Yet nothing has been more pervasive in the history of the human race than scarcity and all the requirements for economizing that go with scarcity.

Regardless of our policies, practices, or institutions - whether they are wise or unwise, noble or ignoble - there is simply not enough to go around to satisfy all our desires to the fullest. ``Unmet needs'' are inherent in these circumstances, whether we have a capitalist, socialist, feudal, or other kind of economy. These various kinds of economics are just differently institutional ways of making trade-offs that are inescapable in any economy.

Economics is not just about dealing with the existing output of goods and services as consumers. It is also, and more fundamentally, about producing that output from scarce resources in the first place - turning inputs into output.

In other words, economics studies the consequences of decisions that are made about the use of land, labor, capital and other resources that go into producing the volume of output which determines a country’s standard of living. Those decisions and their consequences can be more important than the resources and countries like Japan and Switzerland with relatively few natural resources but high standards of living. The values of natural resources per capita in Uruguay and Venezuela are several times what they are in Japan and Switzerland, but real income per capita in Japan and Switzerland is more than double that of Uruguay and several times that of Venezuela.

Not only scarcity but also ``alternative uses'' are at the heart of economics. If each resource had only one use, economics would be much simpler. But water can be used to produce ice or steam by itself or innumerable mixtures and compounds in combination with other things. Similarly, from petroleum comes not only gasoline and heating oil, but also plastics, asphalt and Vaseline. Iron ore can be used to produce steel products ranging from paper clips to automobiles to the frameworks of skyscrapers.

How much of each resource should be allocated to each of its many uses?  Every economy has to answer that question, and each one does, in one way or another, efficiently or inefficiently. Doing so efficiently is what economics is about. Different kinds of economics are essentially different ways of making decisions about the allocation of scarce resources - and those decisions have repercussions on the life of the whole society.

During the days of the Soviet Union, for example, that country’s industries used more electricity than American industries used, even though Soviet industries produced a smaller amount of output than American industries produced. Such inefficiencies in turning inputs into outputs translated into a lower standard of living, in a country richly endowed with natural resources - perhaps more richly endowed than any other country in the world. Russia is, for example, one of the few industrial nations that produces more oil than it consumes. But an abundance of resources does not automatically create an abundance of goods.

Efficiency in production - the rate at which inputs are turned into output - is not just some technicality that economists talk about. It affects the standard of living of whole societies. When visualizing this process, it helps to think about the real things - the iron ore, petroleum, wood and other inputs that go into the production process and the furniture, food and automobiles that come out the other end - rather than think of economic decisions as being simply decisions about money. Although the word ``economics'' suggests money to some people, for a society as a whole money is just an artificial device to get real things done. Otherwise, the government could make us all rich by simply printing more money. It is not money but the volume of goods and services which determines whether a country is poverty stricken or prosperous.