Austrian Economics emanates from the Ludwig Von Mises Institute adjacent to the campus of Auburn University. Based in the axiom of Human Action, Mises realized that the axiom of human action is an a priori synthetic judgment. The axiom of human action says that humans act. This might sound trivial at first glance. At second glance, however, it becomes obvious that the axiom of human action has far-reaching implications.

The axiom of action meets the requirements of an a priori synthetic judgment. First, one cannot observe that humans act in the first place. For doing so, one needs an understanding of what human action is. This knowledge cannot be acquired through experience, because it comes from reason and not from experience.

Second, one cannot deny that humans act, for doing so would result in an intellectual contradiction. Saying ``humans cannot act'' is itself a form of human action, and it would thus contradict the statement’s truth claim.

Mises also realized that, by using formal logic, other truth claims can be deduced from the irrefutably true - propositions that can be validated without taking recourse to experience. Take, for instance, the concept of causality – the idea that each effect has a cause. It is logically implied in the axiom of human action. As Mises put it,

Acting requires and presupposes the category of causality. Only humans see the world in the light of causality is fitted to act. In this sense we may say that causality is a category of action. The category means and ends and presupposes the category cause and effect. In a world without causality and regularity of phenomenon there would be no field for human reasoning and human action. Such a world would be a chaos in which man would be at a loss to find any orientation and guidance. Man is not even capable of imagining the conditions of such a chaotic universe. Where man does not see any causal relation, he cannot act.

Science is a systematic and logical approach to discovering how things in the universe work. It is also the body of knowledge accumulated through the discoveries about all the things in the universe. This is another aspect arising from the fact that science is conducted in a social context. The goal is noble, discovery of reliable knowledge about the world enabling appropriate action. The environment of the activity is fundamentally social, with factors such as who is respected, who is close to power, how persuasively or forcefully participants put forward their views etc., etc. Dr. Tim Ferris.

Yes, science is conducted in a social context which means that there is more transpiring than just science, e.g., who is respected and for what, etc. Similarly, many system design and architecting decisions are taken to placate sponsors or to meet deadlines rather than to maximize system effectiveness. The ``socialness'' of it all highlights the importance of a Standard of Care or at least a lesser Code of Ethics in human activities. Jack Ring.
    
\subsection{Praxeology and Austrian Economics}\index{Praxeology and Austrian Economics}

The Walrasian and the Austrian approaches often come to similar conclusions when it comes to the desirability of markets, but they come to these conclusions using quite different paths. The advantage of the Austrian approach is precisely in the path it takes to come to its conclusions - it maintains several key principles that most of us would have a hard time disagreeing with:

\begin{itemize}
\item Value is in the mind of an individual. It is then by definition subjective and directly unobservable to others.
\item Value is not a physical quantity. Thus, interpersonal comparisons of utility or value are inappropriate.
\item All economic activity is a consequence of individual humans acting on their values.
\end{itemize}

The Walrasian approach often assumes away these principles for the sake of mathematical tractability. This is where the crucial problem arises. To what extent can we believe the conclusions of a model that assumes away the fundamental features of reality as we understand it?  This is actually the most common criticism of the neoclassical defense of markets.

You have probably heard many people blaming the current economic crisis on ``market failure.'' Some would say that markets ``failed'' because real markets are different from the economists’ ``perfect'' models. The logic is as follows: since we can’t trust these market models, neither can we trust the actual markets. There is an error in this logic.

Economics of Exchange. A buyer will purchase an article when he has money available and when he believes that the article has equal or greater utility for him than the amount required to purchase it. Conversely, a seller will sell an article when he believes that the amount of money to be received for the article has greater utility than the article has for him. Thus, an exchange will not take place unless at the time of exchange both parties believe that they will benefit. Exchanges are made when they are thought to result in mutual benefit. This is possible because the objects of exchange are not valued equally by the parties to the exchange.

As an illustration of the economic aspect of exchange, consider the following example. Two workers, upon opening their lunch boxes, discover that one contains a piece of apple pie and the other a piece of cherry pie. Suppose further that Ms. A evaluates her apple pie to have 20 units of utility for her and the cherry pie to have 30 units. Suppose also that Mr. B evaluates his cherry pie to have 20 units of utility for him and the apple pie 40 units. If Ms. A consumes her apple pie and Mr. B consumes his cherry pie, the total utility realized is 20 plus 20, or 40 units. But if the two workers exchange pieces of pie, the resulting utility will be increased to 30 plus 40, or 70 units. The utility in the system can be increased by 30 unites through exchange.

Exchange is possible because consumer utilities are evaluated by the consumer almost entirely, if not entirely, by subjective consideration. Thus, if the workers in the example believe that the exchange has resulted in a gain of net satisfaction to them, no one can deny it. Conversely, at the time of exchange, unless each person valued what he or she had to give less than what he or she was about to receive, an exchange could not occur. Thus, we conclude that an exchange of consumer utilities results in a gain for both parties because the utilities are subjectively evaluated by the participants.

Assuming that each party to an exchange of producer utilities correctly evaluates the objects of exchange in relation to his situation, what makes it possible for each person to gain?  The answer is that the participants are in different economic environments. This fact may be illustrated by an example of a retailer who buys lawn mowers from a manufacturer. For example, at a certain volume of activity the manufacturer finds that he can produce and distribute mowers at a total cost of \$90 per unit and that the retailer buys several mowers at a price of \$110 each. The retailer then finds that by expending an average of \$30 per unit in selling effort, he can sell several mowers to homeowners for \$190 each. Both participants profit by the exchange. The reason the manufacturer profits is that his environment is such that he can sell to the retailer for \$110 several mowers that he cannot sell elsewhere at a higher price, and that he can manufacture lawn mowers for \$90 each. The reason the retailer profits is that his environment is such that he can sell mowers at \$190 each by applying \$30 of selling effort on a mower of certain characteristics which he can buy for \$90 each from the manufacturer in question, but not for less elsewhere.

One may ask: Why does the manufacturer enter the merchandising field and thus increase his profit? or Why doesn’t the retailer enter the manufacturing field?  The answer to these questions is that neither the manufacturer nor the retailer can do so unless each changes the relevant environment. The retailer, for example, lacks physical plant equipment and an organization of engineers and workers competent to build lawn mowers. Also, he may be unable to secure credit necessary to engage in manufacturing, although he may easily secure credit in greater amounts for merchandising activities. It is quite possible that he cannot alter is environment so that he can build mowers for less than \$90. Similar reasoning applies to the manufacturer. Exchange consists essentially of physical activity designed to transfer the control of things from one person to another. Thus, even in exchange, utility is created by altering the physical environment.

Each party in an exchange should seek to give something that has little utility for him but that will have greater utility for the receiver. In this manner, each exchange can result in the greatest gain for each party. Nearly everyone has been a party to such a favorable exchange. When a car becomes stuck in snow, only a slight push may be required to dislodge it. The slight effort involved in the dislodging push may have little utility for the person giving it, so little that he expects no more compensation than a friendly nod. Conversely, it might have great utility for the person whose car was dislodged, so great that he may offer a substantial tip. This is known as the ``range of mutual benefit in exchange.''

The aim of much sales and other research is to find products that not only will have great utility for the buyer but that can be supplied at a low cost, that is, have low utility for the seller. The difference between the utility that a specific good or service has for the buyer and the utility it has for the seller represents the profit or net benefit that is available to divide between buyer and seller. It is called the range of mutual benefit in exchange.

(Figure Here)

Given these insights, the contest in which one would use economic models is quite different from the typical Walrasian approach. In this case, one would say that because markets exist, we may, for illustrative purposes, assume that individuals know the relevant economic characteristics of other individuals in the society. In the typical Walrasian approach, the complete information assumption is a precondition for the existence of efficient markets, while in the Austrian approach, the existence of markets is a precondition for the existence of prices that transform subjective and otherwise unobservable valuations of goods produced and owned by a multitude of individuals into objective and observable metrics.

For many neoclassical economists, the market is a tool (only one of the tools) for allocating production and consumption efficiently. Efficiency here is the state of the world where any change would just make things worse. In this theory, such an “optimal” solution can be reached using means other than the market because of lax assumptions about value and knowledge. More specifically, for a person to determine the optimal allocation of resources in an economy outside of the market process, that person needs to know people’s values, skills, potentials, etc. Thus, in such a model, one needs to assume that these qualities exist as objectively measurable and knowable magnitudes.

Austrians, on the other hand, don’t claim that there is anything like this “optimal” allocation of resources, either within or outside of the market. What they do claim is that, if people want to develop an advanced economy, the market is the way to do this. The path to developing such an economy, the market is the way to do this. The path to developing such an economy is through constant guidance of resource allocation by people’s values reflected in the market prices. In the market, someone will always be dissatisfied with something, but this is not a bad thing. This dissatisfaction is a motive for action and for the improvement of one’s well-being. It is the driving force of the economy.

There are important advantages in being familiar with the Austrian theory. This theory helps one keep in mind fundamental principles such as the subjectivity of value and the incompleteness of information that form the basis for human action. This approach makes it easier to spot errors in one’s economic thinking. One of the common errors is treating economic models as normative standards for reality rather than loose metaphors and illustrations of the logical conclusions resulting from prior theoretical analysis. This error creates a temptation to ``fix'' the reality to fit the model. Often the fix only makes things worse, because it was not the reality that needed fixing. It was, in fact, the economist’s model that did not capture the key features of reality.

Economics is about using scarce resources and available means to achieve the best possible ends. In general, achieving an end is called consumption and applying a means towards an end is called production.

There are four broad categories of means, or factors of production, which are involved in achieving our ends. The first is labor, which refers to our own exertions, whether mental or physical. The second is land, which refers to any of the natural resource existing in nature. The third is time. A certain amount of each of these is required in any production process.

Together, these three factors are called the original factors of production because they exist in nature prior to any human production. The fourth kind of factor is that which is produced, not because it directly brings satisfaction, but because it can be used in a different production process. That fourth factor is capital goods.

Since, all things being equal, people will tend to prefer present over future consumption, it is necessary that a longer production process result in a superior set of consumer goods than a shorter one - enough so to induce people to wait to reap its benefits. The hunter-gatherer will choose hunting over gathering only if he finds the meat he will gain in the future, after presently constructing and using a spear, sufficiently more enticing than the leisure he could enjoy in the present.

An unavoidable feature of capital is that it wears out and therefore must be replaced. An economy that relies on capital must expend work simply to maintain its capital. In an economy that relies on capital, therefore, people must be continually willing to forgo present consumption to maintain their standard of living. ``Since the time spent producing a good could have been consumed immediately as leisure, all production requires that one forgo present consumption for future consumption.''

On the other hand, it is not necessary to save again; this is only necessary for the economy to grow, not to stay the same. The original savings are still around, embodied in the capital goods on which the economy relies. An economy with an abundance of capital goods has a long history of saving and thrift behind it, and as a result has a production process that is long, many-staged, and very productive.

One aspect of any production process not yet mentioned is risk. Since the decision of what to produce takes place in the present whereas consumption is not available until the future, it is always possible that a person could choose incorrectly, and later realize that what he decided to produce was not the best use of his time and resources.

On the free market, goods can be valued in terms of prices, which say what sum of money might be exchanged for them. Prices tend toward the level at which demand equals supply and all the available stock is sold. If the price is higher than this, a seller has the incentive to bid lower to ensure that they sell their stock, and if it is lower, buyers have the incentive to bid higher to make sure they can get the goods they desire.

Consumer goods directly satisfy our desires, so the fact that they are demanded needs no explanation. Their demand and the available supply determine their prices based on the law of supply and demand.

\subsection{The Austrian School and the Entrepreneur}\index{The Austrian School and the Entrepreneur}

Carl Menger is considered the founder of the Austrian School of economics. He described the entrepreneur as a coordinating agent who is both a capitalist and a manager. The entrepreneur owns resources and decides how they will be used. Menger emphasized that entrepreneurs bear uncertainty and take purposeful, decisive action according to the knowledge they have. John Bates Clark and Frank A. Fetter are economists who followed in Menger's approach to economics. Clark believed the entrepreneur must also be the owner of a business. Fetter saw uncertainty-bearing as the key entrepreneurial function. He asserted that an entrepreneur organized and directed production while possessing superior foresight. It is clear that the Austrian School did not start with an emphasis on the entrepreneur as an opportunity discoverer.

While those in the Austrian tradition have always seen the entrepreneur as having a central role in economic affairs, two different strands emerged within the Austrian tradition that led to different conceptions of the entrepreneur. Friedrich von Wieser and F.A. Hayek branched off in emphasizing knowledge, discovery and market process. Wieser saw the entrepreneur as owner, manager, leader, innovator, organizer, and speculator. Hayek emphasized that knowledge is dispersed to individuals throughout the economy. He argued that market competition makes the best use of this dispersed knowledge and brings it to light. Influenced by this strand of thought, Kirzner argues that a competitive market is superior because it best generates entrepreneurial discoveries.

Economists Eugen Böhm-Bawerk, Ludwig von Mises, and Murray Rothbard are considered to compose a different branch of thought than Wieser and Hayek. They emphasize monetary calculation and decision-making under uncertainty. Mises emphasized that the entrepreneur has an anticipative understanding of an uncertain future. Rothbard critiqued Kirzner for not emphasizing the role of entrepreneur as uncertainty-bearer. He also questioned Kirzner's notion that the entrepreneur need not own any resources to perform his function. Rothbard asked, "In what sense can an entrepreneur even make profits if he owns no capital to make profits on?"
