Two classes of goods are recognized by economists: consumer goods and producer goods. Consumer goods are products and services that directly satisfy human wants. Examples of consumer goods are television sets, houses, shoes, books, and health services. Producer goods also satisfy human wants but do so indirectly as a part of the production or the construction process. Broadly speaking, the ultimate end of all production and construction activity is to supply goods and services that people may use or consume to satisfy their wants.

Producer goods are, in the long run, used as a means to an end, namely, that of producing goods and services for human use and consumption. Examples of this class of goods are lathes, bulldozers, ship, and digital computers. Producer goods are an intermediate step in an effort to supply human wants. Such goods are not desired for themselves, but because they may be instrumental in producing something that people can use or consume.

But in the 1870s, Menger boldly applied its implications to the determination of value. He noted that since goods are external to the human person and recognized subjectively as possessing qualities that allow for need satisfaction, they could be differentiated between goods of different order. In Principles, he described first-order goods as being goods that we consume to satisfy needs. These are consumption goods.

Second-order goods are goods required to produce the first-order good, so that while a car may be a first-order good satisfying a felt need for transportation, the second-order goods would include the glass, rubber, chrome, and all the other inputs which make up the car. The third-order goods are all of the goods that are required to produce the second-order goods, and so on, with more complex forms of production being characterized with more distant orders of production.

Nonetheless, the values of all the goods of whatever order are derived from the initial subjective desire on the part of the individual to satisfy a felt need, so that rubber has value not in itself or in the work effort going into its production, but because of the initial human desire for transportation, leading to a human preference for cars with tires. This understanding of goods contrasted greatly with the Classical economist’s notion that the value of economic inputs is based on their technical usefulness in production.
    
\subsection{Utility of Consumer Goods}\index{Utility of Consumer Goods}

People will consider two kinds of utility. One kind embraces the utility of goods and services that they intend to consume personally for the satisfaction they get out of them. Thus, it seems reasonable to believe that the utility people ascribe to goods and services that are consumed directly is in large measure a result of subjective, nonlogical, mental processes.

Marketing analysts apparently find emotional appeals more effective than factual information. An analysis of advertising and sales practices used in selling consumer goods will reveal that they appeal primarily to the senses rather than to reason. Perhaps this is as it should be. If the enjoyment of consumer goods almost exclusively on how one feels about them rather than what one reasons about them, it seems logical to make sales presentations for those things to which customers ascribe utility.

It is not uncommon for a salesperson to call on a prospective customer, describe and explain a certain item, state its price, offer it for sale, and have the offer rejected. This is concrete evidence that the item does not possess sufficient utility at the moment to induce the prospective customer to buy it. In such a situation, the salesperson may be able to induce the prospect to buy it. In such a situation, the salesperson may be able to induce the prospect to listen to further sales talk, during which the prospect may decide to buy on the basis of the original offer. This is evidence that the item now possesses sufficient utility to induce the prospective customer to buy. Because there was no change in the item or the price at which it was offered, there must have been a change in the customer’s attitude or regard for it. The pertinent fact is that a proposition at first undesirable now has become desirable as a result of a change in the customer, not in the proposition.

What about the change?  Several reasons could be advanced. Usually, it would be said that the salesperson persuaded the customer to buy; that is, the salesperson induced the customer to believe something, namely, that the item had sufficient utility to warrant its purchase. There are many aspects to persuasion. It may amount only to calling attention to the availability of an item. A person cannot purchase an item he or she does not know exists. A part of the sales function is to call attention to the things available for sale.

It is observable that persuasive ability is much in demand and is often of inestimable beneficial consequences to all concerned. Persuasion as it applies to the sale of goods is of economic importance to industry. A manufacturer must dispose of the goods he produces. He can increase the salability of his products by building into them greater customer appeal in terms of greater usefulness, greater durability, or greater beauty, or he may elect to accompany his products to market with greater persuasive effort in the form of advertising and sales promotion.
    
\subsection{Utility of Producer Good}\index{Utility of Producer Good}

The second kind of utility that an object or service may have for a person is its utility as a means to an end. Producer goods are not consumed for the satisfaction that can be directly derived from them but as a means of producing consumer goods, usually by facilitating alteration of the physical environment.

Once the kind and amount of consumer goods to be produced has been determined, the kinds and amounts of facilities and producer goods necessary to produce them may be calculated with a high degree of certainty. The energy, ash and other contents of coal, for instance, can be determined very accurately and are the bases of evaluating the utility of the coal. The extent to which producer utility may be considered by logical processes is limed only by technical knowledge and the ability to reason.

Although the utility of the consumer goods is primarily determined subjectively, the utility of the producer goods as a means to an end may be, and usually is, in large measure approached objectively. In this connection consider the satisfaction of the human want for harmonic sounds as in music. Suppose it has been decided that the desire for music can be met by 100,000 compact discs. Then the organization of the artists, the technicians, and the equipment necessary to produce the discs become predominantly objective in character. The amount of material that must be processed to form one disc is calculable to a high degree of accuracy. If a concern has been making discs for some time, it will know the various operations that are to be performed and the unit time for performing them. From these data, the kind and amount of producer service, the amount and kind of labor, and the number of various types of machines are determinable within rather narrow limits. Whereas the determination of the kinds and amounts of consumer goods needed at any one time may depend upon the most subjective of human considerations, the problems associated with their production are quire objective by comparison.