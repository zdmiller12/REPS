Table 3.1 Some Axioms in a Concept of Organizations (Thuesen)

Reference: C.I. Barnard, The Functions of the Executive, Harvard University Press.

    1. All individuals are unique – a product of their heredity and environment – and possessing a limited power of choice.
    2. Individuals are rational – they engage in activity and seek to maximize their satisfactions.
    3. An organization is a system of coordinated human activities directed toward a common objective.
    4. Organizations come into being to accomplish objectives that either cannot be accomplished through individual activity or cannot be accomplished as readily through individual effort.
    5. Individuals contribute their activity to an organization if they believe they are receiving or anticipate receiving more from the organization than they are giving to it.
    6. For an organization to remain in existence, it must be effective – it must make progress towards the attainment of its objective, and it must be efficient – it must distribute more benefits than it requires in burdens as subjectively viewed by the contributors.
    7. All organizations have objectives which may or may not be explicitly stated. The objectives of individuals in an organization may or may not be the same as the objective of the organization.
    8. Organizational objectives may sometimes be detected by noting the activities in which the organization engages.
    9. Most organizations are subordinate to superior organizations which restrict the objectives of the subordinate organization.
    10. The process of establishing subordinate organizations with intermediate objectives is the process of specialization and is ultimately responsible for the effectiveness of the total organization.
    11. The accomplishment of these intermediate objectives should facilitate accomplishment of the total objective.
    12. Organizations are essentially unstable. They fail because of 
        a. A hostile environment
        b. The accomplishment of the objective
        c. Internal problems – the managerial process
    13. Management is the activity of maintaining an organization and it is accomplished through communication.
    14. The size and nature of the basic organizational unit will depend upon
        a. The complexity of purpose or objective
        b. The communication processes
        c. The nature of organizational participants.
    15. All organizations will require
        a. An objective
        b. Communication
        c. Participants willing to serve
    16. The organization must dispense incentives. These may include
        a. Material objects – money, air-conditioned office, etc.
        b. Personal satisfactions – prestige, recognition, opportunity, etc.
        c. Organizational goals – ideals, pride of workmanship, etc.
        d. Associations – social needs, etc.
    17. Activities are coordinated through communication. Authority resides with the recipient of an order. If an order is accepted, the communication was authoritative.
    18. Orders are likely to be accepted and authority granted
        a. If the order is understood, and
        b. It is consistent with the purpose of the organization, and
        c. It is not incompatible with personal interests, and
        d. The recipient is mentally and physically able to comply.
    19. Authority is granted upward in the organization as an economical means of securing cooperation, and thus maintaining the organization.
    20. Informal organizations are an aggregate of personal contacts and interactions without a clearly defined purpose, communication network or incentives.
    21. Informal organizations are always present in formal organizations. They establish attitudes, social norms and they facilitate communication. Society is structured by formal organizations and conditioned by informal organizations.
    22. The function of the manager in an organization is
        a. Formulate the objectives of the organization
        b. Secure the services of individuals, and 
        c. Coordinate their activity through communication.
    23. The responsible manager is one that is governed by the moral code of the organization. Thus, as a good leader, he is consistent, predictable and easy to follow.
    24. Complete individual freedom and organizational activity are not possible. The individual will give of his freedom and participate in an organization if he satisfactions will increase by so doing.
    25. Freedom should mean the freedom to restrict oneself within an organization, commensurate with the common good.