Systems engineering and analysis is promulgated most effectively when supported by insight about important system domains having high-level association.  That is the purpose of four chapters comprising \Cref{part:1} of this textbook.  The relevant domains are the world in which we live, systems thinking and knowledge, organization and enterprise systems, and economics and enterprise economy. Together these provide a prerequisite foundation for the synthesis, analysis, and evaluation of engineered systems; engineered systems being the central focus of this book.
	
The first chapter introduces the world in which we live as the overarching system of primary interest. When partitioned into interconnected natural, human-made, and human-modified sectors, it is found that human activities may be traced and evaluated at high levels with purpose and confidence.
	
\Cref{chap:02} has a science and systems science orientation.  It covers general system definitions and ends with contemporary definitions of systems engineering.  Between these important definitional categories is a conceptual discussion of system elements, a high-level classification of systems, a summary of science and systems science, and a view of technology as the progenitor for engineered systems.
	
Cooperation through organization is essential for human progress. It addressed in \Cref{chap:03} as humankind’s most important innovation. The role of purposeful human action, known as praxeology, is emphasized as being most relevant to systems thinking and enterprise systems.
	
Praxeology also underpins \Cref{chap:04}, focused on economics and enterprise economy.  Here the human action form of economic thinking is featured as the basis for economic progress. Taken together, these last two chapters address organization as humankind’s most important innovation within the pervasive economic setting for enterprises and the economy.

\Cref{part:1} of this textbook has the potential to stand alone as a learning module.  It provides a high-level introduction to systems, science, engineering, and systems engineering that will be suitable for discussion with engineering and technical people.  Subsequent chapters should be considered only after obtaining an understanding of the fundamentals in Part I, but study beyond these fundamentals is optional for those who need only an overview of systems engineering and its foundation in systems analysis.