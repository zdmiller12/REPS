The word ``science'' is derived from the Latin word Scientia, which is knowledge based on demonstrable and reproducible data, according to the Merriam-Webster Dictionary. True to this definition, science aims for measurable results through testing and analysis. Science is based on fact, not opinion or preferences. The process of science is designed to challenge ideas through research. One important aspect of the scientific process is that it focuses only on the natural world, according to the University of California. Anything that is based on faith alone or is considered supernatural does not fit into the definition of science.

\subsection{Deductive and Inductive Reasoning}\index{Deductive and Inductive Reasoning}

During the scientific process, deductive reasoning is used to reach a logical true conclusion. Another type of reasoning, inductive, is also used. Often, deductive reasoning and inductive reasoning are confused. It is important to learn the meaning of each type of reasoning so that proper logic can be identified.

Deductive Reasoning. is a basic form of valid reasoning. Deductive reasoning, or deduction, starts out with a general statement, or hypothesis, and examines the possibilities to reach a specific, logical conclusion, according to the University of California. The scientific method uses deduction to test hypothesis and theories. ``In deductive interference, we hold a theory and based on it we make a prediction of its consequences. That is, we predict what the observations should be if the theory were correct. We go from the general – the theory – to the specific – the observations,'' said Dr. Sylvia Wassertheil-Smoller, a researcher and professor emerita at Albert Einstein College of Medicine.

In deductive reasoning, if something is true of a class of things in general, it is also true for all members of that class. For example, ``All men are mortal. Harold is a man. Therefore, Harold is mortal.''  For deductive reasoning to be sound, the hypothesis must be correct. It is assumed that the premises, ``All men are mortal'' and ``Harold is a man'' are true. Therefore, the conclusion is logical and true.

According to the University of California, deductive inference conclusions are certain provided the premises are true. It’s possible to come to a logical conclusion even if the generalization is not true. If the generalization is wrong, the conclusion may be logical, but it may also be untrue. For example, the argument, ``All bald men are grandfathers. Harold is bald. Therefore, Harold is a grandfather,'' is valid logically but it is untrue because the original statement is false.

A common form of deductive reasoning is the syllogism, in which two statements – a major premise and a minor premise – reach a logical conclusion. For example, the premise ``Every A is B'' could be followed by another premise, ``This C is A.''  Those statements would lead to the conclusion ``This C is B.'' Syllogisms are considered a good way to test deductive reasoning to make sure the argument is valid.

Inductive Rreasoning. is the opposite of deductive reasoning. Inductive reasoning makes broad generalizations from specific observations. ``In inductive inference, we go from the specific to the general. We make many observations, discern a pattern, make a generalization, and infer an explanation or a theory.'' Wassertheil-Smoller told Live Science. ``In science there is a constant interplay between inductive inference (based on observations) and deductive inference (based on theory), until we get closer and closer to the `truth', which we can only approach but not ascertain with complete certainty.''

Even if all of the premises are true in a statement, inductive reasoning allows for the conclusion to be false. Here’s an example: ``Harold is a grandfather. Harold is bald. Therefore all grandfathers are bald.''  The conclusion does not follow logically from the statements.

Inductive reasoning has its place in the scientific method. Scientists use it to form hypothesis and theories. Deductive reasoning allows them to apply the theories to specific situations.

Abductive Reasoning. Another form of scientific reasoning that doesn’t fit in with inductive or deductive reasoning is abductive. Abductive reasoning usually starts with an incomplete set of observations and proceeds to the likeliest possible explanation for the group of observations, according to Butte College. It often entails making an educated guess after observing a phenomenon for which there is no clear explanation.

Abductive reasoning is useful for forming hypotheses to be tested. Abductive reasoning is often used by doctors who make a diagnosis based on test results and by jurors who made decisions based on the evidence presented to them.

Systems Science. Systems sciences are scientific disciplines partly based on systems thinking such as chaos theory, complex systems, control theory, cybernetics, sociotechnical systems theory, systems biology, systems ecology, systems psychology and the already mentioned systems dynamics, systems engineering, and systems theory.

Systems being involves embodying a new consciousness, and expanded sense of self, a recognition that we cannot survive alone, that a future that works for humanity needs also to work for other species and the planet. It involves empathy and love for the greater human family and for all our relationships - plants and animals, earth and sky, ancestors and descendants, and the many peoples and beings that inhabit our Earth. This is the wisdom of many indigenous cultures around the world, this is part of the heritage that we have forgotten and we are in the process of recovering.

Systems being and systems living brings it all together: linking head, heart and hands. The expression of systems being is an integration of our full human capacities. It involves rationality with reverence to the mystery of life, listening beyond words, sensing with our whole being, and expressing our authentic self in every moment of our life. The journey from systems thinking to systems being is a transformative learning process of expansion of consciousness - from awareness to embodiment.

Kathia Laslo, Ph.D., directs Saybrook University’s program in Leadership of Sustainable Ssytems.
NOTE: This post is an excerpt from the plenary presentation “Beyond Systems Thinking: The role of beauty and love in the transformation of our world” by Dr. Karla Lazslo at the 55th Meeting of the International Society for the Systems Sciences at the University of Hull, U.K., on July 21, 2014.

Systems thinking is the process of understanding how things influence one another within a whole. In nature, systems thinking examples include ecosystems in which various elements such as air, water, movement, plants, and animals work together to survive or perish. In organizations, systems consist of people, structures, and processes that work together to make an organization healthy or unhealthy.

Systems thinking has been defined as an approach to problem solving, by viewing “problems” as parts of an overall system, rather than reacting to specific part, outcomes or events and potentially contributing to further development of unintended consequences. Systems thinking is not one thing but a set of habits and practices within a framework that is based on the belief that the component parts of a system can best be understood in the context of relationships with each other and with other systems, rather than in isolation. Systems thinking focuses on cyclical rather than linear cause and effect.

In science systems, it is argued that the only way to fully understand why a problem or element occurs and persists is to understand the parts in relation to the whole. Standing in contrast to Descartes’s scientific reductionism and philosophical analysis, it proposes to view systems in a holistic manner. Consistent with systems philosophy, systems thinking concerns an understanding of a system by examining the linkages and interactions between the elements that compose the entirety of the system.

Science systems thinking attempts to illustrate that events are separated by distance and time and that small catalytic events can cause large changes in complex systems. Acknowledging that an improvement in one area of a system can adversely affect another area of the system, it promotes organizational communication at all levels in order to avoid the silo effect. Systems thinking techniques may be used to study any kind of system – natural, scientific, engineered, human, or conceptual.

\subsection{The Scientific Method}\index{The Scientific Method}

The prevailing scientific thinking in Western cultures today is naturalism. It is assumed that there is no God and that the entire universe can be explained on the basis of physical realities plus time and chance. It is assumed that the laws of physics have never changed and that conditions have been uniform in the past so that recent observations can be compiled and conclusions drawn about the past by simply looking back in time. The scientific method is mostly limited to the study of measurable entities in the physical world. Results should be verifiable by others.

Figure 2.1 Here (New)

\begin{enumerate}
\item Basic Assumptions include underlying philosophy, for example is there an outside intelligence operating, or is the system closed depending only on internal known laws. Are the laws of nature constant everywhere?  Were conditions in the past the same as they are now?  What initial conditions are assumed?  Science does not and cannot take place in a vacuum – an underlying world-view or philosophy is presupposed.
\item The hypothesis is a proposed explanation for an observed phenomenon. The simpler the explanation that fits the facts, the better, known as Occam’s razor. Science assumes we live in a universe. That is, the laws of physics are the same everywhere and, furthermore, they do not vary erratically – nature is predictable, and the universe is rational.
\item If the hypothesis does not explain the known facts or the new data, it is important to carefully examine the initial conditions to if one or more of them is incorrect or suspect. Wrong assumptions a long time ago that have not been challenged cause the weight of tradition to prevail – until there are overwhelming reasons for changing the prevailing scientific paradigm.
‘Science is the only self-correcting human institution, but it is also a process that progresses only by showing itself to be wrong.’ – Alan Sandage
\item Data inputs include measurements and observations. These are systematized and subjected to statistical scrutiny whenever possible.
\item When a theory has been found that seems to the known facts, the theory is then extended into the unknown to make predictions. These predictions are then tested by seeking additional data, exceptions, or confirmations.
\item A new scientific theory or model remains in vogue until new facts are found that contradict the model or whenever a better theory comes along.
\end{enumerate}

When conducting research, scientists use the scientific method to collect measurable, empirical evidence in an experiment related to a hypothesis (often in the form of an if/then statement), the results aiming to support or contradict a theory.
The steps of the scientific method must include: 
Make an observation or observations.
Ask questions about the observations and gather information.

\begin{enumerate}
\item Form a hypothesis – a tentative description of what’s been observed, and make predictions based on that hypothesis.
\item Test the hypothesis and predictions in an experiment that can be reproduced.
\item Analyze the data and draw conclusions; accept or reject the hypothesis or modify the hypothesis if necessary.
\end{enumerate}

Reproduce the experiment until there are no discrepancies between observations and theory. ``Replication of methods and results is an essential step in the scientific method.''

Some key underpinnings to the scientific method are:
\begin{itemize}
\item The hypothesis must be testable and falsifiable, according to North Carolina State University. Falsifiable means that there must be a possible negative answer to the hypothesis.
\item Research must involve deductive reasoning and inductive reasoning. Deductive reasoning is the process of using true premises to reach a logical true conclusion while inductive reasoning takes the opposite approach.
\item An experiment should include a dependent variable (which does not change) and an independent variable (which does change).
\item An experiment should include an experimental group and a control group. The control group is what the experimental group is compared against.
\end{itemize}

Gaining new insights into the nature of systems, Hillary Sillitto, 

Several INCOSE members participated in the IFSR Conversation held April 2018 in Linz, Austria. This article is a report on some interesting results coming out of this activity. INCOSE joined the International Federation for Systems Research in 2012, interfacing through the Systems Science Working Group, and INCOSE members have participated in each Conversation since the year we joined.

One of the major activities of the International Federation for Systems Research (IFSR) is the ``conversation'' held every two years in Linz, Austria, where several teams of typically 6-8 people spend a week discussing different current issues in systems research. The format is a ``conversation'' or ``systemic inquiry'' rather than a conference, and the teams spend most of the time in their own group, exploring their specific topic and attempting to achieve new insights by integrating the different perspectives and worldviews of the different team members. 

Over forty organizations are currently IFSR members. Some are more active than others, and the Conversation this year involved people from INCOSE, ISSS, ASC (American Society of Cybernetics), the IFSR itself, and one representative from the System Dynamics Society (SDS).

The 2018 Conversation addressed four topics: ``Systems Practice'', led by members and associates of Malik Management, focused on challenges set by senior-level input from the Government of Vietnam; ``What is Systems Science?'', led by Gary Smith of INCOSE with a team of INCOSE and IFSR members; ``Active and Healthy Aging'', using Beer’s Viable System Model and a subset of Len Troncale’s System Processes as reference models to understand the challenges facing older members of our communities; and ``Data Driven SE Approaches'', led by Ed Carroll of INCOSE and Sandia Labs, with several INCOSE members, the SDS representative, and several others from Sandia.

Ed Carroll’s team considered the problem of integrating the heterogeneous model types used by different engineering domains, discussed issues such as how to get people to trust models, identified the need for Systems Engineering to shift from a process-centric to an information-centric perspective if MBSE is to succeed, and were inspired by the ``agile manifesto'' to start working up an analogous ``MBSE manifesto''. Their outbrief advocated viewing the model as being the focus, rather than the process of creating it. Others pointed out the tension between this perspective and the verified success of ``shared model building'' as a method for engaging stakeholders, and developing their trust in the model. Someone suggested that ``no-one understands a model except the people who created it''. I look forward to seeing the MBSE manifesto and to the discussions it will undoubtedly provoke about the culture change required in the SE community to take full advantage of the model driven approach while being fully aware of its limitations: not all systems are deterministic, some systems ``have a mind of their own''; and for these, modeling can indicate the range of possible future trajectories but not the precise one that will be followed.

Gary Smith’s team discussed ``What is systems science''. Gary smith made a plausible argument that in historical terms, Systems Science is now where chemistry was before the Periodic Table of the elements - lots of phenomena have been described, many of them understood as individual phenomena, but this knowledge is not yet integrated around a single foundational structure. Further, the current systems science literature in most cases does not clearly distinguish between fundamental ingredients of all systems (think electrons, protons and neutrons), properties of all systems (think properties of atoms and elements due to the electron orbitals) and properties that can be synthesised with combinations of different ``elemental types'' of system – think compounds, crystals, alloys, etc. Most Systems Science literature also does not clearly distinguish between ``how people perceive and interact with systems'', and fundamental ``properties of systems in the natural world''. (Robert Rosen’s book Anticipatory Systems is a notable exception.) We spent the week exploring whether existing systems science knowledge could usefully be organised in this sort of structure, and concluded that it could, and that such a structure offers promise in terms of integrating the seven different worldviews on system we have identified within the INCOSE community. Also, we identified an eighth worldview about systems, that ``systemness'' might be a fundamental organising principle of nature. Our output and subsequent reflections are being posted to a website which will progressively be opened up to SSWG members and then more widely as the content matures.

I participated in the “What is systems science?” team, and also represented INCOSE at the IFSR Board Meeting on the Friday afternoon at the end of the Conversation. The notable points of the Board Meeting were: our old friend Gerhard Chroust stands down as IFSR’s Secretary General after 27 years of service; Gary Metcalf, Jennifer Wilby and Mary Edson also finished their terms of service; new faces join the Board, and Ray Ison takes over from Mary as president; George Mobus has taken over as general Editor of the IFSR book series; and the System Dynamics Society’s membership application was approved. Hillary Sillitto, ESEP, INCOSE Fellow
