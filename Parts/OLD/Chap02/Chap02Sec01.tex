A major part of human nature is the curiosity of people. Curiosity leads to inquiry, and inqooint to increased understanding. “The more you learn, the more acutely aware you become of your ignorance.”

How do we know about the world in which we live – or reality, for that matter?  Where does our knowledge about it come from?  The attempt to answer these questions lead to epistemology, the branch of philosophy dealing with the origin, scope, and validity of human knowledge.

In the epistemological debate, there are two archetypal and actually diametrically opposed concepts: empiricism and rationalism. Empiricism claims that sensory experience (observation) is man’s main (or even sole) source of knowledge, which rationalism claims that his knowledge stems from human reason.

Hardly anyone would deny that there is knowledge that comes to us from sensory experience. Take, for instance, the knowledge that water freezes at zero degrees Celsius. It actually takes observation(s) to acquire such knowledge.

However, in the field of science, which formulates knowledge that applies universally, irrespective of time and place, rationalism holds that empirical knowledge gained through sensory experience doesn’t have the same validity as knowledge deduced from reasoning.
    
\subsection{Human Nature to Inquire}\index{Human Nature to Inquire}

The branches of contemporary science associated with the study of human nature include anthropology, sociology, sociobiology, and psychology, particularly evolutionary psychology, which studies sexual selection in human evolution, as well as developmental psychology. The ``nature versus nurture'' debate is a broadly inclusive and well-known instance of discussion about human nature in the natural science.

This common phrase refers to the distinguishing characteristics – including ways of thinking, feeling, and acting – which humans tend to have naturally, independently of the influence of culture. The questions of what these characteristics are, how fixed they are, and what causes them are amongst the oldest and most important questions in philosophy and science. These questions have particularly important implications in ethics, politics, and theology. This is partly because human nature can be regarded as both a source of norms of conduct of ways of life, as well as presenting obstacles or constraints on living a good life. The complex implications of such questions are also dealt with in art and literature, the question of what it is to be human.

The concept of nature as a standard by which to make judgments was a basic presupposition in Greek philosophy. Specifically, ``almost all'' classical philosophers accepted that a good human life is a life in accordance with nature. (Notions and concepts of human nature from China, Japan, or India are not taken up in the present discussion.)

On this subject, the approach of Aristotle – sometimes considered to be a teleological approach – came to be dominant by late classical and medieval times. This approach understands human nature in terms of final and formal causes. In other words, nature itself (or a nature-creating divinity) has intentions and goals, similar somehow to human intentions and goals, and one of those goals is humanity living naturally. Such understandings of human nature see this nature as an ``idea'', or ``form'' of a human. By this account, human nature really causes humans to become what they become, and so it exists somehow independently of individual humans. This in turn has sometimes been understood as also showing a special connection between human nature and divinity.

However, the existence of this invariable human nature is a subject of much historical debate, continuing into modern times. Against this idea of a fixed human nature, the relative malleability of man has been argued especially strongly in recent centuries – firstly by early modernists such as Thomas Hobbes and Jean-Jacques Rousseau. In Rousseau’s Emile, or On Education, Rousseau wrote: “We do not know what our nature permits us to be.”  Since the early 19th century, thinkers such as Hegel, Marx, Kierkegaard, Nietzsche, Sartre, structuralists, and postmodernists have also sometimes argued against a fixed or innate human nature.

Charles Darwin’s theory of evolution has changed the nature of the discussion, confirming the fact that mankind’s ancestors were not like mankind today. Still more recent scientific perspectives – such as behaviorism, determinism, and the chemical model within modern psychiatry and psychological – claim to be neutral regarding human nature. (As in much of modern science, such disciplines seek to explain with little or no recourse to metaphysical causation.)  They can be offered to explain human nature’s origins and underlying mechanisms, or to demonstrate capacities for change and diversity which would arguably violate the concept of a fixed human nature.

\subsection{How Do We Know?}\index{How Do We Know?}

The notion of ``letting the facts speak for themselves'' without taking recourse to a theory is nonsensical. Mises was aware the people’s ``reasoning may be faulty and the theory incorrect; but thinking and theorizing are not lacking in any action.''

How do we know, and how can we make sure, that we employ a correct theory?  Fortunately, in social science a satisfactory answer can be given to these questions by taking recourse to a priori theory – meaning propositions that provide true knowledge about reality, and whose truth value can be validated independent of experience.

To explain, we have to turn briefly to the Prussian philosopher Immanuel Kant (1724-1804) and his groundbreaking The Critique of Pure Reason (1781). A central outcome of what Kant called transcendental investigation in his discovery of so-called a priori synthetic judgements.
    
\subsection{\textit{A Priori} Theory}\index{\textit{A Priori} Theory}

A priori denotes a proposition (a declarative statement) expressing knowledge that is acquired prior to, or independently from, experience. In contrast, a posteriori denotes knowledge that is acquired through and on the basis of experience.

A synthetic judgment refers to knowledge that is not contained in the subject matter. An example is ``All bodies are heavy.''  Here, the predicate ``heavy'' conveys knowledge that goes beyond the mere concept of ``body'' in general. A synthetic judgment thus yields new knowledge about the subject matter.

Analytical judgments repeat what the concept of the subject matter already presupposes. An example is ``All bodies are extended.''  In order to know that bodies are extended one does not need experience, as this information is already in the concept of ``bodies.''

One would expect that analytical judgements are a priori, while synthetic judgements are posteriori. However, Kant claims that there exist a priori synthetic judgements – knowledge that neither merely repeats the meaning of the concept under review nor requires experience to say something new about the subject matter.

How can a priori synthetic judgements be identified?  According to Kant, a proposition must meet two requirements in order to qualify as an a priori synthetic judgement. First, it must not result from experience, but from reasoning. Second, it cannot be denied without causing an intellectual contradiction.

A priori theory offers an approach for reviewing, criticizing, and possibly revising commonly held theoretical explanations of historical events. When (re)viewed from point of a priori theory, what can be said about the two independent observations?

A priori theory provides true knowledge about the outer world, and the truth of knowledge derived from a priori theory can be validated independent of sensory experience.

By no means less important, a priori knowledge trumps empirical knowledge: ``A proposition of an aprioristic theory can never be refuted by experience.''

Praxeology, the a priori science of human action, and, more specifically, it’s up to now bed-developed part, economics, provides in its field a consummate interpretation of past events recorded and a consummate anticipation of the effects to be expected from future actions of a definite kind.

An a priori theorist can thus decide in advance (that is, without engaging in social experimentation, or testing, for that matter) whether or not a given action – policy measure – can bring about the promised effects.

For instance, we know a priori that issuing flat money does not create economic prosperity, that tax- or debt-financed government spending does not improve society’s material well-being, and that these measures are actually economically harmful.

A priori theory is an intellectually powerful defense against promises made by false theory and its detrimental (even disastrous) economic consequences if put into practice. Students of social sciences should therefore be increasingly encouraged to engage in a priori theory.

In his seminal book Systems Thinking, Systems Practice, Peter Checkland defined systems thinking as thinking about the world through the concept of “system.”  This involves thinking in terms of processes rather than structures, relationships rather than components, interconnections rather than separation. The focus of the inquiry is on the organization and the dynamics generated by the complex interaction of systems embedded in other systems and composed by other systems.