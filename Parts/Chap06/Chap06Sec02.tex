\section{Engineering the System and Product}\index{Engineering the System and Product}

Engineering has always been concerned with the economical use of limited resources to achieve objectives. The purpose of the engineering activities of design and analysis is to determine how physical and conceptual factors may be altered to create the most utility for the least cost, in terms of product cost, product service cost, social cost, and environmental cost. Viewed in this context, engineering should be practiced in an expanded way, with engineering of the system placed ahead of concern for product components thereof.

Classical engineering focuses on physical factors such as the selection and design of physical components and their behaviors and interfaces. Achieving the best overall results requires focusing initially on conceptual factors, such as needs, requirements, and functions. The ultimate system, however, is manifested physically where the main objective is usually considered to be product performance, rather than the design and development of the overall system of which the product is a part. A product cannot come into being and be sustained without a production or construction capability, without support and maintenance, and so on. Engineering the system and product usually requires an interdisciplinary approach embracing both the product and associated capabilities for production or construction, product and production system maintenance, support and regeneration, logistics, connected system relationships, and phase-out and disposal.

A product may also be known as a service such as health care, learning modules, entertainment packets, financial services and controls, and orderly traffic flow. In these service examples, the engineered system is a health care system, an educational system, an entertainment system, a financial system, and a traffic control system.

\subsection{Systems are Known by Their Products}\index{Systems are Known by Their Products}

Systems and their associated products are designed, developed, deployed, renewed, and phased out in accordance with processes that are not as well understood as they might be. The cost-effectiveness of the resulting technical entities can be enhanced by placing emphasis on the following:

\begin{enumerate}
\item Improving methods for determining the scope of needs to be met by the system. All aspects of a systems engineering project are profoundly affected by the scope of needs, so this determination should be accomplished first. Initial consideration of a broad set of needs often yields a consolidated solution that addresses multiple needs in a more cost-effective manner than a separate solution for each need
\item Improving methods for defining product and system requirements as they relate to verified customer needs and external mandates. This should be done early in the design phase, along with a determination of performance, effectiveness, and specification of essential system characteristics
\item Addressing the total system with all its elements from a life-cycle perspective, and from the product to its elements of support and renewal. This means defining the system in functional terms before identifying hardware, software, people, facilities, information, or combinations thereof
\item Considering interactions in the overall system hierarchy. This includes relationships between pairs of system components, between higher and lower levels within the system hierarchy, and between sibling systems or subsystems
\item Organizing and integrating the necessary engineering and related disciplines into a top-down system-engineering effort in a concurrent and timely manner
\item Establishing a disciplined approach with appropriate review, evaluation, and feedback provisions to ensure orderly and efficient progress from the initial identification of needs through phase-out and disposal
\end{enumerate}

Any useful system must respond to identified <emphasis>functional needs</emphasis>. Accordingly, the elements of a system must not only include those items that relate directly to the accomplishment of a given operational scenario or mission profile but must also include those elements of logistics and maintenance support for use should failure of a prime element(s) occur. To ensure the successful completion of a mission, all necessary supporting elements must be available, in place, and ready to respond. System sustainability can help insure that the system continues meeting the functional needs in a competitive manner as the needs and the competition evolves. And, system sustainability contributes to overall sustainability of the environment.

\subsection{Product and System Categories}\index{Product and System Categories}

It is interesting and useful to note that systems are often known by their products. They are identified in terms of the products they propose, produce, deliver, or in other ways bring into being. Examples are manufacturing systems that produce products, construction systems that erect structures, transportation systems that move people or goods, traffic control systems that manage vehicle or aircraft flow, maintenance systems that repair or restore, and service systems that meet the need of a consumer or patient. What the system does is manifested through the product it provides. The product and its companion system are inexorably linked.

As frameworks for study, or baselines, two generic product/system categories are presented and characterized in this section. Consideration of these categories is intended to serve two purposes. First, it will help explain and clarify the topics and steps in the process of bringing engineered or technical systems into being. Then, in subsequent chapters, these categories and examples will provide opportunities for look-back reference to generic situations. Although there are other less generic examples in those chapters, greater understanding of them may be imparted by reference to the categories established in this section.

Single-Entity Product Systems. A single-entity product system, for example, may manifest itself as a bridge, a custom-designed home entertainment center, a custom software system, or a unique consulting session. The product may be a consumable (a nonstandard banquet or a counseling session) or a repairable (a highway or a supercomputer). Another useful classification is to distinguish consumer goods from producer goods, the latter being employed to produce the former. A product, as considered in this textbook, is not an engineered system no matter how complex it might be.

The product standing alone is not an engineered system. Consider a bridge constructed to meet the need for crossing an obstacle (a river, a water body, or another roadway). The engineered system is composed of the bridge structure plus a construction subsystem, a maintenance subsystem, an operating and support subsystem, and a phase-out and demolition process. Likewise, an item of equipment for producer or consumer use is not a system within the definition and description of an engineered system given in 

Manufacturing plants that produce repairable or consumable products, warehouses or shopping centers that distribute products, hospital facilities that provide health care services, and air traffic control systems that produce orderly traffic flow are also single-entity systems when the plant, shopping center, or hospital is the product being brought into being. These entities, in combination with appended and companion subsystems, may rightfully be considered technical systems.

The preceding recognizes the engineered system as more than just the consumable or repairable product, be it a single entity, a population of homogenous entities, or a flow of entities. The product must be treated as part of a system to be engineered, deployed, and operated. Although the product subsystem (including structure or service) directly meets the customer’s need, this need must be functionally decomposed and allocated to the subsystems and components comprising the overall system.

Multiple-Entity Population Systems. Multiple-entity populations, often homogenous in nature, are quite common. Thinking of these populations as being aggregated generic products permits them to be studied probabilistically. However, the engineered system is more than just a single entity in the population, or even the entities as a population. It is composed of the population together with the subsystems of production, maintenance and support, regeneration, and phase-out and disposal.

A set of needs provides justification for bringing the population into being. This set of needs drives the life-cycle phases of acquisition and utilization, made up of design, construction or production, maintenance and support, renovation, and eventually ending with phase-out and demolition/disposal. As with single-entity product systems, the product is the subsystem that directly meets the customer’s need.

Examples of repairable-entity populations include the following: The airlines and the military acquire, operate, and maintain aircraft with population characteristics. In ground transit, vehicles (such as taxicabs, rental automobiles, and commercial trucks) constitute repairable equipment populations. Production equipment types (such as machine tools, weaving looms, and autoclaves) are populations of equipment classified as producer goods.

Also consider repairable (renovatable) populations of structures, often homogenous in nature. In multi-family housing, a population of structures is composed of individual dwelling units constructed to meet the need for shelter at a certain location. In multi-tenant office buildings, the population of individual offices constitutes a population of renovatable entities. And in urban or suburban areas, public clinics constitute a distributed population of structures to provide health care.

The simplest multi-entity populations are called inventories. These inventories may be made up of consumables or repairable. Examples of consumables are small appliances, batteries, foodstuffs, toiletries, publications, and many other entities that are a part of everyday life. Repairable-entity inventories are often subsystems or components for prime equipment. For example, aircraft hydraulic pumps, automobile starters and alternators, and automation controllers are repairable entities that are components of higher-level systems.

Homogenous populations lend themselves to designs that are targeted to the end product or prime equipment, as well as to the population as a whole. Economies of scale, production and maintenance learning, mortality considerations, operational analyses based on probability and statistics, and so on, all apply to the repairable-entity population to a greater or lesser degree. But the system to be brought into being must be larger in scope than the population itself, if the end result is to be satisfactory to the producer and customer.

\subsection{Engineering for Product Competitiveness}\index{Engineering for Product Competitiveness}

Product competitiveness is desired by both commercial and public-sector producers worldwide. Thus, the systems engineering challenge is to bring products and systems into being that meet customer expectations cost-effectively.

Because of intensifying international competition, producers are seeking ways to gain a sustainable competitive advantage in the marketplace. Acquisitions, mergers, and advertising campaigns seem unable to create the intrinsic wealth and goodwill necessary for the long-term health of the organization. Economic competitiveness is essential. Engineering with an emphasis on economic competitiveness must become coequal with concerns for advertising, production distribution, finance, and the like.

Available human and physical resources are dwindling. The industrial base is expanding worldwide, and international competition is increasing rapidly. Many organizations are downsizing, seeking to improve their operations, and considering international partners. Competition has reduced the number of suppliers and subcontractors able to respond. This is occurring at a time when the number of qualified team members required for complex system development is increasing. Consequently, needed new systems are being deferred in favor of extending the life of existing systems.

All other factors being equal, people will meet their needs by purchasing products and services that offer the highest value–cost ratio, subjectively evaluated. This ratio can be increased by giving more attention to the resource-constrained world within which engineering is practiced. To ensure economic competitiveness of the product and enabling system, engineering must become more closely associated with economics and economic feasibility. This is best accomplished through a system life-cycle approach to engineering.

Limiting and Strategic Factors. </title><para>Those factors that stand in the way of attaining objectives are known as <emphasis>limiting factors</emphasis>. An important element of the systems engineering process is the identification of the limiting factors restricting accomplishment of a desired objective. Once the limiting factors have been identified, they are examined to locate strategic factors, those factors that can be altered to make progress possible.

The identification of strategic factors is important, for it allows the decision maker to concentrate effort on those areas in which success is obtainable. This may require inventive ability, or the ability to put known things together in new combination and is distinctly creative in character. The means achieving the desired objective may consist of a procedure, a technical process, or an organizational or managerial change. Strategic factors limiting success may be circumvented by operating on engineering, human, and economic factors individually and jointly.

An important element of the process of defining alternatives is the identification of the limiting factors restricting the accomplishment of a desired objective. Once the limiting factors have been identified, they are examined to locate those strategic factors that can be altered in a cost-efficient way so that a selection from among the alternatives may be made.

Seeking Desirable Emergent Properties. Those factors that stand in the way of attaining objectives are known as limiting factors. An important element of the systems engineering process is the identification of the limiting factors restricting accomplishment of a desired objective. Once the limiting factors have been identified, they are examined to locate <emphasis>strategic factors</emphasis>, those factors that can be altered to make progress possible.

The identification of strategic factors is important, for it allows the decision maker to concentrate effort on those areas in which success is obtainable. This may require inventive ability, or the ability to put known things together in new combinations and is distinctly creative in character. The means achieving the desired objective may consist of a procedure, a technical process, or an organizational or managerial change. Strategic factors limiting success may be circumvented by operating on engineering, human, and economic factors individually and jointly.

An important element of the process of defining alternatives is the identification of the limiting factors restricting the accomplishment of a desired objective. Once the limiting factors have been identified, they are examined to locate those strategic factors that can be altered in a cost-efficient way so that a selection from among the alternatives may be made.

People often acquire diverse products to meet specific needs without companion contributing systems to ensure the best overall results, and without adequately considering the effects of the products on the natural world, on humans, and on other human-made systems. Proper application of systems engineering ensures timely and balanced evaluation of all issues to harmonize overall results from human investments, minimizing problems and maximizing satisfaction.

Maximizing Satisfaction Through Emergent Properties. Maximizing satisfaction requires a focus on emergent properties with linkage to requirements derived from the customer.

An emergent property cannot be a property determined from analysis of constituent part interactions when assembled. No single part provides flight. It only occurs when the parts are correctly assembled, and engines engaged. That situation conforms to the simple description given and all would agree that “dominance in transatlantic business class performance” is also an emergent property but the system assembled to attain that distinction is a whole for which the Dreamliner 787 is but one part.

In the general case, however, it may prove rather difficult to undertake ``total systems design'' of some complex system with particular emergent properties in mind. It has been an idealist target, however, since - if it can be achieved - there is the prospect of "something for nothing," or the whole being greater than the sum of its parts per Derek Hitchins.

The concept of emergence is very useful but may not help as much as expected in some cases. Take a passenger jet airplane as an example. The airplane itself has the emergent property of flight, whereas an airplane engine, on its own, cannot fly. If an airplane engine is considered to be the total system, then the emergent property is thrust. The jet fan does not produce thrust, neither does the jet compressor. However, all together, the jet engine produces thrust.

Thus, I am not sure all would agree about your notion of emergence. A simple description of emergence is: those properties of the whole which may not be ascribedexclusivelyto any of the parts. The property of flight is clearly not emergent: it comes as a linear balance between the forces of drag and thrust in the forward axis and lift and weight on the vertical axis. Newton's First Law applies.

And there is no ``whole is greater than the sum of its parts,'' either, as Aristotle required. So, flight is not emergent. If one were to seek emergent properties from an airliner, then we would be looking at the airliner in operation in some future, competitive environment, and a hoped for emergent property might be "dominance in transatlantic business class performance."

Can one undertake systems design with that emergent property in mind? Yes, one can. Ask the Dreamliner 787 folks.
