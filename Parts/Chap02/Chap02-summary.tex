\section{Summary and Extensions}\index{Summary and Extensions}

Science and systems thinking is increasingly being used to tackle a wide variety of subjects in fields such as computing, engineering, epidemiology, information science, health, manufacture, management, and the environment.

\begin{itemize}
\item Systemology and synthesis. The science of systems or their formation is called systemology. Problems and problem complexities faced by humankind do not organize themselves along disciplinary lines. New arrangements of scientific and professional efforts based on the common attributes and characteristics of needs and problems should contribute to the progress. More attention should be paid to human action and praxeology applications at the macro-level to help understand both the economic and non-economic dimensions of the world in which we live.
\item The formation of interdisciplines began in the middle of the last century and that has brought about an evolutionary synthesis of knowledge. This has occurred not only within science, but between science and technology and between science and humanities. The forward progress of systemology in the study of large-scale complex systems requires a synthesis of science and the humanities in addition to a synthesis of science and technology.
\item When synthesizing human-made systems, unintended effects can be minimized and the natural system can sometimes be improved by engineering the larger human-modified system instead of engineering only the human-made. If system evaluation is applied beyond the human-made, then the boundary of the target system (meant to include both natural and human-made systems) should be adopted as the boundary of the human-modified domain.
\item Systems are as pervasive as the universe in which they exist. They are as grand as the universe itself or as infinitesimal as the atom. Systems appeared first in natural forms, but with the advent of human beings, a variety of human-made systems have come into existence. In recent decades, we have begun to understand structure and characteristics of natural and human-made systems in a scientific way.
\end{itemize}

Upon completion of , the reader will have obtained essential insight into systems and systems thinking, with an orientation toward systems engineering and analysis. The system definitions, classifications, and concepts presented in this chapter are intended to impart a general understanding about the following:

\begin{enumerate}
\item System classifications, similarities, and dissimilarities
\item The fundamental distinction between natural and human-made systems
\item The elements of a system and the position of the system in the hierarchy of systems
\item The domain of systems science, with consideration of cybernetics, general systems theory, and systemology
\item Technology as the progenitor for the creation of technical systems, recognizing its impact on the natural world
\item The transition from the machine or industrial age to the Systems Age, with recognition of its impact upon people and society
\item System complexity and scope and the demands these factors make on engineering in the Systems Age
\item The range of contemporary definitions of systems engineering used within the profession.
\end{enumerate}

%%----------------------------------------------------------------------------------------
%	PROBLEMS
%%----------------------------------------------------------------------------------------

%% SEA CHAPTER 2 - BRINGING SYSTEMS INTO BEING
% SEA Question Location in \label{sea-Chapter#-Problem#}
% ANSWERS NEED UPDATING
\begin{exercises}
    \begin{exercise}
    \label{sea-2-1}
        Identify at least two characteristics that distinguish natural systems from those that are human-made or human-modified.
        % What are some of the characteristics of a human-made or engineered system that distinguish it from a natural system?
    \end{exercise}
    \begin{solution}
        A human–made or engineered system comes into being by purpose-driven human action. It is distinguished from the natural world by characteristics imparted by its human originator, innovator, or designer. The human–made system is made up of elements (materials) extracted from the natural world and it is then embedded therein. Human–made systems may or may not meet human needs in a satisfactory manner. \textbf{Reference:}
    \end{solution}
    
    \begin{exercise}
    \label{sea-2-2}
        Identify at least two possible interfaces between the natural and human worlds for the creation of any arbitrary system.
        % Describe some of the interfaces between the natural world and the human-made world as they pertain to the process of bringing systems/products into being.
    \end{exercise}
    \begin{solution}
        Interfaces between the human–made and the natural world arise from human–made products, systems, and structures for the use of people. An example interface is a system of pipes, pumps, and tanks bringing water from a natural system, such as a lake, or a human– made system like a reservoir, to a city. The human–made water distribution system creates an interface when it is brought into being. Interface-creating entities such as this draw upon natural resources and impact the environment during use and at the end of their useful life. \textbf{Reference:}
    \end{solution}
    
    \begin{exercise}
    \label{sea-2-3}
        Describe what distinguishes human-made from human-modified systems.
        % Identify and describe a natural system of your choice that has been human-modified and identify what distinguishs it from a human-made system.
    \end{exercise}
    \begin{solution}
        A watershed in its natural state is a natural system that receives rainfall, absorbs some rainwater, and accumulates and discharges runoff. This system becomes human-modified if a dam is constructed at a point on the watercourse. The watershed is now a humanmodified system that differs from the original system. Some differences are the new capacity for water storage, a change in the rate of runoff, and some change in the pattern of water absorption into the soil. A change will also occur in the distribution and density of vegetation in the watershed. \textbf{Reference:}
    \end{solution}
    
    \begin{exercise}
    \label{sea-2-5_6}
        Based on the textbook descriptions in ???, identify a system for the following system types and justify your answer based on the system product(s).
        \begin{enumerate}[label=\alph*)]
            \item Single-entity product system.
            \item Multiple-entity product system.
        \end{enumerate}
        % Put a face on the generic single-entity product system of Section 2.1.2 by picking a real structure or service with which you are familiar. Then, rewrite the textbook description based on the characteristics of the entity you picked.
        % Put a face on the generic multiple-entity population system of Section 2.1.2 by picking real equipment with which you are familiar. Then, rewrite the textbook description based on the characteristics of the equipment you picked.
    \end{exercise}
    \begin{solution}
        \begin{enumerate}[label=\alph*)]
            \item todo
            \item todo
        \end{enumerate}
    \end{solution}
    
    \begin{exercise}
    \label{sea-2-7_8}
        Choose any consumer item and identify the producer. Then
        \begin{enumerate}[label=\alph*)]
            \item Identify at least three things that are employed or consumed by the producer to create the consumer item.
            \item Identify any enabling system which is employed by the producer to create the consumer item.
            \item Must an enabling system and a product of that system be engineered jointly? Why or why not?
        \end{enumerate}
        % Pick a consumer good and name the producer good(s) that need to be employed to bring this consumer good to market.
        % Pick a product, describe the enabling system that is required to bring it into being, and explain the importance of engineering the system and product together.
    \end{exercise}
    \begin{solution}
        \begin{enumerate}[label=\alph*)]
            \item todo
            \item todo
            \item todo
        \end{enumerate}
    \end{solution}
    
    \begin{exercise} 
    \label{sea-2-9}
        Identify at least four factors which determine a product's competitiveness. Is product competitiveness important? Why or why not?
        % What are some of the essential factors in engineering for product competitiveness? Why is product competitiveness important?
    \end{exercise}
    \begin{solution}
        The quality, price, ergonomics, and lifetime of a product would all be considered to influence the purchasing behavior of consumers, which is based on product competition. ORIGINAL: The overarching factor in engineering for product competitiveness is the requirement to meet customer expectations cost–effectively. Competitiveness is the assurance of corporate health and advancement in the global marketplace. This desideratum cannot be achieved by advertising, acquisitions, mergers, and outsourcing alone. Product competitiveness requires focus on design characteristics. Product (and system) design is now being recognized by forward-looking enterprises as an underutilized strategic weapon. \textbf{Reference:}
    \end{solution}
    
    \begin{exercise}
    \label{sea-2-10}
        Explain the benefits of system life-cycle thinking and why it is important.
        % What does system life-cycle thinking add to engineering as currently practiced? What are the expected benefits to be gained from this thinking?
    \end{exercise}
    \begin{solution}
        System life–cycle thinking necessitates engineering for the life cycle. This is in contrast to engineering as historically practiced, in which downstream considerations were often deferred or neglected. Life–cycle thinking can help preclude future problems if emphasis is placed on: (a) Improving methods for defining system and product requirements; (b) Addressing the total system with all its elements from a life–cycle perspective; (c) Considering the overall system hierarchy and the interactions between various levels in that hierarchy.\\
        Some of the problems that life–cycle thinking can help alleviate are: (a) The dwindling of available resources by looking ahead and considering timely substitution; (b) The erosion of the industrial base through international competition by emphasizing design–based strategies; (c) The loss of market share by providing the right product at the right price to avoid the need to downsize or merge to synchronize operations; (d) The demand for more complex products, which increases the cost of operations for the producer. \textbf{Reference:}
    \end{solution}
    
    \begin{exercise}
    \label{sea-2-11}
        For the product life-cycles displayed in Figures ???, ???, identify possible sources of feedback or communication between the different phases.
        % Various phases of the product life cycle are shown in Figure 2.1 and expanded in Figure 2.2. Describe some of the interfaces and interactions between the life cycles of the system and the product life cycle.
    \end{exercise}
    \begin{solution}
        The first life cycle involves technological activity beginning with need identification and revolves around product design and development. Consideration is then given to the production or construction of the product or structure. This is depicted in the second life cycle which involves bringing a manufacturing or construction capability into being. The third life cycle concerns the maintenance and logistic support needed to service the product during use and to support the manufacturing capability. Finally, the fourth life cycle addresses the phase–out and disposal of system and product elements and materials. \textbf{Reference:}\\
        The major functions of the system engineering process during conceptual design are the establishment of performance parameters, operational requirements, support policies, and the development of the system specification. As one proceeds through design and development, the functions are primarily system dependent, and may include functional analyses and allocations to identify the major operational and maintenance support functions that the system is to perform. Criteria for system design are established by evaluating different (alternative) design approaches through the accomplishment of system/cost effectiveness analyses and trade–off studies, the conduct of formal design reviews, and preparing system development, process, and material specifications. The production and/or construction phase may entail technical endeavors such as the design of facilities for product fabrication, assembly, and test functions; design of manufacturing processes; selection of materials; and the determination of inventory needs. The major functions during system use and life–cycle support can involve providing engineering assistance in the initial deployment, installation, and checkout of the system in preparation for operational use; providing field service or customer service; and providing support for phase–out and disposal of the system and its product for the subsequent reclamation and recycling of reclaimable components.
    \end{solution}
    
    \begin{exercise}
    \label{sea-2-12}
        Pretend that you are developing a product. How do you convince your peer(s) that the best approach is to \say{design for the life cycle}?
        % What is the full meaning of the phrase “designing for the life cycle”?
    \end{exercise}
    \begin{solution}
        Designing for the life cycle means thinking about the end before the beginning. It questions every design decision on the basis of anticipated downstream impacts. Design for the life cycle is enabled by application of systems engineering defined as an interdisciplinary approach to derive, evolve, and verify a life–cycle balanced system solution which satisfies customer expectations and meets stakeholder expectations. It promotes a top– down, integrated life–cycle approach to bringing a system into being, embracing all of the phases exhibited in ???. \textbf{Reference:}
    \end{solution}
    
    \begin{exercise}
    \label{sea-2-14} 
        Identify one benefit and one consequence of the following design models
        \begin{enumerate}[label=\alph*)]
            \item Waterfall model.
            \item Spiral model.
            \item V-model.
        \end{enumerate}
        % As best you can, identify life-cycle activities that occur in the waterfall model, the spiral model, and the “vee” model. Of these models, pick the one you prefer and explain why.
    \end{exercise}
    \begin{solution}
        \begin{enumerate}[label=\alph*)]
            \item Independent exercise.
            \item Independent exercise.
            \item Independent exercise.
        \end{enumerate}
    \end{solution}
    
    \begin{exercise}
    \label{sea-2-14_part2}
        Of the models described in \Cref{sea-2-14}, do you have a personal preference? Is any one model always the most appropriate? Why or why not?
        % As best you can, identify life-cycle activities that occur in the waterfall model, the spiral model, and the “vee” model. Of these models, pick the one you prefer and explain why.
    \end{exercise}
    \begin{solution}
        \begin{enumerate}[label=\alph*)]
            \item Independent exercise.
        \end{enumerate}
    \end{solution}
    
    \begin{exercise}
    \label{sea-2-15_16}
        Select a design situation of your choice.
        \begin{enumerate}[label=\alph*)]
            \item What are the requirements of the system?
            \item Describe the steps for identifying appropriate technical performance measures.
            \item Describe the relationship between the system requirements and its technical performance measures.
            \item What happens when the technical performance measures disagree with the system requirements?
        \end{enumerate}
        % Design considerations are the first step on the way to deriving technical performance measures. Outline all of the steps, emphasizing the design-dependent parameter concept.
        % How are requirements related to technical performance measures? What is the remedy when requirements and TPMs are not in agreement?
    \end{exercise}
    \begin{solution}
        \begin{enumerate}[label=\alph*)]
            \item After the need has been identified, it should be translated into system operational requirements. In determining system requirements, the engineering design team needs to know what the system is to accomplish, when the system will be needed, how the system is to be utilized, what effectiveness requirements the system should meet, how the system is to be supported during use, and what the requirements are for phase–out and disposal.
            \item TPMs identify the degree to which the proposed design is likely to meet customer expectations. Many parameters may be of importance in a specific design application and most of these are design–dependent. These are appropriately called design–dependent parameters (DDPs). 
            \item Requirements are the driving force for identifying those design considerations that must be measured and expressed as TPMs. TPMs are specific estimated and/or predicted values for DDPs and they may or may not match required values. 
            \item When requirements and TPMs are not in agreement, the system design endeavor must be continued by altering those factors and/or design characteristics upon which design values inherently depend; i.e., DDPs. Alternatively, the customer may be made aware of the discrepancy and be given the opportunity to modify initially stated requirements.
        \end{enumerate}
    \end{solution}
    
    \begin{exercise}
    \label{sea-2-18_20}
        Select a design situation of your choice and refer to figure ??? for the following questions.
        \begin{enumerate}[label=\alph*)]
            \item Identify a top-level requirement and decompose it appropriately.
            \item Identify a lowest-level requirement and explain its relevance for all higher levels.
        \end{enumerate}
        % Pick a top-level requirement and decompose it in accordance with the structure shown in Figure 2.7.
        % Pick up on a design consideration at the lowest level in Figure 2.8. Discuss its position and impact on each of the next higher levels.
    \end{exercise}
    \begin{solution}
        \begin{enumerate}[label=\alph*)]
            \item Independent exercise.
            \item Independent exercise.
        \end{enumerate}
    \end{solution}
    
    \begin{exercise}
    \label{sea-2-22}
        Identify at least three domain manifestations of systems engineering.
        % Identify some of the engineering domain manifestations of systems engineering.
    \end{exercise}
    \begin{solution}
        Formal engineering domain manifestations of systems engineering that are offered as academic degrees are biological systems engineering, computer systems engineering, industrial systems engineering, manufacturing systems engineering, and others. Informal domains exist with employment opportunities in aerospace systems engineering, armament systems engineering, network systems engineering, information systems engineering, health systems engineering, service systems engineering, and many others. \\
        Systems engineering utilizes appropriately applied technological inputs from various engineering disciplines together with management principles in a synergistic manner to create new systems. Traditional engineering domains tend to focus on the bottom–up approach in designing new systems, whereas systems engineering uses the top–down approach. Unlike the traditional disciplines, it adopts a life–cycle approach in the design of new systems. \textbf{Reference:}
    \end{solution}
    
    \begin{exercise}
    \label{sea-2-23}
        Identify at least three obstructions which hinder or prevent the application of systems engineering.
        % What are some of the impediments to the implementation of systems engineering?
    \end{exercise}
    \begin{solution}
        Some organizational impediments to the implementation of systems engineering include: (a) the dominance of disciplines over interdisciplines, (b) a tendency to organize SE in the same manner as the traditional engineering disciplines, (c) an excessive focus on analysis at the expense of synthesis and process, (d) difficulty in integrating the appropriate discipline contributions with the relevant system elements, (e) the lack of sufficient communication, especially where system contributors are geographically dispersed, (f) deficiencies in balancing technologies and tools with planning and management of the activities required to accomplish objectives, (g) an ineffective general organizational environment to enable the systems engineering function to truly impact design and system development. Other impediments related to the above include (a) the lack of a good understanding of customer needs and definition of the system requirements, (b) ignorance of the fact that the majority of the projected life-cycle cost for a given system is committed because of engineering design and management decisions made during the early stages of conceptual and preliminary design, (c) the lack of a disciplined top–down “systems approach” in meeting desired objectives, (d) system requirements defined from a short term perspective and, (e) lack of good planning early, and the lack of subsequent definition and allocation of requirements in a complete and disciplined manner. \textbf{Reference:}
    \end{solution}
    
    \begin{exercise}
    \label{sea-2-24}
        Why is systems thinking and engineering beneficial?
        % What are some of the benefits that may result from the utilization of systems thinking and engineering?
    \end{exercise}
    \begin{solution}
        Some of benefits that accrue from the application of the concepts and principles of systems engineering are: (a) Tailoring involving the modification of engineering activities applied in each phase of the product or system life cycle to adapt them to the particular product or system being brought into being. Its importance lies in that the proper amount of engineering effort must be applied to each phase of the system being developed, and it must be tailored accordingly; (b) Reduction in the life–cycle cost of the system. Often it is perceived that the implementation of systems engineering will increase the cost of the system acquisition. This is misconception since there might be more steps to perform during the early (conceptual and preliminary) system design phases, but this could reduce the requirements in the integration, test and evaluation efforts later in the detail design and development phase; (c) More visibility and a reduction of risks associated with the design decision making process, with a consequent increase in the potential for greater customer satisfaction; (d) promotion of a top–down integrated life–cycle approach for bringing a system into being. \\
        The benefit of systems engineering is needed when the engineering specialists in one of more of the conventional engineering areas may not be sufficiently experienced or capable to ensure that all elements of the system are orchestrated in a proper and timely manner. \textbf{Reference:}
    \end{solution}
\end{exercises}
% SKIPPED
% sea-2-4 Describe the product or prime equipment as a component of the system; provide an explanation of the functions provided by each entity.
% sea-2-13 Select a system of your choice and describe the applicable life-cycle phases and activities, tailoring your description to that system.
% sea-2-17 Pick a design situation of your choice and itemize the multiple criteria that should be addressed.
% sea-2-19 Candidate systems result from design synthesis and become the object of design analysis and evaluation. Explain.
% sea-2-21 Take synthesis, analysis, and evaluation as depicted in Figure 2.9 and then classify each activity exhibited by application of elements in the ten-block morphology of Figure 2.10.
% sea-2-25 Go to the INCOSE web site and find the page about the Systems Engineering Journal. Pick an article that touches upon a topic in this chapter and relate it thereto using no more than one paragraph.
% sea-2-26 Go to the INCOSE web site and identify one individual from the Fellows group who most closely matches your own interest in systems engineering. Say why you would like to meet this person.