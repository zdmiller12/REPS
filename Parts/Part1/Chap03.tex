\chapterimage{leaf-bottom.jpg} % Chapter heading image
\chapter{ORGANIZING HUMAN ACTIVITY SYSTEMS}\label{chap:3}

Organization is human kind’s most important innovation1. It is a human made system that necessitates and requires cooperative action. The human-made world came into existence by and through cooperative collaboration; through purposeful action involving enterprises.

The fundamental unit of humanity is the individual. Each person spends a portion of their time coordinating their effort with that of others. They also spend some of their time in activity which is essentially independent of other individuals. While no activity is completely independent of the activity of all others over all time, in a more immediate sense activity can be considered independent if it does not require some direct coordination with one or more other people. An individual can read to himself or play golf alone. He may also spend the day driving a tractor on his farm.

Organizations consist of the coordinated efforts of individuals and have existed since the beginning of recorded history. Through cooperative action, people have been able to overcome their individual limitations. The high standard of living enjoyed by modern society may be attributed largely to the ability of organizations to change the physical environment more effectively than is possible through individual action alone. In this respect, organization is humankind’s most important innovation.

\section{Humans and Human Nature}\index{Humans and Human Nature}

\subsection{The Nature of the Individual}\index{The Nature of the Individual}

The human being can be treated as a distinct individual. The study of individual differences may be pursued across many dimension1. The most obvious distinctions are physical. Some individuals may be said to be attractive while others may be considered plain or even unattractive. This latter evaluation, however, is a subjective one and influenced by the cultural and social mores of the society.

Individuals are different one from another across dimensions that are more subtle than mere appearances or capabilities. They may differ in intelligence, in aptitudes, in varying attributes of personality and in interests and attitudes. Some students are said to be smarter than others. They receive better grades and they may be more successful after school and accomplish more in life. This success may be partly due to intelligence. The trait of intelligence is apparently more difficult to define than it is to measure. Intelligence has been variously described as adaptability to new circumstances or the ability to deal in abstractions and with complexity. Regardless of the definition, measures of intelligence have been developed that do have predictive validity, and people are known to differ in this important trait.

People also differ in aptitudes. Individuals may possess greater or lesser degrees of mechanical aptitude, which in turn might include such specifics as motor skills and manual dexterity. Other aptitudes may include spatial and perceptual abilities as well as clerical capabilities. People differ in aptitudes and these differences have become recognized in recent years as important in individual job placement. Individuals are also different regarding personality. Some people are emotionally stable while others are not. Some people are nervous while others are calm. Some are dominant while others are quiet and more passive. Personality is a very profound attribute and people differ along this dimension as they do along other dimensions.

Interests and attitudes are other traits in which people differ. Some people are interested in social activity while others are not. Some are oriented more toward economics and practical matters while other people might prefer to deal in theoretical concepts or abstractions. Some people are more interested than others in mystical experiences. In attitudes also, people are different. In classifying attitudes, one might speak of radical as opposed to conservative beliefs, as degrees of support in the established social order. Regarding almost any contemporary controversial issue, we have our opinions, and other individuals with divergent opinions have their prejudices. These are attitudes.

Part of the differences among people may be attributed to the influence of environment and the other part may be due to heredity. The relative effect of each is not as important herein as the combined resultant of their effect. The uniqueness of individuals will be particularly evident and important in the relationship of these individuals to cooperative systems. An objective of a cooperative system may be important to one individual and trivial to another. An activity may appear correct to one person and improper to another. An incentive may be valuable to one person and insignificant to a second. All people are assumed to be different about their subjective evaluation of the value or utility associated with any object, stat, or event. It will subsequently be demonstrated that there are influences acting within cooperative systems to stabilize the values of individuals. A church may attempt to inculcate a code of morality upon its membership. An industrial organization might be concerned and act to improve its image as an employer in a community. In addition to these more overt attempts to structure values, it will be demonstrated that there are also subtle mechanisms acting to stabilize values within cooperative systems. These will be discussed in subsequent chapters. For the present, the uniqueness of individuals is assumed. The attempts to manipulate and stabilize values will be explored, but it will be assumed that the composite value structure of any individual is at least slightly different from those of all other individuals.

Differences within an individual also develop over time. In a physiological sense, body cells die and are replaced. So also may an individual’s personality, interests, and attitudes change. The value system of the individual is usually altered with time. A choice that appears attractive one day may not be so attractive the next. The influence of another individual may have modified this judgement.

A more detailed treatment of individual differences may be found in Leona E. Tyler, The Psychology of Human Differences, New York, Appleton-Century-Crofts, Inc., 1956. 

\subsection{Influence of Classical Greek Philosophy}\index{Influence of Classical Greek Philosophy}

The concept of nature as a standard by which to make judgments was a basic presupposition in Greek philosophy. Specifically, ``almost all'' classical philosophers accepted that a good human life is a life in accordance with nature.[1 ](Notions and concepts of human nature from China, Japan, or India are not taken up in the present discussion.).

On this subject, the approach of Aristotle - sometimes considered to be a teleological approach - came to be dominant by late classical and medieval times. This approach understands human nature in terms of final and formal causes. In other words, nature itself (or a nature-creating divinity) has intentions and goals, similar somehow to human intentions and goals, and one of those goals is humanity living naturally. Such understandings of human nature see this nature as an ``idea'', or ``form'' of a human.[2] By this account, human nature really causes humans to become what they become, and so it exists somehow independently of individual humans. This in turn has sometimes been understood as also showing a special connection between human nature and divinity.

However, the existence of this invariable human nature is a subject of much historical debate, continuing into modern times. Against this idea of a fixed human nature, the relative malleability of man has been argued especially strongly in recent centuries—firstly by early modernists such as Thomas Hobbes and Jean-Jacques Rousseau. In Rousseau's Emile, or On Education, Rousseau wrote: ``We do not know what our nature permits us to be.''[3] Since the early 19th century, thinkers such as Hegel, Marx, Kierkegaard, Nietzsche, Sartre, structuralists, and postmodernists have also sometimes argued against a fixed or innate human nature.

Charles Darwin's theory of evolution has changed the nature of the discussion, confirming the fact that mankind's ancestors were not like mankind today. Still more recent scientific perspectives - such as behaviorism, determinism, and the chemical model within modern psychiatry and psychology - claim to be neutral regarding human nature. (As in much of modern science, such disciplines seek to explain with little or no recourse to metaphysical causation.)[4] They can be offered to explain human nature's origins and underlying mechanisms, or to demonstrate capacities for change and diversity which would arguably violate the concept of a fixed human nature.

Article: Ancient Greek philosophy Classical Greek philosophy

Philosophy in classical Greece is the ultimate origin of the western conception of the nature of a thing. According to Aristotle, the philosophical study of human nature itself originated with Socrates, who turned philosophy from study of the heavens to study of the human things.[5] 

Socrates is said to have studied the question of how a person should best live, but he left no written works. It is clear from the works of his students Plato and Xenophon, and also by what was said about him by Aristotle (Plato's student), that Socrates was a rationalist and believed that the best life and the life most suited to human nature involved reasoning. The Socratic school was the dominant surviving influence in philosophical discussion in the Middle Ages, amongst Islamic, Christian, and Jewish philosophers.

The human soul in the works of Plato and Aristotle has a divided nature, divided in a specifically human way. One part is specifically human and rational, and divided into a part which is rational on its own, and a spirited part which can understand reason. Other parts of the soul are home to desires or passions like those found in animals. In both Aristotle and Plato, spiritedness (thumos) is distinguished from the other passions (epithumiai).[6] The proper function of the "rational" was to rule the other parts of the soul, helped by spiritedness. By this account, using one's reason is the best way to live, and philosophers are the highest types of humans.

Aristotle—Plato's most famous student—made some of the most famous and influential statements about human nature. In his works, apart from using a similar scheme of a divided human soul, some clear statements about human nature are made:

\begin{enumerate}
\item Man is a conjugal animal, meaning an animal which is born to couple when an adult, thus building a household (oikos) and, in more successful cases, a clan or small village still run upon patriarchal lines.[7]
\item Man is a political animal, meaning an animal with an innate propensity to develop more complex communities the size of a city or town, with a division of labor and law-making. This type of community is different in kind from a large family and requires the special use of human reason.[8]
\item Man is a mimetic animal. Man loves to use his imagination (and not only to make laws and run town councils). He says, ``we enjoy looking at accurate likenesses of things which are themselves painful to see, obscene beasts, for instance, and corpses.'' And the ``reason why we enjoy seeing likenesses is that, as we look, we learn and infer what each is, for instance, `that is so and so.'''[9]
\end{enumerate}

For Aristotle, reason is not only what is most special about humanity compared to other animals, but it is also what we were meant to achieve at our best. Much of Aristotle's description of human nature is still influential today. However, the particular teleological idea that humans are ``meant'' or intended to be something has become much less popular in modern times.[10]

For the Socratics, human nature, and all natures, are metaphysical concepts. Aristotle developed the standard presentation of this approach with his theory of four causes. Every living thing exhibits four aspects or ``causes:'' matter, form, effect, and end. For example, an oak tree is made of plant cells (matter), grew from an acorn (effect), exhibits the nature of oak trees (form), and grows into a fully mature oak tree (end). Human nature is an example of a formal cause, according to Aristotle. Likewise, to become a fully actualized human being (including fully actualizing the mind) is our end. Aristotle (Nicomachean Ethics, Book X) suggests that the human intellect () is ``smallest in bulk'' but the most significant part of the human psyche and should be cultivated above all else. The cultivation of learning and intellectual growth of the philosopher, which is thereby also the happiest and least painful life.

\subsection{Effectiveness and Efficiency in Individual Behavior}\index{Effectiveness and Efficiency in Individual Behavior}

Individuals engage in activity. Sometimes the activity appears to be more successful than other times. The individual may accomplish what he set out to accomplish while another time he may not be so successful. Sometimes he accomplishes what he set out to accomplish but he still regrets making the decision to undertake that activity. The cost of accomplishment was too high. Sometimes he is unsuccessful, and this does not bother him. The investment in the activity was not very significant. A measure of the success or lack of success in human behavior is needed to describe these situations. Such a measure will be made in terms of the effectiveness and the efficiency of individual behavior3. This measure will then subsequently e applied to describe the relative success of cooperative activity.

Efficiency has a very precise meaning in the physical sciences and in engineering. The efficiency of a machine or of a process is measured as a percentage and is the ratio of the output of the machine or process divided by the input. This efficiency is always less than one hundred percent because the output is always less than the input. The process or machine consumes some energy in the transformation. Thus, while one hundred percent efficiency would be ideal, it is never attained. The efficiency of cooperative activity will also be measured as the ratio of the output over the input. However, this will be a collective ratio of all the participants, and for the cooperative system to survive, it will be shown that this efficiency must exceed rather than be less than one hundred percent, although the same degree of quantification in measurement is rarely possible.

The efficiency of individual activity is related to this traditional definition of efficiency, but it is personal and subjective. The efficiency of individual activity is dependent upon the cost incurred in undertaking the activity in comparison to the satisfaction achieved. This cost is not a monetary cost so much as a cost of individual utility or satisfaction. The input is a measure of the physiological and/or psychological investment. The output is then a measure of the satisfaction obtained from the activity.

    1 These measures are introduced in Chapter I. Barnard, The Functions of the Executive, Cambridge, Harvard University Press, 1938.

%------------------------------------------------

\section{Our Most Important Innovation}\index{Our Most Important Innovation}


Organization is humankind’s most important innovation. Humans, from their earliest beginning as part of the natural world, found it necessary to collaborate and cooperate. A look back at Figure 1 in the context of the world that we observe, lends credence to the concept of emergence.

A look back at Figure 1 in the context of the world that we observe, lends credence to the concept of emergence . . . 

Figure 3.1 Contributor’s Organizational Benefits and Burdens

There are advantages to be gained from group activity. An effective cooperative system accomplishes objectives far in excess of the simple sum of its parts. Then, the pooling of activity may permit each member of the group to satisfy his wants more expediently than alone and by individual action. All the many patterns of human behavior may be placed into one of these two categories; behavior that is independent of the activity of others, and behavior undertaken as part of a group effort and towards a common objective.

\subsection{Objective of Organized Activity}\index{Objective of Organized Activity}

Objectives pursued by organizations should be directed to the satisfaction of demands resulting from human wants. Therefore, the determination of appropriate objectives for organized activity must be preceded by an effort to determine precisely what these wants are. 

Fabrycky, W.J., Comment offered in 1960 for a graduate course taught by Prof. H.G. Thueson at OSU

Industrial organizations conduct market studies to learn what consumer goods should be produced. City commissions make surveys to ascertain what civic projects will be of most benefit. Highway commissions conduct traffic counts to learn what construction programs should be undertaken.

Organizations come into being as a means for creating and exchanging utility. Their success is dependent on the appropriateness of the series of acts contributed to the system. Most of these acts are purposeful; that is, they are directed to the accomplishment of some objective. These acts are physical in nature and find purposeful employment in the alteration of the physical environment. As a result, utility is created which, through the process of distribution, makes it possible for the cooperative system to endure.

Before the industrial revolution, most productive activity was accomplished in small owner-manager enterprises, usually with a single decision maker and simple organizational objectives. Increased technology and the growth of industrial organizations made necessary the establishment of a hierarchy of objectives.  This, in turn, required a division of the management function until today a hierarchy of decision makers exists in most organizations. Each decision maker is charged with the responsibility of meeting the objectives of his organizational division. Therefore, he may be expected to pursue these objectives in a manner consistent with his view of what is good for the organization as a whole.

The function of the management process is the delineation of organizational objectives and the coordination of activity toward the accomplishment of these objectives. To maintain this system in equilibrium, the decision maker must constantly choose from among a changing set of alternatives. Each member of a set of alternatives may contribute differently to the effectiveness with which organizational objectives are achieved and the contributors satisfied. It is evident from this that managerial talent is a valuable resource.

\subsection{The Benefits of Human Organization}\index{The Benefits of Human Organization}

Organized effort often leads to economy in the accomplishment of an objective. Suppose, for example, that two men adjacent to each other are confronted with the task of lifting a box onto a loading platform, and that each of the two boxes is too heavy for one man to lift but not too heavy for two men to lift. Assume that the only practical way for one man to accomplish his task is to obtain a hand winch with which the task can be accomplished in 30 minutes’ time. If there is no coordination of effort, the cost of getting the two boxes onto the platform will be 60 worker-minutes.

Suppose that the two men had coordinated their efforts to lift the two boxes in turn and the time consumed was 1 minute per box. The two tasks would have been accomplished at the expense of 4 worker-minutes or about one-fifteenth as much time as if there had been no coordination of effort.

Coordination of human effort is so effective a means of labor saving that it may be economical to pay for effort to bring about coordination of effort. In the preceding example, effort directed to bring about coordination would result in a net labor saving of 56 worker-minutes of effort.

As a further illustration of the economics of coordination of human effort, suppose that a water well would have a value of \$100 to each of 100 families in a village of a certain undeveloped country. The head of each of the families recognizes this and on inquiry finds that a well will cost \$1,000. Each family head, being oblivious of the opportunities for coordination, abandons the well-drilling project as unprofitable. If an entrepreneur could bring about a coordination of effort in the village, the net benefit might be as follows: (100 families X \$100) - \$1,000 = \$9,000. This illustrates that entrepreneurship is a worthwhile and necessary activity in most situations involving organized activity.

It is evident that desired ends may be obtained from the environment more easily by joint action than by individual action. For example, the utility of the harmonic sounds in music is usually increased by the precise coordination of the efforts of a group of musicians. The utility of steel is increased by a complex manufacturing process which ultimately results in an automobile. Even friendship is enhanced by participation in certain forms of organized activity.

\subsection{Efficiency of Organization}\index{Efficiency of Organization}

A person who is employed by an industrial organization may be presumed to value the wages and other benefits he gets more highly than the efforts he contributes to gain them. The person who sells material to the organization must value them less than the money he receives for them, or he or she would not sell. The same may be said of the seller of equipment. Similarly, a person who loans money to an organization will, in the long run, receive more in return than he or she advances or will cease to loan money. The customer who comes with money in hand to exchange for the products of the organization may be expected to part with his money only if he values it less than he values the products he can get for it. This situation is illustrated in Figure 3.1.

So that the organization illustrated be successful, not only must the total of the satisfactions exceed the total of contributions, but each contributor’s satisfaction must exceed his contribution as he subjectively evaluates them. In other words, contributors must realize their aspirations to a satisfactory degree, or they cease to contribute. Organizations are essentially devices to which people contribute what they desire less to gain what they desire more. Unless people receive more than they put into an organization, they withdraw from it. For an organization to endure, its efficiency (output divided by input) must exceed unity.

Efficiency, therefore, is a measure of the result of cooperative action for the contributors as subjectively evaluated by them. The effective pursuit of appropriate organizational objectives contributes directly to organizational efficiency. As used here, efficiency is a measure of the want-satisfying power of the cooperative system as a whole. Thus, efficiency is the summation of utilities received from the organization divided by the utilities given to the organization, as subjectively evaluated by each contributor.

%------------------------------------------------

\section{Enterprise and Enterprise Systems}\index{Enterprise and Enterprise Systems}

A history of man is, in part, a chronicle of the attempts of man to shape cooperative systems to accomplish his objectives. From earliest times, cooperative activity contributed to the survival of mankind. Many generations labored only to meet the basic requirements of food and protection from the elements and other forms of life. Gradually this success permitted a small surplus of activity which could be directed towards objectives beyond the barest necessities of life. It may have taken the larger part of the total existence of man before a significant surplus was able to be put aside to create a meaningful culture that could be passed on to succeeding generations.

As this surplus expanded, cooperative systems became more numerous, larger in size and more specialized in objectives. Participants acquired and refined skills unique to a cooperative system. Then communities became formalized, governments were established, armies were formed to protect the participants in these communities, and architectural works were undertaken to glorify these same people and governments. The buildings and monuments of early cooperative systems are the most obvious and permanent record of these systems. Many still exist today in diverse parts of the world as evidence of early cultures.

\subsection{Cooperative Activity}\index{Cooperative Activity}

Individuals engage in purposeful activity, a significant proportion of this activity is undertaken as a part of a cooperative system.
A cooperative system may be described as a group of people interacting, one with another, and pooling their efforts toward a common objective. Cooperative activity is the activity in which these individuals engage while a part of a cooperative system.

Individuals have been described as purposeful – as engaging in activity towards specific goals. So also are cooperative systems purposeful – they have objectives. The objectives of cooperative activity are usually restricted to the achievement of goals that either cannot be accomplished by individual activity or cannot be accomplished as economically through individual action1. In effect, cooperation is a second choice. A cooperative system might be formed to sponsor, conduct, and participate in a charity dance. Such an objective could not be accomplished by an individual because this individual could not interact with other people except through a cooperative system. Another system might be formed to manufacture automobiles. This activity could be accomplished by an individual working under a shade tree but it would probably take half of a lifetime to build one such car. A cooperative system can accomplish this objective more economically; more cars can be built per man-hour of input.

A measure of the material standard of living of a society can be obtained by assessing the extent to which many forms of activity are accomplished by collective rather than individual activity. In a  primitive society, the household furniture and utensils might be built by the individual who uses them. The house in which a man lives might even be built by this same individual, perhaps with the assistance of his immediate family - a small but unspecialized cooperative system. In a more advanced society, the house, the furniture, and the utensils will probably each have been fabricated or manufactured by a specialized cooperative system devoted exclusively to that activity. This will be a more economical method. The objectives of cooperative systems should be reserved for those objectives the individual cannot accomplish, or cannot complete as economically, and they will be either social or physical ends. One system may be established to build automobiles with an anticipated cooperative life of many years. Another may come into existence to sponsor a benefit tea and may last for only a few weeks. Each cooperative system has a goal and consists of a group of people pooling their efforts towards that objective.

1 See Chris Argyris, Integrating the Individual and the Organization, New York, John Wiley, 1964, p. 35, and Chester I. Barnard, op. cit., p. 23.

\subsection{Forms of Cooperation}\index{Forms of Cooperation}

Evidence of cooperative activity throughout the world can be seen in the remains of the early civilizations of the Mediterranean and then up into Europe during the Renaissance. Only within the last two centuries, however, has the emergence of science and the technology permitted the development of the complex enterprise systems that are so characteristic of Western civilization today. This technology and the art of directing complex cooperative systems are yielding a surplus of individual time as well as a high material standard of living. This in turn permits the formation of countless other cooperative systems directed toward social and leisure activities.

The inherent strength of cooperative activity carries no guarantee of the use to which this influence will be put. The dignity and the freedom of the individual may be immersed within the common good. The objective of this cooperative activity may be the exploitation of individuals not a part of the group.

Cooperative systems exist in a variety of forms today. A cooperative system can be considered to be any group undertaking wherein the activity or behavior of an individual must be directly coordinated with the activity of behavior of an individual must be directly coordinated with the activity or behavior of one or more other individuals toward some mutual objective. Systems that may initially come to mind include those of industry and commerce such as factories and banks. The varying levels and forms of governmental cooperative systems exist too.

Cooperative Activity is Pervasive. Cooperative activity is so prevalent and assumes such a variety of forms that is is easy to conclude that organized effort is usually successful. One might infer that the failure of a cooperative system is rather unusual. In reality, the opposite is nearer the truth. It will subsequently be demonstrated that the success of a cooperative system is the exception rather than the rule. The multitudes of observable cooperative systems are the successes remaining from a much larger number of attempts. Most cooperative systems die in infancy.

In Western civilization only a few cooperative systems have survived in essentially the same form more than a few hundred years. Some religious groups, a few universities, and a small number of national and municipal governments have enjoyed extended lives. Within the industrial arena, a few automobile companies remain today where once there were many more. The cooperative systems that do survive usually have to make a deliberate and conscious attempt to perpetuate themselves. The majority of churches actively seek converts. So also do social and fraternal organizations look for new members. Industrial concerns hire new employees and look for new products to manufacture or services to provide. All cooperative systems must maintain both an internal balance and enjoy a raison d’etre. The balance must be continually adjusted and new objectives south when old goals are accomplished. The reasons for the inherent instability of cooperative systems will subsequently be developed. For the present it will suffice to note the cooperative systems exist in a variety of forms, they are very numerous, and the state of cooperative activity need only be contrasted to the state of individual activity.

\subsection{The Plan and the Purpose}\index{The Plan and the Purpose}

It is suggested that there are constructs or explanations which are applicable to all cooperative systems. It is further suggested that an understanding of these constructs will facilitate cooperative activity and enhance both the likelihood of success of the cooperative system and the realization of anticipated satisfactions by the participants in that system.

Universal constructs are probably more characteristic of the subject-matter fields classified as physical sciences. As an example of such a construct, a physicist might express Newton’s second law of motion: the time rate of change of momentum (mass X velocity) is differentiated in regard to time and the mass is assumed to be constant, then the force can be shown to be equal to the product of the mass X the rate of acceleration.

The social sciences, e.g. psychology, economics, sociology, are also able to establish constructs or explanations. These, however, are not quite as precise and do not make up such a complete web of knowledge, and some explanations may even be partially contradictory one to another. One of the reasons for this difficulty is that the social scientist has chosen the more difficult task of understanding man. The fact remains there are more likely to be different and sometimes divergent theories to explain some of the fundamentals within the total sphere of knowledge.

The study of organizations and of the management might be placed in this latter category. There are constructs or explanations but they are not as precise as we might like and there are often exceptions or extenuating circumstances.

It is suggested that an understanding of this model or construct of cooperative systems will permit a better insight into the operation of these systems. The model will permit a better understanding of the behavior of people within cooperative systems and the role of the manager in cooperative systems.

%------------------------------------------------

\section{Conceptualization of Organization}\index{Conceptualization of Organization}

Central to the physics of  STEA are the physical aspects of organization emphasized by 

Chester Barnard. Chester Barnard was best known as the author of The Functions of the Executive, perhaps the 20thcentury’s most influential book on management and leadership. 

Barnard offers a systems approach to the study of organization, which contains a psychological theory of motivation and behavior, a sociological theory of cooperation and complex inter dependencies, and an ideology based on a meritocracy. Barnard’s teachings drew on personal insights as a senior executive of ATT in the 1920s and 1930s, and he emphasized the role of the manager as both a professional and as a steward of the corporation. For leadership to be effective, it had to be perceived as legitimate, Barnard maintained. Barnard sensed that the central challenge of management was balancing both the technological and human dimensions of organization.

Chester Barnard was best known as the author of The Functions of the Executive, perhaps the 20th century’s most influential book on management and leadership. Barnard offers a systems approach to the study of organization, which contains a psychological theory of motivation and behavior, a sociological theory of cooperation and complex inter dependencies, and an ideology based on a meritocracy. Barnard’s teachings drew on personal insights as a senior executive of ATT in the 1920s and 1930s, and he emphasized the role of the manager as both a professional and as a steward of the corporation. For leadership to be effective, it had to be perceived as legitimate, Barnard maintained. Barnard sensed that the central challenge of management was balancing both the technological and human dimensions of organization.

The challenge for the executive was to communicate organizational goals and to win the cooperation of both the formal and the informal organization; but he cautioned against relying exclusively on incentive schemes to win that cooperation. Responsibility in terms of the honor and faithfulness with which managers carry out their responsibilities is the most important function of the executive. Published: 2010

Chester Barnard and the Systems Approach to Nurturing Organizations Andrea Gabor Associate Professor, and Michael R. Bloomberg Professor of Business Journalism, Department of Journalism Baruch College City University of New York One Bernard Baruch Way 55 Lexington at 24th St. New York, NY 10010 (646) 312-3970 AAGabor@aol.com, andrea.gabor@baruch.cuny.edu 1999- Joseph T. Mahoney Investors in Business Education Professor of Strategy, \& Director of Graduate Studies, Department of Business Administration College of Business University of Illinois at Urbana-Champaign 350 Wohlers Hall 1206 South Sixth Street Champaign, IL 61820 (217) 244-8257 josephm@illinois.edu Chester Barnard (1886-1961) was best known as the author of The Functions of the Executive, perhaps the 20th century’s most influential book on management and leadership.1 The book emphasizes competence, moral integrity, rational stewardship, professionalism, and a systems approach, and was written for posterity. For generations, The Functions of the Executive proved to be an inspiration to the leading thinkers in a host of disciplines. Perrow writes that: ``This ... remarkable book contains within it the seeds of three distinct trends of organizational theory that were to dominate the field for the next three decades.''

One was the institutional theory as represented by Philip Selznick [1957]; another was the decision-making school as represented by Herbert Simon [1947]; the third was the human relations school [Mayo, 1933; Roethlisberger \& Dickson, 1939]" (1986: 63). 2 Barnard’s work also influenced sociology’s Parsons and Gouldner and informed the institutional economics of Williamson (1975, 2005). Indeed, Andrews states that: “The Functions of the Executive remains today, as it has been since its publication, the most thought-provoking book on organization and management ever written by a practicing executive” (1968: xxi). Barnard combined the capacity for abstract thought 1 As of July 25th, 2010, Barnard’s (1938) Functions of the Executive had been cited over 8,000 times (Google Scholar). See Bedeian and Wren (2001) for their ranking of the top 25 most influential management books of the 20th century with Taylor (1911) and Barnard (1938) occupying the top two positions. 2 Classic works influenced by Barnard’s The Function of the Executive include: Boulding (1956), Coser (1956), Cyert \& March (1963), Dalton (1959), Downs (1967), Gouldner (1954), Homans (1950), Katz \& Kahn (1966), Likert (1961), March \& Simon (1958), Mayo (1945), McGregor (1960), Merton (1949), Mintzberg (1973), Selznick (1957), Simon (1947), Thompson (1967), and Williamson (1975). Of particular note is the Barnard-Simon connection (Simon, 1991, 1994; Wolf, 1995a). It is worth nothing, however, that although Barnard knew both Roethlisberger and Mayo, he later claimed to have known little about the Hawthorne studies, which were completed before he wrote Functions of the Executive. Barnard did serve as a major influence on Likert (1961) and McGregor (1960). 1 with the ability to apply reason to professional experiences toward developing a “science of organization” (1938: 290).3

The challenge for the executive was to communicate organizational goals and to win the cooperation of both the formal and the informal organization; but he cautioned against relying exclusively on incentive schemes to win that cooperation. Responsibility in terms of the honor and faithfulness with which managers carry out their responsibilities is the most important function of the executive. Published: 2010 URL: 

One was the institutional theory as represented by Philip Selznick [1957]; another was the decision-making school as represented by Herbert Simon [1947]; the third was the human relations school [Mayo, 1933; Roethlisberger \& Dickson, 1939]" (1986: 63). 2 Barnard’s work also influenced sociology’s Parsons and Gouldner and informed the institutional economics of Williamson (1975, 2005). Indeed, Andrews states that: ``The Functions of the Executive remains today, as it has been since its publication, the most thought-provoking book on organization and management ever written by a practicing executive'' (1968: xxi). Barnard combined the capacity for abstract thought 1 As of July 25th, 2010, Barnard’s (1938) Functions of the Executive had been cited over 8,000 times (Google Scholar). See Bedeian and Wren (2001) for their ranking of the top 25 most influential management books of the 20th century with Taylor (1911) and Barnard (1938) occupying the top two positions. 2 Classic works influenced by Barnard’s The Function of the Executive include: Boulding (1956), Coser (1956), Cyert \& March (1963), Dalton (1959), Downs (1967), Gouldner (1954), Homans (1950), Katz \& Kahn (1966), Likert (1961), March \& Simon (1958), Mayo (1945), McGregor (1960), Merton (1949), Mintzberg (1973), Selznick (1957), Simon (1947), Thompson (1967), and Williamson (1975). Of note is the Barnard-Simon connection (Simon, 1991, 1994; Wolf, 1995a). It is worth nothing, however, that although Barnard knew both Roethlisberger and Mayo, he later claimed to have known little about the Hawthorne studies, which were completed before he wrote Functions of the Executive. Barnard did serve as a major influence on Likert (1961) and McGregor (1960). 1 with the ability to apply reason to professional experiences toward developing a ``science of organization'' (1938: 290).3

The Systems Approach to Nurturing Organizations Andrea Gabor Joseph T. Mahoney Baruch College, City University of New York University of Illinois at Urbana Champaign, College of Business.
    
\subsection{Organization Theory per Thusen}\index{Organization Theory per Thusen}

\subsection{Organization Theory per Torgesen}\index{Organization Theory per Torgesen}

Organization of the efforts of individuals is an invention of man. Organizations are often thought of as consisting of or being made up of people. Thought persons are always associated with organizations, it is not persons that are organized but the actions or in fact more clearly the muscular forces of persons. These muscular forces are of course physical.

This concept of organizations is embodied in Barnard’s definition which states, “An organization is the consciously coordinated forces or actions of two or more persons.”  This discussion will be based on this concept which is thoroughly analyzed in Barnard’s The Functions of the Executive. It is the physical aspects of organizations that are to be examined. The author is not unaware of and has often felt what is described as the spirit or esprit de corps of organizations. Such subjective reactions to participants are undoubtedly very important in organizations, but they are subjective and therefore difficult to analyze and evaluate; it is particularly difficult to determine the causes or origins of favorable and unfavorable subjective reactions.

The physical aspects of organizations which are observable and important aspects of organized action can be quite objectively delineated. For some strange reason, few observers have devoted their energies to this task. Rather the emphasis has been placed upon subjective aspects and the viewpoints held about these aspects are as varied as the number of observers and so perhaps of limited use.

It is believed that if attention were first directed to the objective aspects and if these became well understood, it is likely that the subjective aspects of organization could be more profitably dealt with.
There is much to be gained from examinations of the physical manifestations of organizations.

Since this discussion is based on Barnard’s definition of organization which considers the basic ingredient of organization to be the ``efforts or activities of persons,'' it is desirable to consider the physical attributes of persons and personal efforts or activities.

Physical characteristics of persons. A person has certain physical aspects. He is a discrete entity that has weight or mass, temperature, light reflective capacity, color, etc., as do inanimate masses of matter. He further has the capacity to ingest or absorb food, water, air, heat energy, light energy, etc., and the ability to expel waste products, air, heat and light energy. By this process he maintains an equilibrium within the physical environment as long as he lives. The individual also converts physical intake into physical energy. It is this physical energy produced by his muscles that lead to the ``efforts and activities'' which are the crux of organized human effort.

Much has been said about the spirit of an organization. It should be clear that spirit in an observer is essentially a subjective evaluation generated in him by a leader or associates being perceived by virtue of physical energy flows to his sense organs.

The Activities of Persons

The observable external manifestation of a person’s activity, meaning a series of acts, is movement of his body parts. To move a person must produce forces by contracting muscles against his skeletal frame. In other words, he accelerates and displaces some or all of his body parts with physical forces generated within his muscles.

The external activities of persons consist of movement of body parts. They are objective and readily observable.

A person is also capable of mental activity. Mental activity is not directly observable except perhaps by means of an encephalograph. As important as mental activities are in relation to the behavior of persons, the fact remains that for organizational or managerial purposes they are not directly and objectively observable.

Then mental activity leads to action of body parts, an observer may attempt inference of the mental activity has taken place. Inference of the mental activities, that is the inference of thoughts, which have led to a given muscular action is fraught with great uncertainty. Therein lays the difficulty of conveying the thoughts of one person to another. The first person must always code his thoughts in terms that are possible of being decoded by a second person.

There is no way to transfer thoughts from one person to another except by movement of body parts that can be observed by other persons through physical media that can be detected by one or more of the five sense organs.

If a person’s mental activities lead to no physical action, they are in fact of no consequence in so far as other persons are concerned. It should be clear that a manager, a poet, a teacher, a researcher, or a parent influences people by his physical actions only, and not by his mental activities no matter how wise or profound the latter may be. If the manager’s plans are to be put into effect, he must code them in physically manifested verbal or written directions which can be observed by his followers and decoded in their minds into meaningful calls for action. Similarly a researcher must make known through some physical actions understandable to others what truths he has discovered or his efforts go for naught.

Cooperative effort is an essential characteristic of organized effort. When two or more persons coordinate their actions, i.e. their muscular force for a common purpose, the resulting fused actions of forces are said to be organized.

The mental activities of two or more persons may also be joined by proper coding of thoughts. For example, assume that an executive does not know the address of a client but that his secretary does. The executive may find the desired address with considerable expenditure of time by looking through notes in his desk. But he may gain the desired information with much less total expenditure by organized efforts by asking, ``What is the address of the Smith Company which I called upon last week,'' and immediately receive the reply, ``It is located at 6857 North Sylvester Avenue,'' from his secretary who has the address stored in her mind, i.e. memorized.

In this exchange the executive coded his message and transmitted it as sound waves by physically expelling air over his vocal cords. These waves were received by the ears of the secretary, alerted her attention, were decoded for meaning, produced a search of memory for the desired information and a coded message expressed in terms of sound waves which in turn were received by the executive as physical sound waves and decoded by him.

There is little question that communication requires expenditure of energy on the part of the ``sender.'' Also messages are usually transmitted between persons through intervening physical mediums such as air, light waves, electric waves, and physical bodies where senses of touch, taste and smell are involved.

But is there expenditure of muscular energy on the part of the receiver?  Some observers in the fields of psychology and physiology have found that thinking is accompanied by a series of incipient muscular conscious thinking will be absent.

This view is adhered to in this section for it simplifies consideration of cooperative action involving communication. By accepting this view cooperative communication, i.e. communication between two or more persons, requires coordination of muscular forces of sender and receivers to a degree.

\subsection{The Environment of Organizations}\index{The Environment of Organizations}

The total environment may be considered to consist of the universe. Though the universe is inseperable, it is convenient to regard it to consist of three segments, namely the physical, biological, and societal segments. By the physical segment we mean the parts of the environment that have mass such as steel bars, water, air etc., and which embody physical energy such as heat, light power. The biological segment refers to those elements that are characterized by life, that is the capacity to maintain an equilibrium with the balance of the environment by internal self-directed processes, the capacity to base action on experience and by capacity to reproduce its kind. The social segment refers to the interaction of humans. None of the segments can exist alone. For example, there appears to be no example of living beings without physical characteristics. Nor can there be social reactions without life.

Since the total environment is all pervasive, an organization functions in and accomplishes its objectives if at all within the environment.

The question thus arises, how do organizations accomplish their ends. At this point it is well to recall the definitions of organization being used. At this point it is well to recall the definitions of organization being used. Namely, an organization is the consciously coordinated forces of two or more persons.

To understand this definition in toto, it will be helpful to define its elements. The forces of two or more persons are muscular forces external to the persons. These forces are manifested as force applied through distance and are measurable in such units as foot pounds and gram centimeters. By their very nature the forces involved in organization are readily objectively observable.

By the word conscious in the present connection is meant awareness. That is, a person who contributes a force senses or is aware that he is so doing. Not only is a contributor aware of the forces he contributes to an organization but is also usually cognizant in some degree why he is doing so.

The term coordination embodies a spectrum of meanings. But in connection with the definition of organization coordination means that the force or forces supplied by one person is in some relationship with the force or forces contributed to an organization by other persons. The organization in its fundamental aspects consists essentially of the resultant of the several forces supplied at any given instant by the several persons.

This alteration of the physical environment may have utility. And this example is illustrative of the basic method by which utilities are created by organizations. That is, utilities are created in organizations would be sharpened and easier to understand if the underlined word were added to cause the definition to read: An organization is the consciously, continuously coordinated forces or actions of two or more persons. Strictly speaking, this is the definition of a unit organization.

The function of organization is to create utility. Utility is created by altering the physical environment in such a manner that is increases the satisfaction of people.

A violinist draws his bow across the strings setting in motion air waves that may create a satisfaction in the mind of a person whose ears perceive the waves. A delivery service created utility by moving a package from sender to receiver. A machinist creates utility by cutting away part of the metal of a bar with the aid of a machine. In typical factory operations utility is created by a sequence of acts expended upon material.

These steps are typical of the production of useful goods and services. It is apparent any product or service can be produced by similar incremental steps. During some steps continuously coordinated action of two or more persons is used as are the steps in 1 and 3 above. At other time only the forces of individuals are needed either to be expended on the basic material directly or to control machine elements or mechanical power sources of one kind or another.

In the interests of economy the usual practice is to reduce the need for unit organization. This can often be done by supplying outside power sources. For example, in Step 1 a fork truck operated by one man could be employed to move the bar from platform to crane.

\subsection{The Physical Aspects of Control}\index{The Physical Aspects of Control}

A considerable amount of physical energy expended in organizations is directed not to produce the product directly but to direct the activities of persons so that they may produce a desired product. Spontaneous coordination of human efforts rarely produces a useful result and even more rarely a useful pre-determined result.

Consider by the way of analogy an internal combustion engine. It takes in fuel and converts some of the energy in it to physical force energy. Some of this energy is delivered through the crank shaft for external application, some is wasted in internal friction and in heat losses and some of it is utilized in driving certain elements such as the cam shaft and ignition system which regulate the functioning of valves and the ignition of the explosive mixtures in a proper relationship with various other elements of the engine as necessary for the engine to function. It should be noted that the energy consumed in regulation does not contribute directly to the energy delivered to the crank shaft for external use.

The energy expended in organizations to achieve coordinated effort of two or more persons is expended in communication, which is to transfer information from one person to one or more persons.

How is information transferred from one person to another?  The method for transferring information is for the first person to code the information, the ideas in his mind, in a system of symbols that can be transmitted through some physical medium, which can be discerned, decoded, and understood, by a second person.

In the transmission of information as necessary to coordinate human activities and thus in the maintenance of organized action the most usual receptor means are the sight and hearing senses. The receptors involving the senses of touch, tasting, and smelling appear to be used infrequently but the general steps discussed above with respect to sight and hearing.

If, as has been assumed above that communication require expenditure of energy in the form of muscular forces by the transmitter and the receiver, the question arises, do the forces of the transmitter and the receiver produce a resultant force or forces.

Consider voice transmission through air. Most of the energy expended to produce air waves is damped out by air friction and by being absorbed by the surrounding physical environment such as walls, clothing, etc. Only a small portion of the total energy enters the ear of the receiver and this small amount is probably damped out in the process of stimulating the auditory sense receptors. Whatever muscular tension the receiver must exert to ``listen'' seems not to produce forces that join with forces expended by the sender to produce a resultant force as is the case when two persons join forces as in a unit organization to alter the external environment.

Why no resultant forces are produced by the communications efforts of two or more persons may be because they are not usually simultaneous.

The roll of communication in organization is not, so to speak, to do the work of the organization but to regulate or coordinate the physical efforts of contributors who supply the physical forces through which organization objectives will be achieved. In this respect communication plays a roll analogous to that of cam shafts and ignition systems of an engine which regulates the power producing elements of the engine, at the expense of and not contribution to the engine external power output.

\subsection{Other Energy Expenditures in Organization}\index{Other Energy Expenditures in Organization}

Above, it has been indicated that a chief essential ingredient of organizations as defined by Barnard is muscular force of persons.

Also it has been indicated that the objectives of organizations are achieved by altering the physical environment. This is true regardless of whether the ultimate purpose of the organization is to alter principally the physical, biological, or social segment of the environment.

It has been pointed out that the limitations of persons can be overcome by coordinating their forces. For example, two men may be able to lift a greater weight than one. Two persons can look in two directions at once, etc. These are representative of the energies that are employed to achieve the objectives of organizations and require coordination of the forces of persons.

Coordination of forces of persons rarely occurs spontaneously and perhaps never without the expenditure of energy to achieve this purpose.

Energy expended to achieve coordination is largely energy used to transfer information in communication. This energy is in the form of physically encoded ideas transmitted by the muscular forces of the sender through a physical media to a receiver. To be cognizant of the incoming coded message the receiver, it seems, must be under at least some muscular tension and then once having received the encoded message, must be able to decode it to gain an understanding of the idea transmitted. There is no question that there is some loss in information in the process. But in any event the energy expended by the two participants is dissipated in the process of altering the mind of the recipient and apparently has no other tangible effect that might be useful in achieving the objective of the organization. That is to say that the energy is expended in the setting up and maintenance.

To this point nothing has been mentioned about the energy, both physical and mental, that is expended by a contributor in preparation to communicate. This is essentially an individual activity. Consider for example, a manager who prepares a communication for a heads of departments meeting to set forth the production objectives to be reached during the ensuing fiscal year. He may make many calculations, notes, graphs, etc. in arriving at the illustrated presentations he finally makes. This preparation may possibly be done exclusively through the manager’s individual efforts and thus not require coordination of efforts. Certainly preparation of parts of any communication will have to be the result of individual efforts and thus not require coordination of efforts. Certainly preparation of parts of any communication will have to be the result of individual effort. As a second example, consider the common method of communicating technical directions via mechanical drawings. It is apparent that preparations of drawing require mental and physical effort at least some of which is individually done.

Closely associated with but probably not actually a part of formal organization is the intertwined informal organization that seems to always accompany formal organizations because of the human needs of contributors. The activity of informal organizations centers about communication. The communication in question requires expenditures of energy and is carried on essentially as communication necessary to coordinate forces in formal organization and differs only in objective.

%------------------------------------------------

\section{Systems Engineering Organization}\index{Systems Engineering Organization}

An initial step in managing the systems engineering process is to develop an early implementation plan and an organizational structure that will be responsive to program requirements. Systems engineering planning starts with the identification of a customer need and the definition of requirements for a program (project) to design, develop, produce, and deliver a system that will be responsive and affordable. Although every program is somewhat different, its overall planning is usually promulgated through a program management plan (PMP) or equivalent. From this top-level plan, a systems engineering management plan (SEMP), or systems engineering plan (SEP), is derived to guide implementation of the technical activities described in the prior chapters of this textbook.

A SEMP, which should be prepared during the conceptual design phase, provides the necessary guidance for the many design and management plans required for a given program. Included within the SEMP is the identification of systems engineering program tasks, a program work breakdown structure (WBS), task schedule and cost requirements, and the needed organizational structure for program implementation. In developing an organizational approach, it is essential that an environment be established that will allow for the effective and efficient coordination and integration of the various engineering and supporting disciplines that contribute to the overall system design process. Appropriate leadership must also be in place to promote good communications across organizational lines and to foster a truly interdisciplinary approach to system design and development.

The primary objective of systems engineering management is to facilitate the timely integration of numerous design considerations (refer to ) into a functioning system that will be of high value to the customer. Accordingly, the purpose in this chapter is to provide the reader with an understanding of the basic systems engineering planning and organizational requirements for a typical program through consideration of the following factors
\begin{itemize}
\item Systems engineering program planning needs
\item The development of a systems engineering management plan (SEMP) - statement of work (SOW), systems engineering program tasks, work breakdown structure (WBS), scheduling of tasks, projecting costs for program tasks, interfacing with other planning activities
\item Organization for systems engineering - developing the organizational structure, customer/producer/supplier relationships, customer organization and functions, producer organization and functions, supplier organization and functions, and staffing the organization (human resource requirements).
\end{itemize}

The purpose of organization, considered in the systems engineering context, is to fulfill the requirements described in the SEMP and in the system specification (Type A). The goals of the organization are directed toward this end and pertain to two levels of activity: (1) the goals specified by the TPMs, discussed in, which quantify the DDPs making specific the characteristics that must be incorporated into the design of a given system; and (2) the goals of the organization relative to accomplishing the necessary activities (tasks) to ensure that the first objective is attained. A basic question is, How effective and efficient is the organization functioning in the accomplishment of the first goal?

In this instance, the question can be addressed to the applicable systems engineering organization relative to (1) its impact on the ultimate system/product design configuration itself; and (2) its effectiveness and efficiency in performing the 11 tasks described in.2. It is not uncommon to address only the second goal, believing that the organization is doing well in accomplishing the identified tasks when, at the same time, the organization (and its personnel) has not impacted the design at all. Thus, when assessing the organizational capabilities later, the second goal must be evaluated in terms of the first.

Regarding the issue of organizational structure, which is the point of emphasis here, the objective is to develop a complete systems engineering capability that will enable the accomplishment of the 11 program tasks (or functions of an equivalent nature) in an effective manner. This, in turn, requires (1) the establishment of a desired set of metrics against which each of the applicable tasks can be assessed; (2) the development of the necessary processes for the performance of these tasks; and (3) implementation of a data collection and information capability that will enable management to “track” performance and determine just how well the organization is functioning overall. The organizational objective is to establish a disciplined, well-defined approach for the performance of these tasks, establish measurable goals that can be quantitatively expressed and controlled, and initiate a provision for continuous process improvement.

Having identified the systems engineering tasks to be accomplished, along with the associated ``metrics,'' management needs to assess the current status of the organization. At the same time, it may be appropriate to compare the results with other comparable entities. A benchmarking approach can be applied to aid in the development of future goals. Benchmarking can be defined as ``an ongoing activity of comparing one’s own process, product, or service against the best known similar activity, so that challenging but attainable goals can be set, and a realistic course of action implemented to efficiently become and remain the best of the best in a reasonable period of time.'' The ultimate questions are, Where are we today? How do we compare with others relative to the product, process, and/or organization? Where would we like to be in the future?

The organizational goals and objectives must first be responsive to the established system operational requirements and the TPMs for the system being developed (refer to ). Then, additional goals can be established for the systems engineering organization itself. Through the use of benchmarking, future goals can be identified and an organizational growth plan can be developed, which, when implemented, can aid in realizing the desired objectives.

\subsection{Transdisciplinarity Reaching Beyond Disciplines to Find Connections}\index{Transdisciplinarity Reaching Beyond Disciplines to Find Connections}

Considering the foregoing, it is worth clarifying the differences among intradisciplinary, multidisciplinary, interdisciplinary, and transdisciplinary research. Intradisciplinary research involves problems that can be successfully tackled from within a single discipline such as engineering or medicine. Multidisciplinary research involves problems that require cooperation among individuals from different disciplines. Interdisciplinary research involves cooperation among disciplines leading to enrichment of one or more contributing disciplines and occasionally resulting in new discipline. Transdisciplinary research involves looking beyond traditional disciplines to find new connections among disciplines that facilitate knowledge unification. Table 1 compares and contrasts these various forms of research initiatives. 

We live in an era in which the world is becoming increasingly more connected or, as Tom Friedman puts it, ``flat.'' The inevitable consequence of this interconnectedness trend is that problems are becoming much too complex to successfully solve by applying methods from within a single discipline. This recognition is most evident in the growing trend toward multidisciplinary and interdisciplinary collaboration among traditionally independent disciplines. As such collaboration intensifies, existing disciplines are being enriched and occasionally new disciplines are beginning to emerge. At the same time, the knowledge gaps among the disciplines are beginning to surface. What appears to be lacking is a new way of thinking that strives to harmonize traditional disciplines by reaching beyond their traditional boundaries to fill the knowledge gaps not addressed by them. Transdisciplinary thinking promises to reach beyond disciplinary boundaries to identify and overcome knowledge voids and incompatibilities in the quest for knowledge unification. Achieving these objectives is key to fostering new relationships among traditionally independent disciplines and, in so doing, begin to address problems of national and global significance. This paper discusses the aims of transdisciplinarity, the road to transdisciplinarity, successes resulting from transdisciplinary thinking, and recommendations for a research and education agenda embracing trandsdisciplinary thinking. 

Today there is a growing recognition that such collaboration, while essential, is neither straightforward nor easy. This is because the gulf between independent disciplines needs to be bridged before such collaboration can start paying real dividends. Bridging independent disciplines typically requires extending them, reconciling their differences, and unifying the knowledge associated with them in new and novel ways. This recognition inevitably leads to the notion of transdisciplinarity, the quest for and discovery of new connections among disciplines leading to novel approaches for unifying their knowledge. It is not surprising, therefore, that there is a growing interest in transdisciplinary thinking to address problems that appear to be intractable when viewed from the perspective of a single discipline. 

Transactions of the SDPS MARCH 2007, Vol. 11, No. 1, pp. 1-11 

Transdisciplinarity is a global perception of the ultimate connection of multiple (possibly all) disciplines (Nicolescu, 1997). From this perspective, not only science, but all human activities appear as a unitary whole, and part of the unity of the universe. According to Rodriguez Delgado, an eminent Spaniard systemist, unity and diversity (within transdisciplinarity) are not viewed as opposing, but complementary perspectives. 

Despite its obvious allure, transdisciplinarity organization faces many challenges. To begin with, there is no single, agreed upon definition of transdisciplinarity. Academic and societal viewpoints differ. This lack of a common definition is further exacerbated with the formation of new societies, each promoting their own language for discourse. Fortunately, the academic and business communities remain undaunted as evidenced by the growing interest worldwide in infusing transdisciplinary concepts and projects into their educational and research agendas. For example, when it comes to public health, the National Academies (National Academies, 2002) recommends moving from research dominated by a single discipline or a small number of disciplines to transdisciplinary research. They define transdisciplinary research as involving broadly constituted teams of researchers that work across disciplines to develop significant research questions. In these recommendations, transdisciplinary research implies the conception of research questions that transcend individual disciplines and specialized knowledge bases because they are intended to solve applied public health research questions that are, by definition, beyond the purview of any single discipline. In transdisciplinary research, different specialties combine their expertise (and that of community members) to collectively define the health problem and their solutions. The National Academies sum up their position by pointing out that the one qualitatively different and unique aspect of the transdisciplinary process is the holistic blending of expert and community inputs to produce greater integration across disciplines. 

It is worth recognizing that transdisciplinarity originates from the increasing demand for relevance and applicability of academic research to societal challenges (Nicolescu, 2002). Not surprisingly, the two popular definitions of transdisciplinary research today center around academic research and societal challenges. The academic research-oriented definition characterizes transdisciplinarity as ``a special form of interdisciplinarity in which boundaries between and beyond disciplines are transcended and disciplines as well as non-scientific sources are integrated.'' The societal challenge-oriented definition characterizes transdisciplinarity as ``a new form of learning and problem-solving involving cooperation among different parts of society (including academia) to meet complex societal challenges. Solutions devised are a result of collaboration and mutual learning among multiple stakeholders.'' As can be seen from the preceding two definitions, there is no standard definition of transdisciplinarity. What is common to both, however, is the unity of knowledge. Journal of Integrated Design and Process Science MARCH 2007, Vol. 11, No. 1, pp. 2

%------------------------------------------------

\section{Systems Engineering Management}\index{Systems Engineering Management}

Given a comprehensive systems engineering plan and an organizational structure, as described and developed in , the challenge becomes one of implementation. The past provides many examples where relevant and timely planning was accomplished from the beginning, only to find that the subsequent implementation process did not follow the plan. As a result, the plan became impotent by neglect. For systems engineering objectives to be accomplished successfully, a two-step management approach is recommended. First, the planning and organization for systems engineering should be completed as presented in the section. Then, the follow-on program management, implementation and control, and evaluation activities described in this chapter should be activated.

Referring to , the SEMP defines the specific requirements for the implementation of a systems engineering program (or project) with the objective of designing, developing, and bringing a new (or reengineered) system into being. Program goals and objectives, required systems engineering tasks, a work breakdown structure (WBS), task schedules, and cost projections should be included in the SEMP. In addition, an organizational approach is described, along with structure and organizational interface requirements. The SEMP, in conjunction with the system specification (Type A), basically addresses the what requirements from an overall program perspective.

With the what requirements as input, the essential next step is to implement the plan. Accordingly, the objectives of this chapter are to respond to the SEMP requirements by describing the hows as they pertain to plan implementation. Through a study of the material in, the reader should become knowledgeable about the planning and organizational requirements for systems engineering as presented in, and also about the day-to-day program management, control, and evaluation requirements described in this chapter. Remaining topics that should be reviewed and assimilated are the following

\begin{itemize}
\item Establishing specific organizational goals and objectives
\item Outsourcing requirements and the identification of suppliers
\item Providing day-to-day program leadership and direction
\item Implementing a program evaluation and feedback capability
\item Conducting a risk analysis and management function
\end{itemize}

Of special interest in this chapter is an approach for conducting a performance evaluation of a systems engineering organization, including the determination of a level of maturity for the organization. This kind of organizational assessment is in keeping with the increasing necessity for accountability that now exists in most areas of human endeavor.

Successful implementation of the concepts, principles, models, and methods of systems engineering and analysis requires the coordination of numerous technical and managerial endeavors. planning, organizational, and management matters. Although the former dominates the latter, success with the technical is not possible in the absence of the managerial.

Systems engineering may appear to be appropriate and responsive to a given need and related stakeholder interests. However, the best technical approach may not be realized, regardless of how well it embraces systems engineering methods. Effective and efficient program implementation requires timely planning, the establishment of an appropriate organizational structure, a collaborative engineering environment, and continuously applied management controls. To ensure the realization of program deliverables that meet or exceed customer expectations, an appropriate blend of technical and management attention is required. illustrate, the proper implementation of systems engineering begins with the establishment of requirements by a planning process properly initiated during the conceptual design phase of the system life cycle. Inherent within this planning is the identification and scheduling of systems engineering functions and tasks, the development of an effective organizational structure, and the establishment of a sustaining oversight and control capability. Timely feedback regarding overall program visibility and status is necessary. Planning initiated early, coordinated with the development of a systems engineering management plan (SEMP), gives confidence that the systems engineering process can proceed in an effective manner to produce the desired outcome for stakeholders.

The objective of is to provide an overview of essential management functions that pertain to the implementation of a comprehensive systems engineering program. This is accomplished through two closely related chapters. is devoted to the establishment of systems engineering planning and organization, with emphasis on the SEMP. The section that the next section follows, with primary concentration on sustaining management, control, and evaluation activities that are critical to the effective implementation of the intent of systems engineering management. These sections are mutually supportive and should be considered together.

%------------------------------------------------

\section{Summary and Extensions}\index{Summary and Extensions}

Table 3.1 Some Axioms in a Concept of Organizations (Thuesen)

Reference: C.I. Barnard, The Functions of the Executive, Harvard University Press.

    1. All individuals are unique – a product of their heredity and environment – and possessing a limited power of choice.
    2. Individuals are rational – they engage in activity and seek to maximize their satisfactions.
    3. An organization is a system of coordinated human activities directed toward a common objective.
    4. Organizations come into being to accomplish objectives that either cannot be accomplished through individual activity or cannot be accomplished as readily through individual effort.
    5. Individuals contribute their activity to an organization if they believe they are receiving or anticipate receiving more from the organization than they are giving to it.
    6. For an organization to remain in existence, it must be effective – it must make progress towards the attainment of its objective, and it must be efficient – it must distribute more benefits than it requires in burdens as subjectively viewed by the contributors.
    7. All organizations have objectives which may or may not be explicitly stated. The objectives of individuals in an organization may or may not be the same as the objective of the organization.
    8. Organizational objectives may sometimes be detected by noting the activities in which the organization engages.
    9. Most organizations are subordinate to superior organizations which restrict the objectives of the subordinate organization.
    10. The process of establishing subordinate organizations with intermediate objectives is the process of specialization and is ultimately responsible for the effectiveness of the total organization.
    11. The accomplishment of these intermediate objectives should facilitate accomplishment of the total objective.
    12. Organizations are essentially unstable. They fail because of 
        a. A hostile environment
        b. The accomplishment of the objective
        c. Internal problems – the managerial process
    13. Management is the activity of maintaining an organization and it is accomplished through communication.
    14. The size and nature of the basic organizational unit will depend upon
        a. The complexity of purpose or objective
        b. The communication processes
        c. The nature of organizational participants.
    15. All organizations will require
        a. An objective
        b. Communication
        c. Participants willing to serve
    16. The organization must dispense incentives. These may include
        a. Material objects – money, air-conditioned office, etc.
        b. Personal satisfactions – prestige, recognition, opportunity, etc.
        c. Organizational goals – ideals, pride of workmanship, etc.
        d. Associations – social needs, etc.
    17. Activities are coordinated through communication. Authority resides with the recipient of an order. If an order is accepted, the communication was authoritative.
    18. Orders are likely to be accepted and authority granted
        a. If the order is understood, and
        b. It is consistent with the purpose of the organization, and
        c. It is not incompatible with personal interests, and
        d. The recipient is mentally and physically able to comply.
    19. Authority is granted upward in the organization as an economical means of securing cooperation, and thus maintaining the organization.
    20. Informal organizations are an aggregate of personal contacts and interactions without a clearly defined purpose, communication network or incentives.
    21. Informal organizations are always present in formal organizations. They establish attitudes, social norms and they facilitate communication. Society is structured by formal organizations and conditioned by informal organizations.
    22. The function of the manager in an organization is
        a. Formulate the objectives of the organization
        b. Secure the services of individuals, and 
        c. Coordinate their activity through communication.
    23. The responsible manager is one that is governed by the moral code of the organization. Thus, as a good leader, he is consistent, predictable and easy to follow.
    24. Complete individual freedom and organizational activity are not possible. The individual will give of his freedom and participate in an organization if he satisfactions will increase by so doing.
    25. Freedom should mean the freedom to restrict oneself within an organization, commensurate with the common good.

%------------------------------------------------

%% SEA CHAPTER 3 - CONCEPTUAL SYSTEM DESIGN
% SEA Question Location in \label{sea-Chapter#-Problem#}
% NEEDS UPDATING
\begin{exercises}
    \begin{exercise}
    \label{sea-3-1}
        In accomplishing a needs analysis in response to a given deficiency,what type of information would you include? Describe the process that you would use in developing the necessary information.
    \end{exercise}
    \begin{solution}
    \end{solution}
    
    \begin{exercise}
    \label{sea-3-2}
        What is the purpose of the feasibility analysis? What considerations should be addressed in the completion of such an analysis?
    \end{exercise}
    \begin{solution}
    \end{solution}
    
    \begin{exercise}
    \label{sea-3-3}
        Through a review of the literature, describe the QFD approach and how it could be applied in helping to define the requirements for a given system design.
    \end{exercise}
    \begin{solution}
    \end{solution}
    
    \begin{exercise}
    \label{sea-3-4}
        Why is the definition of system operational requirements important? What type of information is included?
    \end{exercise}
    \begin{solution}
    \end{solution}
    
    \begin{exercise}
    \label{sea-3-5}
        What specific challenges exist in defining the operational requirements for a system-of-systems (SOS) configuration? What is meant by interoperability? Provide an example.
    \end{exercise}
    \begin{solution}
    \end{solution}
    
    \begin{exercise}
    \label{sea-3-6}
        hy is it important to define specific mission scenarios (or operational profiles) within the context of the system operational requirements?
    \end{exercise}
    \begin{solution}
    \end{solution}
    
    \begin{exercise}
    \label{sea-3-7}
        What information should be included in the system maintenance concept? How is it developed (describe the steps), and at what point in the system life cycle should it be developed?
    \end{exercise}
    \begin{solution}
    \end{solution}
    
    \begin{exercise}
    \label{sea-3-8}
        How do system operational requirements influence the maintenance concept (if at all)?
    \end{exercise}
    \begin{solution}
    \end{solution}
    
    \begin{exercise}
    \label{sea-3-9}
        How does the maintenance concept affect system/product design? Give some specific examples.
    \end{exercise}
    \begin{solution}
    \end{solution}
    
    \begin{exercise}
    \label{sea-3-10}
        Select a system of your choice and develop the operational requirements for that system. Based on the results,develop the maintenance concept for the system.Construct the necessary operational and maintenance flows,identify repair policies,and apply quantitative effectiveness factors as appropriate.
    \end{exercise}
    \begin{solution}
    \end{solution}
    
    \begin{exercise}
    \label{sea-3-11}
        Refer to Figure 3.12 and assume that a similar infrastructure exists in your own community. Identify the various system capabilities (functions), illustrate (draw) an overall configuration structure (similar to that in the figure), and identify some of the critical metrics required as an input to the design of such.
    \end{exercise}
    \begin{solution}
    \end{solution}
    
    \begin{exercise}
    \label{sea-3-12}
        Refer to Figure 3.13. For a system-of-systems (SOS) configuration of your choice, describe some of the critical requirements in the design of a SOS. 
    \end{exercise}
    \begin{solution}
    \end{solution}
    
    \begin{exercise}
    \label{sea-3-13}
        In evaluating whether or not a two- or three-level maintenance concept should be specified, what factors would you consider in the evaluation process?
    \end{exercise}
    \begin{solution}
    \end{solution}
    
    \begin{exercise}
    \label{sea-3-14}
        In developing the maintenance concept, it is essential that all levels of maintenance be considered on an integrated basis. Why?
    \end{exercise}
    \begin{solution}
    \end{solution}
    
    \begin{exercise}
    \label{sea-3-15}
        Why is the development of technical performance measures (TPMs) important?
    \end{exercise}
    \begin{solution}
    \end{solution}
    
    \begin{exercise}
    \label{sea-3-16}
        Refer to Figure 3.17. Describe the steps that you would complete in developing the information included in the figure. Be specific.
    \end{exercise}
    \begin{solution}
    \end{solution}
    
    \begin{exercise}
    \label{sea-3-17}
        What is the purpose of allocation? To what depth in the system hierarchical structure should allocation be accomplished? How does it impact system design (if at all)?
    \end{exercise}
    \begin{solution}
    \end{solution}
    
    \begin{exercise}
    \label{sea-3-18}
        What is meant by functional analysis? When in the system life cycle is it accomplished? What purpose does it serve? Identify some of the benefits derived. Can a functional analysis be accomplished for any system? Can a functional analysis be accomplished for a system-of-systems (SOS) configuration?
    \end{exercise}
    \begin{solution}
    \end{solution}
    
    \begin{exercise}
    \label{sea-3-19}
        What is meant by a common function in the functional analysis? How are common functions determined? 
    \end{exercise}
    \begin{solution}
    \end{solution}
    
    \begin{exercise}
    \label{sea-3-20}
        How does the functional analysis lead into the definition of specific resource requirements in the form of hardware, software, people, data, facilities, and so on? Briefly describe the steps in the process, and include an example. What is the purpose of the block numbering shown in Figures 3.20, 3.21, and 3.22?
    \end{exercise}
    \begin{solution}
    \end{solution}
    
    \begin{exercise}
    \label{sea-3-21}
        What is the purpose of allocation? To what depth in the system hierarchical structure should allocation be accomplished? How does it impact system design (if at all)? How can allocation be applied for a SOS configuration (if at all)? 
    \end{exercise}
    \begin{solution}
    \end{solution}
    
    \begin{exercise}
    \label{sea-3-22}
        In conceptual design, there are a number of different requirements for predicting or estimating various system metrics. What approach (steps) would you apply in accomplishing such?
    \end{exercise}
    \begin{solution}
    \end{solution}
    
    \begin{exercise}
    \label{sea-3-23}
        What is the purpose of the formal design review? What are some of the benefits derived from the conduct of design reviews? Describe some of the negative aspects.
    \end{exercise}
    \begin{solution}
    \end{solution}
    
    \begin{exercise}
    \label{sea-3-24}
        What are the basic objectives in conducting a conceptual design review?
    \end{exercise}
    \begin{solution}
    \end{solution}
\end{exercises}
% SKIPPED