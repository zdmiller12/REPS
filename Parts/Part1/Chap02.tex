\chapterimage{wireFrame.jpg} % Chapter heading image
\chapter{SCIENTIFIC THINKING AND KNOWLEDGE}\label{chap:2}

In recent decades, humans have begun to understand the underlying structure and characteristics of natural and human-made systems in a scientific way. Chapter One emphasized the human-modified world as primary evidence of the importance of drawing upon each system category.

Some system definitions and system science concepts are presented to provide a basis for the study of systems engineering and analysis. They include the definitions of system characteristics, a classification of systems into various types, consideration of the current state of systems science, and a discussion of the transition to the Systems Age. Finally, the chapter presents technology and the nature and role of engineering in the Systems Age and ends with an account of the development of systems thinking to date with extension beyond the world we know today.

\section{Human Curiosity and Inquiry}\index{Human Curiosity and Inquiry}

A major part of human nature is the curiosity of people. Curiosity leads to inquiry, and inqooint to increased understanding. “The more you learn, the more acutely aware you become of your ignorance.”

How do we know about the world in which we live – or reality, for that matter?  Where does our knowledge about it come from?  The attempt to answer these questions lead to epistemology, the branch of philosophy dealing with the origin, scope, and validity of human knowledge.

In the epistemological debate, there are two archetypal and actually diametrically opposed concepts: empiricism and rationalism. Empiricism claims that sensory experience (observation) is man’s main (or even sole) source of knowledge, which rationalism claims that his knowledge stems from human reason.

Hardly anyone would deny that there is knowledge that comes to us from sensory experience. Take, for instance, the knowledge that water freezes at zero degrees Celsius. It actually takes observation(s) to acquire such knowledge.

However, in the field of science, which formulates knowledge that applies universally, irrespective of time and place, rationalism holds that empirical knowledge gained through sensory experience doesn’t have the same validity as knowledge deduced from reasoning.
    
\subsection{Human Nature to Inquire}\index{Human Nature to Inquire}

The branches of contemporary science associated with the study of human nature include anthropology, sociology, sociobiology, and psychology, particularly evolutionary psychology, which studies sexual selection in human evolution, as well as developmental psychology. The ``nature versus nurture'' debate is a broadly inclusive and well-known instance of discussion about human nature in the natural science.

This common phrase refers to the distinguishing characteristics – including ways of thinking, feeling, and acting – which humans tend to have naturally, independently of the influence of culture. The questions of what these characteristics are, how fixed they are, and what causes them are amongst the oldest and most important questions in philosophy and science. These questions have particularly important implications in ethics, politics, and theology. This is partly because human nature can be regarded as both a source of norms of conduct of ways of life, as well as presenting obstacles or constraints on living a good life. The complex implications of such questions are also dealt with in art and literature, the question of what it is to be human.

The concept of nature as a standard by which to make judgments was a basic presupposition in Greek philosophy. Specifically, ``almost all'' classical philosophers accepted that a good human life is a life in accordance with nature. (Notions and concepts of human nature from China, Japan, or India are not taken up in the present discussion.)

On this subject, the approach of Aristotle – sometimes considered to be a teleological approach – came to be dominant by late classical and medieval times. This approach understands human nature in terms of final and formal causes. In other words, nature itself (or a nature-creating divinity) has intentions and goals, similar somehow to human intentions and goals, and one of those goals is humanity living naturally. Such understandings of human nature see this nature as an ``idea'', or ``form'' of a human. By this account, human nature really causes humans to become what they become, and so it exists somehow independently of individual humans. This in turn has sometimes been understood as also showing a special connection between human nature and divinity.

However, the existence of this invariable human nature is a subject of much historical debate, continuing into modern times. Against this idea of a fixed human nature, the relative malleability of man has been argued especially strongly in recent centuries – firstly by early modernists such as Thomas Hobbes and Jean-Jacques Rousseau. In Rousseau’s Emile, or On Education, Rousseau wrote: “We do not know what our nature permits us to be.”  Since the early 19th century, thinkers such as Hegel, Marx, Kierkegaard, Nietzsche, Sartre, structuralists, and postmodernists have also sometimes argued against a fixed or innate human nature.

Charles Darwin’s theory of evolution has changed the nature of the discussion, confirming the fact that mankind’s ancestors were not like mankind today. Still more recent scientific perspectives – such as behaviorism, determinism, and the chemical model within modern psychiatry and psychological – claim to be neutral regarding human nature. (As in much of modern science, such disciplines seek to explain with little or no recourse to metaphysical causation.)  They can be offered to explain human nature’s origins and underlying mechanisms, or to demonstrate capacities for change and diversity which would arguably violate the concept of a fixed human nature.

\subsection{How Do We Know?}\index{How Do We Know?}

The notion of ``letting the facts speak for themselves'' without taking recourse to a theory is nonsensical. Mises was aware the people’s ``reasoning may be faulty and the theory incorrect; but thinking and theorizing are not lacking in any action.''

How do we know, and how can we make sure, that we employ a correct theory?  Fortunately, in social science a satisfactory answer can be given to these questions by taking recourse to a priori theory – meaning propositions that provide true knowledge about reality, and whose truth value can be validated independent of experience.

To explain, we have to turn briefly to the Prussian philosopher Immanuel Kant (1724-1804) and his groundbreaking The Critique of Pure Reason (1781). A central outcome of what Kant called transcendental investigation in his discovery of so-called a priori synthetic judgements.
    
\subsection{\textit{A Priori} Theory}\index{\textit{A Priori} Theory}

A priori denotes a proposition (a declarative statement) expressing knowledge that is acquired prior to, or independently from, experience. In contrast, a posteriori denotes knowledge that is acquired through and on the basis of experience.

A synthetic judgment refers to knowledge that is not contained in the subject matter. An example is ``All bodies are heavy.''  Here, the predicate ``heavy'' conveys knowledge that goes beyond the mere concept of ``body'' in general. A synthetic judgment thus yields new knowledge about the subject matter.

Analytical judgments repeat what the concept of the subject matter already presupposes. An example is ``All bodies are extended.''  In order to know that bodies are extended one does not need experience, as this information is already in the concept of ``bodies.''

One would expect that analytical judgements are a priori, while synthetic judgements are posteriori. However, Kant claims that there exist a priori synthetic judgements – knowledge that neither merely repeats the meaning of the concept under review nor requires experience to say something new about the subject matter.

How can a priori synthetic judgements be identified?  According to Kant, a proposition must meet two requirements in order to qualify as an a priori synthetic judgement. First, it must not result from experience, but from reasoning. Second, it cannot be denied without causing an intellectual contradiction.

A priori theory offers an approach for reviewing, criticizing, and possibly revising commonly held theoretical explanations of historical events. When (re)viewed from point of a priori theory, what can be said about the two independent observations?

A priori theory provides true knowledge about the outer world, and the truth of knowledge derived from a priori theory can be validated independent of sensory experience.

By no means less important, a priori knowledge trumps empirical knowledge: ``A proposition of an aprioristic theory can never be refuted by experience.''

Praxeology, the a priori science of human action, and, more specifically, it’s up to now bed-developed part, economics, provides in its field a consummate interpretation of past events recorded and a consummate anticipation of the effects to be expected from future actions of a definite kind.

An a priori theorist can thus decide in advance (that is, without engaging in social experimentation, or testing, for that matter) whether or not a given action – policy measure – can bring about the promised effects.

For instance, we know a priori that issuing flat money does not create economic prosperity, that tax- or debt-financed government spending does not improve society’s material well-being, and that these measures are actually economically harmful.

A priori theory is an intellectually powerful defense against promises made by false theory and its detrimental (even disastrous) economic consequences if put into practice. Students of social sciences should therefore be increasingly encouraged to engage in a priori theory.

In his seminal book Systems Thinking, Systems Practice, Peter Checkland defined systems thinking as thinking about the world through the concept of “system.”  This involves thinking in terms of processes rather than structures, relationships rather than components, interconnections rather than separation. The focus of the inquiry is on the organization and the dynamics generated by the complex interaction of systems embedded in other systems and composed by other systems.

%------------------------------------------------

\section{Scientific Thinking in Antiquity}\index{Scientific Thinking in Antiquity}

The application of knowledge gained from the scientific method concerns prediction of the future. If X happens then Y will happen. That is the utility. Y may include probability functions.

\subsection{Classic Greek Philosophy}\index{Classic Greek Philosophy}

Philosophy in classical Greece is the ultimate origin of the western conception of the nature of a thing. According to Aristotle, the philosophical study of human nature itself originated with Socrates, who turned philosophy from study of the heavens to study of the human things.

Socrates is said to have studied the question of how a person should best live, but he left no written works. It is clear from the works of his students Plato and Xenophon, and also by what was said about him by Aristotle (Plato’s student), that Socrates was a rationalist and believed that the best life and the life most suited to human nature involved reasoning. The Socratic school was the dominant surviving influence in philosophical discussion in the Middle Ages, amongst Islamic, Christian, and Jewish philosophers.

The human soul in the works of Plato and Aristotle has a divided nature, divided in a specifically human way. One part is specifically human and rational, and divided into a part which is rational on its own, and a spirited part which can understand reason. Other parts of the soul are home to desires or passions similar to those found in animals. In both Aristotle and Plato, spiritedness (thumos) is distinguished from the other passions (epithumiai). The proper function of the ``rational'' was to rule the other parts of the soul, helped by spiritedness. By this account, using one’s reason is the best way to live, and philosophers are the highest types of human.

Aristotle – Plato’s most famous student – made some of the most famous and influential statements about human nature. In his works, apart from using a similar scheme of a divided human soul, some clear statements about human nature are made:
\begin{enumerate}
\item Man is a conjugal animal, meaning an animal which is born to couple when an adult, thus building a household (oikos) and, in more successful cases, a clan or small village still run upon patriarchal lines.
\item Man is a political animal, meaning an animal with an innate propensity to develop more complex communities the size of a city or town, with a division of labor and law-making. This type of community is different in a kind from a large family and requires the special use of human reason.
\item Man is mimetic animal. Man loves to use his imagination (and not only to make laws and run town councils). He says ``we enjoy looking at accurate likenesses of things which are themselves painful to see, obscene beasts, for instance, and corpses.''  And the ``reason why we enjoy seeing likenesses is that, as we look, we learn and infer what each is, for instance, ‘that is so and so.’''
\end{enumerate}
For Aristotle, reason is not only what is most special about humanity compared to other animals, but it is also what we were meant to achieve at our best. Much of Aristotle’s description of human nature is still influential today. However, the particular teleological idea that humans are ``meant'' or intended to be something has become much less popular in modern times.

For the Socratics, human nature, and all natures, are metaphysical concepts. Aristotle developed the standard presentation of this approach with his theory of four causes. Every living thing exhibits four aspects or ``causes'': matter, form, effect, and end. For example, an oak tree is made of plant cells (matter), grew from an acorn (effect), exhibits the nature of oak trees (form), and grows into a fully mature oak tree (end). Human nature is an example of a formal cause, according to Aristotle. Likewise, to become a fully actualized human being (including fully actualizing the mind) is our end. Aristotle (Nicomachean Ethics, Book X) suggests that the human intellect is ``smallest in bulk'' but the most significant part of the human psyche, and should be cultivated above all else. The cultivation of learning and intellectual growth of the philosopher, which is thereby also the happiest and least painful life.

Although this new realism applied to the study of human life from the beginning – for example, in Machiavelli’s works – the definitive argument for the final rejection of Aristotle was associated especially with Francis Bacon. Bacon sometimes wrote as if he accepted the traditional four causes (``It is a correct position that ``true knowledge is knowledge by causes.'' And causes again are not improperly distributed into kinds: the material, the formal, the efficient, and the final.'') but he adapted these terms and rejected one of the three:

``But of these the final cause rather corrupts than advances the sciences, except such as have to do with human action. The discovery of the formal is despaired of. The efficient and the material (as they are investigated and received, that is, as remote causes, without reference to the latent process leading to the form) are but slight and superficial, and contribute little, if anything, to true and active science.''

This line of thinking continued with Rene Descartes, whose new approach returned philosophy or science to its pre-Socratic focus upon non-human things. Thomas Hobbes, then Grammatist Vico, and David Hume all claimed to be the first to properly use a modern Baconian scientific approach to human things.

Hobbes famously followed Descartes in describing humanity as matter in motion just like machines. He also very influentially described man’s natural state (without science and artifice) as one where life would be “solitary, poor, nasty, brutish, and short.”  Following him, John Locke’s philosophy of empiricism also saw human nature as a tabula rasa. In this view, the mind is at birth a “blank slate” without rules, so data are added, and rules for processing them are formed solely by our sensory experiences.

Jean-Jacques Rousseau pushed the approach of Hobbes to an extreme and criticized it at the same time. He was a contemporary and acquaintance of Hume, writing before the French Revolution and long before Darwin and Freud. He shocked Western civilization with his Second Discourse by proposing that humans had once been solitary animals, without reason or language or communities, and had developed these things due to accidents of pre-history. (This proposal was also less famously made by Giambattista Vico.)  In other words, Rousseau argued that human nature was not only not fixed, but not even approximately fixed compared to what had been assumed before him. Humans are political, and rational, and have language now, but originally, they had none of these things. This in turn implied that living under the management of human reason might not be a happy way to live at all, and perhaps there is no ideal way to live. Rousseau is also unusual in the extent to which he took the approach of Hobbes, asserting that primitive humans were not even naturally social. A civilized human is therefore not only imbalanced and unhappy because of the mismatch between civilized life and human nature, but unlike Hobbes, Rousseau also became well known for the suggestion that primitive humans had been happier, “noble savages.”

Rousseau’s conception of human nature has been seen as the origin of many intellectual and political developments of the 19th and 20th centuries. He was an important influence upon Kant, Hegel, and Marx, and the development of German idealism, historicism, and romanticism.

What human nature did entail, according to Rousseau and the other modernists of the 17th and 18th centuries, were animal-like passions that led humanity to develop language and reasoning, and more complex communities (or communities of any kind, according to Rousseau).

In contrast to Rousseau, David Hume was a critic of the oversimplifying and systematic approach of Hobbes, Rousseau, and some others whereby, for example, all human nature is assumed to be driven by variations of selfishness. Influenced by Hutcheson and Shaftesbury, he argued against oversimplification. On the one hand, he accepted that, for many political and economic subjects, people could be assumed to be driven by such simple selfishness, and he also wrote of some of the more social aspects of ``human nature'' as something which could be destroyed, for example if people did not associate in just societies. On the other hand, he rejected what he called the ``paradox of the sceptics'', saying that no politician could have invented words like \textit{honorable}, \textit{shameful}, \textit{lovely}, \textit{odious}, \textit{noble}, and \textit{despicable}, unless there was not some natural ``original constitution of the mind.''

Hume – like Rousseau – was controversial in his own time for his modernist approach, following the example of Bacon and Hobbes, of avoiding consideration of metaphysical explanations for any type of cause and effect. He was accused to being an atheist. He wrote:

``We needn’t push our researches so far as to ask `Why do we have humanity, i.e. a fellow-feeling with others?'  It’s enough that we experience this as a force in human nature. Our examination of causes must stop somewhere.''

After Rousseau and Hume, the nature of philosophy and science changed, branching into different disciplines and approaches, and the study of human nature changed accordingly. Rousseau’s proposal that human nature is malleable became a major influence upon international revolutionary movements of various kinds, while Hume’s approach has been more typical in Anglo-Saxon countries, including the United States.

\subsection{What Man Has Built}\index{What Man Has Built}

Humanity (???) Twentieth Century, more than our ancestors, must attempt to understand the varied peoples with whom he shares an increasingly small planet. To reach this understanding he needs to know the cultures which molded other people’s outlook, the history that carried them to this point.

How to select the civilizations that must be examined in a limited series of books on the history of the world’s cultures?  That is the subject of Jaques Barzun’s introduction to the Time-Life series entitled The Great Ages of Man. Mr. Barzun, Dean of Faculties and Provost of Columbia University, is one of the pre-eminent cultural historians of this generation. He describes how the “revolution … in our conception of humanity” wrought by the emergence of “dozens of new peoples, new states, and new pasts” has made essential the realization that “nothing human is alien.”

In explaining the criteria for the selection of historic cultures examined in this series, he also suggests the path that present-day cultures may follow in the future.

At the end of this introductory booklet is a comprehensive chronological chart. This shows the meaningful relationships of great cultures the world over – in time, in place, and in the interpenetrations discussed by Dean Barzum. A number of these cultures provide the subject matter for volumes in this series. This overall chart will be found useful in connection with each book; in addition, each book includes an appropriate segment from the chart. THE EDITORS OF TIME-LIFE BOOKS (1965 Time Inc.)

%------------------------------------------------

\section{Contemporary Scientific Thinking}\index{Contemporary Scientific Thinking}

The word ``science'' is derived from the Latin word Scientia, which is knowledge based on demonstrable and reproducible data, according to the Merriam-Webster Dictionary. True to this definition, science aims for measurable results through testing and analysis. Science is based on fact, not opinion or preferences. The process of science is designed to challenge ideas through research. One important aspect of the scientific process is that it focuses only on the natural world, according to the University of California. Anything that is based on faith alone or is considered supernatural does not fit into the definition of science.

\subsection{Deductive and Inductive Reasoning}\index{Deductive and Inductive Reasoning}

During the scientific process, deductive reasoning is used to reach a logical true conclusion. Another type of reasoning, inductive, is also used. Often, deductive reasoning and inductive reasoning are confused. It is important to learn the meaning of each type of reasoning so that proper logic can be identified.

Deductive Reasoning. is a basic form of valid reasoning. Deductive reasoning, or deduction, starts out with a general statement, or hypothesis, and examines the possibilities to reach a specific, logical conclusion, according to the University of California. The scientific method uses deduction to test hypothesis and theories. ``In deductive interference, we hold a theory and based on it we make a prediction of its consequences. That is, we predict what the observations should be if the theory were correct. We go from the general – the theory – to the specific – the observations,'' said Dr. Sylvia Wassertheil-Smoller, a researcher and professor emerita at Albert Einstein College of Medicine.

In deductive reasoning, if something is true of a class of things in general, it is also true for all members of that class. For example, ``All men are mortal. Harold is a man. Therefore, Harold is mortal.''  For deductive reasoning to be sound, the hypothesis must be correct. It is assumed that the premises, ``All men are mortal'' and ``Harold is a man'' are true. Therefore, the conclusion is logical and true.

According to the University of California, deductive inference conclusions are certain provided the premises are true. It’s possible to come to a logical conclusion even if the generalization is not true. If the generalization is wrong, the conclusion may be logical, but it may also be untrue. For example, the argument, ``All bald men are grandfathers. Harold is bald. Therefore, Harold is a grandfather,'' is valid logically but it is untrue because the original statement is false.

A common form of deductive reasoning is the syllogism, in which two statements – a major premise and a minor premise – reach a logical conclusion. For example, the premise ``Every A is B'' could be followed by another premise, ``This C is A.''  Those statements would lead to the conclusion ``This C is B.'' Syllogisms are considered a good way to test deductive reasoning to make sure the argument is valid.

Inductive Rreasoning. is the opposite of deductive reasoning. Inductive reasoning makes broad generalizations from specific observations. ``In inductive inference, we go from the specific to the general. We make many observations, discern a pattern, make a generalization, and infer an explanation or a theory.'' Wassertheil-Smoller told Live Science. ``In science there is a constant interplay between inductive inference (based on observations) and deductive inference (based on theory), until we get closer and closer to the `truth', which we can only approach but not ascertain with complete certainty.''

Even if all of the premises are true in a statement, inductive reasoning allows for the conclusion to be false. Here’s an example: ``Harold is a grandfather. Harold is bald. Therefore all grandfathers are bald.''  The conclusion does not follow logically from the statements.

Inductive reasoning has its place in the scientific method. Scientists use it to form hypothesis and theories. Deductive reasoning allows them to apply the theories to specific situations.

Abductive Reasoning. Another form of scientific reasoning that doesn’t fit in with inductive or deductive reasoning is abductive. Abductive reasoning usually starts with an incomplete set of observations and proceeds to the likeliest possible explanation for the group of observations, according to Butte College. It often entails making an educated guess after observing a phenomenon for which there is no clear explanation.

Abductive reasoning is useful for forming hypotheses to be tested. Abductive reasoning is often used by doctors who make a diagnosis based on test results and by jurors who made decisions based on the evidence presented to them.

Systems Science. Systems sciences are scientific disciplines partly based on systems thinking such as chaos theory, complex systems, control theory, cybernetics, sociotechnical systems theory, systems biology, systems ecology, systems psychology and the already mentioned systems dynamics, systems engineering, and systems theory.

Systems being involves embodying a new consciousness, and expanded sense of self, a recognition that we cannot survive alone, that a future that works for humanity needs also to work for other species and the planet. It involves empathy and love for the greater human family and for all our relationships - plants and animals, earth and sky, ancestors and descendants, and the many peoples and beings that inhabit our Earth. This is the wisdom of many indigenous cultures around the world, this is part of the heritage that we have forgotten and we are in the process of recovering.

Systems being and systems living brings it all together: linking head, heart and hands. The expression of systems being is an integration of our full human capacities. It involves rationality with reverence to the mystery of life, listening beyond words, sensing with our whole being, and expressing our authentic self in every moment of our life. The journey from systems thinking to systems being is a transformative learning process of expansion of consciousness - from awareness to embodiment.

Kathia Laslo, Ph.D., directs Saybrook University’s program in Leadership of Sustainable Ssytems.
NOTE: This post is an excerpt from the plenary presentation “Beyond Systems Thinking: The role of beauty and love in the transformation of our world” by Dr. Karla Lazslo at the 55th Meeting of the International Society for the Systems Sciences at the University of Hull, U.K., on July 21, 2014.

Systems thinking is the process of understanding how things influence one another within a whole. In nature, systems thinking examples include ecosystems in which various elements such as air, water, movement, plants, and animals work together to survive or perish. In organizations, systems consist of people, structures, and processes that work together to make an organization healthy or unhealthy.

Systems thinking has been defined as an approach to problem solving, by viewing “problems” as parts of an overall system, rather than reacting to specific part, outcomes or events and potentially contributing to further development of unintended consequences. Systems thinking is not one thing but a set of habits and practices within a framework that is based on the belief that the component parts of a system can best be understood in the context of relationships with each other and with other systems, rather than in isolation. Systems thinking focuses on cyclical rather than linear cause and effect.

In science systems, it is argued that the only way to fully understand why a problem or element occurs and persists is to understand the parts in relation to the whole. Standing in contrast to Descartes’s scientific reductionism and philosophical analysis, it proposes to view systems in a holistic manner. Consistent with systems philosophy, systems thinking concerns an understanding of a system by examining the linkages and interactions between the elements that compose the entirety of the system.

Science systems thinking attempts to illustrate that events are separated by distance and time and that small catalytic events can cause large changes in complex systems. Acknowledging that an improvement in one area of a system can adversely affect another area of the system, it promotes organizational communication at all levels in order to avoid the silo effect. Systems thinking techniques may be used to study any kind of system – natural, scientific, engineered, human, or conceptual.

\subsection{The Scientific Method}\index{The Scientific Method}

The prevailing scientific thinking in Western cultures today is naturalism. It is assumed that there is no God and that the entire universe can be explained on the basis of physical realities plus time and chance. It is assumed that the laws of physics have never changed and that conditions have been uniform in the past so that recent observations can be compiled and conclusions drawn about the past by simply looking back in time. The scientific method is mostly limited to the study of measurable entities in the physical world. Results should be verifiable by others.

Figure 2.1 Here (New)

\begin{enumerate}
\item Basic Assumptions include underlying philosophy, for example is there an outside intelligence operating, or is the system closed depending only on internal known laws. Are the laws of nature constant everywhere?  Were conditions in the past the same as they are now?  What initial conditions are assumed?  Science does not and cannot take place in a vacuum – an underlying world-view or philosophy is presupposed.
\item The hypothesis is a proposed explanation for an observed phenomenon. The simpler the explanation that fits the facts, the better, known as Occam’s razor. Science assumes we live in a universe. That is, the laws of physics are the same everywhere and, furthermore, they do not vary erratically – nature is predictable, and the universe is rational.
\item If the hypothesis does not explain the known facts or the new data, it is important to carefully examine the initial conditions to if one or more of them is incorrect or suspect. Wrong assumptions a long time ago that have not been challenged cause the weight of tradition to prevail – until there are overwhelming reasons for changing the prevailing scientific paradigm.
‘Science is the only self-correcting human institution, but it is also a process that progresses only by showing itself to be wrong.’ – Alan Sandage
\item Data inputs include measurements and observations. These are systematized and subjected to statistical scrutiny whenever possible.
\item When a theory has been found that seems to the known facts, the theory is then extended into the unknown to make predictions. These predictions are then tested by seeking additional data, exceptions, or confirmations.
\item A new scientific theory or model remains in vogue until new facts are found that contradict the model or whenever a better theory comes along.
\end{enumerate}

When conducting research, scientists use the scientific method to collect measurable, empirical evidence in an experiment related to a hypothesis (often in the form of an if/then statement), the results aiming to support or contradict a theory.
The steps of the scientific method must include: 
Make an observation or observations.
Ask questions about the observations and gather information.

\begin{enumerate}
\item Form a hypothesis – a tentative description of what’s been observed, and make predictions based on that hypothesis.
\item Test the hypothesis and predictions in an experiment that can be reproduced.
\item Analyze the data and draw conclusions; accept or reject the hypothesis or modify the hypothesis if necessary.
\end{enumerate}

Reproduce the experiment until there are no discrepancies between observations and theory. ``Replication of methods and results is an essential step in the scientific method.''

Some key underpinnings to the scientific method are:
\begin{itemize}
\item The hypothesis must be testable and falsifiable, according to North Carolina State University. Falsifiable means that there must be a possible negative answer to the hypothesis.
\item Research must involve deductive reasoning and inductive reasoning. Deductive reasoning is the process of using true premises to reach a logical true conclusion while inductive reasoning takes the opposite approach.
\item An experiment should include a dependent variable (which does not change) and an independent variable (which does change).
\item An experiment should include an experimental group and a control group. The control group is what the experimental group is compared against.
\end{itemize}

Gaining new insights into the nature of systems, Hillary Sillitto, 

Several INCOSE members participated in the IFSR Conversation held April 2018 in Linz, Austria. This article is a report on some interesting results coming out of this activity. INCOSE joined the International Federation for Systems Research in 2012, interfacing through the Systems Science Working Group, and INCOSE members have participated in each Conversation since the year we joined.

One of the major activities of the International Federation for Systems Research (IFSR) is the ``conversation'' held every two years in Linz, Austria, where several teams of typically 6-8 people spend a week discussing different current issues in systems research. The format is a ``conversation'' or ``systemic inquiry'' rather than a conference, and the teams spend most of the time in their own group, exploring their specific topic and attempting to achieve new insights by integrating the different perspectives and worldviews of the different team members. 

Over forty organizations are currently IFSR members. Some are more active than others, and the Conversation this year involved people from INCOSE, ISSS, ASC (American Society of Cybernetics), the IFSR itself, and one representative from the System Dynamics Society (SDS).

The 2018 Conversation addressed four topics: ``Systems Practice'', led by members and associates of Malik Management, focused on challenges set by senior-level input from the Government of Vietnam; ``What is Systems Science?'', led by Gary Smith of INCOSE with a team of INCOSE and IFSR members; ``Active and Healthy Aging'', using Beer’s Viable System Model and a subset of Len Troncale’s System Processes as reference models to understand the challenges facing older members of our communities; and ``Data Driven SE Approaches'', led by Ed Carroll of INCOSE and Sandia Labs, with several INCOSE members, the SDS representative, and several others from Sandia.

Ed Carroll’s team considered the problem of integrating the heterogeneous model types used by different engineering domains, discussed issues such as how to get people to trust models, identified the need for Systems Engineering to shift from a process-centric to an information-centric perspective if MBSE is to succeed, and were inspired by the ``agile manifesto'' to start working up an analogous ``MBSE manifesto''. Their outbrief advocated viewing the model as being the focus, rather than the process of creating it. Others pointed out the tension between this perspective and the verified success of ``shared model building'' as a method for engaging stakeholders, and developing their trust in the model. Someone suggested that ``no-one understands a model except the people who created it''. I look forward to seeing the MBSE manifesto and to the discussions it will undoubtedly provoke about the culture change required in the SE community to take full advantage of the model driven approach while being fully aware of its limitations: not all systems are deterministic, some systems ``have a mind of their own''; and for these, modeling can indicate the range of possible future trajectories but not the precise one that will be followed.

Gary Smith’s team discussed ``What is systems science''. Gary smith made a plausible argument that in historical terms, Systems Science is now where chemistry was before the Periodic Table of the elements - lots of phenomena have been described, many of them understood as individual phenomena, but this knowledge is not yet integrated around a single foundational structure. Further, the current systems science literature in most cases does not clearly distinguish between fundamental ingredients of all systems (think electrons, protons and neutrons), properties of all systems (think properties of atoms and elements due to the electron orbitals) and properties that can be synthesised with combinations of different ``elemental types'' of system – think compounds, crystals, alloys, etc. Most Systems Science literature also does not clearly distinguish between ``how people perceive and interact with systems'', and fundamental ``properties of systems in the natural world''. (Robert Rosen’s book Anticipatory Systems is a notable exception.) We spent the week exploring whether existing systems science knowledge could usefully be organised in this sort of structure, and concluded that it could, and that such a structure offers promise in terms of integrating the seven different worldviews on system we have identified within the INCOSE community. Also, we identified an eighth worldview about systems, that ``systemness'' might be a fundamental organising principle of nature. Our output and subsequent reflections are being posted to a website which will progressively be opened up to SSWG members and then more widely as the content matures.

I participated in the “What is systems science?” team, and also represented INCOSE at the IFSR Board Meeting on the Friday afternoon at the end of the Conversation. The notable points of the Board Meeting were: our old friend Gerhard Chroust stands down as IFSR’s Secretary General after 27 years of service; Gary Metcalf, Jennifer Wilby and Mary Edson also finished their terms of service; new faces join the Board, and Ray Ison takes over from Mary as president; George Mobus has taken over as general Editor of the IFSR book series; and the System Dynamics Society’s membership application was approved. Hillary Sillitto, ESEP, INCOSE Fellow

%------------------------------------------------

\section{Science and Systems Science}\index{Science and Systems Science}

The significant accumulation of scientific knowledge, which began in the eighteenth century and rapidly expanded in the twentieth, made it necessary to classify what was discovered into scientific disciplines. Science began its separation from philosophy almost two centuries ago. It then proliferated into more than 100 distinct disciplines. A relatively recent unifying development is the idea that systems have general characteristics, independent of the area of science to which they belong. In this section, the evolution of a science of systems is presented through an examination of cybernetics, general systems theory, and systemology.

\subsection{General Systems Theory}\index{General Systems Theory}

An even broader unifying concept than cybernetic evolved during the late 1940’s. It was the idea that basic principle common to all systems could be found that go beyond the concept of control and self-regulation. A unifying principle for science and a common ground for interdisciplinary relationships needed in the study of complex systems were being sought. Ludwig von Bertalanffy used the phrase general systems theory around 1950 to describe this endeavor. A related contribution was made by Kenneth Boulding.

General systems theory is concerned with developing a systematic framework for describing general relationships in the natural and the human-made world. The need for a general theory of systems arises out of the problem of communication among various disciplines. Although the scientific method brings similarity between the methods of approach, the results are often difficult to communicate across disciplinary boundaries. Concepts and hypotheses formulated in one area seldom carry over to another, where they could lead to significant forward progress.

One approach to an orderly framework is the structuring of a hierarchy of levels of complexity for individual systems studied in the various fields of inquiry. A hierarchy of levels can lead to a systematic approach to systems that has broad application. Boulding suggested such a hierarchy. It begins with the simplest level and proceeds to increasingly complex levels that usually incorporate the capabilities of all the previous levels, summarized approximately as follows:

\begin{enumerate}
\item The level of static structure or frameworks, ranging from the pattern of the atom to the anatomy of an animal to a map of the earth to the geography of the universe.
\item The level of the simple dynamic system, or clockworks, adding predetermined, necessary motions, such as the pulley, the steam engine, and the solar system.
\item The level of the thermostat or cybernetic system, adding the transmission and interpretation of information.
\item The level of the cell, the open system where life begins to be evident, adding self-maintenance of structure in the midst of a through put of material.
\item The level of the plant, adding a genetic-societal structure with differentiated and mutually dependent parts, “blueprinted” growth, and primitive information receptors.
\item The level of the animal, adding mobility, teleological behavior, and self-awareness using specialized information receptors, a nervous system, and a brain with a knowledge structure.
\item The level of the human, adding self-consciousness, the ability to produce, absorb, and interpret symbols; and understanding of time, relationship, and history.
\item The level of social organization, adding roles, communication channels, the content and meaning of messages, value systems, transcription of image into historical record, art, music, poetry, and complex human emotion.
\item The level of the transcendental system, adding the ultimates and absolutes and unknowables.
\end{enumerate}

The first level in Boulding’s hierarchy is the most pervasive. Static systems are everywhere, and this category provides a basis for analysis and synthesis of systems at higher levels. Dynamic systems with predetermined outcomes are predominant in the natural sciences. At higher levels, cybernetic models are available, mostly in closed-loop form. Open systems are currently receiving scientific attention, but modeling difficulties arise regarding their self-regulating properties. Beyond this level, there is little systematic knowledge available. However, general systems theory provides science with a useful framework within which each specialized discipline may contribute. It allows scientists to compare concepts and similar findings, with its greatest benefit being that of communication across disciplines.

\subsection{Systemology and Synthesis}\index{Systemology and Synthesis}

The science of systems or their information is called systemology. Problems and problem complexes faced by humankind are not organized along disciplinary lines. A new organization of scientific and professional effort based on the common attributes and characteristics of problems will likely accelerate beneficial progress. As systems science is promulgated by the formation and acceptance of interdisciplines, humankind will benefit from systemology and systems thinking.

Disciplines in science and the humanities developed largely by what society permitted scientists and humanists to investigate. Areas that provided the least challenge to cultural, social, and moral beliefs were given priority. The survival of science was also of concern in the progress of certain disciplines. However, recent developments have added to the acceptance of a scientific approach in most areas. Much credit for this can be given to the recent respectability of interdisciplinary inquiry. One of the most important contributions of systemology is that it offers a single vocabulary and a unified set of concepts applicable to many types of systems.

During the 1940s, scientists of established reputation in their respective fields accepted the challenge of attempting to understand a number of common processes in military operations. Their team effort was called operations research, and the focus of their attention was the optimization of operational military systems. After the war, this interdisciplinary area began to take on the attributes of a discipline and a profession. Today a body of systematic knowledge exists for both military and commercial operations. But operations research is not the only science of systems available today. Cybernetics, general systems research, organizational and policy sciences, management science, and the information sciences are others.

Formation of interdisciplines began in the middle of the last century and has brough about an evolutionary synthesis of knowledge. This has occurred not only within science but also between science and technology and between science and the humanities. The forward progress of systemology in the study of large-scale complex systems requires a synthesis of science and the humanities as well as a synthesis of science and technology. Synthesis, sometimes referred to as an interdisciplinary discipline, is the central activity of people often considered to be synthesists.

The science community must build public understanding of and appreciation for science and evidence-based thinking. We must show that evidence-based thinking leads to more reliable policies to create jobs, maintain a healthy environment and improve teaching.

With more than 120,000 members across the world, the American Association for the Advancement of Science is uniquely positioned to bring together voices and ideas from a diverse range of disciplines and backgrounds to lead the charge. But we need your help.

With your support, we will expand our efforts to speak up and draw people — the public and policymakers — to the idea that science is relevant to their lives and can inform their decisions. We will provide training, tools, and resources for scientists to communicate their research and find opportunities to connect with loc

\subsection{Emergence}\index{Emergence}

\subsection{Ethical Principles From the Basics of Science}\index{Ethical Principles From the Basics of Science}

The principles that form the basis of every rational discussion, that is, of every discussion in the search for truth, are in the main ethical principles. I should like to state three such principles:

\begin{enumerate}
\item The principle of fallibility: perhaps I am wrong and perhaps you are right. But we could easily both be wrong.
\item The principle of rational discussion: we want to try, as impersonally as possible, to weight up our reasons for and against a theory; a theory that is definite and criticizable.
\item The principle of approximation to the truth: we can nearly always come closer to the truth in a discussion which avoids personal attacks. It can help us to achieve a better understanding; even in those cases where we do not reach an agreement.
\end{enumerate}

It is worth noting that these three principles are both epistemological and ethical principles. For they imply, among other things, toleration: if I hope to learn from you, and I want to learn in the interest of truth, then I have not only to tolerate you but also to recognize you as a potential equal; the potential unity and willingness to discuss matters rationally. Of importance also is the principle that we can learn much from a discussion, even when it does not lead to agreement: a discussion can help us by shedding light upon some of our errors.

You raise an interesting question, one which has caused argument and confusion over many years. Considering the advanced state of systems engineering, systems thinking, etc., this state of affairs might be considered curious, at the very least.

The problem with defining “systems” is that various disciplines perceive “system” from their own perspectives, and may not always define “system” in an entirely general way. For instance, engineers today often describe complicated artefacts as systems, whereas 20-30 years ago they described them as ‘equipments.’  The artefacts haven’t changed, but somehow the soubriquet ‘system’ has taken over, whether appropriate or not.

Aerospace engineers might describe an aircraft as a system, or even a system of systems, while others might describe the aircraft as a sophisticated artefact, a tool, for the use of a pilot and crew. Together, aircraft and crew comprise a sociotechnical system. Why?  Because only when together, man and machine, is the flying machine complete. Neither man nor machine can fly without the other.

So, an important element of any definition of ‘system’ is the notion of completeness. A system is a complete something that, usually, performs some function(s). A gambling system is a means and method for winning at gambling, but it is only a system if all the moves, all the tactics, are in place. If only one is absent – no system.

Another aspect of ‘system’ in general is organization. By definition, a system is organized. It is not enough for a system to comprise many parts: these parts have to be interconnected, interacting and organized. A system’s “degree of organization” can be measured as entropy. The lower the entropy of a system, the more of its internal energy can be converted for useful external work. (Second Law of Thermodynamics)

Some systems are purposeful: humans as systems can be purposeful. Are all systems purposeful? Apparently not. The solar system is organized, low configuration entropy, etc. – we cannot avow that it is complete – yet it appears to have no explicit purpose, outside of the transcendental, that is. So, purpose need not form part of any definition … nor does the solar system perform any notable function: it just `is.’ So, is function essential?

Within systems there are observed to be levels of organization. For living things, the smallest element that can be described as living in the cell. All living things are made from cells. However, the cells are organized, grouped into tissues; tissues into organs; organs into organ systems; organ systems into organisms (species); species into populations; populations into communities, communities into ecosystem, ecosystems into biomes; biomes into the biosphere. So, an empiricist might say, with some justification, that the definition of “system” places it necessarily in such a hierarchy of organization.

Reductionists, OTOH, seek to explain high-level phenomena in more fundamental, low level explanations. Some argue that the relation between high and low – supervenience - may not allow for reduction. High level properties may be irreducible and may represent new, emergent properties that are more than the sum of their parts. The simple, limited behavior of individual ants results in the sophisticated organization of a colony.

So, another aspect of `system’ is that some systems may exhibit emergent properties, properties/behaviors of the whole that are not evident in the properties/behaviors of the parts. Some pundits seek to divide emergence into `weak’ and `strong,’ where weak is predictable, calculable, and not, therefore, entirely justifiable as emergence. Should emergence appear in the definition of `system?’  Possibly, if only because many natural systems do exhibit emergence (e.g., the Hymenoptera), and some (complex?) manmade systems also exhibit emergence.

Lastly, a system has be open, i.e., it exchanges energy, information and substance with other systems and with its environment. Were a system truly closed, we would be unaware of it.

%------------------------------------------------

\section{Scientific Theory and Laws}\index{Scientific Theory and Laws}

\subsection{What is Scientific Theory?}\index{What is Scientific Theory?}

Imagine for a moment that you are omniscient. Endowed with such knowledge, you would completely understand how the world ``works.''  You would completely understand how light works, how molecules and atoms work, how genetics work, how tectonic plates work, and how the universe came into existence. There would be nothing about the social or natural worlds that you would not understand in its entirety.

Where you endowed with such omniscience, you would have no use whatsoever for ``science.''  You would have no need to study the world in a patient and systematic way; because you would already possess all the knowledge about the world that ``science'' could ever hope to yield. Science would not only bore you to tears; it would appear to be an imperfect and dreadfully tedious means to arrive at the knowledge you already possess.

Unfortunately, however, no human being possesses omniscience. We are born into the world without knowledge about how light works, how tectonic plates work, how atom work, and how the universe came into existence. We also lack perfect knowledge about how capitalism and socialism work, how democracy and monarchy work, and how price controls work.

Our uncertainty about how the social and natural worlds work restricts our ability to act. Our uncertainty about how tectonic plates work restricts our ability to predict and control earthquakes. Our uncertainty about how light works restricts our ability to harness it for our own purposes. And our uncertainty about how monarchy and democracy work restricts our ability to construct political and economic systems that are best suited to our nature. This list could be extended ad infinitum.

We are not without means to overcome our uncertainty about how the world works, however. We are not, like the brute animals, doomed to struggle for our existence in a world that we will never understand or be able to harness for our own purposes. We have reason and memory at our disposal, which, with the aid of our senses, allow us to examine the world and learn how its elements ``work.'' These fantastic mental abilities afford us the means to investigate the world in the hope of overcoming at least a small part of our natural ignorance and uncertainty.

Our fantastic mental abilities do not, however, automatically yield to us infallible knowledge about how the world works. We can misinterpret what is going on, and we can reason unsoundly. Our senses can fail us, and our thinking can become clouded, biased, or myopic. In addition, the world we seek to understand is so fantastically large and complicated, and our time so very scarce, that each of us is severely limited in the amount of knowledge we can individually acquire about how the world works.

Hence, only by working with an learning from other men can we as individual hope to learn more than a tiny fraction about how the world works. By working with and learning from other men, we can take advantage of an intellectual division of labor that allows individuals to investigate very specific aspects of the world and then share the fruits of their investigations with the rest of humanity. This specialization and exchange of ideas allows men to economize their scarce time, learn more about the world than they otherwise could as isolated individual, and serves as a check on the fallible reasoning of each individual.

The concept of ``science'' in the Western world has been to connect a community of individuals who are committed to study the world in a specialized, systematic, and intersubjective verifiable way. Ideally, this scientific community accumulates knowledge about how the world works as individuals learn from the specialized investigations of their colleagues and build on them, and as the scientific community critiques and refines their theories through time.

The process by which the scientific community investigates the world is not a magic or automatic path to enlightenment or omniscience, however. The theories that are fashionable in the scientific community at any specific moment may or may not accurately describe the actual working of the world. Communities of individual scholars, like the individual scholars themselves, can fall victim to intellectual error. They can misinterpret what is going on, and they can reason unsoundly. Their senses can fail them, and their thinking can become clouded, biased, or myopic.

The critical and inexorable problem that the community of scholars faces, therefore, is knowing whether the theories it currently embraces accurately and completely describe the workings of the world. This uncertainty about the accuracy of their scientific theories stems, once again, from the fact that no member of the scientific community is omniscient. No member of the scientific community is in a position to say with certainty that any theory does or does not accurately describe the workings of the world.

If even one member of the scientific community were omniscient, it would be possible to appeal to that member as an objective assessor of scientific theories. In that case, the omniscient assessor would not trouble himself with describing the world using the clumsy word ``theory,'' however. He would say ``the world works thusly,'' or ``the world does not work thusly.'' If such a person existed, moreover, the practice of ``science'' would cease altogether, because certain knowledge about the world could be obtained from the omniscience person without the need to tediously and imperfectly study the world ``scientifically.''

Because the scientific community does not count omniscient members among its number, its members have developed a ``scientific method'' to try to deal with their uncertainty about their theories. The ``scientific method,'' which consists of developing hypotheses and ``testing'' those hypotheses against empirical experience, does not provide the scientific community with certain knowledge, however. It merely serves a rather low hurdle that assists in weeding out what most scientists would consider implausible, unverifiable, and silly theories.

A theory’s ability to clear this low hurdle by no means can be interpreted as ``verifying'' a theory, or ``proving'' its truth, however, because alternative theories could always be imagined that would also be consistent with the empirical ``facts.''  The scientific method does not provide the scientific community with a means to determine which theory, if any, out of the limitless set of alternative theories that could be dreamed up to explain the same empirical phenomena is ``correct.''  Nor does the scientific method provide the scientific community with a means to know for certain that its members are not misinterpreting the empirical evidence. Only an omniscient being could know these things for certain.

Because empirical evidence doesn’t not ``speak for itself,'' and because scientists are not omniscient (and thus cannot know if they are ``correctly'' interpreting empirical evidence), scientists can never know for certain if their theories correctly describe physical reality. This means that any theory that relies on the interpretation of empirical evidence can never be more than a subjective statement of belief about how a part of the world works, based on some empirical evidence.

This definition is unavoidable, because no scientist is in an omniscient position to know for certain whether he has interpreted empirical evidence correctly, or whether his theory is the ``correct'' one out of the infinite set of alternative theories that could be imagined to explain a given phenomenon.

This is not to say that scientific theories rely on the interpretation of empirical evidence are useless or meaningless, simply because they are subjective statements of belief. Nor does it imply that all empirically derived scientific theories are equally plausible, or that they all must be deemed “equal” in some other way, simply because they are all subjective statements of belief about how the world works. On the contrary, a theory that relies on empirical evidence is nothing more than an ``expert opinion'' about how a part of the world works, but it can nevertheless be useful - sometimes amazingly useful, in fact – even when it is known to be ``incorrect'' in some respects (e.g., Newtonian physics). Moreover, individual are free to evaluate the plausibility of scientific theories on their own, which means that they are free to accord some empirical theories more plausibility than others.

The fact that scientific theories are subjective statements of belief does mean, however, that the scientist who claims that his empirically derived theory is a ``fact'' or ``undeniably certain'' does not understand the limitations of his method. He is deluding himself – and anyone else who believes his claims – if he thinks he is able to ``prove'' his empirically derived theory to be ``irrefutably true.''  Only an omniscient being could possibly know for certain that a specific theory out of the infinite set of alternative theories that could be imagined to explain a given phenomenon is ``correct.''  But, again, an omniscient being would not bother with the clumsy and inefficient methods of science. He would merely say, ``the world works thusly,'' or ``the world does not work thusly.''  He certainly would not bother ``testing'' his ideas against empirical experience, because he would already know the outcome. Hence, the fact that the scientist bother to ``test'' his theories and hypotheses reveals his lack of omniscience, and it also reveals, a fortiori, his inability to know for certain whether he is interpreting empirical evidence ``correctly.''

In order to move beyond making subjective statements of belief about how parts of the world work the scientist would either need to become omniscient himself or consult someone who is omniscient, or else he would need to move beyond gathering and interpreting empirical evidence. Because the former options are, presumably, not open to him, the scientist’s only viable option is to discover “facts” about the world, or parts of the world, that cannot possibly be thought to be false, and which are not open to misinterpretation. In other words, the scientist would have to transform himself from an empiricist into a ``rationalist'' who was concerned to discover fundamental truths about the world (i.e., a priori truths about the world) and elucidate them by means of a deductive and rationalistic method. Only then would the scientist be in a position to say that he has found “facts” about parts of the world that are ``indisputably true.''

By dogmatically endorsing the ``scientific method'' as the only means to acquire knowledge about the world, the empirically minded scientist tacitly admits that it is possible to discover fundamental truths about the world without going out and ``testing'' them. For, the proposition ``all hypotheses and theories must be `tested’ against empirical experience'' purports to be objectively and universally true, yet the proposition itself has not and can never be ``tested.''  Therefore, the proposition is self-contradictory and thus false, a fact that establishes that it is indeed possible to discover irrefutable and demonstrable truths about the world without going out and testing them.

Thus, absolute certainty in science cannot be acquired by means of the “scientific method” and the collection and interpretation of empirical evidence. For beings that lack omniscience, collection and interpretation of empirical evidence can only yield imperfect and subjective beliefs about how the world “works.”  Instead, absolute certainly in science can only be acquired by discovering propositions about the world that can be known to be true a priori – propositions that cannot be thought to be false.

This observation, in a nutshell, forms the foundation and is the great strength of the Austrian School of economics, which stands virtually alone in the contemporary would as a bastion for thinkers who are unsatisfied with imperfect and subjective approaches to science.

System thinking is not enough. We must also engage in systems feeling and making. We must start during conceptual design and continue through the multiple years of integration and maintenance of operational systems.

SysML helps us make the evidence that system thinking has occurred. SysML must be augmented by ways of communicating intended system feeling, particularly for systems that include humans as active components. Example attributes might be engages, influences, inspires, learns, etc. as well as trusted, trustworthy, alert, etc.

While designing a system model we must ensure that these necessary and sufficient emotions and judgements are in play and evolving in the right direction(s).

An extension to SysML is not likely, rather associated stories and a set of key metrics may be appropriate. Azad Madni and others such as IDEO and other design laboratories have shown us examples of such stories. Tom Love champions performing conceptual design sans words, a flow of cartoons only.

How shall we evolve an ontology for the Feeling aspect of systems as an adjunct to SysML, particularly for highly autonomous systems that emulate human emotions? Jack.

\subsection{Acknowledging Scientific Risk}\index{Acknowledging Scientific Risk}

The knowledge that is built buy science is always open to question and revision. No scientific idea is ever once-and-for-all ``proved.''  Why not?  Well, science is constantly seeking new evidence, which could reveal problems with our current understandings. Ideas that we fully accept today may be rejected or modified considering new evidence discovered tomorrow. For example, up until 1938, paleontologists accepted the idea that coelacanths (an ancient fish) went extinct at the time that they last appear in the fossil record – about 80 million years ago. But that year, a live coelacanth was discovered off the coast of South Africa, causing scientists to revise their ideas and begin to investigate how this animal survives in the deep sea.

Even though they are subject to change, scientific ideas are reliable. The ideas that have gained scientific acceptance have done so because they are supported by many lines of evidence. These scientific explanations continually generate expectations that hold true, allowing us to figure out how entities in the natural world are likely to behave (e.g., how likely it is that a child will inherit a particular genetic disease) and how we can harness that understanding to solve problems (e.g., how electricity, wire, glass, and various compounds can be fashioned into a working light bulb). For example, scientific understandings of motion and gases allow us to build airplanes that reliably get us from one airport to the next. Though the knowledge used to design airplanes is technically provisional, time and time again, that knowledge has allowed us to produce airplanes that fly. We have good reason to trust scientific ideas: they work!

Systems science - systemology (Greco. Systema, logos) or systems theory is an interdisciplinary field that studies the nature of systems - from simple to complex - in nature, society, and science itself. The field aims to develop interdisciplinary foundations that are applicable in a variety of areas, such as engineering, biology, medicine, and social sciences.

Systems science covers formal sciences such as complex systems, cybernetics, dynamical systems theory, and systems theory, and applications in the field of the natural and social sciences and engineering, such a control theory, operations research, social systems theory, systems biology, system dynamics, human factors, systems ecology, systems engineering and systems psychology. Themes commonly stressed in system science are (a) holistic view, (b) interaction between a system and its embedding environment, and (c)complex (often subtle) trajectories of dynamic behavior that sometimes are stable (and thus reinforcing), while at various `boundary conditions’ can become wildly unstable (and thus destructive). Concerns about Earth-scale biosphere/geosphere dynamics is an example of the nature of problems in which systems science seeks to contribute meaningful insights.

Since the emergence of general systems research in the 1950s, systems thinking and systems science have developed into many theoretical frameworks.
Systems notes of Henk Bikker, TU Delft, 1991

A theory is almost never proven, though a few theories do become scientific laws. One example would be the laws of conservation of energy, which is the first law of thermodynamics. Dr. Linda Boland, a neurobiologist and chairperson of the biology department at the University of Richmond, Virginia told Live Science that this is her favorite scientific law. ``This is one that guides much of my research on cellular electrical activity and it states that energy cannot be created nor destroyed, only changed in form. This law continually reminds me of the many forms of energy,'' she said.

Laws are generally considered to be without exception, though some laws have been modified over time after further testing found discrepancies. This does not mean theories are not meaningful. For a hypothesis to become a theory, rigorous testing must occur, typically across multiple disciplines by separate groups of scientists. Saying something is ``just a theory'' is a layperson’s term that has no relationship to science. To most people a theory is a hunch. In science a theory is the framework for observations and facts, Jaime Tanner, a professor of biology at Marlboro College, told Life Science.

Debate on whether `scientific consensus’ is to be given special weight seems to turn on the two-sides of the injunction from Richard Feynman:

“When someone says science teaches such and such, he is using the word incorrectly. Science doesn’t teach it; experience teaches it. If they say to you science has shown such and such, you might ask, “How does science show it – how did the scientists find out – how, what, where?”  Not science has shown, but this experiment, this effect, has shown. And you have as much right as anyone else, upon hearing about the experiments (but we must listen to all the evidence), to judge whether a reusable conclusion has been arrived at.
    • (Feynman, “The Pleasure of Finding things out”, page 187, emphasis added)
    
Is there any dispute that both aspects are required for the integrity of the scientific process?

\begin{enumerate}
\item Everyone has the right to judge for themselves whether `a reusable conclusion’ has been arrived at, \textit{but}
\item This must be done after listening to all the evidence, including what, how, and where scientists obtained the evidence
\end{enumerate}

I have not studied this topic in detail, but from a common-sense point of view, it does seem that many people (groups) want to declare their beliefs to be true, with these beliefs often coming from what they wish to be true based on ulterior motives or to conform with a specific narrative.

Kant wins. Marx wins. Everything has become politicized. Death of reason and dialogue. Victory to mindless rhetoric. Say it enough and people will believe it is true.

A person’s right to believe as he/she wishes does not imply an obligation on anyone else to accept that belief, no matter how many people choose to accept it. A group claiming it has consensus on a viewpoint cannot claim truth any more than a jury declaring “not guilty” proves the defendant is innocent of the act. A claim of consensus is not a form of higher or moral authority over others. It is not an act of reason. It is an act of intellectual treason.

Remember, it was consensus that all swans were white. It was consensus that the heavens revolved around the Earth, while simultaneously the world was flat. Until, of course, that those pesky non-believers and intellectual saboteurs revealed the errors by having the guts to find black swans and circumnavigate the Earth.

My lay viewpoint is that we would do well to eliminate the concept of “consensus” from our consciousness when speaking of science. Consensus does not imply truth.

We might also wish to recall Warfield’s definition of Clan-Think when touting consensus.

This makes interesting demands on education.

RE: “the practice of science has achieved considerable success by setting goals of sharing information to enable independent replication, debate, and testing of predictions – that is, valuing objective evidence over personal opinion or ideology or theology.”

Is there some other mode of the practice of science  that does not require independent replication, debate, and testing of predictions?

I ask this because it seems you are saying that humans are also practicing science when a sufficient number simply agree on some assertion without performing independent replication, debate, and testing.

Are you saying that replication, debate, and testing of predictions does not have to be done physically but can be done conceptually AND does not have to be done independently but is even better if done collegially?
Seems to me if you eliminated the concept of consensus from the practice of science you would be left with “independent replication, debate, and testing of predictions.”

If this would not be ‘science’ as practiced for the past few hundred years, then viola! The scientific method triumphs by detecting and highlighting that which is false science.

Please tell us what you think of Cornell Prof. Derek Cabrera’s rule that (if I am representing his claim accurately) we should reserve the label ‘theory’ for an idea that has been independently tested and we should use the label ‘hypothesis’ for an idea that has not yet been – even if highly popular among those who have awarded themselves the label scientist.

Could it be that a model of an intended system as produced by a system engineering activity is a hypothesis waiting to be vetted by those who develop and deploy actual systems and measure their effects?  Then, if updated to a sufficient degree of fidelity to the measurable system does that model become a theory?

%------------------------------------------------

\section{Valuing Faith-Based Thinking}\index{Valuing Faith-Based Thinking}

%------------------------------------------------

\section{Summary and Extensions}\index{Summary and Extensions}

Science and systems thinking is increasingly being used to tackle a wide variety of subjects in fields such as computing, engineering, epidemiology, information science, health, manufacture, management, and the environment.

\begin{itemize}
\item Systemology and synthesis. The science of systems or their formation is called systemology. Problems and problem complexities faced by humankind do not organize themselves along disciplinary lines. New arrangements of scientific and professional efforts based on the common attributes and characteristics of needs and problems should contribute to the progress. More attention should be paid to human action and praxeology applications at the macro-level to help understand both the economic and non-economic dimensions of the world in which we live.
\item The formation of interdisciplines began in the middle of the last century and that has brought about an evolutionary synthesis of knowledge. This has occurred not only within science, but between science and technology and between science and humanities. The forward progress of systemology in the study of large-scale complex systems requires a synthesis of science and the humanities in addition to a synthesis of science and technology.
\item When synthesizing human-made systems, unintended effects can be minimized and the natural system can sometimes be improved by engineering the larger human-modified system instead of engineering only the human-made. If system evaluation is applied beyond the human-made, then the boundary of the target system (meant to include both natural and human-made systems) should be adopted as the boundary of the human-modified domain.
\item Systems are as pervasive as the universe in which they exist. They are as grand as the universe itself or as infinitesimal as the atom. Systems appeared first in natural forms, but with the advent of human beings, a variety of human-made systems have come into existence. In recent decades, we have begun to understand structure and characteristics of natural and human-made systems in a scientific way.
\end{itemize}

Upon completion of , the reader will have obtained essential insight into systems and systems thinking, with an orientation toward systems engineering and analysis. The system definitions, classifications, and concepts presented in this chapter are intended to impart a general understanding about the following:

\begin{enumerate}
\item System classifications, similarities, and dissimilarities
\item The fundamental distinction between natural and human-made systems
\item The elements of a system and the position of the system in the hierarchy of systems
\item The domain of systems science, with consideration of cybernetics, general systems theory, and systemology
\item Technology as the progenitor for the creation of technical systems, recognizing its impact on the natural world
\item The transition from the machine or industrial age to the Systems Age, with recognition of its impact upon people and society
\item System complexity and scope and the demands these factors make on engineering in the Systems Age
\item The range of contemporary definitions of systems engineering used within the profession.
\end{enumerate}

%------------------------------------------------

%% QUESTIONS, PROBLEMS, AND EXERCISES
% SEA Question Location in \label{sea-Chapter#-Problem#}
