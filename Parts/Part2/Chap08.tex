\chapter{SYSTEMS ENGINEERING PROCESS PHASES}\label{chap:8}

There are several phases through which the system design and development process must invariably pass. Foremost among them is identification and understanding of the customer-stated problem, with special emphasis on what the system is intended to do. This is followed by the solicitation of system-level requirements, technological innovation to discover and define feasible solutions, the design and development of system components, the fabrication of a prototype and/or engineering model, and the validation of system design through test and eval..

The coordinated sections that follow partition the system design and development process into major phases. Conceptual design, preliminary design, detail design and development, and test and evaluation may be incorrectly considered to be distinct from each other, rather than as part of a seamless process. This partition is made for convenience and communication purposes. The proper and timely application of feedback, iteration, and successive improvement will act to ensure integration of the essential activities.

System design is not systems engineering per se, but it is almost as pervasive. From the perspective of synthesis, system design reaches almost as far and deep as systems engineering. It is nominally comprised of conceptual, preliminary, and detail design. These are artificial categories that, along with test and evaluation, make up the four chapters in of this textbook. Included in these chapters is a high degree of connectivity among the design categories and an elaboration upon the theme of ``bringing into being,'' promulgated in Chapter 6.

The engine that drives the systems engineering process is system design. System design is a process itself, with a prominent position in systems engineering. It is an essential activity ensuring the orderly realization of the final configuration and composition of a system. It is also the basis for incorporating the designs for companion and post-realization capabilities such as maintenance, support, sustainment, recycling, and disposal.

Accordingly, the overall purpose of this chapter is to impart an in-depth understanding of system design as a process; a process that is greater than the sum of its artificial categories as identified by process phases.

\section{System Life-Cycle Phases}\index{System Life-Cycle Phases}

A newly identified human need, or an evolving need, reveals a new set of system requirements. If a decision is made to seek a solution for the need, then a decision is needed whether to consider other needs in designing the solution. Based on an initial determination regarding the scope of needs, the basic phases of conceptual design and onward through system retirement and phase-out are then applicable, as described in the paragraphs that follow. The scope of needs may contract or expand, but the scope should be stabilized as early as possible during conceptual design, preferably based on an evaluation of value and cost by the customer.

\subsection{Chronology of Life-Cycle Phases}\index{Chronology of Life-Cycle Phases}

Figure 8.1 8.18 illustrates the major life-cycle process phases and selected milestones for a generic system. This is the model that will serve as a frame of reference for material presented in subsequent sections. Included are the basic steps in the systems engineering process (i.e., requirements analysis, functional analysis and allocation, synthesis, trade-off studies, design evaluation, etc.

Figure 8.1 Here (From SEA 2.4)

Program phases described in are not intended to convey specific tasks, or time periods, or levels of funding, or numbers of iterations. Individual program requirements will vary from one application to the next. The figure exhibits an overall process that needs to be followed during system acquisition and deployment. Regardless of the type, size, and complexity of the system, there is a conceptual design requirement (i.e., to include requirements analysis), a preliminary design requirement, and so on. Also, to ensure maximum effectiveness, the concepts presented in must be properly ``tailored'' to the particular system application being addressed.

shows the basic steps in the systems engineering process to be iterative in nature, providing a top-down definition of the system, and then proceeding down to the subsystem level (and below as necessary). Focused on the needs, and beginning with conceptual design, the completion of Block 0.2 defines the system in functional terms (having identified the ``what'' from a requirements perspective). These ``whats'' are translated into an applicable set of ``hows'' through the iterative process of functional partitioning and requirements allocation, together with conceptual design synthesis, analysis, and evaluation. This conceptual design phase is where the initial configuration of the system (or system architecture) is defined.

During preliminary design, completion of Block 1.1 defines the system in refined functional terms providing a top-down definition of subsystems with preparation for moving down to the component level. Here the ``whats'' are extracted from (provided by) the conceptual design phase. These ``whats'' are translated into an applicable set of ``hows'' through the iterative process of functional partitioning and requirements allocation, together with preliminary design synthesis, analysis, and evaluation. This preliminary design phase is where the initial configuration of subsystems (or subsystem architecture) is defined.

Blocks 1.1–1.7 are an evolution from Blocks 0.1–0.8, Blocks 2.1–2.5 are an evolution from Blocks 1.1–1.7, and Blocks 3.1–3.6 are an evolution from Blocks 2.1–2.5. The overall process reflected in the figure constitutes an evolutionary design and development process. With appropriate feedback and design refinement provisions incorporated, the process should eventually converge to a successful design. The functional definition of the system, its subsystems, and its components serves as the baseline for the identification of resource requirements for production and then operational use (i.e., hardware, software, people, facilities, data, elements of support, or a combination thereof).

\subsection{Checklist of Life\-Cycle Steps}\index{Checklist of Life\-Cycle Steps}

This section addresses certain steps in the systems engineering process and, in doing so, provides basic insight and knowledge about the following:

\begin{enumerate}
\item Identifying and translating a problem or deficiency into a definition of need for a system that will provide a preferred solution
\item Accomplishing advanced system planning and architecting in response to the identified need
\item Developing system operational requirements describing the functions that the system must perform to accomplish its intended purpose(s) or mission(s);
\item Conducting exploratory studies leading to the definition of a technical approach for system design
\item Proposing a maintenance concept for the sustaining support of the system throughout its planned life cycle
\item Identifying and prioritizing technical performance measures (TPMs) and related criteria for design
\item Accomplishing a system-level functional analysis and allocating requirements to various subsystems and components
\item Performing systems analysis and producing trade-off studies
\item Developing a system specification
\item Conducting a conceptual design review
\end{enumerate}

The completion of the steps above constitutes the system definition process at the conceptual level. Although the depth, effort, and cost of accomplishing these steps may vary, the process is applicable to any type or category of system, complex or simple, large or small. It is important that these steps, which encompass the front end of the systems engineering process, be thoroughly understood. Collectively, they serve as a learning objective with the goal being to provide a comprehensive step-by-step approach for addressing this critical early phase of the systems engineering process.
    
\subsection{A Hypothetical Example}\index{A Hypothetical Example}

System life-cycle phases and their associated steps will be more easily explained and visualized if presented in connection with a hypothetical but realistic example. Accordingly, this is the time to offer and describe an example that will lend itself as a basis for discussing the material in this chapter. The chosen example is that of engineering a solution to the problem possessed by a municipality that exists due to the division of the community into east and west by the presence of a river. This will be referred to as a river crossing problem.

Assume that a regional transportation authority is faced with the problem of providing for increased two-way traffic flow across a river that divides a growing municipality (to illustrate the overall process, this particular example is developed further in, and under preliminary design in 

In considering alternative system design approaches, different technology applications are investigated. For instance, in response to the river crossing problem (identified in ), alternative design concepts may include a tunnel under the river, a bridge spanning the river, an airlift capability over the river, the use of barges and ferries on the river, or possibly re-routing the river itself. Then a feasibility study would be accomplished to determine a preferred approach. In performing such a study, one must address limiting factors such as geological and geotechnical, atmospheric and weather, hydrology and water flow, as well as the projected capability of each alternative to meet life-cycle cost objectives. In this case, the feasibility results might tentatively indicate that some type of bridge structure spanning the river appears to be best.

Returning to the regional public transportation authority facing the problem of providing for capability that will allow for a significant increase in the two-way traffic flow across a river dividing a growing municipality (a <emphasis>what</emphasis>). Further study of the problem (i.e., the current deficiency) revealed requirements for the two-way flow of private vehicles, taxicabs, buses, rail and rapid transit cars, commercial vehicles, large trucks, people on motor cycles and bicycles, and pedestrians across the river. Through advanced system planning and consideration of possible architectures, various river crossing concepts were proposed and evaluated for physical and economic feasibility. These included going under the river, on the river, spanning the river, over the river, or possibly re-routing the river itself. Feasibility considerations determined that the river is not a good candidate for rerouting, both physically and due to its role in providing navigable traffic flow upstream and downstream.

Results from the study indicated that the most attractive approach is the construction of some type of a bridge structure spanning the river (a how). From this point on, it is necessary to delve further into the operational requirements leading to the selection and evaluation of a bridge type (suspension, pier and superstructure, causeway, etc.) by considering some detailed ``design-to'' factors as below

While some of the specific design-to qualitative and quantitative factors introduced in this example may vary from one project to the next, this example is presented with the intention of illustrating those considerations that must be addressed early in conceptual design as it pertains to the river crossing problem.
    
\subsection{Other Hypothetical Examples}\index{Other Hypothetical Examples}


These additional illustrations. The five illustrations presented, derived from the results of feasibility analyses conducted earlier, are representative of typical ``needs'' for which system operational requirements must be defined at the inception of a program, and must serve as the basis for all subsequent program activities. These requirements must not only be implemented within the bounds of the specific system configuration in question but must also consider all possible external interfaces that may exist.

For example, as illustrated in, one may be dealing with a number of different systems, all of which are closely related and may have a direct impact on each other. Also, there may be some ``sharing'' of capabilities across the board; for example, the air and ground transportation systems utilizing some of the same components which operate as part of the communication systems. Thus, the design of any new system must consider all possible impacts that it will have on other systems within the same <emphasis>SOS</emphasis> configuration (as shown in the figure), as well as those possible impacts that the other systems may have on the new system.

The methodology employed is basically the same for any system, whether the subject is a relatively small item as part of the river crossing bridge, installed in an aircraft or on a ship, a factory, or a large one-of-a-kind project such as the community hospital involving design and construction. In any case, the system must be defined in terms of its projected mission, performance, operational deployment, life cycle, utilization, effectiveness factors, and the anticipated environment.
    
%------------------------------------------------

\section{Conceptual System Design}\index{Conceptual System Design}

Conceptually sound system design focuses on what the system is intended to do before determining what the system is, with form following function. This focus is most effective when based on essential design dependent parameters, recognizing the concurrent life-cycle factors of production, support, maintenance, phase-out, recycle, and disposal. It invokes iterating synthesis, analysis, and evaluation as in Chapter 6. These considerations are germane to system and product design as embedded within the systems engineering process and its inherent life cycle.

\subsection{Problem Definition and Need Identification}\index{Problem Definition and Need Identification}

The systems engineering process usually begins with the identification of a ``want'' or ``desire'' for something based on some ``real'' or ``perceived'' deficiency. For instance, suppose that a current system capability is not adequate in terms of meeting certain required performance goals, is not available when needed, cannot be properly supported, or is too costly to operate. Or, there is a lack of capability to communicate between point A and point B, at a desired rate X, with reliability of Y, and within a specified cost of Z. Or, a regional transportation authority is faced with the problem of providing for increased two-way traffic flow across a river that divides a growing municipality (to illustrate the overall process, this particular example is developed further in, and under preliminary design in).

It is important to commence by first defining the ``problem'' and then defining the need for a specific system capability that (hopefully) is responsive. It is not uncommon to first identify some ``perceived'' need which, in the end, doesn’t really solve the problem at hand. In other words, why is this particular system capability needed? Given the problem definition, a new system requirement is defined along with the priority for introduction, the date when the new system capability is required for customer use, and an estimate of the resources necessary for its acquisition. To ensure a good start, a comprehensive statement of the problem should be presented in specific qualitative and quantitative terms and in enough detail to justify progressing to the next step. It is essential that the process begin by defining a ``real'' problem and its importance.

The necessity for identifying the need may seem to be basic or self-evident; however, a design effort is often initiated as a result of a personal interest or a political whim, without the requirements first having been adequately defined. In the software and information technology area, in particular, there is a tendency to accomplish considerable coding and software development at the detailed level before adequately identifying the real need. In addition, there are instances when engineers sincerely believe that they know what the customer needs, without having involved the customer in the ``discovery'' process. The ``design-it-now-fix-it-later'' philosophy often prevails, which, in turn, leads to unnecessary cost and delivery delay.

Defining the problem is often the most difficult part of the process, particularly if there is a rush to get underway. The number of false starts and the resulting cost commitment can be significant unless a good foundation is laid from the beginning, as illustrated by . A complete description of the need, expressed in quantitatively related criteria whenever possible, is essential. It is important that the problem definition reflects true customer requirements, particularly in an environment of limited resources.

Having defined the problem completely and thoroughly, a needs analysis should be performed with the objective of translating a broadly defined ``want'' into a more specific system-level requirement. The questions are as follows: What is required of the system in ``functional'' terms? What functions must the system perform? What are the ``primary'' functions? What are the ``secondary'' functions? What must be accomplished to alleviate the stated deficiency? When must this be accomplished? Where is it to be accomplished? How many times or at what frequency must this be accomplished?

There are many basic questions of this nature, making it important to describe the customer requirements in a functional manner to avoid a premature commitment to a specific design concept or configuration. Unless form follows function, there is likely to be an unnecessary expenditure of valuable resources. The ultimate objective is to define the whats first, deferring the hows until later.

Identifying the problem and accomplishing a needs analysis in a satisfactory manner can best be realized through a team approach involving the customer, the ultimate consumer or user (if different from the customer), the prime contractor or producer, and major suppliers, as appropriate. The objective is to ensure that proper and effective communications exist between all parties involved in the process. Above all, the ``voice of the customer'' must be heard, providing the system developer(s) an opportunity to respond in a timely and appropriate manner.

\subsection{Advanced System Planning and Architecting}\index{Advanced System Planning and Architecting}

Given an identified ``need'' for a new or improved system, the advanced stages of system planning and architecting can be initiated. Planning and architecting are essential and coequal activities for bringing a new or improved capability into being. The overall ``program requirements'' for bringing the capability into being initiate an advanced system planning activity and the development of a <emphasis>program management plan</emphasis> (PMP), shown as the second block in . While the specific nomenclature for this top-level plan may differ with each program, the objective is to prepare a ``management-related'' plan providing the necessary guidance for all subsequent managerial and technical activities.

Referring to, the PMP guides the development of requirements for implementation of a systems engineering program and the preparation of a systems engineering management plan (SEMP), or system engineering plan (SEP). The ``technical requirements'' for the system are simultaneously determined. This involves development of a system-level architecture (functional first and physical later) to include development of system operational requirements, determination of a functional architecture, proposing alternative technical concepts, performing feasibility analysis of proposed concepts, selecting a maintenance and support approach, and so on, as is illustrated by. The results lead to the preparation of the system specification (Type A). The preparation of the SEMP and the system specification should be accomplished concurrently in a coordinated manner. The two documents must ``talk'' to each other and be mutually supportive.

It can be observed from and  that the identified requirements are directly aligned and supportive of the activities and milestones shown in . The system specification (Type A) contains the highest-level architecture and forms the basis for the preparation of all lower-level specifications in a top-down manner. These lower-level specifications include development (Type B),product (Type C), process (Type D), and material (Type E) specifications, and are described further in . The systems engineering management plan is addressed in detail in . The systems engineering process and the steps illustrated in the lower part of are described in the remaining sections of this chapter.

Conceptual design is the first and most important phase of the system design and development process. It is an early and high-level life-cycle activity with the potential to establish, commit, and otherwise predetermine the function, form, cost, and development schedule of the desired system and its product(s). The identification of a problem and an associated definition of need provide a valid and appropriate starting point for conceptual system design.	

Selection of a path forward for the design and development of a preferred system architecture that will ultimately be responsive to the identified customer need is a major purpose of conceptual design. Establishing this foundation early, as well as initiating the early planning and evaluation of alternative technological approaches, is a critical initial step in the implementation of the systems engineering process. Systems engineering, from an organizational perspective, should take the lead in the solicitation of system 

\subsection{System Design and Feasibility Analysis}\index{System Design and Feasibility Analysis}

Having justified the need for a new system, it is necessary to (1) identify various system-level design approaches or alternatives that could be pursued in response to the need; (2) evaluate the feasible approaches to find the most desirable in terms of performance, effectiveness, maintenance and sustaining support, and life-cycle economic criteria; and (3) recommend a preferred course of action. There may be many possible alternatives; however, the number of these must be narrowed down to those that are physically feasible and realizable within schedule requirements and available resources.

In considering alternative system design approaches, different technology applications are investigated. For instance, in response to the river crossing problem (identified in ), alternative design concepts may include a tunnel under the river, a bridge spanning the river, an airlift capability over the river, the use of barges and ferries on the river, or possibly re-routing the river itself. Then a feasibility study would be accomplished to determine a preferred approach. In performing such a study, one must address limiting factors such as geological and geotechnical, atmospheric and weather, hydrology and water flow, as well as the projected capability of each alternative to meet life-cycle cost objectives. In this case, the feasibility results might tentatively indicate that some type of bridge structure spanning the river appears to be best.

At a more detailed level, in the design of a communications system, is a fiber optics technology or the conventional twisted-wire approach preferred? In aircraft design, to what extent should the use of composite materials be considered? In automobile design, should high-speed electronic circuitry in a certain control application be incorporated or should an electromechanical approach be utilized? In the design of a data transmission capability, should a digital or an analogue format be used? In the design of a process, to what extent should embedded computer capabilities be incorporated? Included in the evaluation process are considerations pertaining to the type and maturity of the technology, its stability and growth potential, the anticipated lifetime of the technology, the number of supplier sources, and so on.

It is at this early stage of the system life cycle that major decisions are made relative to adopting a specific design approach and related technology application. Accordingly, it is at this stage that the results of such design decisions can have a great impact on the ultimate behavioral characteristics and life-cycle cost of a system (refer again to ). Technology applications are evaluated and, in some instances where there is not enough information available (or a good solution is not readily evident), research may be initiated with the objective of developing new knowledge to enable other approaches. Finally, it must be agreed that the ``need'' should dictate and drive the ``technology,'' and not vice versa.

The identification of alternatives and feasibility considerations will significantly impact the operational characteristics of the system and its design for constructability, producibility, supportability, sustainability, disposability, and other design characteristics. The selection and application of a given technology has reliability and maintainability implications, may impact human performance, may affect construction or manufacturing and assembly operations in terms of the processes required, and may significantly impact the need for system maintenance and support. Each will certainly affect life-cycle cost differently. Thus, it is essential that life-cycle considerations be an inherent part of the process of determining the feasible set of system design alternatives.

\subsection{System Operational Requirements}\index{System Operational Requirements}

Once the need and technical approach have been defined, it is necessary to translate this into some form of an ``operational scenario,'' or a set of operational requirements. At this point, the following questions may be asked: What are the anticipated types and quantities of equipment, software, personnel, facilities, information, and so on, required, and where are they to be located? How is the system to be utilized, and for how long? What is the anticipated environment at each operational site (user location)? What are the expected interoperability requirements (i.e., interfaces with other ``operating'' systems in the area)? How is the system to be supported, by whom, and for how long? The answer to these and comparable questions leads to the definition of system operational requirements, the follow-on maintenance and support concept, and the identification of specific design-to criteria, and related guidelines.</para>

Defining Operational requirements. System operational requirements should be identified and defined early, carefully, and as completely as possible, based on an established need and selected technical approach. The operational concept and scenario as defined herein is identified in and . It should include the following:

\begin{enumerate}
\item Identification of the prime and alternate or secondary missions of the system. What is the system to accomplish? How will the system accomplish its objectives? The mission may be defined through one or a set of scenarios or operational profiles. It is important that the <emphasis>dynamics</emphasis> of system operating conditions be identified to the extent possible.
\item Performance and physical parameters: Definition of the operating characteristics or functions of the system (e.g., size, weight, speed, range, accuracy, flow rate, capacity, transmit, receive, throughput, etc.). What are the critical system performance parameters? How are they related to the mission scenario(s)?
\item Operational deployment or distribution: Identification of the quantity of equipment, software, personnel, facilities, and so on and the expected geographical location to include transportation and mobility requirements. How much equipment and associated software is to be distributed, and where is it to be located and for how long? When does the system become fully operational?
\item Operational life cycle (horizon): Anticipated time that the system will be in operational use (expected period of sustainment). What is the total inventory profile throughout the system life cycle? Who will be operating the system and for what period of time?
\item Utilization requirements: Anticipated usage of the system and its elements (e.g., hours of operation per day, percentage of total capacity, operational cycles per month, facility loading). How is the system to be used by the customer, operator, or operating authority in the field?
\item Effectiveness factors: System requirements specified as figures-of-merit (FOMs) such as cost/system effectiveness, operational availability (Ao), readiness rate, dependability, logistic support effectiveness, mean time between maintenance (MTBM), failure rate (), maintenance downtime (MDT), facility utilization (in percent), operator skill levels and task accomplishment requirements, and personnel efficiency. Given that the system will perform, how effective or efficient is it? How are these factors related to the mission scenario(s)?
\item Environmental factors: Definition of the environment in which the system is expected to operate (e.g., temperature, humidity, arctic or tropics, mountainous or flat terrain, airborne, ground, or shipboard). This should include a range of values as applicable and should cover all transportation, handling, and storage modes. How will the system be handled in transit? To what will the system be subjected during its operational use, and for how long? A complete environmental profile should be developed.
\end{enumerate}

In addition to defining operational requirements that are system specific, the system being developed may be imbedded within an overall higher-level structure making it necessary to give consideration to interoperability requirements. For example, an aircraft system may be contained within a higher-level airline transportation system, which is part of a regional transportation capability, and so on. There may be both ground and marine transportation systems within the same overall structure, where major interface requirements must be addressed when system operational requirements are being defined.

In some instances, there may be both vertical and horizontal impacts when addressing the system in question, within the context of some larger overall configuration. There are two important questions to be addressed: <emphasis>What is the potential impact of this new system on the other systems in the same SOS configuration? What are the external impacts from the other systems within the same SOS structure on this new system?

Illustrating System Operational Requirements. Further consideration of system operational requirements (as presented in is provided through five sample illustrations, each covering different degrees or levels of detail. The first illustration is an extension of the river crossing problem introduced in . The second illustration, covering operational requirements in more depth, is an aircraft system with worldwide deployment. The third illustration is a communication system with ground and airborne applications. The fourth illustration deals with commercial airline capability for a metropolitan area. The fifth illustration considers a hospital as part of a community healthcare system. Finally, it is noted that there may exist many other applications and situations to which the illustrated methodology applies.<

The intent of these examples is to encourage consideration of operational requirements at a greater depth than before and to do so early in the system life cycle when the specification of such requirements will have the greatest impact on design, as emphasized in. While it may be easier to delay such considerations until later in the system design process, the consequences of such are likely to be very costly in the long term. The objective in systems engineering is to ``force'' considerations of operational requirements as early as practicable in the design process.

\subsection{System Maintenance and Support}\index{System Maintenance and Support}

In addressing system requirements, the normal tendency is to deal primarily with those elements of the system that relate directly to the ``performance of the mission,'' that is, prime equipment, operator personnel, operational software, operating facilities, and associated operational data and information. At the same time, too little attention is given to system maintenance and support and the sustainment of the system throughout its planned life cycle. In general, the emphasis has been directed toward only part of the system and not the entire system as an entity, which has led to costly results in the past.

To realize the overall benefits of systems engineering, it is essential that <emphasis>all</emphasis> elements of the system be considered on an integrated basis from the beginning. This includes not only the prime mission-related elements of the system but the maintenance and support infrastructure as well. The prime system elements must be designed in such a way that they can be effectively and efficiently supported through the entire system life cycle, and the maintenance and support infrastructure must be responsive to this requirement. This, in turn, means that one should also address the design characteristics as they pertain to transportation and handling equipment, test and support equipment, maintenance facilities, the supply chain process, and other applicable elements of logistic support.

The maintenance and support concept developed during the conceptual design phase evolves from the definition of system operational requirements described in . It constitutes a before-the-fact series of illustrations and statements leading to the definition of reliability, maintainability, human factors and safety, constructability and producibility, supportability, sustainability, disposability, and related requirements for design. It constitutes an ``input'' to the design process, whereas the maintenance plan (developed later) defines the follow-on requirements for system support based on a known design configuration and the results of the supportability analysis presented in 

The maintenance and support concept is reflected by the network and the activities and their interrelationships, illustrated in . The network exists whenever there are requirements for corrective and/or preventive maintenance at any time and throughout the system life cycle. By reviewing these requirements, one should address such issues as the levels of maintenance, functions to be performed at each level, responsibilities for the accomplishment of these functions, design criteria pertaining to the various elements of support (e.g., type of spares and levels of inventory, reliability of the test equipment, personnel quantities and skill levels), and the effectiveness factors and ``design-to'' requirements for the overall maintenance and support infrastructure. Although the design of the prime elements of the system may appear to be adequate, the overall ability of the system to perform its intended mission objective highly depends on the effectiveness of the support infrastructure as well.

While there may be some variations that arise, depending on the type and nature of the system, the maintenance and support concept generally includes the following items

Levels of maintenance: Corrective and preventive maintenance may be performed on the system itself (or an element thereof) at the site where the system is operating and used by the customer, in an intermediate shop near the customer’s operational site, and/or at a depot or manufacturer’s facility. Maintenance level pertains to the division of functions and tasks for each area where maintenance is performed. Anticipated frequency of maintenance, task complexity, personnel skill-level requirements, special facility needs, supply chain requirements, and so on, dictate to a great extent the specific functions to be accomplished at each level. Depending on the nature and mission of the system, there may be two, three, or four levels of maintenance; however, for the purposes of further discussion, maintenance may be classified as organizational, intermediate, and manufacturer/depot/supplier. 
describes the basic criteria and differences between these levels.

Repair policies: Within the constraints illustrated in there are a number of possible policies specifying the extent to which the repair of an element or component of a system should be accomplished (if at all). A repair policy may dictate that an item should be designed such that, in the event of failure, it should be nonrepairable, partially repairable, or fully repairable. Stemming from the operational requirements described in (refer to the five system illustrations), an initial ``repair policy'' for the system being developed should be established with the objective of providing some early guidelines for the design of the different components that make up the system. Referring to the example of the repair policy, illustrated in, it can be seen that there are numerous quantitative factors, which were initially derived from the definition of system operational requirement, that provide ``design-to'' guidelines as an input to the overall design process; for example, the system shall be designed such that the MTBM shall be 175 hours or greater, the MDT shall be 2 hours or less, the MLH/OH shall not exceed 0.1, and so on. A repair policy should be initially developed and established during the conceptual design phase, and subsequently updated as the design progresses and the results of the level-of-repair and supportability analyses become available.

Organizational responsibilities: The accomplishment of maintenance may be the responsibility of the customer, the producer (or supplier), a third party, or a combination thereof. In addition, the responsibilities may vary, not only with different components of the system but also as one progresses in time through the system operational use and sustaining support phase. Decisions pertaining to organizational responsibilities may affect system design from a diagnostic and packaging standpoint, as well as dictate repair policies, product warranty provisions, and the like. Although conditions may change, some initial assumptions are required at this point in time.

Maintenance support elements: As part of the initial maintenance concept, criteria must be established relating to the various elements of maintenance support. These elements include supply support (spares and repair parts, associated inventories, and provisioning data), test and support equipment, personnel and training, transportation and handling equipment, facilities, data, and computer resources. Such criteria, as an input to design, may cover self-test provisions, built-in versus external test requirements, packaging and standardization factors, personnel quantities and skill levels, transportation and handling factors, constraints, and so on. The maintenance concept provides some initial system design criteria pertaining to the activities illustrated in, and the final determination of specific logistic and maintenance support requirements will occur through the completion of a supportability analysis as design progresses.

Effectiveness requirements: These constitute the effectiveness factors associated with the support capability. In the supply support area, they may include a spare-part demand rate, the probability that a spare part will be available when required, the probability of mission success given a designated quantity of spares in the inventory, and the economic order quantity as related to inventory procurement. For test equipment, the length of the queue while waiting for test, the test station process time, and the test equipment reliability are key factors. In transportation, transportation rates, transportation times, the reliability of transportation, and transportation costs are of significance. For personnel and training, one should be interested in personnel quantities and skill levels, human error rates, training rates, training times, and training equipment reliability. In software, the number of errors per mission segment, per module of software, or per line of code may be important measures. For the supply chain overall, reliability of service, item processing time, and cost per item processed may be appropriate metrics to consider. These factors, as related to a specific system-level requirement, must be addressed. It is meaningless to specify a tight quantitative requirement applicable to the repair of a prime element of the system when it takes 6 months to acquire a needed spare part (for example). The effectiveness requirements applicable to the supply chain and support capability must complement the requirements for the system overall.

Environment: Definition of environmental requirements as they pertain to the maintenance and support infrastructure is equally important. This includes the impact of external factors such as temperature, shock and vibration, humidity, noise, arctic versus tropical applications, operating in mountainous versus flat terrain country, shipboard versus ground conditions, and so on, on the design of the maintenance and support infrastructure. In addition, it is also necessary to address possible ``outward'' environmental impact(s) of the maintenance and support infrastructure on other systems and on the environment in general (with the ``design for sustainability'' as being a major objective).

The maintenance concept provides the foundation that leads to the design and development of the maintenance and support infrastructure and defines the specific design-to requirements for the various elements of support (e.g., the supply support capability, transportation and handling equipment, test and support equipment, and facilities). These requirements, as they apply to system life-cycle support, can have a significant ``feedback'' effect (impact) on the prime elements of the system as well. Thus, the definition of system operational requirements and the development of the maintenance concept must be accomplished concurrently and early during the conceptual design phase. The combined result forms the basis for development of much of the material throughout the subsequent sections of this text, particularly with regard to the subject areas included in , that is, design for reliability, design for maintainability, design for usability (human factors), design for supportability, and others.

In summary, when defining the maintenance concept, it is important that consideration be given to the interfaces that may exist between the support requirements and infrastructure for the new system being developed and those comparable requirements for other systems that may be contained within the same overall system-of-systems configuration (refer to ). The requirements for this new system must first be defined, the impacts on (and from) the other systems evaluated, major conflicting areas noted, and finally modifications be incorporated as required. Care must be taken to ensure that the requirements for this new system are not compromised in any way. Finally, the selection of the ultimate maintenance and support infrastructure configuration must be justified on the basis of the LCC. What may seem to be a least-cost approach in providing maintenance support at the local level may not be such when considering the costs associated with all of the supply chain activities for the system in question.

\subsection{The System Architecture}\index{The System Architecture}

With the definition of system operational requirements, the maintenance and support concept, and the identification and prioritization of the TPMs, the basic system architecture has been established. The architecture deals with a top-level system structure (configuration), its operational interfaces, anticipated utilization profiles (mission scenarios), and the environment within which it is to operate.

Architecture describes how various requirements for the system interact. This, in turn, leads into a description of the functional architecture, which evolves from a functional analysis and is a description of the system in functional terms. From this analysis, and through the requirements allocation process and the definition of the various resource requirements necessary for the system to accomplish its mission, the physical architecture is defined. Through application of this process, one is able to evolve from the whats to the hows.

Many different trade-offs are possible as the system design progresses. Decisions must be made regarding the evaluation and selection of appropriate technologies, the evaluation and selection of commercial off-the-shelf (COTS) components, subsystem and component packaging schemes, possible degrees of automation, alternative test and diagnostic routines, various maintenance and support policies, and so on. Later in the design cycle, there may be alternative manufacturing processes, alternative factory maintenance plans, alternative logistic support structures, and alternative methods of material phase-out, recycling, and/or disposal.

One must first define the problem and then identify the design criteria or measures against which the various alternatives will be evaluated (i.e., the applicable TPMs), select the appropriate evaluation techniques, select or develop a model to facilitate the evaluation process, acquire the necessary input data, evaluate each of the candidates under consideration, perform a sensitivity analysis to identify potential areas of risk, and finally recommend a preferred approach. This process is illustrated in , and can be tailored and applied at any point in the life cycle as illustrated in and. Only the depth of the analysis and evaluation effort will vary, depending on the nature of the problem.

Referring to , trade-off analysis involves synthesis. Synthesis refers to the combining and structuring of components to create a feasible system configuration. Synthesis is design. Initially, synthesis is used in the development of preliminary concepts and to establish relationships among various components of the system. Later, when sufficient functional definition and decomposition have occurred, synthesis is used to further define the hows at a lower level. Synthesis involves the creation of a configuration that could be representative of the form that the system will ultimately take, although a final configuration should not be assumed at this early point in the design process.

Given a synthesized configuration, its characteristics need to be evaluated in terms of the system requirements initially specified. Changes are incorporated as required, leading to a preferred design configuration. This iterative process of synthesis, analysis, evaluation, and design refinement leads initially to the establishment of the functional baseline, then the allocated baseline, and finally the product baseline (refer to ). A good description of these configuration baselines, combined with a disciplined approach to baseline management, is essential for the successful implementation of the systems engineering process.

Throughout the conceptual system design phase (commencing with the needs analysis), one of the major objectives is to develop and define the specific ``design-to'' requirements for the system as an entity. The results from the activities described in are combined, integrated, and included in a system specification (Type A). This specification constitutes the top ``technical-requirements'' document that provides overall guidance for system design from the beginning. Referring to , this specification is usually prepared at the conclusion of conceptual design. Further, this top-level specification provides the baseline for the development of all lower-level specifications to include development (Type B), product (Type C), process (Type D), and material (Type E) specifications. While there may be a variety of formats used in the preparation of the system specification, an example of one approach is presented in .

\subsection{Conceptual Design Review}\index{Conceptual Design Review}

Design progresses from an abstract notion to something that has form and function, is firm, and can ultimately be reproduced in designated quantities to satisfy a need. Initially, a need is identified. From this point, design evolves through a series of stages (i.e., conceptual design, preliminary system design, and detail design and development). In each major stage of the design process, an evaluative function is accomplished to ensure that the design is correct at that point before proceeding with the next stage. The evaluative function includes both the informal day-to-day project coordination and data/documentation review and the formal design review. Design information is released and reviewed for compliance with the basic system-equipment requirements (i.e., performance, reliability, maintainability, usability, supportability, sustainability, etc., as defined by the system specification). If the requirements are satisfied, the design is approved as is. If not, recommendations for corrective action are initiated and discussed as part of the formal design review.

The formal design review constitutes a coordinated activity (including a meeting or series of meetings) directed to satisfy the interests of the design engineer and the technical discipline support areas (reliability, maintainability, human factors, logistics, manufacturing engineering, quality assurance, and program management). The purpose of the design review is to formally and logically cover the proposed design from the total system standpoint in the most effective and economical manner through a combined integrated review effort. The formal design review serves a number of purposes.

\begin{enumerate}
\item It provides a formalized check (audit) of the proposed system/subsystem design with respect to specification requirements. Major problem areas are discussed and corrective action is taken as required
\item It provides a common baseline for all project personnel. The design engineer is provided the opportunity to explain and justify his or her design approach, and representatives from the various supporting organizations (e.g., maintainability, logistic support, and marketing) are provided the opportunity to learn of the design engineer’s problems. This serves as an excellent communication medium and creates a better understanding among design and support personnel
\item It provides a means for solving interface problems and promotes the assurance that all system elements will be compatible, internally and externally
\item It provides a formalized record of what design decisions were made and the reasons for making them. Analyses, predictions, and trade-off study reports are noted and are available to support design decisions. Compromises to performance, reliability, maintainability, human factors, cost, and/or logistic support are documented and included in the trade-off study reports
\item It promotes a higher probability of mature design, as well as the incorporation of the latest techniques (where appropriate). Group review may identify new ideas, possibly resulting in simplified processes and ultimate cost savings
\end{enumerate}

The formal design review, when appropriately scheduled and conducted in an effective manner, leads to reduction in the producer’s risk relative to meeting specification requirements and results in improvement of the producer’s methods of operation. Also, the customer often benefits from the receipt of a better product.

Design reviews are generally scheduled before each major evolutionary step in the design process, as illustrated in . In some instances, this may entail a single review toward the end of each stage (i.e., conceptual, preliminary system design, and detail design and development). For the other projects, where a large system is involved and the amount of new design is extensive, a series of formal reviews may be conducted on designated elements of the system. This may be desirable to allow for the early processing of some items while concentrating on the more complex, high-risk items.

Although the number and type of design reviews scheduled may vary from program to program, four basic types are readily identifiable and are common to most programs. They include the conceptual design review (i.e., system requirements review), the system design review, the equipment/software design review, and the critical design review. Of particular interest relative to the activities discussed in this section is the conceptual design review, which is dedicated to the review and validation of system operational requirements, maintenance and support concept, specified TPMs, and the functional analysis and allocation of requirements at the system level. Referring to , this review is usually conducted at the end of the conceptual design phase and prior to the accomplishment of preliminary design.

\subsection{Conceptual Design for the River Crossing Problem}\index{Conceptual Design for the River Crossing Problem}

Write paras about CD for the RCP

%------------------------------------------------

\section{Preliminary System Design}\index{Preliminary System Design}

The conceptual design process presented leads to the selection of a tentatively preferred, conceptual system design architecture or configuration. Top-level requirements, as defined there, enable early design evolution that follows in the preliminary system design phase. This phase of the life cycle progresses by addressing the definition and development of the preferred system concept and the allocated requirements for subsystems and the major elements thereof.

An essential purpose of preliminary design is to demonstrate that the selected system concept will conform to performance and design specifications, and that it can be produced and/or constructed with available methods, and that established cost and schedule constraints can be met. Some products of preliminary design include the functional analysis and allocation of requirements at the subsystem level and below, the identification of design criteria as an ``input'' to the design process, the application of models and analytical methods in conducting design trade-offs, the conduct of formal design reviews throughout the system development process, and planning for the <emphasis>detail design and development phase.

Referring to , implementation of the systems engineering process continues in a manner consistent with the steps followed earlier. Specifically, the functional analysis and allocation in  is extended to the next lower level in the system hierarchical structure in , specific design requirements in , and the design review process introduced in  is expanded in . Of particular significance is the utilization of continuous <emphasis>feedback</emphasis>, as shown in  and . The feedback loop is important for verifying that the initially specified requirements are indeed valid and providing a mechanism allowing for changes that enable corrective action and/or system improvement.

This secton leads directly to the material on detail design and development in and system test, evaluation, and validation in . It addresses the following steps in the systems engineering process:

\begin{itemize}
\item Developing design requirements for subsystems and major system elements from system-level requirements
\item Preparing <emphasis>development, product, process, and material specifications applicable to subsystems
\item Accomplishing functional analysis and allocation to and below the subsystem level
\item Establishing detailed design requirements and developing plans for their handoff to engineering domain specialists
\item Identifying and utilizing appropriate engineering design tools and technologies
\item Conducting trade-off studies to achieve design and operational effectiveness
\item Conducting design reviews at predetermined points in time
\end{itemize}

Completion of these steps implements the process illustrated by Blocks 1.1–1.7 in . While the depth, level of effort, and costs of accomplishing this may vary from one application to the next, the process outlined is applicable to the development of any type and size system. As a learning objective, the goal is to provide the reader with a comprehensive and valid approach for addressing preliminary design.

\subsection{Requirements Allocation to Preliminary Design}\index{Requirements Allocation to Preliminary Design}

Preliminary design</emphasis> requirements evolve from “system” design requirements, which are determined through the definition of system operational requirements, the maintenance and support concept, and the identification and prioritization of TPMs. These requirements are documented through the preparation of the <emphasis>system specification</emphasis> (Type <emphasis>A</emphasis>)<emphasis>,</emphasis> prepared in the conceptual design phase (refer to  and ). These requirements become the criteria by which preliminary design alternatives are judged.

The whats initiating conceptual design produce hows from the conceptual design evaluation effort applied to feasible conceptual design concepts. Next, the hows are taken into preliminary design through the means of allocated requirements. There they become whats and drive preliminary design to address hows at this lower level. This is a cascading process following the pattern exhibited in . It emanates from a process giving attention to what the system is intended to do before determining what the system is.

Requirements for the design of subsystems and the major elements of the system are defined through an extension of the functional analysis and allocation, the conduct of design trade-off studies, and so on. This involves an iterative process of top-down/bottom-up design, which continues until the next lowest level of system components are identified and configured.

Consider once again the regional public transportation authority facing the need to increase the capacity for two-way traffic flow across a river that separates a growing municipality. From the results of the conceptual design phase, a bridge spanning the river is selected from among other mutually exclusive river crossing alternatives. Each preliminary bridge design alternative is evaluated through consideration and analysis of its subsystem components. For example, if the pier and superstructure alternative is under evaluation, it will be the abutments, piers, and superstructure that have to be synthesized, sized, and evaluated. Trade-off and optimization is accomplished to determine the pier spacing that will minimize the sum of the first cost of piers and of superstructure, estimated cost of maintenance and support, projected end-of-life cost, and total life-cycle cost. The optimized result provides the basis for comparing this preliminary design alternative on an equivalent basis with other bridge design alternatives, to presented in 


In the river crossing example, each preliminary design alternative is evaluated through careful consideration and analysis of its subsystem components. For instance, if the pier and superstructure alternative is under evaluation, it will be the abutments, piers, and superstructure that will have to be synthesized, sized, and costed. Trade off and optimization should be accomplished to determine the pier spacing that will minimize the sum of the first cost of piers and superstructure. Then, with this optimized first-cost, life-cycle costing should be applied for maintenance and operations, to provide a basis for comparing this preliminary design with the other bridge types.

Lower-level requirements then emanate from the allocated requirements for the tentatively chosen (preliminary) bridge design. These lower-level requirements become the design criteria for subsystems and components of the pier and superstructure bridge. In this particular illustration, the major subsystems identified for the river crossing bridge include the basic road and railway bed, passenger walkway and bicycle path, toll facilities, the maintenance and support infrastructure, and others. Subsystems are further broken down into their respective system elements, such as, superstructure, substructures, piles and footings, foundations, retaining walls, and construction materials. These lower-level requirements are then documented through development, product</emphasis>, process, and/or material specifications. Design and development of these specific system elements is addressed in the detail design and development phase of .

\subsection{Development, Product, and Process Specifications}\index{Development, Product, and Process Specifications}

The technical requirements for the system and its elements are documented through a series of specifications, as indicated in . This series commences with the preparation of the system specification (Type A) prepared in the conceptual design phase (refer to  and ). This, in turn, leads to one or more subordinate specifications and/or standards covering applicable subsystems, configuration items, equipment, software, and other components of the system. In addition, there may be any number of supplemental ANSI (American National Standards Institute), EIA (Electronic Industries Alliance), IEC (International Electrotechnical Commission), ISO (International Organization for Standardization), and related standards that are required in support of the basic program-related specifications.

Although the individual specifications for a given program may assume a different set of designations, a generic approach is used throughout this text. The categories assumed herein are described below and illustrated in . Referring to the figure, the development of a specification tree is recommended for each program, showing a hierarchical relationship in terms of which specification has ``preference'' in the event of conflict. Further, it is critical that all specifications and standards be prepared in such a way as to ensure that there is a traceability of requirements from the top down. Preparation of the development specification (Type B), product specification (Type C), and so on, must include the appropriate TPM requirements that will support an overall system-level requirement; for example, operational availability (Ao) of 0.98. The traceability of requirements, through a specification tree, is particularly important in view of current trends pertaining to increasing globalization, greater outsourcing, and the increasing utilization of external suppliers, where variations often occur in implementing different practices and standards.

\begin{itemize}
\item System specification (Type A): includes the technical, performance, operational, and support characteristics for the system as an entity; the results of a feasibility analysis, operational requirements, and the maintenance and support concept; the appropriate TPM requirements at the system level; a functional description of the system; design requirements for the system; and an allocation of design requirements to the subsystem level (refer to 
\item Development specification (Type B): includes the technical requirements (qualitative and quantitative) for any new item below the system level where research, design, and development are needed. This may cover an item of equipment, assembly, computer program, facility, critical item of support, data item, and so on. Each specification must include the performance, effectiveness, and support characteristics that are required in the evolving of design from the system level and down.
\item Product specification (Type C): includes the technical requirements (qualitative and quantitative) for any item below the system level that is currently in inventory and can be procured “off the shelf.” This may cover any commercial off-the-shelf (COTS) equipment, software module, component, item of support, or equivalent
\item Process specification (Type D): includes the technical requirements (qualitative and quantitative) associated with a process and/or a service performed on any element of a system or in the accomplishment of some functional requirement. This may include a manufacturing process (e.g., machining, bending, and welding), a logistics process (e.g., materials handling and transportation), an information handling process, and so on
\item Material specification (Type E): includes the technical requirements that pertain to raw materials (e.g., metals, ore, and sand), liquids (e.g., paints and chemical compounds), semifabricated materials (e.g., electrical cable and piping), and so on.</para></listitem></orderedlist>
\end{itemize}

Each applicable specification must be direct, complete, and written in performance-related terms and must describe the appropriate design requirements in terms of the whats; that is, the function(s) that the item in question must perform. Further, the specification must be properly tailored to its application, and care must be taken to ensure that it is not overspecifying or underspecifying. While individual programs may vary in applying a different set of designations, or specific content within each specification, it is important that a complete top-down approach be implemented encompassing the requirements for the entire system and all of its elements.

\subsection{Functional Analysis and Allocation}\index{Functional Analysis and Allocation}

With the basic objectives in accomplishing a functional analysis described in and the process for the development of functional flow block diagrams (FFBDs) covered further in , the next step is to extend the functional analysis from the system level down to the subsystem and below as required. The depth of such an analysis (i.e., the breakdown in developing FFBDs from the system level to the second level, third level, and so on) will vary depending on the degree of visibility desired, whether new or existing design is anticipated, and/or to the level at which the designer wishes to establish some specific design-to requirements as an input. As mentioned earlier, it is important to establish the proper architecture describing structure, interrelationships, and related requirements.

The Functional Analysis Process. Referring to and A.2–A.7 in , there are a variety of illustrations showing a breakdown of functions into subfunctions and ultimately describing major subsystems. shows the general sequence of steps leading from the system level down to a communications subsystem (refer to Block 9.5.1). While this shows only one of the many subsystems required to meet an airborne transportation need, the same approach can be applied in defining other subsystems. The development of operational FFBDs can then lead to the development of maintenance FFBDs, as shown in . Given completion of the operational and maintenance FFBDs that reflect the whats, one must next determine the hows; that is, how will each function be accomplished? This is realized by evaluating each individual block of an FFBD, defining the necessary inputs and expected outputs, describing the external controls and constraints, and determining the mechanisms or the physical resources required for accomplishing the function; that is, equipment, software, people, facilities, data/information, or various combinations thereof. An example of the process is presented in 

Referring to the figure, note that just one of the blocks in is addressed, where the resource requirements are identified as mechanisms. As there may be a number of different approaches for accomplishing a given function, trade-off studies are conducted with a preferred approach being selected. The result leads to the determination and compilation of the resource requirements for each function, and ultimately for all of the functions included in the functional analysis.

In performing the analysis process depicted in , a documentation format, similar to that illustrated in (or something equivalent), should be used. In the figure, the functions pertaining to system design and development are identified along with required inputs, expected outputs, and anticipated resource requirements. While this particular example is ``qualitative'' by nature, there are many functions where specific metrics (i.e., TPMs) can be applied in the form of design-to ``constraints.'' Referring to , for example, there is a functional requirement at the system level that states, ``Operate the system in the user environment,'' represented by Block 9.0. Assuming that there is a system TPM requirement for operational availability (Ao) of 0.985 (refer to ), this measure of effectiveness will constitute a design requirement for the function in Block 9.0, and the appropriate resources must be applied accordingly. The same approach can be applied in relating all of the TPM requirements, which are specified for the system, to one or more functions. Accordingly, the purpose of including  is to promote a disciplined approach in accomplishing a functional analysis.

In conducting trade-off studies pertaining to the best approach in responding to a functional requirement (i.e., the mechanisms), the results may point toward the selection of hardware, software, people, facilities, data, or various combinations thereof. gives an example where the requirements for hardware, software, and the human are identified. Stemming from the functional analysis (Block 0.2 in ), the individual design and development steps for each is shown, and a plan is prepared for the acquisition of these system elements. From a systems engineering perspective, it is essential that these activities be coordinated and integrated, across the life cycles, from the beginning. In other words, an ongoing “communication(s)” must exist throughout the design and development of the hardware, software, and human elements of the system.

\subsection{Requirements Allocation}\index{Requirements Allocation}

Referring to , lower-level elements of the system are defined through the functional analysis and subsequently by partitioning (or grouping) similar functions into logical subdivisions, identifying major subsystems, configuration items, units, assemblies, modules, and so forth (refer to ). presents an overview of this process, evolving from a functional definition of System XYZ to the packaging of the system into three units; that is, Units A, B, and C.

Given the packaging concept shown in , it is now appropriate to determine the ``design-to'' requirements for each one of the three units. This is accomplished through the process of allocation (or apportionment). In the development of design goals at the unit level, priorities are established based on the TPMs for the system shown in , and both quantitative and qualitative design requirements are determined. Such requirements then lead to the incorporation of the appropriate design characteristics (attributes) in the design of Units A, B, and C. Such design characteristics, as shown in the hierarchy in , should be ``tailored'' in response to the relative importance of each as it impacts the system-level requirements. These design characteristics are initially viewed from the top down and are compared. Trade-off studies are conducted to evaluate the interaction effects, and those characteristics most significant in meeting the overall system objectives are selected.

presents an example resulting from the allocation process described. System-level TPMs are identified along with the design-to metrics for each of the three units, as well as the requirements for Assemblies 1, 2, and 3 within Unit B. The requirements at the system level have been allocated downward. Further, the requirements established at the unit level, when combined, must be compatible with the higher-level requirements. Thus, there is a top-down/bottom-up relationship, and there may be trade-offs conducted comprehensively at the unit level in order to achieve the proper balance of requirements overall. Meeting these quantitative design-to requirements leads to the incorporation of the proper characteristics in the design of the item in question.

The objective of including with all of its metrics is to emphasize the process and its importance early in system design. Although the metrics shown are primarily related to reliability, maintainability, availability, and design-to-life-cycle-cost factors, it should be noted that in the allocation process, one needs to include all performance factors, human factors, physical features, producibility and supportability factors, sustainability and disposability factors, and so on. Some of these measures are covered in detail in ), and the interrelationships should become clearer after reviewing those sectons.

In situations where there are a number of different systems in a system-of-systems (SOS) configuration, or where there are common functions, the allocation process becomes a little more complex. One needs to not only comply with the top-down/bottom-up traceability requirements for each system in the overall configuration but consider the compatibility (or operability) requirements among the systems as well. Referring to , for example, there are two systems, System ABC and System XYZ, within a given SOS structure, with a ``common'' function being shared by each. Given such, several possibilities may exist

\begin{enumerate}
\item One of the systems already exists, is operational, and the design is basically ``fixed,'' while the other system is new and in the early stages of design and development. Assuming that System ABC is operational, the design characteristics for the unit identified as being ``Common'' in the figure and required as a functioning element of ABC are essentially ``fixed.'' This, in turn, may have a significant impact on the overall effectiveness of System XYZ. In order to meet the overall requirements for XYZ, more stringent design input factors may have to be placed on the design of new Units C and D for that system. Or the common unit must be modified to be compatible with higher-level XYZ requirements, which (in turn) would likely impact the operational effectiveness of System ABC. Care must be taken to ensure that the overall requirements for both System ABC and System XYZ will be met. This can be accomplished by establishing a fixed requirement for the ``common unit'' and by modifying the design requirements for Assemblies 1, 2, and 3 within that unit to meet allocated requirements from both ABC and XYZ
\item Both System ABC and System XYZ are new and are being developed concurrently. The allocation process illustrated in is initially accomplished for each of the systems, ensuring that the overall requirements at the system level are established. The design-to characteristics for the ``common unit'' are compared and a single set of requirements is identified for the unit through the accomplishment of trade-off analyses, and the requirements for each of the various system units are then modified as necessary across-the-board while ensuring that the top system-level requirements are maintained. This may constitute an iterative process of analysis, feedback, and so on
\end{enumerate}

As a final point, those quantitative and qualitative requirements for the various elements of the system (i.e., subsystems, units, and assemblies) must be included in the appropriate specification, as identified in ). Further, there must be a top-down/bottom-up ``traceability'' of requirements throughout the overall hierarchical structure for each of the systems in question.

Applying the Functional Analysis. A major objective of systems engineering is to develop a complete set of requirements in order to define a single ``baseline'' from which all lower-level requirements may evolve; that is, to develop a functional baseline in conceptual design and later an allocated baseline in the preliminary system design phase (refer to ). The results of the functional analysis constitute a required input for a number of design-related activities that occur subsequently. Most important is breaking the system (and its elements) down into functional entities through functional packaging and the development of an open-architecture configuration. A prime objective is to develop a configuration that can be easily upgraded as required (through new technology insertions) and easily supported throughout its life cycle (through a modularized approach in maintenance). In addition, the functional analysis provides a foundation upon which many of the subsequent analytical tasks and associated documentation are based, some of which are listed below:

\begin{enumerate}
\item Reliability analysis: reliability models and block diagrams; failure mode, effects, and criticality analysis (FMECA); fault-tree analysis (FTA); reliability prediction (refer to 
\item Maintainability analysis: maintainability models; reliability-centered maintenance (RCM); level-of-repair analysis (LORA); maintenance task analysis (MTA); total productive maintenance (TPM); maintainability prediction (refer to 
\item Human factors analysis: operator task analysis (OTA); operational sequence diagrams (OSDs); safety/hazard analysis; personnel training requirements (refer to 
\item Maintenance and logistic support: supply chain and supportability analysis leading to the definition of maintenance and support requirements—spares/repair parts and associated inventories, test and support equipment, transportation and handling equipment, maintenance personnel, facilities, technical data, information (refer to 
\item Producibility, disposability, and sustainability analysis (refer to 
\item Affordability analysis: life-cycle and total ownership cost (refer to 
\end{enumerate}

All of these activities, as described throughout of this textbook, are dependent and based on functional analysis as an essential input.

Preliminary Design Criteria. The basic design objective(s) for the system and its elements must (1) be compatible with the system operational requirements, maintenance and support concept, and the prioritized TPMs; (2) comply with the allocated design-to criteria described in .2; and (3) meet all of the requirements in the various applicable specifications. The particular design characteristics to be incorporated will vary from one instance to the next, depending on the type and complexity of the system and the mission or purpose that it is intended to accomplish. In all cases, the design team activity must address the downstream life-cycle outcomes considering the phases of production and construction, system utilization and sustaining support, and retirement and material recycling/disposal. While all considerations in system design must be addressed (refer to ), a few require some additional emphasis. These are clustered in Part IV.

Design Engineering Activities. The day-to-day design activities begin with the implementation of the appropriate planning that was initiated in the conceptual design phase (i.e., the program management and system engineering management plans identified in ). This includes the establishment of the design team and the initiation of specific design tasks, the ongoing liaison and working with various responsible designers throughout the project organization, the development of design data, the accomplishment of periodic design reviews, and the initiation of corrective action as necessary.</para>

Trade Off Studies and Design Definition. </title><para>As the design evolves, the system synthesis, analysis, and evaluation process, described in and , continues. Proposed configurations for subsystems and major elements of the system are synthesized, trade-off studies are conducted, alternatives are evaluated, and a preferred design approach is selected. This process continues throughout the conceptual design, preliminary system design, and the detail design and development phases, leading to the definition of the system configuration down to the detailed component level (refer to 

This iterative process of systems analysis is shown in a generic and simplified context in . Referring to the figure (which complements ), a key challenge is the application of the appropriate analytical techniques and models. As conveyed in the previous section, knowledge of the analytical methods described in  of this text is required to accomplish the steps reflected by Blocks 4 and 5 (5a and 5b) of . Further, when selecting and/or developing a specific analytical model/tool to facilitate the analysis effort, care must be exercised to ensure that the right computer-based tool is selected for the application intended.

shows the application of a number of different analytical models used in the evaluation of alternative maintenance and logistic support policies. This illustration presents just one of a number of examples where there may be a multiple mix (or combination) of tools used to evaluate a specific design configuration. While the approach conveyed in the figure may not appear to be unique, it should be noted that there are many different computer-based models that are available in the commercial market and are advertised as solutions to a wide variety of problems. Most of these models were developed on a relatively ``independent'' or ``isolated'' basis in terms of selected platform, context or computer language, input data requirements, varying degrees of ``user friendliness,'' and so on. In general, many of the models advertised today do not ``talk to each other,'' are too complex, require too much input data, and can only be effectively used in the ``downstream'' portion of the life cycle during the detail design and development phase when there are a lot of data available.

From a systems engineering perspective, a good objective is to select or develop an integrated design workstation (incorporating a Macro-CAD approach) that can be utilized in all phases of the system life cycle and that can be adapted to the different levels of design definition as one progresses from the conceptual design to the detail design and development phase (refer to ). For example, this workstation should incorporate the right tools that can be applied at a high level in conceptual design and again at a more in-depth level in detail design and development. Theoretically, it should be possible to utilize any one (or all) of the tools described throughout Parts III and IV of this text at any time in the system life cycle, where they can be ``tailored'' to the degree of design definition as required.

\subsection{Design Review and Evaluation}\index{Design Review and Evaluation}

The basic objectives and benefits of the design review, evaluation, and feedback process are as described in , and include two facets of the activity shown in . First, there is an ongoing informal review and evaluation of the results of the design, accomplished on a day-to-day basis, where the responsible designer provides applicable technical data and information to all project personnel as the design progresses. Through subsequent review, discussion, and feedback, the proposed design is either approved or recommended changes are submitted for consideration. Second, there is a structured series of formal design reviews conducted at specific times in the overall system development process. While the specific types, titles, and scheduling of these formal reviews will vary from one program to the next, it is assumed herein that formal reviews will include the following:

The Conceptual Design is usually scheduled toward the end of the conceptual design phase and prior to entering the preliminary system design phase of the program. The objective is to review and evaluate the requirements and the functional baseline for the system, and the material to be covered through this review should include the results from the feasibility analysis, system operational requirements, the maintenance and support concept, applicable prioritized TPMs, the functional analysis (top level for the system), system specification (Type <emphasis>A</emphasis>), a systems engineering management plan (SEMP), a test and evaluation master plan (TEMP), and supporting design criteria and data/documentation. Refer to  and 

System design reviews</emphasis> are generally scheduled during the preliminary system design phase when functional requirements and allocations are defined, preliminary design layouts and detailed specifications are prepared, system-level trade-off studies are conducted, and so on. These reviews are oriented to the overall system configuration (as subsystems and major system elements are defined), rather than to individual equipment items, software, and other lower-level components of the system. There may be one or more formal reviews scheduled, depending on the size of the system and the complexity of design. System design reviews may cover a variety of topics, including the following: functional analysis and the allocation of requirements; development, product, process, and material specifications (Types <emphasis>B</emphasis>, <emphasis>C</emphasis>, <emphasis>D</emphasis>, and <emphasis>E</emphasis>); applicable TPMs; significant design criteria for major system elements; trade-off study and analysis reports; predictions; and applicable design data (layouts, drawings, parts/material lists, supplier reports, and data)

Equipment/software design reviews</emphasis> are scheduled during the detail design and development phase and usually cover such topics as product, process, and material specifications (Types <emphasis>C</emphasis>, <emphasis>D</emphasis>, and <emphasis>E</emphasis>); design data defining major subsystems, equipment, software, and other elements of the system as applicable (assembly drawings, specification control drawings, construction drawings, installation drawings, logic drawings, schematics, materials/parts lists, and supplier data); analyses, predictions, trade-off study reports, and other related design documentation; and engineering models, laboratory models, mock-ups, and/or prototypes used to support a specific design configuration

The critical design review</emphasis> is generally scheduled after the completion of detailed design, but prior to the release of firm design data for production and/or construction. Design is essentially ``fixed'' at this point, and the proposed configuration is evaluated in terms of adequacy, producibility, and/or constructability. The critical design review may include the following topics: a complete package of final design data and documentation; applicable analyses, trade-off study reports, predictions, and related design documentation; detailed production/construction plans; operational and sustainability plans; detailed maintenance plans; and a system retirement and material recycling/disposal plan. The results of the critical design review describe the final system configuration product baseline prior to entering into production and/or construction.

The review, evaluation, and feedback process is continuous throughout system design and development and, as indicated in and , encompasses conceptual, preliminary, and detail design. The objective in is to introduce the review process and describe some of the benefits that can be derived from it, to provide an overall spectrum of activity and the types and scheduling of reviews, and to describe some specifics leading to further implementing the process in

%------------------------------------------------

\section{Detail Design and Development}\index{Detail Design and Development}

Having established the top-level requirements for the overall system as in Section 8.2 and the preliminary design requirements as in Section 8.3, the design process from this point forward is essentially evolutionary by nature. Referring to Figure 4.9, the design team has been established with the overall objective of integrating the various system elements into a final system configuration. Such elements include not only mission-related hardware and software, but people, real estate and facilities, data/information, consumables, and the resources necessary for the operation and sustaining support of the system throughout its planned life cycle. The integration of these elements is emphasized in Figure 5.2, which is an extension of Figure 4.9.

Include Figure 5.2 Here

Returning again to the example of a regional public transportation authority facing the problem of increasing the capacity for two-way traffic flow across a river to connect communities. The output from the Conceptual Design phase (in Chapter 3) was the tentative selection of an over-the-water approach from among several mutually exclusive river-crossing concepts. Its effect was to initiate and support Preliminary Design phase activities (in Chapter 4) to investigate a number of competing bridge configuration alternatives. The pier and superstructure configuration was indicated there as being the preferred approach. Choice of this approach now enables the Detail Design phase, with focus on static and operational subsystems and associated components.

Allocated requirements emanate from the tentatively chosen (preliminary) design and are of a lower-level appropriate for use in the design and development of components comprising the pier and superstructure bridge system. These lower-level requirements become the design criteria for subsystems and components, components that include the piers and superstructure plus other system components such as abutments, footings, toll collection, maintenance capability, and so forth.

Preliminary design evolves from ``system'' design decisions, which are determined through the definition of system operational requirements, the maintenance and support concept, and the identification and prioritization of TPMs. They are the criteria by which preliminary design alternatives are judged on the way to selecting a preferred preliminary design. These requirements are then documented through the preparation of the system specification (Type A), described in Section 3.9. From this top-level specification, requirements for the design of subsystems and the major elements of the system are defined through an extension of the functional analysis and allocation, the conduct of design trade-off studies, and so on. This involves an iterative process of top-down/bottom-up design, which continues until the next lowest level of system components are identified and configured.

Lower-level requirements then emanate from the allocated requirements from the tentatively chosen (preliminary) bridge design. These lower-level requirements become the design criteria for subsystems and components of the pier and superstructure bridge, components that include the piers and superstructure plus other system components such as abutments, footings, toll booths, and maintenance capability. These lower-level requirements are then specified through development, product, process, and/or material specifications.</title><section id=”ch05lev1fm” role=”fm”><title id=”ch05lev1fm.title”/></section>

The detail design and development phase of the system life cycle is a continuation of the iterative development process illustrated in  and . The definition of system requirements, the establishment of a top-level functional baseline, and the preparation of a system specification (Type A) are described in . This, in turn, leads to the definition and development of subsystems and major elements of the system as described in . Functional analysis, the allocation of design-to requirements below the system level, the accomplishment of synthesis and trade-off studies, the preparation of lower-level specifications (Types B-E), and the conduct of formal design reviews all provide a foundation upon which to base detailed design decisions that go down to the component level.

With the functional baseline developed as an output of the conceptual system design phase and with the allocated baseline derived during the preliminary design phase, the design team may now proceed with reasonable confidence in the realization of specific components as well as the ``make-up'' of the system configuration at the lowest level in the hierarchy. Realization includes the accomplishment of activities that (1) describe subsystems, units, assemblies, lower-level components, software modules, people, facilities, elements of maintenance and support, and so on, that make up the system and address their interrelationships; (2) prepare specifications and design data for all system components; and (3) acquire and integrate the selected components into a final system configuration.

This section addresses eight essential steps in the detail design and development process, while simultaneously providing an understanding of the intricacy of detail design and development within the systems engineering context. This includes

\begin{itemize}
\item Developing design requirements for all lower-level components of the system
\item Implementing the necessary technical activities to fulfill all design objectives
\item Integrating system elements and activities
\item Selecting and utilizing design tools and aids
\item Preparing design data and documentation
\item Developing engineering and prototype models
\item Implementing a design review, evaluation, and feedback capability
\item Incorporating design changes as appropriate
\end{itemize}

It is important for these steps to be thoroughly understood and reviewed, as they fall within the overall systems engineering process. As a learning objective, the intent is to provide a relatively comprehensive approach that addresses the detailed aspects of design. The chapter summary and extensions contribute insights by going beyond the general references cited in and by calling attention to applicable design standards and supporting documentation.

Specific requirements at this stage in the system design process are derived from the system specification (Type <emphasis>A</emphasis>) and evolve through applicable lower-level specifications (Types B–E), as illustrated in . Included within these specifications are applicable design-dependent parameters (DDPs), technical performance measures (TPMs), and supporting design-to criteria leading to identification of the specific characteristics (attributes) that must be incorporated into the design configuration of elements and components. This is influenced through the requirements allocation process illustrated in , where the appropriate built-in characteristics must be such that the allocated quantitative requirements in the figure will be met.

Given this top-down approach for establishing requirements at each level in the system hierarchical structure (refer to ), the design process evolves through the iterative steps of synthesis, analysis, and evaluation, and to the definition of components leading to the establishment of a <emphasis>product baseline</emphasis>, as shown in . At this point, the procurement and acquisition of system components begin, components are tested and integrated into a next higher entity (e.g., subassembly, assembly, and unit), and a physical model (or replica) of the system is constructed for test and evaluation. The integration, test, and evaluation steps constitute a bottom-up approach, and should result in a configuration that can be assessed for compliance with the initially specified customer requirements. This top-down/bottom-up approach is guided by the steps in the ``vee'' process model shown in 

Progressing through the system design and development process in an expeditious manner is essential in today’s competitive environment. Minimizing the time that it takes from the initial identification of a need to the ultimate delivery of the system to the customer is critical. This requires that certain design activities be accomplished on a concurrent basis. (an extension of ) further illustrates the importance of ``concurrency'' in system design.

Referring to , the designer(s) must think in terms of the four life cycles and their interrelationships, concurrently and in an integrated manner, in lieu of the sequential approach to design often followed in the past. The realization of this necessity became readily apparent in the late 1980s and resulted in concepts promulgated as <emphasis>simultaneous engineering, concurrent engineering, integrated product development</emphasis> (IPD), and others. These concepts must be inherent within the systems engineering process if the benefits of that process are to be realized.

\subsection{The Evolution of Detail Design}\index{The Evolution of Detail Design}

The evolution of detail design is based on the results from the requirements established during the conceptual and preliminary system design phases. As an example, for the river crossing problem addressed by a bridge system, the primary top-level requirements were identified in conceptual design (refer to  and ). These requirements were decomposed further in 

Illustration 1, expanded during the preliminary design phase through the functional analysis and allocation of requirements down to major subsystems to include the roadway and railbed, passenger walkway and bicycle path, toll collection facilities, and the maintenance and support infrastructure ; and then the more detailed design-to requirements for the various lower-level elements of the system are defined (in this chapter) down to the bridge substructures, foundations, piles and footings, retaining walls, toll collectors, lighting, and so on. This top-down/bottom-up process is illustrated in 

Referring to the figure, the design-to requirements are identified from the ``top down,'' with the cross-hatched block defined in conceptual design along with the allocation and requirements for the major subsystems noted by the shaded blocks (superstructure, substructure, etc.). These requirements are further expanded, through functional analysis and allocation, during the preliminary design phase to define the specific requirements for the lower-level elements of the bridge represented by the white blocks in the figure (road and railway deck, piles/piers, footings, toll facilities, etc.). The basic ``requirements'' for the river crossing bridge are defined and allocated (i.e., ``driven'') from the ``top down,'' while the detailed design and follow-on construction is accomplished from the ``bottom up.'' This is why one often invokes a top-down/bottom-up process versus assuming only a ``bottom-up'' approach (refer to 

Having established the basic top level requirements for the overall system as in and the preliminary design requirements as in the design process from this point forward is essentially evolutionary by nature as described earlier. Referring to , the design team has been established with the overall objective of integrating the various system elements into a final system configuration. Such elements include not only mission-related hardware and software but also people, real estate and facilities, data/information, consumables, and the materials and resources necessary for the operation and sustaining support of the system throughout its planned life cycle. The integration of these elements is emphasized in , an extension of 

Detail design evolution follows the basic sequence of activities shown in . The process is iterative, proceeding from system-level definition to a product configuration that can be constructed or produced in multiple quantities. There are “checks and balances” in the form of reviews at each stage of design progression and a feedback loop allows for corrective action as necessary. These reviews may be relatively informal and occur continuously, as compared with the formal design reviews scheduled at specific milestones. In this respect, the process is similar to the synthesis, analysis, evaluation, and product definition, accomplished in the preliminary design stage, except that the requirements are at a lower level (i.e., units, assemblies, subassemblies, etc.). As the level of detail increases, actual definition is accomplished through the development of data describing the item being designed. These data may be presented in the form of a digital description of the item(s) in an electronic format, design drawings in physical and electronic form, material and part lists, reports and analyses, computer programs, and so on.

The actual process of design iteration may occur through the use of the World Wide Web (WWW), a local area network (LAN), the use of telecommunications and compressed video conferences, or some equivalent form of communication. The design configuration selected may be the best possible in the eyes of the responsible designer. However, the results are practically useless unless properly documented, so that others can first understand what is being conveyed and then be able to translate and convert the output into an entity that can be constructed (as is the case for the river crossing bridge) or produced in multiple quantities (see on product and system categories).

\subsection{Integrating System Elements and Activities}\index{Integrating System Elements and Activities}

An important output from the functional analysis and allocation process is the identification of various elements of the system and the need for hardware (equipment), software, people, facilities, materials, data, or combinations thereof. The objective is to conduct the necessary trade-off analyses to determine the best way to respond to the hows. The result may take the form illustrated in and , which identify the major elements of a system.

Given the basic configuration of system elements, the designer must decide how best to meet the need in selecting a specific approach in responding to an equipment need, a software need, and so on. For example, there may be alternative approaches in selecting a specific resource, such as illustrated in , with the following steps taken (in order of precedence) in arriving at a satisfactory result:

\begin{enumerate}
\item Select a standard component that is commercially available and for which there are a number of viable suppliers; for example, a commercial off-the-shelf (COTS) item, or equivalent. The objective is, of course, to gain the advantage of competition (at reduced cost) and to provide the assurance that the appropriate maintenance and support will be readily available in the future and throughout the system life cycle when required, or
\item Modify an existing commercially available off-the-shelf item by providing a mounting for the purposes of installation, adding an adapter cable for the purposes of compatibility, providing a software interface module, and so on. Care must be taken to ensure that the proposed modification is relatively simple and inexpensive and doesn’t result in the introduction of a lot of additional problems in the process, or
\item Design and develop a new and unique component to meet a specific functional requirement. This approach will require that the component selected be properly integrated into the overall system design and development process in a timely and effective manner
\end{enumerate}

The most cost-effective solution seems to favor the utilization of commercial off-the-shelf (COTS) components, as the acquisition cost, item availability time, and risks associated with meeting a given system technical requirement are likely to be less. In any event, the decision-making process may occur at the subsystem level early in preliminary design, at the configuration-item level, and/or at the unit level. These decisions will be based on factors including the functions to be performed, availability and stability of current technology, number of sources of supply and supplier response times, reliability and maintainability requirements, supportability requirements, cost, and others. From the resulting decisions, the requirements for component acquisition (procurement) will be covered through the preparation of either a product specification (Type C) or <emphasis>process specification</emphasis> (Type D) for COTS components, or a development specification (Type B) for newly designed or modified components.

When the ultimate decision specifies the design and development of a new, or unique, system element, the follow-on design activity will include a series of steps, which evolve from the definition of need to the integration and test of the item as part of the system validation process. For example, the functional analysis in leads to the identification of <emphasis>hardware</emphasis>, <emphasis>software</emphasis>, and <emphasis>human</emphasis> requirements. The development of hardware usually leads to the identification of units, assemblies, modules, and down to the component-part level. The software acquisition process often involves the identification of computer software configuration items (CSCIs), computer software components (CSCs) and computer software units (CSUs), the development of code, and CSC integration and testing. The development of human requirements includes the identification of individual tasks, the combining of tasks into position descriptions, the determination of personnel quantities and skill levels, and personnel training requirements.

While  addresses only the three ``mini'' life cycles mentioned (i.e., hardware, software, and human), there may be other resource requirements evolving from the functional analysis to include facilities, data/documentation, elements of the maintenance and support infrastructure, and so on.. Associated with each will be a life cycle unique to the particular application.

There must be some form and degree of communication and integration throughout and across these component life cycles on a continuing basis. Quite often, design engineers become so engrossed in the development of hardware that they neglect the major interfaces and the impact that hardware design decisions have on the other elements of the system. Software engineers, like hardware engineers, often operate strictly within their own domain without considering the necessary interfaces with the hardware development process. On occasion, the hardware development process will lead to the need for software at a lower level. Further, consideration of the human being, as an element of the system, is often ignored in the design process altogether. Because of these practices in the past, the integration of the various system elements has not occurred until late in the detail design and development phase during systems integration and testing (refer to Block 2.3 in , often resulting in incompatibilities and the need for last-minute costly modifications in order to ``make it work!''

Thus, a primary objective of systems engineering is to ensure the proper coordination and timely integration of all system elements (and the activities associated with each) from the beginning. This can be accomplished with a good initial definition of requirements through the preparation of well-written specifications, followed by a structured and disciplined approach to design. This includes the scheduling of an appropriate number of formal design reviews to ensure the proper communications across the project and to check that all elements of the system are compatible at the time of review. In the event of incompatibilities, the hardware design effort should not be allowed to proceed without first ensuring that the software is compatible; the software development effort should not be allowed to proceed without first ensuring compatibility with the hardware; and so on.

\subsection{Design Tools and Aids}\index{Design Tools and Aids}

The successful completion of the design process depends on the availability of the appropriate tools and design aids that will help the design team in accomplishing its objectives in an effective and efficient manner. The application of computer-aided engineering (CAE) and computer-aided design (CAD) tools enables the projection of many different design alternatives throughout the life cycle. At the early stages of design, it is often difficult to visualize a system configuration (or element thereof) in its true perspective, whereas simulating a three-dimensional view of an item is possible through the use of CAD. In some instances, the validation of system requirements can be accomplished through the use of simulation methods during the preliminary system design and detail design and development phases prior to the introduction of hardware, software, and so on. Through the proper use of these and related technologies (to include the appropriate application of selected analytical models), the design team is able to produce a robust design more quickly, while reducing the overall program technical risks.

As an additional aid to the designer, physical three-dimensional scale models or <emphasis>mock-ups</emphasis> are sometimes constructed to provide a realistic simulation of a proposed system configuration. Models, or mock-ups, can be developed to any desired scale and to varying depths of detail depending on the level of emphasis desired. Mock-ups can be developed for large as well as small systems and may be constructed of heavy cardboard, wood, metal, or a combination of different materials. Mock-ups can be developed on a relatively inexpensive basis and in a short period of time when employing the right materials and personnel services. Industrial design, human factors, or model-shop personnel are usually available in many organizations, are well-oriented to this area of activity, and should be utilized to the greatest extent possible. Some of the uses and values of a mock-up are that they:

\begin{enumerate}
\item Provide the design engineer with the opportunity of experimenting with different facility layouts, packaging schemes, panel displays, cable runs, and so on, before the preparation of final design data. A mock-up or engineering model of the proposed river crossing bridge can be developed to better visualize the overall structure, its location, interfaces with the communities on each side of the river, and so on
\item Provide the reliability–maintainability–human factors engineer with the opportunity to accomplish a more effective review of a proposed design configuration for the incorporation of supportability characteristics. Problem areas readily become evident
\item Provide the maintainability-human factors engineer with a tool for use in the accomplishment of predictions and detailed task analyses. It is often possible to simulate operator and maintenance tasks to acquire task sequence and time data
\item Provide the design engineer with an excellent tool for conveying his or her final design approach during a formal design review.</para></listitem>
\item Serve as an excellent marketing tool
\item Can be employed to facilitate the training of system operator and maintenance personnel
\item Can be utilized by production and industrial engineering personnel in developing fabrication and assembly processes and procedures and in the design of factory tooling and associated test fixtures
\item Can serve as a tool at a later stage in the system life cycle for the verification of a modification kit design prior to the preparation of formal data and the development of kit hardware, software, and supporting materials
\end{enumerate}

In the domain of software development, in particular, designers are oriented toward the building of ``one-of-a-kind'' software packages. The issues in software development differ from those in other areas of engineering in that mass production is not the normal objective. Instead, the goal is to develop software that accurately portrays the features that are desired by the user (customer). For example, in the design of a complex workstation display, the user may not at first comprehend the implications of the proposed command routines and data format on the screen. When the system is ultimately delivered, problems occur, and the ``user interface'' is not acceptable for one reason or another. Changes are then recommended and implemented, and the costs of modification and rework are usually high.

The alternative is to develop a ``protoype'' early in the system design process, design the applicable software, involve the user in the operation of the prototype, identify areas that need improvement, incorporate the necessary changes, involve the user once again, and so on. This iterative and evolutionary process of software development, accomplished throughout the preliminary and detail design phases, is referred to as <emphasis>rapid prototyping</emphasis>. Rapid prototyping is a practice that is often implemented and is inherent within the systems engineering process, particularly in the development of large software-intensive systems.

\subsection{Incorporating Design Changes}\index{Incorporating Design Changes}

After a baseline has been established as a result of a formal design review, changes are frequently initiated for any one of a number of reasons such as to correct a design deficiency, improve a product, incorporate a new technology, improve the level of sustainability, respond to a change in operational requirements, compensate for an obsolete component, and so on. Changes may be initiated from within the project, or as a result of some new externally imposed requirement.

At first, it may appear that a change is relatively insignificant in nature and that it may constitute a change in the design of a prime equipment item, a software modification, a data revision, and/or a change in some process. However, what might initially appear to be minor often turns out to have a great impact on and throughout the system hierarchical structure. For instance, a change in the design configuration of prime equipment (e.g., a change in size, weight, repackaging, and added performance capability) will probably affect related software, design of test and support equipment, type and quantity of spares/repair parts, technical data, transportation and handling requirements, and so on. A change in software will likely have an impact on associated prime equipment, technical data, and test equipment. A change in any one item will likely have an impact on many other elements of the system. Further, if there are numerous changes being incorporated at the same time, the entire system configuration may be severely compromised in terms of maintaining some degree of requirements traceability.

Past experience with a variety of systems has indicated that many of the changes incorporated are introduced late in detail design and development, during production or construction, and/or early in the system utilization and sustaining support phase. As shown in , the accomplishment of changes this far downstream in the life cycle can be very costly; for example, a small change in an equipment item can result in a subsequent change in software, technical data, facilities, the various elements of support, or a production process.

While the incorporation of changes (for one reason or another) is certainly inevitable, the process for doing this must be formalized and controlled to ensure traceability from one configuration baseline to another. Also, it is necessary to ensure that the incorporation of a change is consistent with the requirements in the system specification (Type A). Referring to , a proposed change is initially submitted through the preparation of an engineering change proposal (ECP), which, in turn, is reviewed by a Change Control Board (or Configuration Control Board). Each proposed change must be thoroughly evaluated in terms of its impact on other elements of the system, the specified TPMs, life-cycle cost, and the various considerations that have been addressed throughout the earlier stages of design (e.g., reliability, maintainability, human factors, producibility, sustainability, and disposability). Approved changes will then lead to the development of the required modification kit, installation instructions, and the ultimate incorporation of the change in the system configuration. Accordingly, there needs to be a highly disciplined configuration management (CM) process from the beginning and throughout the entire system life cycle. This is particularly important in the successful implementation of the systems engineering process.

%------------------------------------------------

\section{Design Evaluation and Validation}\index{Design Evaluation and Validation}

The basic objectives and benefits of the design review, evaluation, and feedback process are as described in , and the categories (types) and scheduling of formal design reviews are shown in . As mentioned in , the design review and evaluation activity is continuous and includes both (1) an informal ongoing iterative day-to-day process of review and evaluation, and (2) the scheduling of formal design reviews at discrete points in design and throughout system acquisition. The purpose of conducting any type of a review is to assess if (and just how well) the design configuration, as envisioned at the time, is in compliance with the initially specified quantitative and qualitative requirements. The technical performance measures (TPMs), identified and prioritized in the conceptual design phase (refer to ) and allocated to the various elements of the system (refer to ), must be measured to ensure compliance with these specified requirements. Further, these critical measures must be ``tracked'' from one review to the next.

Five key TPMs have been selected in to illustrate the tracking of TPMs. It is assumed that the specific quantitative requirements have been established, along with an allowable “target” range for each of the values. For instance, the system operational availability (Ao) must be at least 0.90, but can be higher if feasible. As a result of a system design review, based on a prediction associated with the design configuration at the time, it was determined that Ao was only around 0.82. Given this, there is a need to develop a formal design plan indicating the steps that will be necessary to realize the required reliability growth in order to comply with the 0.90 requirement. Each of the specified TPMs must be monitored in a similar manner.

For relatively large systems, there may be several distinct TPM requirements that must be met and there should be priorities established to indicate relative degrees of importance. While all requirements may be important, there is a tendency on the part of some designers to favor one TPM requirement over another. Given that system design requires a ``team'' approach and that the different team members may represent different design disciplines, there may be a greater amount of effort expended to ensure that the design will meet one requirement at the expense of not complying with another. For instance, if the system being developed is an aerospace system, then weight, size, and speed are critical and the impact from an aeronautical engineering perspective is high. On the other hand, meeting an MLH/OH requirement may not be of concern to the aeronautical engineer assigned to the design team, but is highly important when considering the maintainability characteristics in design.

To ensure that all of the requirements are met, or at least seriously addressed, various design team members may be assigned to ``track'' specific TPMs throughout the design process. illustrates the relationships between TPMs and responsible design disciplines (in terms of likely interest) and indicates where the ``tracking'' assignments may be specified. In other words, a reliability engineer will likely ``track'' the requirements pertaining to the MTBF, a maintainability engineer will be interested in MLH/OH, MTBM, and LCC, and so on. From a systems engineering perspective, all of these factors must be ``tracked'' to the extent indicated by the distributed priorities.

As the designer evolves through the detail design and development process, including the conduct of the essential trade-off analyses leading to the selection of specific components, he/she must be cognizant of considerations other than the obvious TPMs. Two of these are presented here as examples.

The expected life of each of the components chosen for incorporation into the design is a consideration of importance. If the specified component ``life'' is less than the planned life cycle of the system (i.e., a ``critical useful-life item''), then this leads to the requirement for a scheduled maintenance plan for component replacement. If, when combining and integrating the various components into a larger assembly, the expected life is less than the planned life cycle of the system, then once again there may be a need for a preplanned system maintenance cycle or a redesign for mitigation of this burden. In any case, the designer must track and address the issue of obsolescence in design (as in ) ``before the fact'' instead of leaving it to chance later downstream in the system life cycle.

Another consideration to be tracked in today’s environment pertains to implementing the overall system design process effectively, in a limited amount of time, and at reduced cost. Shortening the acquisition process (cycle) continues to be a desired objective. At the same time, there has been a tendency to continue the design process, on an evolutionary basis, for as long as possible and incorporate design improvements up to the last minute before delivering the system to the customer. As desirable as this might appear, a superb system delivered late and over budget may be considered a failure by the customer.

Design reviews are scheduled periodically. Any review depends on the depth of planning, organization, and preparation prior to the review itself. An extensive amount of coordination is needed, involving the following factors:

\begin{enumerate}
\item Identification of the items to be reviewed
\item A selected date for the review
\item The location or facility where the review is to be conducted
\item An agenda for the review (including a definition of the basic objectives)
\item A design review board representing the organizational elements and disciplines affected by the review. Basic design functions, reliability, maintainability, human factors, quality control, manufacturing, sustainability, and logistic support representation are included. Individual organizational responsibilities should be identified. Depending on the type of review, the customer and/or individual equipment suppliers may be included
\item Equipment (hardware) and/or software requirements for the review. Engineering models, prototypes and/or mock-ups may be required to facilitate the review process
\item Design data requirements for the review. This may include all applicable specifications, lists, drawings, predictions and analyses, logistic data, computer data, and special reports
\item Funding requirements. Planning is necessary in identifying sources and a means for providing the funds for conducting the review
\item Reporting requirements and the mechanism for accomplishing the necessary follow-up actions stemming from design review recommendations. Responsibilities and action-item time limits must be established
\end{enumerate}

The design review involves a number of different discipline areas and covers a wide variety of design data and in some instances the presence of hardware, software, and/or other selected elements of the system. In order to fulfill its objective expeditiously (i.e., review the design to ensure that all system requirements are met in an optimum manner), the design review must be well organized and firmly controlled by the design review board chairperson. Design review meetings should be brief and to the point and must not be allowed to drift away from the topics on the agenda. Attendance should be limited to those who have a direct interest and can contribute to the subject matter being presented. Specialists who participate should be authorized to speak and make decisions concerning their area of specialty. Finally, the design review must make provisions for the identification, recording, scheduling, and monitoring of any subsequent corrective action that is required. Specific responsibility for follow-up action must be designated by the chairperson of the design review board.

I know about exercising a) the system model then b) the system as is being realized then c) the system as interoperating with its real context all for the purpose of discovering dynamic and integrity limits. The key is the viability of the test scenarios. In a system consisting of hundreds of variables and hundreds factorial of data states no contrived test cases are going to find the integrity limits even if they cover 100\% of the algorithms.
Is it time to understand systems integrity assessment as an analysis practice. J

System test, evaluation, and validation activities should be established during the conceptual design phase of the life cycle, concurrently with the definition of the overall system design requirements. From that point on, the test and evaluation effort continues by the testing of individual components, the testing of various system elements and major subsystems, and then by the testing of the overall system as an integrated entity. The objective is to adopt a ``progressive'' approach that will lend itself to continuous implementation and improvement as the system design and development process evolves.

Test and evaluation activities discussed in this chapter can be aligned initially with the design activities described in and then extended through the production/construction and the system utilization and support phases. These chapters address an evolutionary treatment of the system design process and ``look ahead'' to downstream outcomes where the results will have defined a specific configuration, supported by a comprehensive design database and augmented by supplemental analyses.

The next step is that of validation. Validation, as defined herein, refers to the steps and the process needed to ensure that the system configuration, as designed, meets all requirements initially specified by the customer. The process of validation is somewhat evolutionary in nature. Referring to , the activities noted by Blocks 0.6, 1.5, 2.3, and 8.1.4 invoke an ongoing process of review, evaluation, feedback, and the ultimate verification of requirements. These activities have been primarily directed to address early design concepts and various system elements. However, up to this point, a total integrated approach for the validation of the system and its elements, as an integrated entity, has not been fully accomplished. Final system validation occurs when the system performs effectively and efficiently when operating within its associated higher-level system-of-systems (SOS) configuration (as applicable).

The purpose of this chapter is to present an integrated approach for system test and evaluation and to facilitate the necessary validation of the proposed system configuration to provide assurance that it will indeed meet customer requirements. This chapter addresses the following topics:

\begin{itemize}
\item Determining the requirements for system test, evaluation, and validation
\item Describing the categories of system test and evaluation
\item Planning for system test and evaluation
\item Preparing for system test and evaluation
\item Conducting the system test, collecting and analyzing the test data, comparing the results with the initially specified requirements, preparing a test report
\item Incorporating system modifications as required
\end{itemize}

Upon completion of this chapter, the reader should have acquired an understanding of how the evaluation and validation of system design should be accomplished. A section on summary and extensions closes the chapter, with suggested references for further study also provided. Also, several relevant website addresses are offered.

%------------------------------------------------

\section{Summary and Extensions}\index{Summary and Extensions}

%------------------------------------------------

%% QUESTIONS, PROBLEMS, AND EXERCISES
% SEA Question Location in \label{sea-Chapter#-Problem#}
\begin{exercises}
    \begin{exercise}
    \label{sea-8-1}
    
    \end{exercise}
    \begin{solution}
    \end{solution}

\end{exercises}
% SKIPPED