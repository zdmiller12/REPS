\section{Modeling Quality Function Deployment}\index{Modeling Quality Function Deployment}

A useful tool that can be applied to aid in the establishment and prioritization of TPMs is the <emphasis>quality function deployment</emphasis> (QFD) model. QFD constitutes a team approach to help ensure that the “voice of the customer” is reflected in the ultimate design. The purpose is to establish the necessary requirements and to translate those requirements into technical solutions. Customer requirements are defined and classified as <emphasis>attributes</emphasis>, which are then weighted based on the degree of importance. The QFD method provides the design team an understanding of customer desires, forces the customer to prioritize those desires, and enables a comparison of one design approach against another. Each customer attribute is then satisfied by a technical solution.1<footnoteref preference="1" label="1" role="generated" linkend="ch03fn01"/></para>
<para>The QFD process involves constructing one or more matrices, the first of which is often referred to as the <emphasis>House of Quality</emphasis> (HOQ). A modified version of the HOQ is presented in . Starting on the left side of the structure is the identification of customer needs and the ranking of those needs in terms of priority, the levels of importance being specified quantitatively. This side reflects the <emphasis>whats</emphasis> that must be addressed. The top part of the HOQ identifies the designer’s <emphasis>technical</emphasis> response relative to the attributes that must be incorporated into the design in order to respond to the needs (i.e., the voice of the customer). This part constitutes the <emphasis>hows</emphasis>, and there should be at least one technical solution for each identified customer need. The interrelationships among attributes (or technical correlations) are identified, as well as possible areas of conflict. The center part of the HOQ conveys the strength or impact of the proposed technical response on the identified requirement. The bottom part allows a comparison between possible alternatives, and the right side of the HOQ is used for planning purposes.</para>
<para>The QFD method is used to facilitate the translation of a prioritized set of subjective customer requirements into a set of system-level requirements during conceptual design. A similar approach may be used to subsequently translate system-level requirements into a more detailed set of requirements at each stage in the design and development process. In, the <emphasis>hows</emphasis> from one house become the <emphasis>whats</emphasis> for a succeeding house. Requirements may be developed for the system, subsystem, component, manufacturing process, support infrastructure, and so on. The objective is to ensure the required justification and traceability of requirements from the top down. Further, requirements should be stated in <emphasis>functional</emphasis> terms.</para>
<para>Although the QFD method may not be the only approach used in helping define the specific requirements for system design, it does constitute an excellent tool for creating the necessary visibility from the beginning. One of the largest contributors to <emphasis>risk</emphasis> is the lack of a good set of requirements and an adequate system specification. Inherent within the system specification should be the identification and prioritization of TPMs. The TPM, its associated metric, its relative importance, and benchmark objective in terms of what is currently available will provide designers with the necessary guidance for accomplishing their task. This is essential for establishing the appropriate levels of design emphasis, for defining the criteria as an input to the design, and for identifying the levels of possible risk should the requirements not be met. Again, care must be taken to ensure that these system-level requirements are not adversely impacted by external factors from other systems within the same overall SOS structure.