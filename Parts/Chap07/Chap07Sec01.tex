\section{Models and Indirect Experimentation}\index{Models and Indirect Experimentation}

The only accurate representation of reality is reality itself. Accordingly, all representations or abstractions of reality offered for various purposes are properly designated as models of reality. Fortunately, indirect experimentation may be accomplished through models. Direct experimentation would require manipulating reality, often an impossibility.
The use of models is ubiquitous in engineering, but data on model use and user sophistication in engineering-driven firms is remarkably sparse. Furthermore, some predict a significant change in engineering workflow through model-based system engineering (MBSE). Furthermore, MBSE has been trending over the last decade, arguing for a central model to serve as a means of coordinating system design. Despite benefits that modeling provides, the prevalence of effective modeling in enterprises remains an open question.
</para></para></section></section><para>When used as a noun, the word “model” implies representation. An aeronautical engineer may construct a wind-tunnel model of a possible configuration for a proposed aircraft type. An architect might represent a proposed building with a scale model of the building. An industrial engineer may use templates to represent a proposed layout of equipment in a factory. The word “model” may also be used as an adjective, carrying with it the implication of ideal. From this motivation comes the desire for optimizing operations through decision models.

\subsection{Classical Categories of Models}\index{Classical Categories of Models}

Models are designed to represent a system under study, by an idealized example of reality, to explain the essential relationships involved. They can be classified by distinguishing physical, analogue, schematic, and mathematical types. Physical models look like what they represent, analogue models behave like the original, schematic models graphically describe a situation or process, and mathematical models symbolize the principles of a situation being studied. These model types are used available for use in systems engineering and analysis.</para>
<section id="ch07lev3sec1"><title id="ch07lev3sec1.title">Physical Models. </title><para><inst></inst>Physical models are geometric equivalents, either as miniatures, enlargements, or duplicates made to the same scale. Globes are one example. They are used to demonstrate the shape and orientation of continents, water bodies, and other geographic features of the earth. A model of the solar system is used to demonstrate the orientation of the sun and planets in space. A model of an atomic structure would be similar in appearance but at the other extreme in dimensional reproduction. Each of these models represents reality and is used for demonstration.</para>
<para>Some physical models are used in the simulation process. An aeronautical engineer may test a specific tail assembly design with a model airplane in a wind tunnel. A pilot plant might be built by a chemical engineer to test a new chemical process for the purpose of locating operational difficulties before full-scale production. An environmental chamber is often used to create conditions anticipated for a component under test.</para>
<para>The use of templates in plant layout is an example of experimentation with a physical model. Templates are either two- or three-dimensional replicas of machinery and equipment that are moved about on a scale-model area. The relationship of distance is important, and the templates are manipulated until a desirable layout is obtained. Such factors as noise generation, vibration, and lighting are also important but are not a part of the experimentation and must be considered separately.</para></section>
<section id="ch07lev3sec2"><title id="ch07lev3sec2.title">Analogue Models.</title><para><inst></inst>Analogue comes from the Greek word <emphasis>analogia</emphasis>, which means proportion. This explains the concept of an <emphasis>analogue model</emphasis>; the focus is on similarity in relations. Analogues are usually meaningless from the visual standpoint.</para>
<para>Analogue models can be physical in nature, such as where electric circuits are used to represent mechanical systems, hydraulic systems, or even economic systems. Analogue computers use electronic components to model power distribution systems, chemical processes, and the dynamic loading of structures. The analogue is represented by physical elements. When a digital computer is used as a model for a system, the analogue is more abstract. It is represented by symbols in the computer program and not by the physical structure of the computer components.</para>
<para>The analogue may be a partial subsystem, or it may be an almost complete representation of the system under study. For example, the tail assembly design being tested in a wind tunnel may be complete in detail but incomplete in the properties being studied. The wind tunnel test may examine only the aerodynamic properties and not the structural, weight, or cost characteristics of the assembly. From this it is evident that only those features of an analogue model that serve to describe reality should be considered. These models, like other types, suffer from certain inadequacies.</para></section>
<section id="ch07lev3sec3"><title id="ch07lev3sec3.title">Schematic Models.</title><para><inst></inst>A schematic model is developed by reducing a state or event to a chart or diagram. The schematic model may or may not look like the real-world situation it represents. It is usually possible to achieve a much better understanding of the real-world system described by the model through use of an explicit coding process employed in the construction of the model. The execution of a football play may be diagrammed on a game board with a simple code. It is the idealized aspect of this schematic model that permits this insight into the football play.</para>
<para>An organization chart is a common schematic model. It is a representation of the state of formal relationships existing between various members of the organization. A human–machine chart is another example of a schematic model. It is a model of an event, that is, the time-varying interaction of one or more people and one or more machines over a complete work cycle. A flow process chart is a schematic model that describes the order or occurrence of several events that constitute an objective, such as the assembly of an automobile from a multitude of component parts.</para>
<para>In each case, the value of the schematic model lies in its ability to describe the essential aspects of the existing situation. It does not include all extraneous actions and relationships but rather concentrates on a single facet. Thus, the schematic model is not in itself a solution but only facilitates a solution. After the model has been carefully analyzed, a proposed solution can be defined, tested, and implemented.</para></section>
<section id="ch07lev3sec4"><title id="ch07lev3sec4.title">Mathematical Models.</title><para><inst></inst>A mathematical model employs the language of mathematics and, like other models, may be a description and then an explanation of the system it represents. Although its symbols may be more difficult to comprehend than verbal symbols, they do provide a much higher degree of abstraction and precision in their application. Because of the logic it incorporates, a mathematical model may be manipulated in accordance with established mathematical procedures.</para>
<para>Almost all mathematical models are used either to predict or to control. Such laws as Boyle’s law, Ohm’s law, and Newton’s laws of motion are formulated mathematically and may be used to predict certain outcomes when dealing with physical phenomena. Outcomes of alternative courses of action may also be predicted if a measure of evaluation is adopted. For example, a linear programming model may predict the profit associated with various production quantities within a multiproduct process. Mathematical models may be used to control an inventory. In quality control, a mathematical model may be employed to monitor the proportion of defects that will be accepted from a supplier. Such models maintain control over a state of reality.</para>
<para>Mathematical models directed to the study of systems differ from those traditionally used in the physical sciences in two important ways. First, because the system being studied usually involves social and economic factors, these models must often incorporate probabilistic elements to explain their random behavior. Second, mathematical models formulated to explain existing or planned operations incorporate two classes of variables: those under the control of a decision maker and those not directly under control. The objective is to select values for controllable variables so that some measure of effectiveness is optimized. Thus, these models are of great benefit in systems engineering and systems analysis.

\subsection{Direct and Indirect Experimentation}\index{Direct and Indirect Experimentation}

In direct experimentation, the object, state, or event, and/or the environment are subject to manipulation, and the results are observed. For example, a family might rearrange the furniture in their living room by this method. They move the furniture and observe the results. This process may then be repeated with a second move and perhaps a third, until all logical alternatives have been exhausted. Eventually, one such move is subjectively judged best; the furniture is returned to this position, and the experiment is completed. Direct experimentation, such as this, may be applied to the rearrangement of equipment in a factory. Such a procedure is time-consuming, disruptive, and costly. Hence, simulation or indirect experimentation is employed with templates representing the equipment to be moved.</para>
<para>Direct experimentation in aircraft design would involve constructing a full-scale prototype that would be flight tested under real conditions. Although this is an essential step in the evolution of a new design, it would be costly as the first step. The usual procedure is evaluating several proposed configurations by building a model of each and then testing in a wind tunnel. This is the process of indirect experimentation or <emphasis>simulation</emphasis>. It is extensively used in situations where direct experimentation is not economically feasible.</para>
In systems analysis, indirect experimentation is affected through the formulation and manipulation of decision models. This makes it possible to determine how changes in those aspects of the system under control of the decision maker affect the modeled system. Indirect experimentation enables the systems analyst to evaluate the probable outcome of a given decision without changing the operational system itself. In effect, indirect experimentation in the study of operations provides a means for making quantitative information available to the decision maker without disturbing the operations under his or her control.

\subsection{Simulation Through Indirect Experimentation}\index{Simulation Through Indirect Experimentation}

Models and their manipulation (the process of simulation) are useful tools in systems analysis. A <emphasis>model</emphasis> may be used as a representation of a system to be brought into being, or to analyze a system already in being. Experimental investigation using a model yields design or operational decisions in less time and at less cost than direct manipulation of the system itself. This is particularly true when it is not possible to manipulate reality because the system is not yet in existence, or when manipulation is costly and disruptive as with complex human-made systems.</para>
	<para>Models and the process of simulation provide a convenient means of obtaining information about a system being designed or a system in being. In component design, it is customary and feasible to build several prototypes, test them, and then modify the design based on the test results. This is often not possible in systems engineering because of the cost involved and the length of time required over the system life cycle. A major part of the design process requires decisions based on a model of the system rather than decisions derived from the system that does not yet exist. But then, reality cannot be called a model.
	In most design and operational situations, the objective sought is the optimization of an effectiveness or performance measure. Rarely, if ever, can this be done by direct experimentation with a system under development or a system in being. Also, there is no available theory by which the best model for a given system simulation can be selected. The choice of an appropriate model is determined as much by the experience of the systems analyst as the system itself.</para>
<para>The primary use of simulation in systems engineering is to explore the effects of alternative system characteristics on system performance without actually producing and testing each candidate system. Most models used will fit the classification given earlier, and many will be mathematical. The type used will depend on the questions to be answered. In some instances, simple schematic diagrams will suffice. In others, mathematical or probabilistic representations will be needed. In many cases, simulation with the aid of an analogue or digital computer will be required.</para>
<para>In most systems engineering and analysis undertakings, several models are usually formulated. These models form a hierarchy ranging from considerable aggregation to extreme detail. At the start of a systems project, knowledge of the system is sketchy and general. As the design progresses, this knowledge becomes more detailed and, consequently, the models used for simulation should be detailed