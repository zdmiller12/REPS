\section{Preliminary System Design}\index{Preliminary System Design}

The conceptual design process presented leads to the selection of a tentatively preferred, conceptual system design architecture or configuration. Top-level requirements, as defined there, enable early design evolution that follows in the preliminary system design phase. This phase of the life cycle progresses by addressing the definition and development of the preferred system concept and the allocated requirements for subsystems and the major elements thereof.

An essential purpose of preliminary design is to demonstrate that the selected system concept will conform to performance and design specifications, and that it can be produced and/or constructed with available methods, and that established cost and schedule constraints can be met. Some products of preliminary design include the functional analysis and allocation of requirements at the subsystem level and below, the identification of design criteria as an ``input'' to the design process, the application of models and analytical methods in conducting design trade-offs, the conduct of formal design reviews throughout the system development process, and planning for the <emphasis>detail design and development phase.

Referring to , implementation of the systems engineering process continues in a manner consistent with the steps followed earlier. Specifically, the functional analysis and allocation in  is extended to the next lower level in the system hierarchical structure in , specific design requirements in , and the design review process introduced in  is expanded in . Of particular significance is the utilization of continuous <emphasis>feedback</emphasis>, as shown in  and . The feedback loop is important for verifying that the initially specified requirements are indeed valid and providing a mechanism allowing for changes that enable corrective action and/or system improvement.

This secton leads directly to the material on detail design and development in and system test, evaluation, and validation in . It addresses the following steps in the systems engineering process:

\begin{itemize}
\item Developing design requirements for subsystems and major system elements from system-level requirements
\item Preparing <emphasis>development, product, process, and material specifications applicable to subsystems
\item Accomplishing functional analysis and allocation to and below the subsystem level
\item Establishing detailed design requirements and developing plans for their handoff to engineering domain specialists
\item Identifying and utilizing appropriate engineering design tools and technologies
\item Conducting trade-off studies to achieve design and operational effectiveness
\item Conducting design reviews at predetermined points in time
\end{itemize}

Completion of these steps implements the process illustrated by Blocks 1.1–1.7 in . While the depth, level of effort, and costs of accomplishing this may vary from one application to the next, the process outlined is applicable to the development of any type and size system. As a learning objective, the goal is to provide the reader with a comprehensive and valid approach for addressing preliminary design.

\subsection{Requirements Allocation to Preliminary Design}\index{Requirements Allocation to Preliminary Design}

Preliminary design</emphasis> requirements evolve from “system” design requirements, which are determined through the definition of system operational requirements, the maintenance and support concept, and the identification and prioritization of TPMs. These requirements are documented through the preparation of the <emphasis>system specification</emphasis> (Type <emphasis>A</emphasis>)<emphasis>,</emphasis> prepared in the conceptual design phase (refer to  and ). These requirements become the criteria by which preliminary design alternatives are judged.

The whats initiating conceptual design produce hows from the conceptual design evaluation effort applied to feasible conceptual design concepts. Next, the hows are taken into preliminary design through the means of allocated requirements. There they become whats and drive preliminary design to address hows at this lower level. This is a cascading process following the pattern exhibited in . It emanates from a process giving attention to what the system is intended to do before determining what the system is.

Requirements for the design of subsystems and the major elements of the system are defined through an extension of the functional analysis and allocation, the conduct of design trade-off studies, and so on. This involves an iterative process of top-down/bottom-up design, which continues until the next lowest level of system components are identified and configured.

Consider once again the regional public transportation authority facing the need to increase the capacity for two-way traffic flow across a river that separates a growing municipality. From the results of the conceptual design phase, a bridge spanning the river is selected from among other mutually exclusive river crossing alternatives. Each preliminary bridge design alternative is evaluated through consideration and analysis of its subsystem components. For example, if the pier and superstructure alternative is under evaluation, it will be the abutments, piers, and superstructure that have to be synthesized, sized, and evaluated. Trade-off and optimization is accomplished to determine the pier spacing that will minimize the sum of the first cost of piers and of superstructure, estimated cost of maintenance and support, projected end-of-life cost, and total life-cycle cost. The optimized result provides the basis for comparing this preliminary design alternative on an equivalent basis with other bridge design alternatives, to presented in 


In the river crossing example, each preliminary design alternative is evaluated through careful consideration and analysis of its subsystem components. For instance, if the pier and superstructure alternative is under evaluation, it will be the abutments, piers, and superstructure that will have to be synthesized, sized, and costed. Trade off and optimization should be accomplished to determine the pier spacing that will minimize the sum of the first cost of piers and superstructure. Then, with this optimized first-cost, life-cycle costing should be applied for maintenance and operations, to provide a basis for comparing this preliminary design with the other bridge types.

Lower-level requirements then emanate from the allocated requirements for the tentatively chosen (preliminary) bridge design. These lower-level requirements become the design criteria for subsystems and components of the pier and superstructure bridge. In this particular illustration, the major subsystems identified for the river crossing bridge include the basic road and railway bed, passenger walkway and bicycle path, toll facilities, the maintenance and support infrastructure, and others. Subsystems are further broken down into their respective system elements, such as, superstructure, substructures, piles and footings, foundations, retaining walls, and construction materials. These lower-level requirements are then documented through development, product</emphasis>, process, and/or material specifications. Design and development of these specific system elements is addressed in the detail design and development phase of .

\subsection{Development, Product, and Process Specifications}\index{Development, Product, and Process Specifications}

The technical requirements for the system and its elements are documented through a series of specifications, as indicated in . This series commences with the preparation of the system specification (Type A) prepared in the conceptual design phase (refer to  and ). This, in turn, leads to one or more subordinate specifications and/or standards covering applicable subsystems, configuration items, equipment, software, and other components of the system. In addition, there may be any number of supplemental ANSI (American National Standards Institute), EIA (Electronic Industries Alliance), IEC (International Electrotechnical Commission), ISO (International Organization for Standardization), and related standards that are required in support of the basic program-related specifications.

Although the individual specifications for a given program may assume a different set of designations, a generic approach is used throughout this text. The categories assumed herein are described below and illustrated in . Referring to the figure, the development of a specification tree is recommended for each program, showing a hierarchical relationship in terms of which specification has ``preference'' in the event of conflict. Further, it is critical that all specifications and standards be prepared in such a way as to ensure that there is a traceability of requirements from the top down. Preparation of the development specification (Type B), product specification (Type C), and so on, must include the appropriate TPM requirements that will support an overall system-level requirement; for example, operational availability (Ao) of 0.98. The traceability of requirements, through a specification tree, is particularly important in view of current trends pertaining to increasing globalization, greater outsourcing, and the increasing utilization of external suppliers, where variations often occur in implementing different practices and standards.

\begin{itemize}
\item System specification (Type A): includes the technical, performance, operational, and support characteristics for the system as an entity; the results of a feasibility analysis, operational requirements, and the maintenance and support concept; the appropriate TPM requirements at the system level; a functional description of the system; design requirements for the system; and an allocation of design requirements to the subsystem level (refer to 
\item Development specification (Type B): includes the technical requirements (qualitative and quantitative) for any new item below the system level where research, design, and development are needed. This may cover an item of equipment, assembly, computer program, facility, critical item of support, data item, and so on. Each specification must include the performance, effectiveness, and support characteristics that are required in the evolving of design from the system level and down.
\item Product specification (Type C): includes the technical requirements (qualitative and quantitative) for any item below the system level that is currently in inventory and can be procured “off the shelf.” This may cover any commercial off-the-shelf (COTS) equipment, software module, component, item of support, or equivalent
\item Process specification (Type D): includes the technical requirements (qualitative and quantitative) associated with a process and/or a service performed on any element of a system or in the accomplishment of some functional requirement. This may include a manufacturing process (e.g., machining, bending, and welding), a logistics process (e.g., materials handling and transportation), an information handling process, and so on
\item Material specification (Type E): includes the technical requirements that pertain to raw materials (e.g., metals, ore, and sand), liquids (e.g., paints and chemical compounds), semifabricated materials (e.g., electrical cable and piping), and so on.</para></listitem></orderedlist>
\end{itemize}

Each applicable specification must be direct, complete, and written in performance-related terms and must describe the appropriate design requirements in terms of the whats; that is, the function(s) that the item in question must perform. Further, the specification must be properly tailored to its application, and care must be taken to ensure that it is not overspecifying or underspecifying. While individual programs may vary in applying a different set of designations, or specific content within each specification, it is important that a complete top-down approach be implemented encompassing the requirements for the entire system and all of its elements.

\subsection{Functional Analysis and Allocation}\index{Functional Analysis and Allocation}

With the basic objectives in accomplishing a functional analysis described in and the process for the development of functional flow block diagrams (FFBDs) covered further in , the next step is to extend the functional analysis from the system level down to the subsystem and below as required. The depth of such an analysis (i.e., the breakdown in developing FFBDs from the system level to the second level, third level, and so on) will vary depending on the degree of visibility desired, whether new or existing design is anticipated, and/or to the level at which the designer wishes to establish some specific design-to requirements as an input. As mentioned earlier, it is important to establish the proper architecture describing structure, interrelationships, and related requirements.

The Functional Analysis Process. Referring to and A.2–A.7 in , there are a variety of illustrations showing a breakdown of functions into subfunctions and ultimately describing major subsystems. shows the general sequence of steps leading from the system level down to a communications subsystem (refer to Block 9.5.1). While this shows only one of the many subsystems required to meet an airborne transportation need, the same approach can be applied in defining other subsystems. The development of operational FFBDs can then lead to the development of maintenance FFBDs, as shown in . Given completion of the operational and maintenance FFBDs that reflect the whats, one must next determine the hows; that is, how will each function be accomplished? This is realized by evaluating each individual block of an FFBD, defining the necessary inputs and expected outputs, describing the external controls and constraints, and determining the mechanisms or the physical resources required for accomplishing the function; that is, equipment, software, people, facilities, data/information, or various combinations thereof. An example of the process is presented in 

Referring to the figure, note that just one of the blocks in is addressed, where the resource requirements are identified as mechanisms. As there may be a number of different approaches for accomplishing a given function, trade-off studies are conducted with a preferred approach being selected. The result leads to the determination and compilation of the resource requirements for each function, and ultimately for all of the functions included in the functional analysis.

In performing the analysis process depicted in , a documentation format, similar to that illustrated in (or something equivalent), should be used. In the figure, the functions pertaining to system design and development are identified along with required inputs, expected outputs, and anticipated resource requirements. While this particular example is ``qualitative'' by nature, there are many functions where specific metrics (i.e., TPMs) can be applied in the form of design-to ``constraints.'' Referring to , for example, there is a functional requirement at the system level that states, ``Operate the system in the user environment,'' represented by Block 9.0. Assuming that there is a system TPM requirement for operational availability (Ao) of 0.985 (refer to ), this measure of effectiveness will constitute a design requirement for the function in Block 9.0, and the appropriate resources must be applied accordingly. The same approach can be applied in relating all of the TPM requirements, which are specified for the system, to one or more functions. Accordingly, the purpose of including  is to promote a disciplined approach in accomplishing a functional analysis.

In conducting trade-off studies pertaining to the best approach in responding to a functional requirement (i.e., the mechanisms), the results may point toward the selection of hardware, software, people, facilities, data, or various combinations thereof. gives an example where the requirements for hardware, software, and the human are identified. Stemming from the functional analysis (Block 0.2 in ), the individual design and development steps for each is shown, and a plan is prepared for the acquisition of these system elements. From a systems engineering perspective, it is essential that these activities be coordinated and integrated, across the life cycles, from the beginning. In other words, an ongoing “communication(s)” must exist throughout the design and development of the hardware, software, and human elements of the system.

\subsection{Requirements Allocation}\index{Requirements Allocation}

Referring to , lower-level elements of the system are defined through the functional analysis and subsequently by partitioning (or grouping) similar functions into logical subdivisions, identifying major subsystems, configuration items, units, assemblies, modules, and so forth (refer to ). presents an overview of this process, evolving from a functional definition of System XYZ to the packaging of the system into three units; that is, Units A, B, and C.

Given the packaging concept shown in , it is now appropriate to determine the ``design-to'' requirements for each one of the three units. This is accomplished through the process of allocation (or apportionment). In the development of design goals at the unit level, priorities are established based on the TPMs for the system shown in , and both quantitative and qualitative design requirements are determined. Such requirements then lead to the incorporation of the appropriate design characteristics (attributes) in the design of Units A, B, and C. Such design characteristics, as shown in the hierarchy in , should be ``tailored'' in response to the relative importance of each as it impacts the system-level requirements. These design characteristics are initially viewed from the top down and are compared. Trade-off studies are conducted to evaluate the interaction effects, and those characteristics most significant in meeting the overall system objectives are selected.

presents an example resulting from the allocation process described. System-level TPMs are identified along with the design-to metrics for each of the three units, as well as the requirements for Assemblies 1, 2, and 3 within Unit B. The requirements at the system level have been allocated downward. Further, the requirements established at the unit level, when combined, must be compatible with the higher-level requirements. Thus, there is a top-down/bottom-up relationship, and there may be trade-offs conducted comprehensively at the unit level in order to achieve the proper balance of requirements overall. Meeting these quantitative design-to requirements leads to the incorporation of the proper characteristics in the design of the item in question.

The objective of including with all of its metrics is to emphasize the process and its importance early in system design. Although the metrics shown are primarily related to reliability, maintainability, availability, and design-to-life-cycle-cost factors, it should be noted that in the allocation process, one needs to include all performance factors, human factors, physical features, producibility and supportability factors, sustainability and disposability factors, and so on. Some of these measures are covered in detail in ), and the interrelationships should become clearer after reviewing those sectons.

In situations where there are a number of different systems in a system-of-systems (SOS) configuration, or where there are common functions, the allocation process becomes a little more complex. One needs to not only comply with the top-down/bottom-up traceability requirements for each system in the overall configuration but consider the compatibility (or operability) requirements among the systems as well. Referring to , for example, there are two systems, System ABC and System XYZ, within a given SOS structure, with a ``common'' function being shared by each. Given such, several possibilities may exist

\begin{enumerate}
\item One of the systems already exists, is operational, and the design is basically ``fixed,'' while the other system is new and in the early stages of design and development. Assuming that System ABC is operational, the design characteristics for the unit identified as being ``Common'' in the figure and required as a functioning element of ABC are essentially ``fixed.'' This, in turn, may have a significant impact on the overall effectiveness of System XYZ. In order to meet the overall requirements for XYZ, more stringent design input factors may have to be placed on the design of new Units C and D for that system. Or the common unit must be modified to be compatible with higher-level XYZ requirements, which (in turn) would likely impact the operational effectiveness of System ABC. Care must be taken to ensure that the overall requirements for both System ABC and System XYZ will be met. This can be accomplished by establishing a fixed requirement for the ``common unit'' and by modifying the design requirements for Assemblies 1, 2, and 3 within that unit to meet allocated requirements from both ABC and XYZ
\item Both System ABC and System XYZ are new and are being developed concurrently. The allocation process illustrated in is initially accomplished for each of the systems, ensuring that the overall requirements at the system level are established. The design-to characteristics for the ``common unit'' are compared and a single set of requirements is identified for the unit through the accomplishment of trade-off analyses, and the requirements for each of the various system units are then modified as necessary across-the-board while ensuring that the top system-level requirements are maintained. This may constitute an iterative process of analysis, feedback, and so on
\end{enumerate}

As a final point, those quantitative and qualitative requirements for the various elements of the system (i.e., subsystems, units, and assemblies) must be included in the appropriate specification, as identified in ). Further, there must be a top-down/bottom-up ``traceability'' of requirements throughout the overall hierarchical structure for each of the systems in question.

Applying the Functional Analysis. A major objective of systems engineering is to develop a complete set of requirements in order to define a single ``baseline'' from which all lower-level requirements may evolve; that is, to develop a functional baseline in conceptual design and later an allocated baseline in the preliminary system design phase (refer to ). The results of the functional analysis constitute a required input for a number of design-related activities that occur subsequently. Most important is breaking the system (and its elements) down into functional entities through functional packaging and the development of an open-architecture configuration. A prime objective is to develop a configuration that can be easily upgraded as required (through new technology insertions) and easily supported throughout its life cycle (through a modularized approach in maintenance). In addition, the functional analysis provides a foundation upon which many of the subsequent analytical tasks and associated documentation are based, some of which are listed below:

\begin{enumerate}
\item Reliability analysis: reliability models and block diagrams; failure mode, effects, and criticality analysis (FMECA); fault-tree analysis (FTA); reliability prediction (refer to 
\item Maintainability analysis: maintainability models; reliability-centered maintenance (RCM); level-of-repair analysis (LORA); maintenance task analysis (MTA); total productive maintenance (TPM); maintainability prediction (refer to 
\item Human factors analysis: operator task analysis (OTA); operational sequence diagrams (OSDs); safety/hazard analysis; personnel training requirements (refer to 
\item Maintenance and logistic support: supply chain and supportability analysis leading to the definition of maintenance and support requirements—spares/repair parts and associated inventories, test and support equipment, transportation and handling equipment, maintenance personnel, facilities, technical data, information (refer to 
\item Producibility, disposability, and sustainability analysis (refer to 
\item Affordability analysis: life-cycle and total ownership cost (refer to 
\end{enumerate}

All of these activities, as described throughout of this textbook, are dependent and based on functional analysis as an essential input.

Preliminary Design Criteria. The basic design objective(s) for the system and its elements must (1) be compatible with the system operational requirements, maintenance and support concept, and the prioritized TPMs; (2) comply with the allocated design-to criteria described in .2; and (3) meet all of the requirements in the various applicable specifications. The particular design characteristics to be incorporated will vary from one instance to the next, depending on the type and complexity of the system and the mission or purpose that it is intended to accomplish. In all cases, the design team activity must address the downstream life-cycle outcomes considering the phases of production and construction, system utilization and sustaining support, and retirement and material recycling/disposal. While all considerations in system design must be addressed (refer to ), a few require some additional emphasis. These are clustered in Part IV.

Design Engineering Activities. The day-to-day design activities begin with the implementation of the appropriate planning that was initiated in the conceptual design phase (i.e., the program management and system engineering management plans identified in ). This includes the establishment of the design team and the initiation of specific design tasks, the ongoing liaison and working with various responsible designers throughout the project organization, the development of design data, the accomplishment of periodic design reviews, and the initiation of corrective action as necessary.</para>

Trade Off Studies and Design Definition. </title><para>As the design evolves, the system synthesis, analysis, and evaluation process, described in and , continues. Proposed configurations for subsystems and major elements of the system are synthesized, trade-off studies are conducted, alternatives are evaluated, and a preferred design approach is selected. This process continues throughout the conceptual design, preliminary system design, and the detail design and development phases, leading to the definition of the system configuration down to the detailed component level (refer to 

This iterative process of systems analysis is shown in a generic and simplified context in . Referring to the figure (which complements ), a key challenge is the application of the appropriate analytical techniques and models. As conveyed in the previous section, knowledge of the analytical methods described in  of this text is required to accomplish the steps reflected by Blocks 4 and 5 (5a and 5b) of . Further, when selecting and/or developing a specific analytical model/tool to facilitate the analysis effort, care must be exercised to ensure that the right computer-based tool is selected for the application intended.

shows the application of a number of different analytical models used in the evaluation of alternative maintenance and logistic support policies. This illustration presents just one of a number of examples where there may be a multiple mix (or combination) of tools used to evaluate a specific design configuration. While the approach conveyed in the figure may not appear to be unique, it should be noted that there are many different computer-based models that are available in the commercial market and are advertised as solutions to a wide variety of problems. Most of these models were developed on a relatively ``independent'' or ``isolated'' basis in terms of selected platform, context or computer language, input data requirements, varying degrees of ``user friendliness,'' and so on. In general, many of the models advertised today do not ``talk to each other,'' are too complex, require too much input data, and can only be effectively used in the ``downstream'' portion of the life cycle during the detail design and development phase when there are a lot of data available.

From a systems engineering perspective, a good objective is to select or develop an integrated design workstation (incorporating a Macro-CAD approach) that can be utilized in all phases of the system life cycle and that can be adapted to the different levels of design definition as one progresses from the conceptual design to the detail design and development phase (refer to ). For example, this workstation should incorporate the right tools that can be applied at a high level in conceptual design and again at a more in-depth level in detail design and development. Theoretically, it should be possible to utilize any one (or all) of the tools described throughout Parts III and IV of this text at any time in the system life cycle, where they can be ``tailored'' to the degree of design definition as required.

\subsection{Design Review and Evaluation}\index{Design Review and Evaluation}

The basic objectives and benefits of the design review, evaluation, and feedback process are as described in , and include two facets of the activity shown in . First, there is an ongoing informal review and evaluation of the results of the design, accomplished on a day-to-day basis, where the responsible designer provides applicable technical data and information to all project personnel as the design progresses. Through subsequent review, discussion, and feedback, the proposed design is either approved or recommended changes are submitted for consideration. Second, there is a structured series of formal design reviews conducted at specific times in the overall system development process. While the specific types, titles, and scheduling of these formal reviews will vary from one program to the next, it is assumed herein that formal reviews will include the following:

The Conceptual Design is usually scheduled toward the end of the conceptual design phase and prior to entering the preliminary system design phase of the program. The objective is to review and evaluate the requirements and the functional baseline for the system, and the material to be covered through this review should include the results from the feasibility analysis, system operational requirements, the maintenance and support concept, applicable prioritized TPMs, the functional analysis (top level for the system), system specification (Type <emphasis>A</emphasis>), a systems engineering management plan (SEMP), a test and evaluation master plan (TEMP), and supporting design criteria and data/documentation. Refer to  and 

System design reviews</emphasis> are generally scheduled during the preliminary system design phase when functional requirements and allocations are defined, preliminary design layouts and detailed specifications are prepared, system-level trade-off studies are conducted, and so on. These reviews are oriented to the overall system configuration (as subsystems and major system elements are defined), rather than to individual equipment items, software, and other lower-level components of the system. There may be one or more formal reviews scheduled, depending on the size of the system and the complexity of design. System design reviews may cover a variety of topics, including the following: functional analysis and the allocation of requirements; development, product, process, and material specifications (Types <emphasis>B</emphasis>, <emphasis>C</emphasis>, <emphasis>D</emphasis>, and <emphasis>E</emphasis>); applicable TPMs; significant design criteria for major system elements; trade-off study and analysis reports; predictions; and applicable design data (layouts, drawings, parts/material lists, supplier reports, and data)

Equipment/software design reviews</emphasis> are scheduled during the detail design and development phase and usually cover such topics as product, process, and material specifications (Types <emphasis>C</emphasis>, <emphasis>D</emphasis>, and <emphasis>E</emphasis>); design data defining major subsystems, equipment, software, and other elements of the system as applicable (assembly drawings, specification control drawings, construction drawings, installation drawings, logic drawings, schematics, materials/parts lists, and supplier data); analyses, predictions, trade-off study reports, and other related design documentation; and engineering models, laboratory models, mock-ups, and/or prototypes used to support a specific design configuration

The critical design review</emphasis> is generally scheduled after the completion of detailed design, but prior to the release of firm design data for production and/or construction. Design is essentially ``fixed'' at this point, and the proposed configuration is evaluated in terms of adequacy, producibility, and/or constructability. The critical design review may include the following topics: a complete package of final design data and documentation; applicable analyses, trade-off study reports, predictions, and related design documentation; detailed production/construction plans; operational and sustainability plans; detailed maintenance plans; and a system retirement and material recycling/disposal plan. The results of the critical design review describe the final system configuration product baseline prior to entering into production and/or construction.

The review, evaluation, and feedback process is continuous throughout system design and development and, as indicated in and , encompasses conceptual, preliminary, and detail design. The objective in is to introduce the review process and describe some of the benefits that can be derived from it, to provide an overall spectrum of activity and the types and scheduling of reviews, and to describe some specifics leading to further implementing the process in