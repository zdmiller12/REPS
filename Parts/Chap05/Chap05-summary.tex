\section{Summary and Extensions}\index{Summary and Extensions}

In this chapter, some system definitions and systems science concepts were presented to provide a basis for the study of systems engineering and analysis. They include definitions of system characteristics, a classification of systems into various types, consideration of the current state of systems science, and a discussion of the transition to the Systems Age. Finally, the chapter presents technology and the nature and role of engineering in the Systems Age and ends with a number of commonly accepted definitions of systems engineering.</para>

Upon completion of , the reader will have obtained essential insight into systems and systems thinking, with an orientation toward systems engineering and analysis. The system definitions, classifications, and concepts presented in this chapter are intended to impart a general understanding about the following:

\begin{enumerate}
\item System classifications, similarities, and dissimilarities
\item The fundamental distinction between natural and human-made systems
\item The elements of a system and the position of the system in the hierarchy of systems
\item The domain of systems science, with consideration of cybernetics, general systems Theory, and systemology
\item Technology as the progenitor for the creation of technical systems, recognizing its impact on the natural world
\item The transition from the machine or industrial age to the Systems Age, with recognition of its impact upon people and society
\item System complexity and scope and the demands these factors make on engineering in the Systems Age
\item The range of contemporary definitions of systems engineering used within the profession
\end{enumerate}

Although this book focuses on the engineering of systems and on systems analysis, it would not be intellectually prudent to begin the discussion at that level. Upon examination, it is evident that both the engineering and the analysis aspects of the focus are directed to systems. Accordingly, this chapter is devoted to helping the reader gain essential insight into systems in general, and systems thinking in particular, with orientation toward the engineering and analysis of technical systems.

System definitions, a discussion of system elements, and a high-level classification of systems provide an opening panorama. It is here that a consideration of the origin of systems provides an orientation to natural and human-made domains as an overarching dichotomy. The importance of this dichotomy cannot be overemphasized in the study and application of systems engineering and analysis. It is the suggested frame of reference for considering and understanding the interface and impact of the human-made world on the natural world and on humans.

Individuals interested in obtaining as in-depth appreciation for this interface and the mitigation of environmental impacts are encouraged to read T.E. Graedel and B.R. Allenby, Industrial Ecology, 2nd ed., Prentice Hall, 2003. Also of contemporary interest is the issue of sustainability treated as part of an integrated approach to sustainable engineering by P. Stasinopoulos, M.H. Smith, K. Hargroves, and C. Desha, Whole System Design, Earthscan Publishing, 2009. These works are recommended as an extension to this chapter (as well as Chapter 16), because they illuminate and address the sensitive interface between the natural and the human-made.

%%----------------------------------------------------------------------------------------
%	PROBLEMS
%%----------------------------------------------------------------------------------------

%% SEA CHAPTER 5 - DETAIL DESIGN AND DEVELOPMENT
% SEA Question Location in \label{sea-Chapter#-Problem#}
\begin{exercises}
    \begin{exercise}
    \label{sea-5-1}
        What are the basic differences between conceptual design, preliminary system design, and detailed design and development? Are these stages of design applicable to the acquisition of all systems? Explain.
    \end{exercise}
    \begin{solution}
    \end{solution}
    
    \begin{exercise}
    \label{sea-5-2}
        Design constitutes a team effort. Explain why. What constitutes the make-up of the design team? How can this be accomplished? How does systems engineering fit in to the process?
    \end{exercise}
    \begin{solution}
    \end{solution}
    
    \begin{exercise}
    \label{sea-5-3}
        Briefly describe the role of systems engineering in the overall design process as it is described in Chapters 3, 4, and 5.
    \end{exercise}
    \begin{solution}
    \end{solution}
    
    \begin{exercise}
    \label{sea-5-4}
        Refer to Figure 5.1. What are some of the advantages of the concurrent approach in design? Identify some of the problems that could occur in its implementation.
    \end{exercise}
    \begin{solution}
    \end{solution}
    
    \begin{exercise}
    \label{sea-5-5}
        Refer to Figures 4.4 and 4.8 (Chapter 4). As the systems engineering manager on a given program, what steps would you take to ensure that the proper integration of requirements occurs across the three life cycles (hardware, software, human) from the beginning?
    \end{exercise}
    \begin{solution}
    \end{solution}
    
    \begin{exercise}
    \label{sea-5-6}
        As a designer, one of your tasks constitutes the selection of a component to fulfill a specific design objective. What priorities would you consider in the selection process (if any)?
    \end{exercise}
    \begin{solution}
    \end{solution}
    
    \begin{exercise}
    \label{sea-5-7}
        Why are design standards (as applied to component parts and processes) important?
    \end{exercise}
    \begin{solution}
    \end{solution}
    
    \begin{exercise}
    \label{sea-5-8}
        Why are engineering documentation and the establishment of a design database necessary?
    \end{exercise}
    \begin{solution}
    \end{solution}
    
    \begin{exercise}
    \label{sea-5-9}
        Refer to Figure 5.5. When accomplishing the necessary trade-offs, there may be some confusion as to which of the three options to pursue. Describe what information is required as an input in order to evolve into a “clear-cut” approach.
    \end{exercise}
    \begin{solution}
    \end{solution}
    
    \begin{exercise}
    \label{sea-5-10}
        Describe how the application of CAD, CAM, and CAS tools can facilitate the system design process. Identify some benefits. Address some of the problems that could occur in the event of misapplication.
    \end{exercise}
    \begin{solution}
    \end{solution}
    
    \begin{exercise}
    \label{sea-5-11}
        How can CAD, CAM, and CAS tools be applied to validate the design? Provide an example of two.
    \end{exercise}
    \begin{solution}
    \end{solution}
    
    \begin{exercise}
    \label{sea-5-12}
        What is the purpose of developing a physical model of the system, or an element thereof, early in the system design process?
    \end{exercise}
    \begin{solution}
    \end{solution}
    
    \begin{exercise}
    \label{sea-5-13}
        What are some of the differences between a mock-up, an engineering model, and a prototype?
    \end{exercise}
    \begin{solution}
    \end{solution}
    
    \begin{exercise}
    \label{sea-5-14}
        Select a system (or an element of a system) of your choice and develop a design review checklist that you can use for evaluation purposes. (Refer to Figure 5.8 and Appendix B.)
    \end{exercise}
    \begin{solution}
    \end{solution}
    
    \begin{exercise}
    \label{sea-5-15}
        What are some of the benefits that can be acquired through implementation of a formal design review process?
    \end{exercise}
    \begin{solution}
    \end{solution}
    
    \begin{exercise}
    \label{sea-5-16}
        Refer to Figure 5.9. The predicted LCC value for the system, at the time of a system design review, is around $500K, which is well above the $420K design-to requirement. What steps would you take to ensure that the ultimate requirement will be met at (or before) the critical design review? Be specific.
    \end{exercise}
    \begin{solution}
    \end{solution}
    
    \begin{exercise}
    \label{sea-5-17}
        Refer to Figure 5.10. As a systems engineering manager, how would you ensure that all of the TPM requirements are being properly “tracked”?
    \end{exercise}
    \begin{solution}
    \end{solution}
    
    \begin{exercise}
    \label{sea-5-18}
        What determines whether or not a given design review has been successful?
    \end{exercise}
    \begin{solution}
    \end{solution}
    
    \begin{exercise}
    \label{sea-5-19}
        In evaluating the feasibility of an ECP, what considerations need to be addressed?
    \end{exercise}
    \begin{solution}
    \end{solution}
    
    \begin{exercise}
    \label{sea-5-20}
        Assume that an ECP has been approved by the CCB. What steps need to be taken in implementing the proposed change?
    \end{exercise}
    \begin{solution}
    \end{solution}
    
    \begin{exercise}
    \label{sea-5-21}
        What is configuration management? When can it be implemented? Why is it important?
    \end{exercise}
    \begin{solution}
    \end{solution}
    
    \begin{exercise}
    \label{sea-5-22}
        Why is baseline management so important in the implementation of the systems engineering process?
    \end{exercise}
    \begin{solution}
    \end{solution}
\end{exercises}
% SKIPPED