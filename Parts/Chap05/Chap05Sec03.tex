\section{Engineering for the Systems Age}\index{Engineering for the Systems Age}

In the Systems Age, successful accomplishment of engineering objectives usually requires a combination of technical specialties and expertise. Engineering in the Systems Age must be a team activity where various individuals involved are cognizant of the important relationships between specialties and between economic factors, ecological factors, political factors, and societal factors. The engineering decisions of today require consideration of these factors in the early stage of system design and development, and the results of such decisions have a definite impact on these factors. Conversely, these factors usually impose constraints on the design process. Thus, technical expertise must include not only the basic knowledge of individual specialty fields of engineering but also knowledge of the context of the system being brought into being.

Although relatively small products, such as a wireless communication device, an electrical household appliance, or even an automobile, may employ a limited number of direct engineering personnel and supporting resources, there are many large-scale systems that require the combined input of specialists representing a wide variety of engineering and related disciplines. An example is that of a ground-based mass-transit system.

Civil engineers are required for the layout and/or design of the railway, tunnels, bridges, and facilities. Electrical engineers are involved in the design and provision of automatic controls, traction power, substations for power distribution, automatic fare collection, digital data systems, and so on. Mechanical engineers are necessary in the design of passenger vehicles and related mechanical equipment. Architectural engineers provide design support for the construction of passenger terminals. Reliability and maintainability engineers are involved in the design for system availability and the incorporation of supportability characteristics. Industrial engineers deal with the production and utilization aspects of passenger vehicles and human components. Test engineers evaluate the system to ensure that all performance, effectiveness, and system support requirements are met. Engineers in the planning and marketing areas are required to keep the public informed, to explain the technical aspects of the system, and to gather and incorporate public input. General systems engineers are required to ensure that all aspects of the system are properly integrated and function together as a single entity.

Although the preceding example is not all-inclusive, it is evident that many different disciplines are needed. In fact, there are some large projects, such as the development of a new aircraft, where the number of engineers needed to perform engineering functions is in the thousands. In addition, the different engineering types often range in the hundreds. These engineers, forming a part of a large organization, must not only be able to communicate with each other but must also be conversant with such interface areas as purchasing, accounting, personnel, and legal.

Another major factor associated with large projects is that considerable system development, production, evaluation, and support are often accomplished at supplier (sometimes known as subcontractor) facilities located throughout the world. Often there is a prime producer or contractor who is ultimately responsible for the development and production of the total system as an entity, and there are numerous suppliers providing different system components. Thus, much of the project work and many of the associated engineering functions may be accomplished at dispersed locations, often worldwide.

\subsection{Systems Engineering Definition}\index{Systems Engineering Definition}

To this day, there is no commonly accepted definition of Systems Engineering (SE) in the literature. Almost a half-century ago, Hendrick W. Bode, writing on ``The Systems Approach'' in Applied Science-Technological Progress, said that ``It seems natural to begin the discussion with an immediate formal definition of Systems Engineering. However, Systems Engineering is an amorphous, slippery subject that does not lend itself to such formal, didactic treatment. One does much better with a broader, more loose-jointed approach. Some writers have, in fact, sidestepped the issue by saying that Systems Engineering is what systems engineers do.''

The definition of system engineering and the systems approach is usually based on the background and experience of the individual or the performing organization. The variations are evident from the following five published definitions:

\begin{itemize}
\item “An interdisciplinary approach and means to enable the realization of successful systems.”
\item “An interdisciplinary approach encompassing the entire technical effort to evolve into and verify an integrated and life-cycle balanced set of systems people, product, and process solutions that satisfy customer needs. Systems engineering encompasses (a) the technical efforts related to the development, manufacturing, verification, deployment, operations, support, disposal of, and user training for, system products and processes; (b) the definition and management of the system configuration; (c) the translation of the system definition into work breakdown structures; and (d) development of information for management decision making.”
\item “The application of scientific and engineering efforts to (a) transform an operational need into a description of system performance parameters and a system configuration through the use of an iterative process of definition, synthesis, analysis, design, test, and evaluation; (b) integrate related technical parameters and ensure compatibility of all physical, functional, and program interfaces in a manner that optimizes the total system definition and design; and (c) integrate reliability, maintainability, safety, survivability, human engineering, and other such factors into the total engineering effort to meet cost, schedule, supportability, and technical performance objectives.”
\item “An interdisciplinary collaborative approach to derive, evolve, and verify a life-cycle balanced system solution which satisfies customer expectations and meets public acceptability.”
\item “An approach to translate operational needs and requirements into operationally suitable blocks of systems. The approach shall consist of a top-down, iterative process of requirements analysis, functional analysis and allocation, design synthesis and verification, and system analysis and control. Systems engineering shall permeate design, manufacturing, test and evaluation, and support of the product. Systems engineering principles shall influence the balance between performance, risk, cost, and schedule.”
\end{itemize}

Although the definitions vary, there are some common threads. Basically, systems engineering is good engineering with special areas of emphasis. Some of these are following:

\begin{enumerate}
\item A top-down approach that views the system as a whole. Although engineering activities in the past have adequately covered the design of various system components (representing a bottom-up approach), the necessary overview and understanding of how these components effectively perform together is frequently overlooked.
\item A life-cycle orientation that addresses all phases to include system design and development, production and/or construction, distribution, operation, maintenance and support, retirement, phase-out, and disposal. Emphasis in the past has been placed primarily on design and system acquisition activities, with little (if any) consideration given to their impact on production, operations, maintenance, support, and disposal. If one is to adequately identify risks associated with the up-front decision-making process, then such decisions must be based on life-cycle considerations.
\item A better and more complete effort is required regarding the initial definition of system requirements, relating these requirements to specific design criteria, and the follow-on analysis effort to ensure the effectiveness of early decision making in the design process. The true system requirements need to be well defined and specified and the traceability of these requirements from the system level downward needs to be visible. In the past, the early “front-end” analysis as applied to many new systems has been minimal. The lack of defining an early “baseline” has resulted in greater individual design efforts downstream.
\item An interdisciplinary or team approach throughout the system design and development process to ensure that all design objectives are addressed in an effective and efficient manner. This requires a complete understanding of many different design disciplines and their interrelationships, together with the methods, techniques, and tools that can be applied to facilitate implementation of the system engineering process.
\end{enumerate}

Systems engineering is not a traditional engineering discipline in the same sense as civil engineering, electrical engineering, industrial engineering, mechanical engineering, reliability engineering, or any of the other engineering specialties. It should not be organized in a similar manner, nor does the implementation of systems engineering (or its methods) require extensive resources. However, a well-planned and highly disciplined approach must be followed. The systems engineering process involves the use of appropriate technologies and management principles in a synergetic manner. Its application requires synthesis and focus on process, along with a new “thought process” that is compatible with the needs of the Systems Age.

\subsection{Promulgating Systems Engineering Within the Engineering Profession}\index{Promulgating Systems Engineering Within the Engineering Profession}

From its modest beginning more than a half-century ago, Systems Engineering is now gaining international recognition as an effective technology based interdisciplinary process for bringing human-made systems into being, and for improving systems already in being. Certain desirable academic and professional attributes are coming into clear view. Others require further study, development, testing, and implementation.

This section summarizes the heritage from which Systems Engineering entered the 21st century. Several emerging attributes of Systems Engineering education and professional practice are addressed. These include the necessary but not sufficient academic and professional activities of technical societies, degree programs and program accreditation, certification and licensing, knowledge generation and publications, recognition and honors, and considerations regarding maturity. Special attention is directed to those attributes that should be developed further to enable Systems Engineering to serve society will in this century.

Conceptually sound system design derives from focusing on what the system is intended to do before determining what the system is, with form following function. This focus is most effective when based on essential design dependent parameters, recognizing the concurrent life-cycle factors of production, support, maintenance, phase-out, and disposal. It invokes integrating and iterating synthesis, analysis, and evaluation. These considerations are germane to system and product design when embedded within the systems engineering process. The purpose of this presentation is to provide an overview of the embedded relationship of design dependent parameters as key controllables in the effective, and orderly process of bringing cost-effective systems, products, structures, and services (the human-made world) into being.  Draft Thales Copyright 2011 12 Jan 2011 Page 1 of 8