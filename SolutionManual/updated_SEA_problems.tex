%% SEA CHAPTER 1 - SYSTEMS SCIENCE AND ENGINEERING
\chapter{SYSTEMS SCIENCE AND ENGINEERING}

% SEA Question Location in \label{sea-Chapter#-Problem#}
\begin{exercises}
    \begin{exercise}
    \label{sea-1-26}
        Describe how systems thinking differs from systems engineering.
    \end{exercise}
    \begin{solution}
    \end{solution}
    
    \begin{exercise}
    \label{sea-1-1}
        For a system with which you are familiar, justify why it is a system according to the definition in \Cref{sec:sectorsComprisingWorld}.
    \end{exercise}
    \begin{solution}
    \end{solution}
    
    \begin{exercise} 
    \label{sea-1-2}
        Describe the components, attributes, and relationships in the system you used in Problem \ref{sea-1-1}.
    \end{exercise}
    \begin{solution}
    \end{solution}
    
    \begin{exercise}
    \label{sea-1-3}
        Name any system which includes a material that transforms over the system's life cycle and identify its structural components, operating components, and flow components.
    \end{exercise}
    \begin{solution}
    \end{solution}
    
    \begin{exercise} 
    \label{sea-1-4_5}
        Name any complex system and
        \begin{enumerate}[label=\alph*)]
            \item Define the hierarchy related to the system.
            \item Define the system boundaries.
        \end{enumerate}
    \end{exercise}
    \begin{solution}
    \end{solution}
    
    \begin{exercise} 
    \label{sea-1-6_7_8}
        Identify and contrast
        \begin{enumerate}[label=\alph*)]
            \item Physical versus conceptual systems.
            \item Static versus dynamic systems.
            \item Closed versus open systems.
        \end{enumerate}
    \end{exercise}
    \begin{solution}
    \end{solution}
    
    \begin{exercise} 
    \label{sea-1-15}
        For any system of the following types, name any system property
        \begin{enumerate}[label=\alph*)]
            \item Dynamic system.
            \item Steady-state system.
        \end{enumerate}
    \end{exercise}
    \begin{solution}
    \end{solution}
    
    \begin{exercise} 
    \label{sea-1-9}
        For each of the following systems, define a unique system and describe it in terms of components, attributes, and relationships
        \begin{enumerate}[label=\alph*)]
            \item Natural system.
            \item Human-made system.
            \item Human-modified system.
        \end{enumerate}
    \end{exercise}
    \begin{solution}
    \end{solution}
    
    \begin{exercise} 
    \label{sea-1-10_11_12}
        For the Human-made system in Question \ref{sea-1-9}
        \begin{enumerate}[label=\alph*)]
            \item Identify the system's purpose(s) and potential metrics to present its value.
            \item Describe the system's state at any arbitrary time during operation, at least one system behavior, and an overview of the system's process.
            \item Name any two related components of the system, define the purpose of each component as it relates to the other, and the necessary attributes of the component pair such that they contribute to the purpose(s) of the entire system.
        \end{enumerate}
    \end{exercise}
    \begin{solution}
    \end{solution}
    
    \begin{exercise} 
    \label{sea-1-14}
        For the Human-modified system in Question \ref{sea-1-9}, name some positive and negative impact(s) of the modification to the natural system.
    \end{exercise}
    \begin{solution}
    \end{solution}
    
    \begin{exercise} 
    \label{sea-1-13}
        Give examples of each of the following
        \begin{enumerate}[label=\alph*)]
            \item First-order relationship.
            \item Second-order relationship.
            \item Redundance.
        \end{enumerate}
    \end{exercise}
    \begin{solution}
    \end{solution}
    
    \begin{exercise} 
    \label{sea-1-16}
        Name any system that operates at equilibrium and another system that degrades over time.
    \end{exercise}
    \begin{solution}
    \end{solution}
    
    \begin{exercise} 
    \label{sea-1-17}
        The United States government, for example, can be divided and described as three individual entities of the executive, legislative, and judicial branches. Create an argument for why a government of this structure should either be considered a single system or three systems.
    \end{exercise}
    \begin{solution}
    \end{solution}
    
    \begin{exercise} 
    \label{sea-1-19}
        Give any example of cybernetics and define why the example is appropriate.
    \end{exercise}
    \begin{solution}
    \end{solution}
    
    \begin{exercise} 
    \label{sea-1-21}
        Do all systems at higher levels of Boulding's Hierarchy necessarily incorporate the lower levels of the hierarchy? If not, provide a specific system example.
    \end{exercise}
    \begin{solution}
    \end{solution}
    
    \begin{exercise} 
    \label{sea-1-22}
        Describe a novel system that may be necessary for society 50 to 100 years in the future and
        \begin{enumerate}[label=\alph*)]
            \item Define the system requirements.
            \item Define the system objectives.
        \end{enumerate}
    \end{exercise}
    \begin{solution}
    \end{solution}
    
    \begin{exercise} 
    \label{sea-1-25}
        For the system described in Question \ref{sea-1-22}
        \begin{enumerate}[label=\alph*)]
            \item Identify factors which led to the need for a new system.
            \item Identify other societal factors which may evolve in parallel and lead to other changes or innovations.
        \end{enumerate}
    \end{exercise}
    \begin{solution}
    \end{solution}
    
    \begin{exercise} 
    \label{sea-1-23}
        Compare and contrast systemology and synthesis.
    \end{exercise}
    \begin{solution}
    \end{solution}
    
    \begin{exercise} 
    \label{sea-1-24}
        Classify a technical system.
    \end{exercise}
    \begin{solution}
    \end{solution}
    
    \begin{exercise} 
    \label{sea-1-27}
        Compare and contrast the attributes of the Machine (Industrial) Age and the Systems Age.
    \end{exercise}
    \begin{solution}
    \end{solution}
    
    \begin{exercise} 
    \label{sea-1-28}
        Identify key differences between synthetic and analytical thinking. Is one method of thinking always preferable? Why or why not?
    \end{exercise}
    \begin{solution}
    \end{solution}
    
    \begin{exercise} 
    \label{sea-1-29}
        What challenges make the Systems Age unique from other periods of human evolution?
    \end{exercise}
    \begin{solution}
    \end{solution}
    
    \begin{exercise} 
    \label{sea-1-30}
        Compare and contrast systems engineering with other engineering disciplines.
    \end{exercise}
    \begin{solution}
    \end{solution}
    
    \begin{exercise} 
    \label{sea-1-32}
        Identify a system which required an interdisciplinary approach to develop \textit{or} to implement. What disciplines were required and why?
    \end{exercise}
    \begin{solution}
    \end{solution}
    
    \begin{exercise} 
    \label{sea-1-33}
        Identify an interdiscipline and the disciplines from which it was derived.
    \end{exercise}
    \begin{solution}
    \end{solution}
    
    \begin{exercise} 
    \label{sea-1-35_36_38}
        For the following organizations, summarize their mission statements
        \begin{enumerate}[label=\alph*)]
            \item \href{http://isss.org/world/index.php}{International Society for the Systems Sciences}
            \item \href{https://www.incose.org/}{International Council on Systems Engineering}
            \item \href{https://omegalpha.org/}{Omega Alpha Association}
        \end{enumerate}
    \end{exercise}
    \begin{solution}
    \end{solution}
    
    \begin{exercise} 
    \label{sea-1-37}
        Explain how the goals of ISSS and INCOSE differ.
    \end{exercise}
    \begin{solution}
    \end{solution}

\end{exercises}
% SKIPPED
% sea-1-18
% sea-1-20
% sea-1-31
% sea-1-34

%------------------------------------------------

%% SEA CHAPTER 2 - BRINGING SYSTEMS INTO BEING
\chapter{BRINGING SYSTEMS INTO BEING}

% SEA Question Location in \label{sea-Chapter#-Problem#}
\begin{exercises}
    \begin{exercise}
    \label{sea-2-1}
        Identify at least two characteristics that distinguish natural systems from those that are human-made or human-modified. 
    \end{exercise}
    \begin{solution}
    \end{solution}
    
    \begin{exercise}
    \label{sea-2-2}
        Identify at least two possible interfaces between the natural and human worlds for the creation of any arbitrary system.
    \end{exercise}
    \begin{solution}
    \end{solution}
    
    \begin{exercise}
    \label{sea-2-3}
        Describe what distinguishes human-made from human-modified systems.
    \end{exercise}
    \begin{solution}
    \end{solution}
    
    \begin{exercise}
    \label{sea-2-5_6}
        Based on the textbook descriptions in ???, identify a system for the following system types and justify your answer based on the system product(s).
        \begin{enumerate}[label=\alph*)]
            \item Single-entity product system.
            \item Multiple-entity product system.
        \end{enumerate}
    \end{exercise}
    \begin{solution}
    \end{solution}
    
    \begin{exercise}
    \label{sea-2-7_8}
        Choose any consumer item and identify the producer. Then
        \begin{enumerate}[label=\alph*)]
            \item Identify at least three things that are employed or consumed by the producer to create the consumer item.
            \item Identify any enabling system which is employed by the producer to create the consumer item.
            \item Must an enabling system and a product of that system be engineered jointly? Why or why not?
        \end{enumerate}
    \end{exercise}
    \begin{solution}
    \end{solution}
    
    \begin{exercise} 
    \label{sea-2-9}
        Identify at least four factors which determine a product's competitiveness. Is product competitiveness important? Why or why not?
    \end{exercise}
    \begin{solution}
        The quality, price, ergonomics, and lifetime of a product would all be considered to influence the purchasing behavior of consumers, which is based on product competition.
    \end{solution}
    
    \begin{exercise}
    \label{sea-2-10}
        Explain the benefits of system life-cycle thinking and why it is important. 
    \end{exercise}
    \begin{solution}
    \end{solution}
    
    \begin{exercise}
    \label{sea-2-11}
        For the product life-cycles displayed in figures ???, ???, identify possible sources of feedback or communication between the different phases.
    \end{exercise}
    \begin{solution}
    \end{solution}
    
    \begin{exercise}
    \label{sea-2-12}
        Pretend that you are developing a product. How do you convince your peer(s) that the best approach is to \say{design for the life cycle}?
    \end{exercise}
    \begin{solution}
    \end{solution}
    
    \begin{exercise}
    \label{sea-2-14} 
        Identify one benefit and one consequence of the following design models
        \begin{enumerate}[label=\alph*)]
            \item Waterfall model.
            \item Spiral model.
            \item V-model.
        \end{enumerate}
    \end{exercise}
    \begin{solution}
    \end{solution}
    
    \begin{exercise}
    \label{sea-2-14_part2}
        Of the models described in \ref{sea-2-14}, do you have a personal preference? Is any one model always the most appropriate? Why or why not? 
    \end{exercise}
    \begin{solution}
    \end{solution}
    
    \begin{exercise}
    \label{sea-2-15_16}
        Select a design situation of your choice.
        \begin{enumerate}[label=\alph*)]
            \item What are the requirements of the system?
            \item Describe the steps for identifying appropriate technical performance measures.
            \item Describe the relationship between the system requirements and its technical performance measures.
            \item What happens when the technical performance measures disagree with the system requirements?
        \end{enumerate}
    \end{exercise}
    \begin{solution}
    \end{solution}
    
    \begin{exercise}
    \label{sea-2-18_20}
        Select a design situation of your choice and refer to figure ??? for the following questions.
        \begin{enumerate}[label=\alph*)]
            \item Identify a top-level requirement and decompose it appropriately.
            \item Identify a lowest-level requirement and explain its relevance for all higher levels.
        \end{enumerate}
    \end{exercise}
    \begin{solution}
    \end{solution}
    
    \begin{exercise}
    \label{sea-2-22}
        Identify at least three domain manifestations of systems engineering.  
    \end{exercise}
    \begin{solution}
    \end{solution}
    
    \begin{exercise}
    \label{sea-2-23}
        Identify at least three obstructions which hinder or prevent the application of systems engineering.
    \end{exercise}
    \begin{solution}
    \end{solution}
    
    \begin{exercise}
    \label{sea-2-24}
        Why is systems thinking and engineering beneficial?
    \end{exercise}
    \begin{solution}
    \end{solution}

\end{exercises}
% SKIPPED
% sea-2-4
% sea-2-13
% sea-2-17
% sea-2-19
% sea-2-21
% sea-2-25
% sea-2-26

%------------------------------------------------

%% SEA CHAPTER 3 - CONCEPTUAL SYSTEM DESIGN
\chapter{CONCEPTUAL SYSTEM DESIGN}

% SEA Question Location in \label{sea-Chapter#-Problem#}
\begin{exercises}
    \begin{exercise}
    \label{sea-3-1}
    
    \end{exercise}
    \begin{solution}
    \end{solution}

\end{exercises}
% SKIPPED

%------------------------------------------------

%% SEA CHAPTER 4 - PRELIMINARY SYSTEM DESIGN
\chapter{PRELIMINARY SYSTEM DESIGN}

% SEA Question Location in \label{sea-Chapter#-Problem#}
\begin{exercises}
    \begin{exercise}
    \label{sea-4-1}
    
    \end{exercise}
    \begin{solution}
    \end{solution}

\end{exercises}
% SKIPPED

%------------------------------------------------

%% SEA CHAPTER 5 - DETAIL DESIGN AND DEVELOPMENT
\chapter{DETAIL DESIGN AND DEVELOPMENT}

% SEA Question Location in \label{sea-Chapter#-Problem#}
\begin{exercises}
    \begin{exercise}
    \label{sea-5-1}
    
    \end{exercise}
    \begin{solution}
    \end{solution}

\end{exercises}
% SKIPPED

%------------------------------------------------

%% SEA CHAPTER 6 - SYSTEM TEST, EVALUATION, AND VALIDATION
\chapter{SYSTEM TEST, EVALUATION, AND VALIDATION}

% SEA Question Location in \label{sea-Chapter#-Problem#}
\begin{exercises}
    \begin{exercise}
    \label{sea-6-1}
    
    \end{exercise}
    \begin{solution}
    \end{solution}

\end{exercises}
% SKIPPED

%------------------------------------------------

%% SEA CHAPTER 7 - ALTERNATIVES AND MODELS IN DECISION MAKING
\chapter{ALTERNATIVES AND MODELS IN DECISION MAKING}

% SEA Question Location in \label{sea-Chapter#-Problem#}
\begin{exercises}
    \begin{exercise}
    \label{sea-7-1}
    
    \end{exercise}
    \begin{solution}
    \end{solution}

\end{exercises}
% SKIPPED

%------------------------------------------------

%% SEA CHAPTER 8 - MODELS FOR ECONOMIC EVALUATION
\chapter{MODELS FOR ECONOMIC EVALUATION}

% SEA Question Location in \label{sea-Chapter#-Problem#}
\begin{exercises}
    \begin{exercise}
    \label{sea-8-1}
        How much money must be invested to accumulate \$10,000 in 8 years at 6\% compounded annually?
    \end{exercise}
    \begin{solution}
    \begin{equation}
        P=P/F,6,8=\$10,000
    \end{equation}
    \end{solution}
    
    \begin{exercise}
    \label{sea-8-2}
        What amount will be accumulated by each of the following investments?
        \begin{enumerate}[label=\alph*)]
            \item \$8,000 at 7.2\% compounded annually over 10 years.
            \item \$52,000 at 8\% compounded annually over 5 years.
        \end{enumerate}
    \end{exercise}
    \begin{solution}
    \end{solution}
    
    \begin{exercise}
    \label{sea-8-3}
        What is the present equivalent amount of a year-end series of receipts of \$6,000 over 5 years at 8\% compounded annually?
    \end{exercise}
    \begin{solution}
    \end{solution}
    
    \begin{exercise}
    \label{sea-8-4}
        What is the present equivalent of a year-end series of receipts starting with a first-year base of \$1,000 and increasing by 8\% per year to year 20 with an interest rate of 8\%?
    \end{exercise}
    \begin{solution}
    \end{solution}
    
    \begin{exercise}
    \label{sea-8-5}
        What is the present equivalent of a year-end series of receipts starting with a first-year base of \$1 million and decreasing by 25\% per year to year 4 with an interest rate of 6\%?
    \end{exercise}
    \begin{solution}
    \end{solution}
    
    \begin{exercise}
    \label{sea-8-6}
        What interest rate compounded annually is involved if \$4,000 results in \$10,000 in 6 years?
    \end{exercise}
    \begin{solution}
    \end{solution}
    
    \begin{exercise}
    \label{sea-8-7}
        How many years will it take for \$4,000 to grow to \$7,000 at an interest rate of 10\% compounded annually?
    \end{exercise}
    \begin{solution}
    \end{solution}
    
    \begin{exercise}
    \label{sea-8-8}
        What interest rate is necessary for a sum of money to double itself in 8 years? What is the approximate product of $i$ and $n$ ($i$ as an integer) that establishes the doubling period? How accurate is this product of $i$ and $n$ for estimating the doubling period?
    \end{exercise}
    \begin{solution}
    \end{solution}
    
    \begin{exercise}
    \label{sea-8-9}
        An asset was purchased for \$52,000 with the anticipation that it would serve for 12 years and be worth \$6000 as scrap. After 5 years of operation, the asset was sold for \$18,000. The interest rate is 14\%.
        \begin{enumerate}[label=\alph*)]
            \item What was the anticipated annual equivalent cost of the asset?
            \item What was the actual annual equivalent cost of the asset?
        \end{enumerate}
    \end{exercise}
    \begin{solution}
    \end{solution}
    
    \begin{exercise}
    \label{sea-8-10}
        An epoxy mixer purchased for \$33,000 has an estimated salvage value of \$5,000 and an expected life of 3 years. An average of 200 pounds per month will be processed by the mixer.
        \begin{enumerate}[label=\alph*)]
            \item Calculate the annual equivalent cost of the mixer with an interest rate of 8\%.
            \item Calculate the annual equivalent cost per pound mixed with an interest rate of 12\%.
        \end{enumerate}
    \end{exercise}
    \begin{solution}
    \end{solution}
    
    \begin{exercise}
    \label{sea-8-11}
        The table below shows the receipts and disbursements for a given venture. Determine the desirability of the venture for a 14\% interest rate, based on the present equivalent comparison and the annual equivalent comparison.
        \begin{table}[h]
        \centering
        \begin{tabular}{c r r}
        \toprule
        \textbf{End of the Year} & \textbf{Receipts (\$)} & \textbf{Disbursements (\$)}\\
        \midrule
        0 & 0 & 20,000 \\
        1 & 6,000 & 0 \\
        2 & 5,000 & 4,000 \\
        3 & 5,000 & 0 \\
        4 & 12,000 & 1,000 \\
        \bottomrule
        \end{tabular}
        %\caption{Table caption}
        \label{tab:example} % Unique label used for referencing the table in-text
        %\addcontentsline{toc}{table}{Table \ref{tab:example}} % Uncomment to add the table to the table of contents
        \end{table}
    \end{exercise}
    \begin{solution}
    \end{solution}
    
    \begin{exercise}
    \label{sea-8-12}
        A microcomputer-based controller can be installed for \$30,000 and will have a \$3,000 salvage value after 10 years and is expected to decrease energy consumption cost by \$4,000 per year.
        \begin{enumerate}[label=\alph*)]
            \item What rate of return is expected if the controller is used for 10 years?
            \item For what life will the controller give a return of 15\%?
        \end{enumerate}
    \end{exercise}
    \begin{solution}
    \end{solution}
    
    \begin{exercise}
    \label{sea-8-13}
        Transco plans on purchasing a bus for \$75,000 that will have a capacity of 40 passengers. As an alternative, a larger bus can be purchased for \$95,000 which will have a capacity of 50 passengers. The salvage value of either bus is estimated to be \$8,000 after a 10-year life. If an annual net profit of \$400 can be realized per passenger, which alternative should be recommended using a management-suggested interest rate of 15\%? Using the actual cost of money at 7.5\%?
    \end{exercise}
    \begin{solution}
    \end{solution}
    
    \begin{exercise}
    \label{sea-8-14}
        An office building and its equipment are insured to \$7,100,000. The present annual insurance premium is \$0.85 per \$100 of coverage. A sprinkler system with an estimated life of 20 years and no salvage value can be installed for \$180,000. Annual maintenance and operating cost is estimated to be \$3,600. The premium will be reduced to \$0.40 per \$100 coverage if the sprinkler system is installed.
        \begin{enumerate}[label=\alph*)]
            \item Find the rate of return if the sprinkler system is installed.
            \item With interest at 12\%, find the payout period for the sprinkler system.
        \end{enumerate}
    \end{exercise}
    \begin{solution}
    \end{solution}
    
    \begin{exercise}
    \label{sea-8-15}
        The design of a system is to be pursued from one of two available alternatives. Each alternative has a life-cycle cost associated with an expected future. The costs for the corresponding futures are given in the table below (in millions of dollars). If the probabilities of occurrence of the futures are 30\%, 50\%, and 20\%, respectively, which alternative is most desirable from an expected cost viewpoint, using an interest rate of 10\%?
        \begin{table}[h]
        \centering
        \begin{tabular}{l D{.}{.}{1} D{.}{.}{1} D{.}{.}{1} D{.}{.}{1} D{.}{.}{1} D{.}{.}{1} D{.}{.}{1} D{.}{.}{1} D{.}{.}{1} D{.}{.}{1} D{.}{.}{1} D{.}{.}{1}}
        \toprule
        \textbf{Design 1} & \multicolumn{12}{c}{\textbf{Years}} \\
        \midrule
        Future & 1 & 2 & 3 & 4 & 5 & 6 & 7 & 8 & 9 & 10 & 11 & 12 \\
        \midrule
        Optimistic & 0.4 & 0.6 & 5.0 & 7.0 & 0.8 & 0.8 & 0.8 & 0.8 & 0.8 & 0.8 & 0.8 & 0.8 \\
        Expected & 0.6 & 0.8 & 1.0 & 5.0 & 10.0 & 1.0 & 1.0 & 1.0 & 1.0 & 1.0 & 1.0 & 1.0 \\
        Pessimistic & 0.8 & 0.9 & 1.0 & 7.0 & 10.0 & 1.2 & 1.2 & 1.2 & 1.2 & 1.2 & 1.2 & 1.2 \\
        \midrule
        \textbf{Design 2} & \multicolumn{12}{c}{\textbf{Years}} \\
        \midrule
        Future & 1 & 2 & 3 & 4 & 5 & 6 & 7 & 8 & 9 & 10 & 11 & 12 \\
        \midrule
        Optimistic & 0.4 & 0.4 & 0.4 & 1.0 & 3.0 & 2.5 & 2.5 & 2.5 & 2.5 & 2.5 & 2.5 & 2.5 \\
        Expected & 0.6 & 0.8 & 1.0 & 3.0 & 6.0 & 3.0 & 3.0 & 3.0 & 3.0 & 3.0 & 3.0 & 3.0 \\
        Pessimistic & 0.6 & 0.8 & 1.0 & 5.0 & 6.0 & 3.1 & 3.1 & 3.1 & 3.1 & 3.1 & 3.1 & 3.1 \\
        \bottomrule
        \end{tabular}
        %\caption{Table caption}
        \label{tab:sea-8-15} % Unique label used for referencing the table in-text
        %\addcontentsline{toc}{table}{Table \ref{tab:example}} % Uncomment to add the table to the table of contents
        \end{table}
    \end{exercise}
    \begin{solution}
    \end{solution}
    
    \begin{exercise}
    \label{sea-8-16}
        Prepare a decision evaluation matrix for the design alternatives in \ref{sea-8-15}, and then choose the alternative that is best under the following decision rules: Laplace, maximax, maximin, and Hurwicz with $\alpha=0.6$. Assume that the choice is under uncertainty.
    \end{exercise}
    \begin{solution}
    \end{solution}
    
    \begin{exercise}
    \label{sea-8-17}
        A campus laboratory can be climate conditioned by piping chilled water from a central refrigeration plant. Two competing proposals are being considered for the piping system, as outlined in the table. On the basis of a 10-year life, find the number of hours of operation per year for which the cost of the two systems will be equal if the interest rate is 9\%.
        \begin{table}[h]
        \centering
        \begin{tabular}{l D{.}{.}{2} D{.}{.}{2}}
        \toprule
        {} & \textbf{6" System} & \textbf{8" System}\\
        \cmidrule{2-3}
        {} & \multicolumn{2}{c}{Horsepower} \\
        \cmidrule{2-3}
        Motor size & 6 & 3 \\
        \midrule
        {} & \multicolumn{2}{c}{Cost (\$)} \\
        \cmidrule{2-3}
        Pump and pipe installation & 32,000 & 44,000 \\
        Motor installation & 4,500 & 3,000 \\
        Energy per hour of operation & 3.20 & 2.00 \\
        \midrule
        Salvage value & 5,000 & 6,000 \\
        \bottomrule
        \end{tabular}
        %\caption{Table caption}
        \label{tab:sea-8-17} % Unique label used for referencing the table in-text
        %\addcontentsline{toc}{table}{Table \ref{tab:example}} % Uncomment to add the table to the table of contents
        \end{table}
    \end{exercise}
    \begin{solution}
    \end{solution}
    
    \begin{exercise}
    \label{sea-8-18}
        Replacement fence posts for a cattle ranch are currently purchased for \$4.20 each. It is estimated that equivalent posts can be cut from timber on the ranch for a variable cost of \$1.50 each, which is made up of the value of the timber plus labor cost. Annual fixed cost for required equipment is estimated to be \$1,200. If 1,000 posts will be required each year, What will be the annual saving if posts are cut?
    \end{exercise}
    \begin{solution}
    \end{solution}
    
    \begin{exercise}
    \label{sea-8-19}
        An equipment operator can buy a maintenance component from a supplier for \$960 per unit delivered. Alternatively, operator can rebuild the component for a variable cost of \$460 per unit. It is estimated that the additional fixed cost would be \$80,000 per year if the component is rebuilt. Find the number of units per year for which the cost of the two alternatives will break even.
    \end{exercise}
    \begin{solution}
    \end{solution}
    
    \begin{exercise}
    \label{sea-8-20}
        A marketing company can lease a fleet of automobiles for its sales personnel for \$35 per day plus \$0.18 per mile for each vehicle. As an alternative, the company can pay each salesperson \$0.45 per mile to use his or her own automobile. If these are the only costs to the company, how many miles per day must a salesperson drive for the two alternatives to break even?
    \end{exercise}
    \begin{solution}
    \end{solution}
    
    \begin{exercise}
    \label{sea-8-21}
        An electronics manufacturer is considering the purchase of one of two types of laser trimming devices. The sales forecast indicated that at least 8,000 units will be sold per year. Device A will increase the annual fixed cost of the plant by \$20,000 and will reduce variable cost by \$5.60 per unit. Device B will increase the annual fixed cost by \$5,000 and will reduce variable cost by \$3.60 per unit. If variable costs are now \$20 per unit produced, which device should be purchased?
    \end{exercise}
    \begin{solution}
    \end{solution}
    
    \begin{exercise}
    \label{sea-8-22}
        Machine A costs \$20,000, has zero salvage value at any time, and has an associated labor cost of \$1.15 for each piece produced on it. Machine B costs \$36,000, has zero salvage value at any time, and has an associated labor cost of \$0.90. Neither machine can be used except to produce the product described. If the interest rate is 10\% and the annual rate of production is 20,000 units, how many years will it take for the cost of the two machines to break even?
    \end{exercise}
    \begin{solution}
    \end{solution}
    
    \begin{exercise}
    \label{sea-8-23}
        An electronics manufacturer is considering two methods for producing a circuit board. The board can be hand-wired at an estimated cost of \$9.80 per unit and an annual fixed equipment cost of \$10,000. A printed equivalent can be produced using equipment costing \$180,000 with a service life of 8 years and salvage value of \$12,000. It is estimated that the labor cost will be \$3.20 per unit and that the processing equipment will cost \$4,000 per year to maintain. If the interest rate is 8\%, how many circuit boards must be produced each year for the two methods to break even?
    \end{exercise}
    \begin{solution}
    \end{solution}
    
    \begin{exercise}
    \label{sea-8-24}
        It is estimated that the annual sales of labor-saving device will be 20,000 the first year and increase by 10,000 per year until 50,000 units are sold during the fourth year. Proposal A is to purchase manufacturing equipment costing \$120,000 with an estimated salvage value of \$15,000 at the end of 4 years. Proposal B is to purchase equipment costing \$280,000 with an estimated salvage value of \$32,000 at the end of 4 years. The variable manufacturing cost per unit under proposal A is estimated to be \$8.00, but is estimated to be only \$0.26 under proposal B. If the interest rate is 9\%, which proposal should be accepted for a 4-year production period?
    \end{exercise}
    \begin{solution}
    \end{solution}
    
    \begin{exercise}
    \label{sea-8-25}
        The fixed operating cost of a machine center (capital recovery, interest, maintenance, space charges, supervision, insurance, and taxes) is $F$ dollars per year. The variable cost of operating the center (power, supplies, and other items, but excluding direct labor) is $V$ dollars per hour of operation. If $N$ is the number of hours the center is operated per year, $TC$ the annual total cost of operating the center, $TC_h$ the hourly cost of operating the center, $t$ the time in hours to process 1 unit of product, and $M$ the center cost of processing 1 unit, write expressions for the following
        \begin{enumerate}[label=\alph*)]
            \item $TC$
            \item $TC_h$
            \item $M$
        \end{enumerate}
    \end{exercise}
    \begin{solution}
    \end{solution}
    
    \begin{exercise}
    \label{sea-8-26}
        In \ref{sea-8-25}, $F=\$60,000$ per year, $t=0.2$ hour, $V=\$50$ per hour, and $N$ varies from 1,000 to 10,000 in increments of 1,000.
        \begin{enumerate}[label=\alph*)]
            \item Plot values of $M$ as a function of $N$.
            \item Write an expression for the total cost of direct labor and machine cost per unit $TC_h$ using the symbols in \ref{sea-8-25} and letting $W$ equal the hourly cost of direct labor.
        \end{enumerate}
    \end{exercise}
    \begin{solution}
    \end{solution}
    
    \begin{exercise}
    \label{sea-8-27}
        A certain firm has the capacity to produce 800,000 units per year. At present it is operating at 75\% of capacity. The income per unit is \$0.10 regardless of output. Annual fixed costs are \$28,000, and the variable cost is \$0.06 per unit. Find the annual profit or loss at this capacity and the capacity for which the firm will break even.
    \end{exercise}
    \begin{solution}
    \end{solution}
    
    \begin{exercise}
    \label{sea-8-28}
        An arc welding machine that is used for a certain joining process costs \$90,000. The machine has a life of 5 years and a salvage value of \$10,000. Maintenance, taxes, insurance, and other fixed costs amount to \$5,000 per year. The cost of power and supplies is \$28.00 per hour of operation and the total operator cost (direct and indirect) is \$65.00 per hour. If the cycle time per unit of product is 60 min and the interest rate is 8\%, calculate the cost per unit for the following unit outputs per year.
        \begin{enumerate}[label=\alph*)]
            \item 200 units
            \item 600 units
            \item 1,800 units
        \end{enumerate}
    \end{exercise}
    \begin{solution}
    \end{solution}
    
    \begin{exercise}
    \label{sea-8-29}
        A certain processing center has the capacity to assemble 650,000 units per year. At present, it is operating at 65\% of capacity. The annual income is \$416,000. Annual fixed costs are \$192,000 and the variable costs are \$0.38 per unit assembled.
        \begin{enumerate}[label=\alph*)]
            \item What is the annual profit or loss attributable to the center?
            \item At what volume of output does the center break even?
            \item What will be the profit or loss at 70\%, 80\%, and 90\% of capacity on the basis of constant income per unit and constant variable cost per unit?
        \end{enumerate}
    \end{exercise}
    \begin{solution}
    \end{solution}
    
    \begin{exercise}
    \label{sea-8-30}
        Chemco operates two plants, A and B, which produce the same product. The capacity of plant A is 60,000 gallons while that of B is 80,000 gallons. The annual fixed cost of plant A is \$2,600,000 per year and the variable cost is \$32 per gallon. The corresponding values for plant B are \$2,800,000 and \$39 per gallon. At present, plant A is being operated at 35\% of capacity and plant B is being operated at 40\% of capacity.
        \begin{enumerate}[label=\alph*)]
            \item What would be the total cost of production of plants A and B?
            \item What are the total cost and the average unit cost of the total output of both plants?
            \item What would be the total cost to the company and cost per gallon if all production were transferred to plant A?
            \item What would be the total cost to the company and cost per gallon if all production were transferred to plant B?
        \end{enumerate}
    \end{exercise}
    \begin{solution}
    \end{solution}

\end{exercises}
% SKIPPED

%------------------------------------------------

%% SEA CHAPTER 9 - OPTIMIZATION IN DESIGN AND OPERATIONS
\chapter{OPTIMIZATION IN DESIGN AND OPERATIONS}

% SEA Question Location in \label{sea-Chapter#-Problem#}
\begin{exercises}
    \begin{exercise}
    \label{sea-9-1}
        Specify the dimensions of the sides of a rectangle of perimeter $p$ so that the area it encloses will be maximum.
    \end{exercise}
    \begin{solution}
    \end{solution}
    
    \begin{exercise}
    \label{sea-9-2}
        The cost per unit produced at a certain facility is represented by the function
        \begin{equation}
            UC = 2x^2-10x+50
        \end{equation}
        where $x$ is in thousands of units produced. For what value of $x$ would unit cost be minimized (other than zero)? What is the minimum cost at this volume? Show that the value found is truly a minimum.
    \end{exercise}
    \begin{solution}
    \end{solution}
    
    \begin{exercise}
    \label{sea-9-3}
        Advertising expenditures have been found to relate to profit approximately in accordance with the function
        \begin{equation}
            P = x^3-100x^2+3,125x
        \end{equation}
        where $x$ is the expenditure in thousands of dollars. What advertising expenditure would produce the maximum profit? What profit is expected at this expenditure? Show that the derived result is truly a maximum.
    \end{exercise}
    \begin{solution}
    \end{solution}
    
    \begin{exercise}
    \label{sea-9-4}
        The cost of producing and selling a certain item is for the first 1,000 units, for a production range between 1,000 and 2,500 units, and $\$205x-\$97.500$ for more than 2,500 units, where $x$ is the number of units produced. If the selling price is \$200 per unit, and all units produced are sold,find the level of production that will maximize profit.
    \end{exercise}
    \begin{solution}
    \end{solution}
    
    \begin{exercise}
    \label{sea-9-5}
        Ethyl acetate is made from acetic acid and ethyl alcohol. Let $x=$ pounds of acetic acid input, $y=$ of ethyl alcohol input, and $z=$ of ethyl acetate output. The relationship of output to input is
        \begin{equation}
            \frac{z^2}{(1.47x-z)(1.91y-z)}=3.9
        \end{equation}
        \begin{enumerate}[label=\alph*)]
            \item Determine the output of ethyl acetate per pound of acetic acid,where the ratio of acetic acid of ethyl alcohol is 2.0,1.0,and 0.67,and graph the result.
            \item Graph the cost of material per pound of ethyl acetate for each of the ratios given and determine the ratio for which the material cost per pound of ethyl acetate is a minimum if acetic acid costs \$0.80 per pound and ethyl alcohol costs \$0.92 per pound.
        \end{enumerate}
    \end{exercise}
    \begin{solution}
    \end{solution}
    
    \begin{exercise}
    \label{sea-9-6}
        It has been found that the heat loss through the ceiling of a building is 0.13 Btu per hour per square foot of area per degree Fahrenheit. If the $2,200ft^2$ ceiling is insulated,the heat loss in Btu per hour per degree temperature difference per square foot of area is taken as
        \begin{equation}
            \frac{1}{(\frac{1}{0.13})(\frac{t}{0.27})}
        \end{equation}
        where tis the thickness in inches.The in-place cost of insulation 2,4,and 6 in thick is \$0.18, \$0.30,and \$0.44 per square foot,respectively.The building is heated to 75$^{\circ}$F 3,000 hrs per year by a gas furnace with an efficiency of 50\%.The mean outside temperature is 45°F and the natural gas used in the furnace costs \$4.40 per and has a heating value of 2,000 Btu per What thickness of insulation, if any, should be used if the interest rate is 10\% and the resale value of the building 6 years hence is enhanced \$850 if insulation is added,regardless of the thickness?
    \end{exercise}
    \begin{solution}
    \end{solution}
    
    \begin{exercise}
    \label{sea-9-7}
        An overpass is being considered for a certain crossing. The superstructure design under consideration will be made of reinforced concrete and will have a weight per foot depending on the span between piers in accordance with $W=32(S)+1,850$. Piers will be made of steel and will cost \$250,000 each. The superstructure will be erected at a cost of \$3.20 per pound. If the number of piers required is to be one less than the number of spans, find the number of piers that will result in a minimum total cost for piers and superstructure if $L=1,250$ft. 
    \end{exercise}
    \begin{solution}
    \end{solution}
    
    \begin{exercise}
    \label{sea-9-8}
        Two girder designs are under consideration for a bridge for a 1,200-foot crossing.The first is expected to result in a superstructure weight per foot of $22(S)+800$, where $S$ is the span between piers. The second should result in superstructure weight per foot of $20(S)+1,000$. Piers and two required abutments are estimated to cost \$220,000 each. The superstructure will be erected at a cost of \$0.55 per pound. Choose the girder design that will result in a minimum cost and specify the optimum number of piers.
    \end{exercise}
    \begin{solution}
    \end{solution}
    
    \begin{exercise}
    \label{sea-9-9}
        What is the cost advantage of choosing the best girder design for the bridge described in \ref{sea-9-8}? If the number of piers is determined from the best girder design alternative, but the other design alternative is adopted, what cost penalty is incurred?
    \end{exercise}
    \begin{solution}
    \end{solution}
    
    \begin{exercise}
    \label{sea-9-10}
        A used automobile can be purchased by a student to provide transportation to and from school for \$5,500 as is (i.e.,the auto will have no warranty). First-year maintenance cost is expected to be \$350 and the maintenance costs will increase by \$100 per year thereafter. Operation costs for the automobile will be \$1,200 for every year the auto is used and its salvage value decreases by 15\% per year.
        \begin{enumerate}[label=\alph*)]
            \item What is the economic life without considering the time value of money?
            \item With interest at 16\%,what is the economic life?
        \end{enumerate}
    \end{exercise}
    \begin{solution}
    \end{solution}
    
    

\end{exercises}
% SKIPPED

%------------------------------------------------

%% SEA CHAPTER 10 - QUEUING THEORY AND ANALYSIS

% SEA Question Location in \label{sea-Chapter#-Problem#}
\begin{exercises}
    \begin{exercise}
    \label{sea-10-1}
    
    \end{exercise}
    \begin{solution}
    \end{solution}

\end{exercises}
% SKIPPED

%------------------------------------------------

%% SEA CHAPTER 11 - CONTROL CONCEPTS AND METHODS

% SEA Question Location in \label{sea-Chapter#-Problem#}
\begin{exercises}
    \begin{exercise}
    \label{sea-11-1}
    
    \end{exercise}
    \begin{solution}
    \end{solution}

\end{exercises}
% SKIPPED

%------------------------------------------------

%% SEA CHAPTER 12 - DESIGN FOR OPERATIONAL FEASIBILITY

% SEA Question Location in \label{sea-Chapter#-Problem#}
\begin{exercises}
    \begin{exercise}
    \label{sea-12-1}
    
    \end{exercise}
    \begin{solution}
    \end{solution}

\end{exercises}
% SKIPPED

%------------------------------------------------

%% SEA CHAPTER 13 - DESIGN FOR MAINTAINABILITY

% SEA Question Location in \label{sea-Chapter#-Problem#}
\begin{exercises}
    \begin{exercise}
    \label{sea-13-1}
    
    \end{exercise}
    \begin{solution}
    \end{solution}

\end{exercises}
% SKIPPED

%------------------------------------------------

%% SEA CHAPTER 14 - DESIGN FOR USABILITY (HUMAN FACTORS)

% SEA Question Location in \label{sea-Chapter#-Problem#}
\begin{exercises}
    \begin{exercise}
    \label{sea-14-1}
    
    \end{exercise}
    \begin{solution}
    \end{solution}

\end{exercises}
% SKIPPED

%------------------------------------------------

%% SEA CHAPTER 15 - DESIGN FOR LOGISTICS AND SUPPORTABILITY

% SEA Question Location in \label{sea-Chapter#-Problem#}
\begin{exercises}
    \begin{exercise}
    \label{sea-15-1}
    
    \end{exercise}
    \begin{solution}
    \end{solution}

\end{exercises}
% SKIPPED

%------------------------------------------------

%% SEA CHAPTER 16 - DESIGN FOR PRODUCIBILITY, DISPOSABILITY, AND SUSTAINABILITY

% SEA Question Location in \label{sea-Chapter#-Problem#}
\begin{exercises}
    \begin{exercise}
    \label{sea-16-1}
    
    \end{exercise}
    \begin{solution}
    \end{solution}

\end{exercises}
% SKIPPED

%------------------------------------------------

%% SEA CHAPTER 17 - DESIGN FOR AFFORDABILITY (LIFE-CYCLE COSTING)

% SEA Question Location in \label{sea-Chapter#-Problem#}
\begin{exercises}
    \begin{exercise}
    \label{sea-17-1}
    
    \end{exercise}
    \begin{solution}
    \end{solution}

\end{exercises}
% SKIPPED

%------------------------------------------------

%% SEA CHAPTER 18 - SYSTEMS ENGINEERING PLANNING AND ORGANIZATION

% SEA Question Location in \label{sea-Chapter#-Problem#}
\begin{exercises}
    \begin{exercise}
    \label{sea-18-1}
    
    \end{exercise}
    \begin{solution}
    \end{solution}

\end{exercises}
% SKIPPED

%------------------------------------------------

%% SEA CHAPTER 19 - PROGRAM MANAGEMENT, CONTROL, AND EVALUATION

% SEA Question Location in \label{sea-Chapter#-Problem#}
\begin{exercises}
    \begin{exercise}
    \label{sea-19-1}
    
    \end{exercise}
    \begin{solution}
    \end{solution}

\end{exercises}
% SKIPPED

%------------------------------------------------